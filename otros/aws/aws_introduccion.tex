\chapter{Introducción}

Los servicios que utilizamos al navegar por internet, al consultar páginas web, o al utilizar \textit{apps} en el móvil, realmente están alojados en un conjunto de ordenadores (denominados \textbf{servidores}), que están almacenados en un \textbf{CPD} (centro de procesamiento de datos).

Hace unos años había empresas pequeñas que contaban con sus propios servidores en sus oficinas, abaratando parte del coste del mantenimiento de los servidores, ya que no tenían que alquilar el espacio o el propio servidor. Si se quedaba pequeño, compraban uno nuevo, o reajustaban los recursos.

Con el avance de internet, han ido surgiendo distintos proveedores que ofrecen servicios para que las empresas puedan contratarlos en condiciones más favorables y en CPDs profesionales. Veremos diferentes tipos de computación y diferenciaremos distintos tipos de “computación en la nube”.



\chapter{Sistemas de computación propios}
Dado que tener el servidor en una oficina puede suponer distintos riesgos (pérdida de electricidad, robo de datos, acceso al servidor por parte de personas, tener la necesidad de tener IP estática en la oficina), muchas empresas deciden delegar la parte pública de sus servicios en empresas que cumplen con estándares de seguridad para tener servidores de manera segura.

\infobox{Hoy en día se sigue teniendo servidores en las oficinas, pero son para tareas internas, de desarrollo o de control de dominios}

Para los servicios externos, dependiendo del tamaño de la empresa, hoy en día lo habitual es delegar parte del mantenimiento o de los servicios en un proveedor externo.

Algunos ejemplos de sistemas de computación propios (cuando nos referimos a hardware propio) puede ser:

\begin{itemize}
    \item \textbf{CPD propio}: Esto sólo está al alcance de empresas muy grandes, que necesitan de muchos servidores y que tienen la necesidad de montar su propio CPD parar sus propios servicios.

    \item \textbf{Alquiler de un rack en un CPD profesional}: Existen empresas que se encargan de crear CPDs donde alquilan armarios donde poder colocar nuestros servidores físicos. La empresa nos proporcionará el espacio, el ancho de banda que contratemos, y puede que otros servicios extra de red (firewall perimetral, sistemas para apaciguar ataques de denegación de servicios...). Por otro lado, el hardware del servidor lo proporcionaremos nosotros, y cualquier posible rotura del mismo será cosa nuestra.

    En este tipo de sitios se necesita pedir cita previa para acceder a los servidores, y sólo ciertas personas pre-autorizadas previamente podrán entrar.

    \item \textbf{Alquiler de máquinas virtuales}: Con el \textbf{boom} de la virtualización, surgieron compañías que ofrecen la posibilidad de contratar unos recursos (con computación, RAM y almacenamiento limitados) que están virtualizados dentro de la infraestructura del proveedor. Esto hace que sea barato de contratar, pero toda la gestión del sistema operativo y del software sea por nuestra cuenta.
\end{itemize}


Dependiendo de las necesidades de la empresa, se hará uso de un sistema u otro. El problema en estos casos, es que normalmente era necesario un equipo de informáticos que tuviesen los conocimientos adecuados dependiendo del tipo de infraestructura que tuviésemos.

Aunque hoy en día este tipo de computación sigue siendo barato y funcional, el término “computación en la nube” ha hecho que ya no sea tan habitual contar con estos sistemas.



\chapter{Computación en la nube}

La computación en la nube se puede resumir como el uso de una red de servidores al que de manera remota se puede solicitar una serie de recursos que nos proporcionará de manera casi instantánea y sin la gestión activa por nuestra parte de hardware o software.

En lugar de solicitar un recurso físico como tal, “la nube”, que está compuesta por un conjunto de servidores que a través de la configuración


Por poner sólo unos ejemplos:




\begin{itemize}
    \item \textbf{Software as a Service} (SaaS): En este caso se contrata lo necesario para poder hacer uso de un software y todo los datos que se vayan a almacenar con él. El proveedor nos da acceso a una instancia del software que está configurada y que permitirá el uso con unos recursos limitados. De esta manera, no nos tendremos que preocupar del hardware en el que está instalado, o la propia instalación del software. Ejemplos: servicio blog; sistema ERP para una empresa; sistema de base de datos con alta disponibilidad y backups incluidos...

    \item \textbf{Platform as a service} (PaaS):
\end{itemize}