\newcommand{\ClassPath}{../../yukibook.cls}
\documentclass{\ClassPath/yukibook}


\begin{document}

    \yukibook{Introducción a Laravel} % Title
    {Rubén Gómez Olivencia}  % Author
    {2023}    % Year
    {} % Name of degree
    {} % catch phrase
    {} % the phrase's author
    {img/logo.png} %cover
    {ff5449}
    {} %mini-title

    \coverpage
    \graphicspath{{../../yukibook.cls/}}
    \licensepage
    %
    \tableofcontents

    %--------------------------------------------------------------------------
    % Start your parts, chapters and sections here
    %--------------------------------------------------------------------------
    \graphicspath{{img/}}

    \part{Introducción}
    \chapter{Introducción}

La \textbf{informática} es un área de la ciencia que abarca distintas disciplinas teóricas (como la creación de algoritmos, teoría de computación, teoría de la información, ...) y disciplinas prácticas (diseño de hardware, implementación de software).

A la hora de crear programas (o \textit{software}), podemos identificarlos de distintos tipos:
\begin{itemize}
    \item \textbf{Software de sistema}: Programas o aplicaciones que pertenecen al sistema y nos ayudan a mejorarlo, administrarlo ... Pueden ser aplicaciones de monitorización, de auditoría de logs, \textit{drivers}, ...

    \item \textbf{Software de desarrollo}: En este caso serán aplicaciones que nos ayudarán a crear otras aplicaciones. Por ejemplo: librerías de funciones, compiladores, \textit{debuggers}, IDEs...

    \item \textbf{Aplicaciones de usuario}: Son aplicaciones que los usuarios finales utilizarán en su día a día. Podríamos diferenciarlas como:
    \begin{itemize}
        \item \textbf{Aplicaciones generalistas}: Son aquellas que cualquier tipo de usuario utilizará en cualquier momento. Son creadas con un propósito específico, pero que no hay que tener grandes conocimientos para usarlas. Por ejemplo: navegadores web, clientes de correo, aplicaciones de ofimática simple, calculadora, calendario, ...

        \item \textbf{Aplicaciones de uso específico}: En este caso son aplicaciones creadas para un usuario específico, con una utilidad muy concreta y que normalmente deben existir conocimientos para utilizarla.

        Pueden ser aplicaciones no muy complejas, pero cuya utilidad, o lo que hagan, tenga importancia y conste de procesos complejos. Por ejemplo: aplicaciones CAD, sistemas de virtualización, aplicaciones científicas (R, JupyterLab), aplicaciones empresariales, ...
    \end{itemize}
\end{itemize}

En esta asignatura veremos distintos tipos de software especializado dentro de la gestión empresarial como son los ERP y los CRM, que podríamos englobar como \textbf{sistemas de información}.

\chapter{Sistemas de información}

Un sistema de información, de manera generalizada, es aquel que ayuda a administrar, recolectar, recuperar, procesar, almacenar y distribuir información relevante para ser usados dentro de los procesos fundamentales de una organización.

Normalmente estos sistemas de información son fáciles de usar, tienen cierto grado de flexibilidad (se pueden adaptar a las empresas), permiten guardar y recuperar información de manera rápida y sencilla.

De esta manera, la información resultante será más valiosa para la propia organización, ya que tendrá una “imagen” más amplia y habiendo podido relacionar más información que de no haber utilizado este tipo de software.

\section{Componentes}
Un sistema de información debe contar con los siguientes componentes básicos, que deben interactuar entre sí de manera adecuada para un buen funcionamiento global:
\begin{itemize}
    \item El \textbf{hardware}, equipo físico utilizado para procesar y almacenar datos.
    \item El \textbf{software} y los procedimientos utilizados para transformar y extraer información.
    \item Los \textbf{datos} que representan las actividades de la empresa.
    \item La \textbf{red} que permite compartir recursos entre computadoras y dispositivos.
    \item Las \textbf{personas} que desarrollan, mantienen y utilizan el sistema.
\end{itemize}

El último punto es muy importante, ya que de nada sirve tener la mejor herramienta, en el mejor hardware, si luego las personas que van a hacer uso de ella no tienen los conocimientos suficientes.

\errorbox{\textbf{Las personas que utilizan los sistemas de información deben tener los conocimientos adecuados para su correcta utilización.}}

Es por eso que las personas que hagan uso del sistema de información deberán ser entrenadas y/o tener manuales para su correcto uso, así como \textbf{también tener en cuenta sus opiniones para mejorar los procesos internos de la empresa}.

\section{Datos vs información}

Los datos reflejan hechos recogidos en la organización y que están todavía sin procesar (reflejan valores o resultados de mediciones). Estos datos serán hechos o cifras sobre algún tema específico concreto, que a simple vista no tienen por qué decir nada.

Por otro lado, la información se obtiene una vez se han procesado, agregado y/o presentado de manera adecuada esos datos para que puedan ser útiles y de esta manera obtener un valor que de otra manera no se podría obtener.

El ejemplo más claro entre datos e información se puede obtener en cualquier \textbf{estudio científico}, en el que a tras la obtención de unos datos, a través del método científico se llega una conclusión y con ello información.

Por ejemplo, mediciones del dióxido de carbono (CO$_{2}$) en la atmósfera, se obtienen  datos y se llega a la siguiente imagen que es la información:

\begin{center}
    \includegraphics[width=0.7\linewidth]{co2.png}
    \captionof{figure}{Mediciones de CO$_{2}$ en los últimos miles de años. Fuente: \href{https://climate.nasa.gov/en-espanol/signos-vitales/dioxido-de-carbono/}{NASA}.}
\end{center}



\section{Objetivo}

El objetivo de los sistemas de información, y en este caso, los utilizados para la gestión empresarial, es el de realizar acciones de manera más rápida y eficiente, por lo que también debería ser más económico para la empresa.

El uso de las tecnologías de la información y la comunicación en las empresas se ha convertido en un \textbf{elemento esencial como motor vertebrador y fuente de ventajas competitivas}.

Hoy en día una empresa que no haga uso de la informática está condicionando su estrategia empresarial, y es bastante probable que esté perdiendo oportunidades de negocio, así como la posibilidad de desarrollar sus productos y servicios.

Es por eso que el uso de la informática y de \textit{software} especializado de gestión empresarial puede ayudar a las empresas en:

\begin{itemize}
    \item Obtener ventajas competitivas.
    \item Mejorar la eficiencia interna de la empresa: reducir costes, mejorar la productividad, mejorar la organización de la información, ...
    \item Mejorar y facilitar la toma de decisiones a través de la recopilación de la información.
    \item Para desarrollar nuevas estrategias de negocios.
\end{itemize}

\section{Requisitos}

Para que la información sea útil en la toma de decisiones dentro de una organización, debe cumplir una serie de requisitos:

\begin{itemize}
    \item \textbf{Exactitud}: debe ser precisa y libre de errores.
    \item \textbf{Comprensión}: inteligible por el usuario.
    \item \textbf{Completitud}: debe contener todos aquellos hechos que pudieran ser importantes.
    \item \textbf{Economicidad}: el coste para obtener la información debe ser menor que el beneficio.
    \item \textbf{Confianza}: garantizar tanto la calidad de los datos utilizados, como la de las fuentes de información.
    \item \textbf{Relevancia}: ha de ser útil para la toma de decisiones.
    \item \textbf{Nivel de detalle}: se debe proporcionar con la presentación y el formato adecuados, para que resulte sencilla y fácil de manejar.
    \item \textbf{Oportunidad}: se debe entregar la información a la persona que corresponde y en el momento adecuado.
    \item \textbf{Verificabilidad}: la información ha de poder ser contrastada y comprobada en todo momento.
\end{itemize}

\warnbox{A tener en cuenta: \textbf{el exceso de información también puede ser contraproducente}.}

\section{Actividades}

A la hora de hacer uso de un sistema de información, las actividades que se pueden realizar con él se pueden resumir en:

\begin{itemize}
    \item \textbf{Recopilación}: Es la recogida de datos en bruto. Estos datos pueden ser de dentro de la organización, del exterior, recopilados de manera automática o de manera manual.
    \item \textbf{Almacenamiento}: Los datos deben ser guardados de manera estructurada para su posterior uso. Por otro lado, \textbf{nos debemos asegurar que su persistencia no corra peligro}, por lo que deberemos contar con un sistema de almacenamiento que sea capaz de asegurar posibles problemas. Para ello deberemos tener un sistema en \textbf{alta disponiblidad}, y con un buen sistema de \textbf{backups} configurado.

    También hay que asegurar que \textbf{el acceso a los datos estará limitado y asegurado mediante sistemas de control de acceso y de autenticación}.

    \item \textbf{Procesamiento}: Es el punto clave en el que los datos se convierten en información, de esta manera cumpliendo la labor de ayudar a la organización en la toma de decisiones.

    \item \textbf{Distribución}: El sistema permitirá distribuir la información entre las personas que la necesiten.
\end{itemize}


\section{Tipos de sistemas de información}

Aunque existen distintos tipos de sistemas de información, y su clasificación se puede realizar teniendo en cuenta distintas funcionalidades y/o objetivos, nos vamos a centrar en dos tipos:

\begin{itemize}
    \item \textbf{ERP}: \textit{Enterprise Resource Planning} o planificación de recursos en la empresa. Se trata de los sistemas de gestión integrados que permiten dar soporte a la totalidad de los procesos de una empresa: control económico financiero, logística, producción, mantenimiento, recursos humanos, ...

    \item \textbf{CRM}: \textit{Customer Relationship Management}, sistemas para gestionar las relaciones con los clientes y el soporte a todos los contactos comerciales.
\end{itemize}
    \chapter{Modelo-Vista-Controlador}

La arquitectura \textbf{Modelo-Vista-Controlador} es un patrón de diseño de software que separa las funciones que el software realiza en tres capas principales:

\begin{itemize}
    \item \textbf{Modelo de datos}: Es la representación de la información que la que la aplicación interactúa, tanto para obtener la información como para ser actualizada.

    El modelo de datos normalmente equivale al diseño de base de datos, donde cada modelo representa a una entidad en un diseño de base de datos relacional.

    El controlador es el encargado de realizar las peticiones al modelo, ya se actualizaciones o la obtención de información.

    \item \textbf{Controlador}: Responde a acciones del usuario (o eventos), que normalmente desencadenan en una acción al modelo de datos (ya sea obtención de datos, actualización, borrado...).

    El controlador hace de intermediario entre la vista y el modelo.

    \item \textbf{Vista}: Es la parte que muestra al usuario los datos obtenidos y con la que este interactúa. Esta interacción generará posibles acciones que irán al controlador para volver a empezar el ciclo.
\end{itemize}

\section{Interacción de los componentes}

Aunque existen distintas implementaciones de la arquitectura Modelo-Vista-Controlador, el flujo de acciones suele ser similar al siguiente:

{
\begin{minipage}{0.56\linewidth}
\begin{enumerate}
    \item El usuario interactúa con la interfaz de usuario de alguna forma (por ejemplo, el usuario pulsa un botón, enlace, etc.).

    \item El controlador recibe (por parte de los objetos de la interfaz-vista) la notificación de la acción solicitada por el usuario. El controlador gestiona el evento que llega, frecuentemente a través de un gestor de eventos (handler) o callback.

    \item El controlador realiza una petición al modelo, ya sea para solicitar información o para actualizarla. El modelo debe confirmar si la acción se ha realizado de manera correcta o no.

    \item El controlador delega en la vista la información obtenida para que sea visualizada.

    \item La interfaz se mantiene a la espera de una nueva interacción para comenzar de nuevo el ciclo.
\end{enumerate}
\end{minipage}
\hfill
\begin{minipage}{0.4\linewidth}
    \includegraphics[width=\linewidth]{mvc.png}
\end{minipage}
}

\vspace{10pt}
Tal como se ha dicho, pueden existir distintas implementaciones, pero de manera generalizada y simplificada este sería el esquema básico de interacción.

    \part{Crear entorno Laravel}
    \chapter{Crear primer proyecto en Laravel}

A la hora de crear un proyecto en Laravel lo primero que deberíamos hacer es visitar la \href{https://laravel.com/docs/10.x/installation}{documentación}, ya que nos dará distintas opciones dependiendo del sistema operativo en el que nos encontremos. Aparte, podremos ver si ha habido cambios desde la última vez que hayamos creado un proyecto.

\section{Servicios a utilizar}

Antes de crear el proyecto, debemos tomar una serie de decisiones para nuestro \textit{stack} de aplicación. Laravel cuenta con distintos servicios, algunos de ellos necesarios y otros optativos, por lo que deberemos tenerlos en cuenta.

Los servicios entre los que deberemos decidir son:

\begin{itemize}
    \item \textbf{Sistema Gestor de Base de Datos a utilizar}: Laravel permite el uso de distintos sistemas de bases de datos relacionales como son \href{https://dev.mysql.com/downloads/mysql/}{MySQL}, \href{https://www.postgresql.org/}{PostgreSQL} y \href{https://mariadb.org/}{MariaDB}. Por defecto hace uso de \textbf{MySQL}.
    \item \textbf{Sistema de caché}: Podemos hacer uso de distintos sistemas para cachear desde la sesión a información obtenida de la base de datos y también HTML. Por defecto, \textbf{Laravel cachea la sesión en el sistema de ficheros}, pero eso puede ser lento, por lo que se permite hacer uso de sistemas \textbf{clave-valor} para el almacenamiento de información para acelerar el rendimiento de la aplicación web. Se puede elegir \href{https://www.memcached.org/}{Memcached} o \href{https://redis.io/}{Redis} entre otros.
\end{itemize}

Otros servicios que podemos instalar y que nos darán ciertas funcionalidades son:

\begin{itemize}
    \item \textbf{\href{https://github.com/axllent/mailpit}{Mailpit}}: Es un sistema para controlar los emails que envía nuestra aplicación durante el desarrollo. En lugar de enviarlos a las cuentas finales, se quedan almacenados y se pueden visualizar a través de una web que a modo de buzón de correo. También ofrece una API.

    \item Uso de  \href{https://min.io/}{MinIO} para simular el \textbf{almacenamiento en la nube S3}. De esta manera no tendremos que crear un Bucket de pruebas.

    \item Sistema de \textbf{búsqueda \textit{full-text}} en la base de datos gracias a \href{https://laravel.com/docs/10.x/scout#introduction}{Scout} y haciendo uso del backend \href{https://www.meilisearch.com/}{MeiliSearch}.

    \item Creación y automatización de \textbf{tests} utilizando \href{https://www.selenium.dev/}{Selenium}.
\end{itemize}

Estas son algunas de los servicios que podríamos configurar antes de comenzar a crear nuestra aplicación. Para comenzar de manera sencilla nos centraremos únicamente en la elección de la base de datos, dejando el resto de servicios para más adelante.


\section{Instalación mediante Sail y Docker}

En la \href{https://laravel.com/docs/10.x/installation}{documentación} de Laravel nos explica cómo realizar la instalación de distintos modos teniendo en cuenta el sistema operativo, los servicios iniciales que nos interesan y el sistema de instalación que mejor se adapte a nuestro entorno.


El sistema es similar utilizando GNU/Linux, Windows y MacOS, con la salvedad de que en Windows deberíamos instalar Docker Desktop y \textit{Windows Subsystem for Linux} (WSL). De manera generalizada, es necesario tener instalado:

\begin{itemize}
    \item Entorno GNU/Linux
    \item Docker
    \item Docker Compose
    \begin{mycode}{Instalar Docker Compose}{console}{}
ruben@vega:~$ sudo apt install docker-compose
\end{mycode}
\end{itemize}

Para realizar la instalación sólo vamos a elegir tener el servicio de MySQL, para simplificarlo, tal como se ha comentado previamente. Para ello, deberemos ejecutar lo siguiente en el directorio donde nos interese crear el directorio del proyecto.

\begin{mycode}{Usamos el instalador de Laravel}{console}{{\small }}
ruben@vega:~$ curl -s "https://laravel.build/example-app?with=mysql" | bash
\end{mycode}

Este comando lo que va a hacer es descargarse un script que va a ejecutar lo siguiente:
\begin{enumerate}
    \item Se va a asegurar que Docker está corriendo
    \item Va a levantar un contenedor con la imagen “laravelsail/php82-composer” que nos va a crear un directorio llamado \configdir{example-app} con un proyecto limpio de Laravel usando MySQL como SGBD.
    \item Si no tenemos la imagen de MySQL la descarga.
\end{enumerate}


\subsection{Ventajas}

Este sistema de instalación permite realizar un despliegue sencillo en un equipo donde tengamos instalado Docker, con todas las ventajas que ello ofrece, aparte de la posibilidad de elegir los servicios que necesitemos inicialmente.

Podríamos resumir las ventajas en la siguiente lista:

\begin{itemize}
    \item Instalación rápida con un único comando.
    \item Ventajas de usar Docker: todos los desarrolladores usan el mismo contenedor/entorno de desarrollo.
    \item No es necesario tener nada más que Docker instalado en el equipo anfitrión (ni PHP, composer, servicios web, ...).
\end{itemize}

\subsection{Desventajas}
Aunque las ventajas durante el desarrollo de aplicaciones son notables, también pueden existir algunas desventajas:

\begin{itemize}
    \item El servidor web que se arranca por defecto no es el más recomendado para despliegues en producción, obteniendo mejor rendimiento con el servidor web \href{https://nginx.org/en/}{Nginx}.

    \item En un principio puede resultar “raro” desarrollar dentro de un contenedor.
\end{itemize}


\section{Iniciar servicios}

Una vez terminada la descarga del código y tras realizar las acciones que necesita, el propio asistente nos avisa de qué tenemos que realizar para que nuestro entorno arranque.

\begin{mycode}{Arrancamos los servicios}{console}{}
ruben@vega:~$ cd example-app && ./vendor/bin/sail up -d
[+] Running 2/2
  Container example-app-mysql-1         Started    0.4s
  Container example-app-laravel.test-1  Started    0.7s
\end{mycode}

Y con ello podemos ir al puerto 80 a través de nuestro navegador y veremos la página principal para comprobar que todo ha ido bien.

\begin{center}
    \includegraphics[frame,width=0.7\linewidth]{intro.png}
\end{center}


\chapter{Variables de entorno}
Todo proyecto de Laravel cuenta con unas variables de entorno del proyecto. Es un fichero de configuración situado en la raíz del proyecto que se llama \configfile{.env} y en él se encuentran las credenciales para acceder a la base de datos, servidor SMTP, ...

Dado que hemos generado el entorno a través del asistente, se ha rellenado con la configuración por defecto, entre las que nos encontramos que la aplicación tiene \textbf{el modo debug activado}. Durante el desarrollo nos va a ayudar para poder hacer \textit{debugging} mientras realizamos acciones, pero esta opción debería estar deshabilitada al poner la aplicación en producción.

Por último, es conveniente recordar que este fichero \textbf{nunca debería estar en un repositorio público}, ya que contiene información sensible como lo son los credenciales de acceso a bases de datos o servicios externos.

\errorbox{\textbf{Cuidado con versionar el fichero “\texttt{.env}”, ya que contiene información sensible}}
    \chapter{Usar Visual Studio Code con Laravel}

\href{https://code.visualstudio.com/}{Visual Studio Code} es un entorno de desarrollo integrado (IDE) desarrollado por Microsoft y con licencia MIT, lo que lo hace Software Libre y que cualquiera pueda ver el código fuente, así como realizar modificaciones.

El problema es que Microsoft no ha liberado todo el código fuente, y los binarios que ofrece para descargar hacen uso de ese software, así como la inclusión de sistemas de telemetría. Es por eso que existe un proyecto llamado \href{https://vscodium.com/}{VSCodium} que ofrece los binarios libres de ese código.

Entre las ventajas que ofrece este IDE podemos destacar:

\begin{itemize}
    \item Se puede programar para muchos lenguajes de programación, no está especializado en uno sólo.

    \item Es extensible mediante \textit{plugins}. Hoy en día existen infinidad de extensiones para todo tipo de desarrollos.

    \item Es multiplataforma.

    \item Altamente configurable.

    \item Configurando la cuenta de GitHub, se puede sincronizar las configuraciones entre distintos dispositivos.

    \item Existe una versión \href{https://vscode.dev/}{online}.
\end{itemize}

\section{Extensiones recomendadas}

Para desarrollar con Laravel, aunque se puede extender a cualquier proyecto que haga uso de un entorno Docker, es recomendable utilizar una serie de extensiones para facilitar el desarrollo con ellos. De todas maneras, Visual Studio Code nos va a recomendar extensiones a medida que lo usemos, ya que observará el tipo de desarrollo que estamos realizando.

Entre las extensiones que se recomiendan están:
\begin{itemize}
    \item \href{https://marketplace.visualstudio.com/items?itemName=ms-vscode-remote.vscode-remote-extensionpack}{Remote Development}: Nos instala un grupo de extensiones para poder trabajar contra un servidor remoto.

    \item \href{https://marketplace.visualstudio.com/items?itemName=onecentlin.laravel-extension-pack}{Laravel Extension Pack}: Es una “meta-extensión”, ya que incluye a otras extensiones creadas especialmente para ayudar durante el desarrollo de Laravel.

    \item \href{https://marketplace.visualstudio.com/items?itemName=xdebug.php-pack}{PHP Extension Pack}: Es un conjunto de extensiones que nos va a permitir trabajar de manera más cómoda durante el desarrollo de código PHP.

    \item \href{https://marketplace.visualstudio.com/items?itemName=formulahendry.auto-close-tag}{Auto Close Tag}: Muy útil durante el desarrollo de HTML, ya que cuando creamos una etiqueta, automáticamente nos crea la etiqueta de cerrado.
\end{itemize}

Existe una infinidad de extensiones que nos pueden ayudar durante el desarrollo,

\section{Conexión al servidor}
Si hacemos uso de una máquina virtual para el desarrollo, por no usar GNU/Linux en la máquina anfitriona donde hemos instalado el contenedor de Laravel, Visual Studio Code nos permite conectarnos por SSH a un servidor donde vayamos a realizar el desarrollo.

Para conectarnos usaremos la extensión recién instalada “Remote explorer”, nos aseguramos que estamos en la opción “Remotes (Tunnels/SSH)” y crearemos una nueva conexión SSH al servidor a través del icono “+”, en el que nos pedirá realizar la conexión SSH:

\begin{center}
    \includegraphics[frame,width=0.5\linewidth]{visual_studio_code_ssh.png}
\end{center}

\warnbox{Es recomendable hacer uso del \hyperlink{ssh_clave_publica_privada}{sistema de certificados de clave pública/privada} de SSH para realizar la conexión}

\begin{center}
    \includegraphics[width=0.9\linewidth]{visual_studio_code_ssh2.png}
\end{center}

Nos pedirá dónde queremos guardar la configuración, dejando la ruta por defecto, que es el fichero \configfile{.ssh/config} dentro de la “home” de nuestro usuario. Si hemos realizado la configuración de los certificados de clave pública/privada, no nos pedirá la contraseña.


\section{Conexión al contenedor}

Gracias a la extensión “remote development” instalada previamente, vamos a poder trabajar \textbf{dentro del contenedor de Laravel}. De esta manera Visual Studio Code va a tener acceso al intérprete de PHP para poder ayudarnos durante el desarrollo.

Hacer uso de esta funcionalidad es muy útil ya que al estar dentro del contenedor, estamos dentro del entorno de desarrollo de manera “inmersiva”, pudiendo instalar componentes o ejecutar órdenes dentro del contenedor.

Para realizar la conexión, deberemos ver los contenedores en la extensión “Remote Explorer”, en el apartado “\textbf{Dev Containers}”:

\begin{center}
    \includegraphics[frame,width=0.5\linewidth]{visual_studio_code_container.png}
\end{center}

Al acceder al contenedor, en este caso “example-app\_laravel.test\_1”, Visual Studio nos debería abrir el directorio principal donde está situada la aplicación Laravel, \configdir{/var/www/html}.

    \part{Funcionalidad básica}
    \chapter{Introducción}

Ahora que ya tenemos el entorno creado, es momento de empezar a añadir funcionalidad básica a nuestra aplicación y comenzar a crear nuestra aplicación. Para estos ejemplos se ha decidido crear una pequeña aplicación a modo de blog, con posts y comentarios.

\chapter{Artisan}
\href{https://laravel.com/docs/10.x/artisan}{Artisan} es la interfaz de línea de comandos que vamos a utilizar para realizar todo tipo de interacción entre el proyecto y el propio \textit{framework} Laravel. Esta interfaz nos va a permitir, entre otras cosas:

\begin{itemize}
    \item Crear modelos y controladores.
    \item Crear una sesión a la base de datos.
    \item Controlar el estado de los “migrations”.
    \item Hacer uso de los “seeds” en la base de datos.
    \item Limpiar la caché de objetos.
\end{itemize}

Cada comando contará con su ayuda, por lo que es recomendable ir mirando la ayuda y así conocer las distintas opciones para cada uno de ellos.


\chapter{Crear modelo}
Un blog tiene una serie de “Posts”, que son las entradas que los usuarios introducen en el blog. De momento vamos a ignorar el apartado de usuarios, para simplificarlo. Una entrada del blog contará con los siguientes atributos:

\begin{itemize}
    \item Título
    \item Texto
    \item Si está publicada o no
\end{itemize}

Para crear el modelo, ejecutaremos el siguiente comando. Este comando lo debemos ejecutar dentro del contenedor Docker y dentro de la ruta donde se encuentra el proyecto Laravel, que es \configdir{/var/www/html}:

\begin{mycode}{Crear Modelo}{console}{}
root@1b29e46c10ae:/var/www/html# php artisan make:model Post -crms
\end{mycode}

Este comando nos va a crear el modelo Post junto con:
\begin{itemize}
    \item \textbf{Controlador} de tipo “resource”, lo que va a permitir realizar acciones “\textbf{CRUD}” (\textit{create}, \textit{read}, \textit{update} y \textit{delete}), necesarias en cualquier aplicación web.
    \item \textbf{\textit{Migration}}: Un fichero para realizar la migración del modelo en la base de datos.
    \item \textbf{\textit{Seed}}: Un fichero de tipo “semilla” para introducir datos en la base de datos.
\end{itemize}

\chapter{Entendiendo las “\textit{migrations}” de base datos}

Hoy en día son muchos los \textit{frameworks} que hacen uso de sistemas de tipo \textbf{\textit{migration}} a la hora de interactuar en el tiempo con la base de datos. Podríamos definirlo como un \textbf{sistema de control de versiones para el esquema de base de datos}.

Este sistema permite ir evolucionando el esquema de base de datos (tablas, columnas de las tablas, funciones...) a medida que el propio código fuente de la aplicación va evolucionando. De esta manera, si tenemos el código en un punto concreto, con el sistema \textbf{migrations} nos va a crear la base de datos tal como se necesita en ese punto.

Al crear nuestro proyecto Laravel, ya contamos con una serie de ficheros de migraciones para la base de datos. Estos ficheros se encuentran en \configdir{app/database/migrations/}, teniendo cada fichero un formato similar a \configfile{YYYY_mm_dd_HHMMSS_comentario.php} siendo:

\begin{itemize}
    \item \textbf{YYYY}: el año que se ha creado el fichero.
    \item \textbf{mm}: el mes que se ha creado el fichero.
    \item \textbf{dd}: el día que se ha creado el fichero.
    \item \textbf{HHMMSS}: la hora, minuto y segundo.
    \item \textbf{comentario}: un pequeño comentario sobre el contenido del fichero.
\end{itemize}

De esta manera, los migrations se van a poder ejecutar en orden de fecha de creación, donde normalmente suele ser:
\begin{itemize}
    \item \textbf{De más antiguo a más nuevo}: Se van creando las tablas, y se van añadiendo modificaciones. Es el ciclo normal de de desarrollo, y este es el sistema de uso habitual.
    \item \textbf{De más nuevo a más antiguo}: Se vuelve atrás en el proyecto, eliminando modificaciones. Utilizado para ir a una versión antigua del proyecto.
\end{itemize}

Vamos a utilizar como ejemplo el primer fichero que existe en el directorio, que es para hacer uso de la tabla de usuarios del sistema de autenticación de Laravel. El fichero tiene una clase que extiende de la clase \textbf{Migration} con dos funciones:

\begin{mycode}{Fichero Migration}{PHP}{}
<?php
use Illuminate\Database\Migrations\Migration;
use Illuminate\Database\Schema\Blueprint;
use Illuminate\Support\Facades\Schema;

return new class extends Migration {
    public function up(): void {
        Schema::create('users', function (Blueprint $table) {
            $table->id();
            $table->string('name');
            $table->string('email')->unique();
            $table->timestamp('email_verified_at')->nullable();
            $table->string('password');
            $table->rememberToken();
            $table->timestamps();
        });
    }

    public function down(): void {
        Schema::dropIfExists('users');
    }
};
\end{mycode}

La función \inlineconsole{up()} se ejecutará cuando realizamos la migración, mientra que la función \inlineconsole{down()} se usará cuando realicemos un “\textbf{\textit{rollback}}” (echar para atrás una migración).


\warnbox{\textbf{Por convenio, el nombre de los modelos suelen ser en singular, mientras que las tablas se deben crear en plural. \href{https://laravel.com/docs/10.x/eloquent\#table-names}{Pero se puede cambiar el nombre de la tabla}.}}

\section{Opciones de las migraciones}

En la \href{https://laravel.com/docs/10.x/migrations#tables}{documentación oficial} se explican cómo funcionan los \textit{migrations} y las funcionalidades básicas y avanzadas que tienen.

Teniendo en cuenta lo visto en el punto anterior, podemos visualizar que las acciones del \textit{migration} contiene varias líneas, y vamos a destacar lo siguiente para el fichero \configfile{2014_10_12_000000_create_users_table.php}:

\begin{itemize}
    \item Crea una tabla llamada “\textbf{users}”, que contiene varias columnas
    \item \textbf{id}: es un alias al método \textbf{bigIncrements}. Va a generar una columna de tipo “\textit{big integer}” sin signo, que se va a ir incrementando y que va a ser \textbf{clave primaria}.

    \item \textbf{string}: existen varias columnas de tipo “string”, que son “name”, “email” y “password”. Es lo equivalente a “varchar”, sin indicar en este caso el número de longitud. Se le puede indicar como segundo parámetro.
    \item \textbf{unique()}: el contenido de este campo (en el ejemplo el \textbf{email}) debe ser único en la tabla.
    \item \textbf{timestamp}: crea un campo de tipo TIMESTAMP.
    \item \textbf{nullable}: permite que este campo sea \textbf{null}.

    \item \textbf{timestamps()}: Este es un método especial que crea dos campos en la base de datos: “\textbf{created\_at”} y “\textbf{updated\_at}”. De esta manera sabemos cuándo se ha creado y modificado el registro en la base de datos.
\end{itemize}

\exercisebox{Añade a la migración del modelo Post, la generación de los campos: “título”, “texto” y “publicado”. Recuerda mirar la documentación oficial.}

\section{Uso de las migraciones}

Una vez tenemos distintos ficheros de migraciones, hay que saber cómo aplicarlos y qué sucede con ellos. De nuevo, en la \href{https://laravel.com/docs/10.x/migrations#running-migrations}{documentación} aparecen distintos ejemplos, de los cuales se van a destacar sólo unos a continuación:

\subsection{Desplegar migraciones}
Para realizar el despliegue de todas las migraciones debemos ejecutar el siguiente comando:

\begin{mycode}{Ejecutar migraciones}{console}{}
root@1b29e46c10ae:/var/www/html# php artisan migrate
   INFO  Preparing database.
Creating migration table ............................. 52ms DONE

   INFO  Running migrations.
2014_10_12_000000_create_users_table ..............   108ms DONE
2014_10_12_100000_create_password_reset_tokens_table  127ms DONE
2019_08_19_000000_create_failed_jobs_table .........   88ms DONE
2019_12_14_000001_create_personal_access_tokens_table 140ms DONE
2023_09_26_094514_create_posts_table ...............   74ms DONE
\end{mycode}

\subsection{Comprobar estado de las migraciones}

Para comprobar el estado de las migraciones podemos realizarlo de la siguiente manera:
\begin{mycode}{Estado de las migraciones}{console}{}
root@1b29e46c10ae:/var/www/html# php artisan migrate:status

Migration name ................................ Batch / Status
2014_10_12_000000_create_users_table ................. [1] Ran
2014_10_12_100000_create_password_reset_tokens_table . [1] Ran
2019_08_19_000000_create_failed_jobs_table ........... [1] Ran
2019_12_14_000001_create_personal_access_tokens_table  [1] Ran
2023_09_26_094514_create_posts_table ................. [1] Ran
\end{mycode}


Si queremos ver a nivel de base de datos qué ha pasado, podemos ejecutar una sesión y visualizar la propia base de datos. Veremos cómo se ha creado la base de datos, las tablas, y una tabla especial llamada \textbf{migrations}, que contiene qué ficheros se han desplegado.

\begin{mycode}{Ejecutar migraciones}{mysql}{}
root@1b29e46c10ae:/var/www/html# php artisan db

mysql> use example_app;
Database changed

mysql> show tables;
+------------------------+
| Tables_in_example_app  |
+------------------------+
| failed_jobs            |
| migrations             |
| password_reset_tokens  |
| personal_access_tokens |
| posts                  |
| users                  |
+------------------------+
6 rows in set (0.00 sec)

mysql> select * from migrations;
+----+-------------------------------------------------------+-------+
| id | migration                                             | batch |
+----+-------------------------------------------------------+-------+
|  1 | 2014_10_12_000000_create_users_table                  |     1 |
|  2 | 2014_10_12_100000_create_password_reset_tokens_table  |     1 |
|  3 | 2019_08_19_000000_create_failed_jobs_table            |     1 |
|  4 | 2019_12_14_000001_create_personal_access_tokens_table |     1 |
|  5 | 2023_09_26_094514_create_posts_table                  |     1 |
+----+-------------------------------------------------------+-------+
5 rows in set (0.00 sec)
\end{mycode}


\subsection{\textit{Rollback} la última migración}

En un momento dado nos puede interesar echar atrás la última migración, y para ello contamos con la opción \textbf{\textit{rollback}}. Este sistema puede que deshaga las migraciones de varios ficheros.

\begin{mycode}{Ejecutar migraciones}{mysql}{}
root@1b29e46c10ae:/var/www/html# php artisan migrate:rollback

INFO  Rolling back migrations.
2023_09_26_094514_create_posts_table ................  27ms DONE
2019_12_14_000001_create_personal_access_tokens_table  26ms DONE
2019_08_19_000000_create_failed_jobs_table ..........  25ms DONE
2014_10_12_100000_create_password_reset_tokens_table   24ms DONE
2014_10_12_000000_create_users_table ................  27ms DONE
\end{mycode}

En este caso, como el migrate hizo todos los ficheros, el \textit{rollback} se ha ejecutado de todos los ficheros pero \textbf{en orden inverso al de creación}.


\subsection{Limpiar, \textit{reset} y recarga de migraciones}

Para asegurar que el sistema de migraciones está funcionando bien, para hacer pruebas, o para realizar despliegues limpios quizá nos interese borrar todas las migraciones de la aplicación o realizar una recarga de las mismas.

\begin{itemize}
    \item \textbf{db:wipe}: borra todas las tablas, vistas y tipos.
    \item \textbf{migrate:fresh}: borra todas las tablas de la base de datos y aplica de nuevo todas migraciones.
    \item \textbf{migrate:reset}: deshace todas las migraciones de la aplicación. Básicamente es dejar la base de datos limpia. \textbf{En este caso no se borra la tabla “\textit{migration}”}.
    \item \textbf{migrate:refresh}: deshace todas las migraciones de la aplicación y las vuelve a aplicar en orden.
\end{itemize}


\section{Uso de las semillas}

A la hora de crear una aplicación es posible que nos interese que tras realizar un primer despliegue existan datos en la base de datos. Ya sea porque estos datos son necesarios para el correcto funcionamiento de la aplicación o para darle una funcionalidad básica.

Para poblar de datos la base de datos existe el sistema de semillas, o \textbf{\textit{seeds}}. Este sistema funciona a través de sus propios ficheros, que se pueden crear por modelo (tal como hemos hecho en este capítulo), o de manera general en una semilla propia.

Con la generación del modelo se ha creado también el fichero  al que vamos a añadirle el código necesario para que cree un primer post de pruebas: \configfile{app/database/seeders/PostSeeder.php}.

\begin{mycode}{\textit{Seed} del PostSeeder.php}{php}{}
<?php
use Illuminate\Database\Console\Seeds\WithoutModelEvents;
use Illuminate\Database\Seeder;
use Illuminate\Support\Facades\DB;

class PostSeeder extends Seeder {
    public function run(): void {
        DB::table('posts')->insert([
            "titulo"=>"Primer post",
            "texto"=>"Este es el texto del primer post",
            "publicado"=>true,
            "created_at"=>now(),
        ]);
    }
}
\end{mycode}

Para poder hacer uso del modelo “\textbf{DB}” es necesario hacer uso de la librería \configfile{Illuminate\Support\Facades\DB}. Ahora sólo queda ejecutar el \textit{seed} tal como se explica en la \href{https://laravel.com/docs/10.x/seeding}{documentación}:

\begin{mycode}{Ejecutar el seed}{console}{}
root@1b29e46c10ae:/var/www/html# php artisan db:seed PostSeeder
INFO  Seeding database.
\end{mycode}

Si se comprueba la base de datos, se verá cómo en la tabla aparecen los datos del \textit{seed}.


\chapter{Rutas de la aplicación}

Aunque ya tenemos un controlador y datos en la aplicación, hasta ahora son inaccesibles. Lo único que vemos en la aplicación es la página de bienvenida al proyecto y si ponemos cualquier ruta en la URL nos aparece un error “404 Not Found”.

Esto es debido al sistema de enrutado de la aplicación, que sólo permite acceder al \textit{path} “/” que nos muestra la plantilla de bienvenida. Esta configuración se puede ver en el fichero \configfile{routes/web.php}.

\begin{mycode}{Rutas de la aplicación web de Laravel}{php}{}
<?php
use Illuminate\Support\Facades\Route;

Route::get('/', function () {
    return view('welcome');
});
\end{mycode}

Cualquier intento de acceso a algo que no sea esa ruta dará un error 404. Este es un sistema de seguridad para controlar a qué se tiene acceso en la aplicación, y por eso que debemos modificar este fichero para poder acceder a nuestro nuevo controlador.

\begin{mycode}{Añadiendo rutas para el nuevo controlador}{php}{}
<?php
// ...
use App\Http\Controllers\PostController;
Route::controller(PostController::class)->group(function () {
    Route::get('/posts', 'index')->name('posts.index');
    Route::get('/posts/{post}', 'show')->name('posts.show');
});
\end{mycode}

Este código indica que se va a utilizar la clase “PostController” para el grupo de las rutas que aparecen en ese trozo de código. Si vamos al fichero \configfile{App\Http\Controllers\PostController.php} veremos que por defecto todas las funciones están vacías, y es por eso que no nos devuelve ningún dato.

Por lo tanto, la idea es:

\begin{itemize}
    \item \textbf{/posts}: irá a la función “index” del controlador. Esta función normalmente lista el contenido de la tabla de base de datos que hace referencia al modelo. En nuestro caso, mostrará todos los posts del blog (normalmente en formato paginado).
    \item \textbf{/posts/\{post\}}: esta ruta será la utilizada cuando queramos ir a ver un registro del modelo concreto. En este caso “\{post\}” indicará el “id” dentro de la base de datos que se le pasará a la función “show”.
\end{itemize}

En el siguiente apartado, cuando modifiquemos el controlador quedará más claro.

\section{Tipos de rutas}

Hay que entender que las rutas funcionan en base al protocolo HTTP. Esto quiere decir que existen distintas maneras de acceder a la misma URL dependiendo del tipo de petición que se realice en base a lo que realicemos con el navegador.

Normalmente, cuando navegamos por internet, las peticiones que se realizan son de tipo \textbf{GET}, ya que estamos pidiendo información al servidor web. En cambio, cuando rellenamos un formulario y le damos a enviar, se hace uso del “verbo” \textbf{POST}, ya que se envían datos al servidor.

Las peticiones HTTP que se pueden utilizar son:

\begin{itemize}
    \item \textbf{GET}: Se realiza una petición a la ruta especificada. Estas peticiones sólo obtienen información.
    \item \textbf{POST}: Se envían datos al servidor, que van incluidos dentro del cuerpo de la petición. Lo habitual cuando utilizamos formularios. Se utiliza para crear nuevos recursos.
    \item \textbf{PUT}: Similar a POST, pero en este caso suele estar orientado a modificar datos previamente creados.
    \item \textbf{PATCH}: Como PUT, sobreescribe completamente un recurso existente.
    \item \textbf{DELETE}: Borra el recurso especificado.
\end{itemize}

Es conveniente mirar la \href{https://laravel.com/docs/10.x/routing}{documentación} cuando queramos realizar algún tipo de petición distinto de GET, ya que nos ayudará a comprender mejor qué es lo que está sucediendo.

\chapter{Controladores y Vistas}
Ahora que ya tenemos las rutas creadas, es momento de que los datos se visualicen en la aplicación. Para ello es necesario entender cómo funciona el sistema de plantillas utilizado por Laravel, llamado \textbf{\href{https://laravel.com/docs/10.x/blade}{Blade}}, que junto con el sistema de \textbf{enrutado} visto previamente, relaciona la URL a la que se llama con el controlador y la vista correspondientes.

\section{Obtener datos en el controlador}

El ejemplo va a consistir en obtener todos los posts de la base de datos y hacer un listado con ellos. Por otro lado, al seleccionar un post concreto, se mostrará dicho post. Para ello vamos a modificar el controlador para modificar las dos funciones que se utilizan en las rutas:

\begin{mycode}{Funciones modificadas en el controlador Post}{php}{}
<?php
// ...
use App\Models\Post;
// ...
class PostController extends Controller{
    public function index(){
        $posts = Post::orderBy('created_at')->get();
        return view('posts.index',['posts' => $posts]);
    }
    //...
    public function show(Post $post){
        return view('posts.show',['post'=>$post]);
    }
\end{mycode}

El problema de este código es que estamos llamando a unas vistas que todavía no hemos creado, y les estamos pasando como variables a la vista los datos obtenidos dentro de un array. Podremos pasar tantas variables como queramos.


\section{Generar vista}

El sistema de plantillas y vistas Blade se guardan en la ruta \configdir{resources/views}, y en el primer caso lo que estamos diciendo es que haga uso de “posts.index”, que quiere decir el fichero “index.blade.php” del directorio “posts”. Por lo tanto, deberemos crear un fichero en la ruta \configfile{resources/views/posts/index.blade.php}, que corresponde a la vista que estamos llamando.

\infobox{\textbf{Es recomendable para cada Modelo/Controlador crear un directorio de vistas}}

Ahora es momento de visualizar los datos en la vista. Para ello, recorreremos el listado obtenido y lo visualizaremos, todo ello en la vista. El sistema de plantillas \href{https://laravel.com/docs/10.x/blade}{Blade} permite introducir funcionalidad similar a PHP en la vista mezclado con HTML. También permite incrustar código PHP directamente, pero intentaremos evitarlo.

El sistema de plantillas tiene una serie de palabras reservadas similar a la de los lenguajes de programación más habituales. En este ejemplo se va a recorrer con un bucle for la lista, se crea una variable de indexación, y así poder visualizar los atributos:

\begin{mycode}{Vista “index.blade.php”}{html+smarty}{}
<ul>
  {{--esto es un comentario: recorremos el listado de posts--}}
  @foreach ($posts as $post)
    {{-- visualizamos los atributos del objeto --}}
    <li>
      <a href="{{route('posts.show',$post)}}"> {{$post->titulo}}</a>.
      Escrito el {{$post->created_at}}
    </li>
  @endforeach
</ul>
\end{mycode}

Si ahora visualizamos la ruta “/posts” obtendremos el listado. Es importante destacar que para el enlace que nos lleva a visualizar un post concreto \textbf{se ha hecho uso del sistema ed rutas} al que se le pasa como parámetro el post.

\exercisebox{Crea la vista para visualizar toda la información del post en “show.blade.php”.}


\chapter{\textit{Soft Deleting}}

Laravel, a través de su ORM Eloquent, nos permite hacer uso del sistema “\textit{\href{https://laravel.com/docs/10.x/eloquent\#soft-deleting}{soft deleting}}”, que en lugar de borrar los registros de la base de datos, lo que hace es marcarlo como borrado. Esto lo hace a través de una columna en la base de datos, indicando con una fecha cuándo se ha borrado.

Es habitual hacer uso de estos sistemas, por si el borrado ha sido erróneo, y de esta manera poder recuperar registros (ya que realmente no se han borrado).

Para hacer uso de este sistema debemos indicarlo en el modelo, para ello le indicaremos que se va a usar “\textbf{SoftDeletes}”:

\begin{mycode}{Indicar en el modelo el uso de Softdeletes}{php}{}
<?php
//...
use Illuminate\Database\Eloquent\SoftDeletes;
class Post extends Model{
    use SoftDeletes;
    //...
}
\end{mycode}

Y también debemos indicarlo en la generación de la base de datos (o en un nuevo \textit{migration}). De esta manera, se creará la columna correspondiente que es necesaria.

\begin{mycode}{Indicar en el “migration” el uso de Softdeletes}{php}{}
<?php
//...
public function up(): void {
    Schema::create('posts', function (Blueprint $table) {
        $table->id();
        $table->string("titulo",128);
        $table->string("texto",5000);
        $table->boolean("publicado");
        $table->softDeletes();
        $table->timestamps();
    });
}
\end{mycode}

Si ejecutamos los \textit{migrations}, veremos que la tabla tiene un campo “\textbf{deleted\_at}”, que por defecto estará a NULL. Si ahora borramos un registro, se actualizará esa columna con la fecha del momento en el que se ha realizado la acción de borrado. \textbf{Estos registros pueden ser recuperados}.



\chapter{Debug}
Durante el desarrollo es habitual hacer uso de sistemas de \textit{debug}, por ejemplo para poder ver el contenido de variables y parar la ejecución del algoritmo que estamos programando.

Laravel cuenta con una función llamada \inlineconsole{dd()} que podemos utilizar en cualquier momento. Por ejemplo, si lo usamos en el controlador creado previamente:

\begin{mycode}{Llamar a Tinker con Artisan}{php}{}
<?php
 public function index(){
    $posts = Post::all();
    dd($posts);
    return view('posts.index',['posts' => $posts]);
}
\end{mycode}

En este caso, se ejecutará la petición de obtener todos los \textit{posts}, y acto seguido la función \inlineconsole{dd($posts)} lo que hará será mostrar por pantalla el contenido de la variable y terminará la ejecución del código.


\chapter{Consola Tinker}

Hoy en día muchos \textit{frameworks} tienen algún sistema de consola interactiva con la que poder utilizar las funcionalidades del mismo. De esta manera, podemos realizar comprobaciones, interactuar con los modelos, objetos... pero sin tener que hacerlo desde el código de la web.

En el caso de Laravel la consola se llama \href{https://laravel.com/docs/10.x/artisan#tinker}{Tinker}, y se puede llamar de dos maneras, dependiendo desde dónde lo hagamos:
\begin{itemize}
    \item Si lo realizamos desde dentro del contenedor, usaremos Artisan de la siguiente manera:
\begin{mycode}{Llamar a Tinker con Artisan}{console}{}
root@1b29e46c10ae:/var/www/html# php artisan tinker
Psy Shell v0.11.21 (PHP 8.2.10 — cli) by Justin Hileman
\end{mycode}

    \item Si lo hacemos desde nuestro servidor anfitrión, usaremos Sail:
\begin{mycode}{Arrancamos los servicios}{console}{}
ruben@vega:~$ cd example-app && ./vendor/bin/sail tinker
Psy Shell v0.11.21 (PHP 8.2.10 — cli) by Justin Hileman
\end{mycode}
\end{itemize}

Una vez dentro, podremos hacer uso de los modelos, por ejemplo, para ver los datos que tenemos en la base de datos.

\begin{mycode}{Arrancamos los servicios}{psysh}{}
> Post::all();
[!] Aliasing 'Post' to 'App\Models\Post' for this Tinker session.
= Illuminate\Database\Eloquent\Collection {#7247
    all: [
    App\Models\Post {#7249
        id: 1,
        titulo: "Primer post",
        texto: "Este es el texto del primer post",
        publicado: 1,
        created_at: "2023-10-01 16:57:30",
        updated_at: null,
    },
    ],
}
\end{mycode}

    \part{Usar Bootstrap en Laravel}
    \chapter{Instalar dependencias}

Laravel tenía soporte nativo de Bootstrap, pero decidió sustituirlo por \href{https://tailwindcss.com/}{Tailwind}. Eso no quita que podamos usar Bootstrap, pero necesitaremos realizar la instalación de dependencias.

\begin{mycode}{Rutas de la aplicación web de Laravel}{console}{}
root@1b29e46c10ae:/var/www/html# composer require laravel/ui --dev
root@1b29e46c10ae:/var/www/html# php artisan ui bootstrap --auth
root@1b29e46c10ae:/var/www/html# npm install
root@1b29e46c10ae:/var/www/html# npm run build
\end{mycode}

De esta manera no sólo hemos instalado las dependencias necesarias para hacer uso de Bootstrap, si no que también nos ha generado unas vistas para el sistema de autenticación en \configdir{resources/views/auth} y una plantilla general para la aplicación.


\chapter{Plantilla general}

Anteriormente se ha mencionado que Blade es un sistema de plantillas para Laravel. Esto significa que es capaz de generar unos componentes de vistas que a su vez incorporan otras vistas, de esta manera generando plantillas que se pueden reutilizar ahorrando código y simplificando la aplicación.

Con lo realizado previamente se ha generado una plantilla general en el fichero \configfile{resources/views/layouts/app.blade.php}, que se puede dividir en dos apartados:

\begin{itemize}
    \item \textbf{Cabecera “nav”}: es la cabecera de la aplicación. Aparece el nombre de la aplicación y a la derecha tiene enlaces para hacer login o registrarse en la aplicación con el sistema de autenticación.

    \item \inlineconsole{@yield('content')}: este apartado será sustituido por la vista desde la que se llame a esta plantilla.
\end{itemize}

De esta manera, en todas las vistas de la aplicación que llamemos a la plantilla, no tendremos que escribir el código de la cabecera. Lógicamente, \textbf{es posible añadir nuevos apartados a esta vista} para cumplir con el objetivo final de la aplicación.

\section{Cómo usar la plantilla}

Para poder hacer uso de la plantilla, debemos indicarlo en las correspondientes vistas. Como ejemplo, se va a utilizar la vista creada en el capítulo anterior, la que muestra por pantalla los \textit{posts} del blog.

La vista modificada quedaría:

\begin{mycode}{Vista modificada usando la plantilla}{html+smarty}{}
@extends('layouts.app')

@section('content')
  <div class="container">
    <ul>
      {{--esto es un comentario: recorremos el listado de posts--}}
      @foreach ($posts as $post)
        {{-- visualizamos los atributos del objeto --}}
        <li>{{$post->titulo}}. Escrito el {{$post->created_at}}</li>
      @endforeach
    </ul>
  </div>
@endsection
\end{mycode}

Tal como se puede ver, lo primero que se indica es que esta vista “extiende” de la plantilla correspondiente. Posteriormente, lo que se hace es indicar una sección de la plantilla que va a ser sustituida por el contenido que aparece entre “@section” y “@endsection”.

    \part{Métodos \textit{create}, \textit{update}, \textit{delete}}
    \chapter{Crear rutas necesarias}

Una aplicación web normalmente nos va a permitir crear datos, no sólo visualizarlos. Por lo tanto vamos a tener que crear la vista de un formulario que el usuario pueda utilizar para crear datos a través del controlador.

Tal como hemos dicho, las funcionalidades de la aplicación empiezan por crear una ruta a la que el usuario puede acceder. En este caso, se podrían crear las rutas necesarias para visualizar el formulario de creación, obtener los datos para la creación, edición y actualización de datos...

En lugar de eso el sistema de rutas de Laravel nos permite simplificarlo, y si tenemos un modelo que sabemos que es de tipo “resource”, nos permite crear todas las rutas necesarias para la gestión de los datos. Por lo tanto, las rutas quedarían de la siguiente manera:

\begin{mycode}{Rutas simplificadas para un modelo de tipo “resource”}{php}{}
<?php
//...
Route::resources([
    'posts' => PostController::class,
]);
\end{mycode}

Si miramos las rutas generadas, veremos que nos ha creado todas las rutas necesarias para interactuar con los posts:

\begin{mycode}{Mirando todas las rutas creadas}{console}{}
root@1b29e46c10ae:/var/www/html# php artisan route:list
GET|HEAD   posts ............ posts.index › PostController@index
POST       posts ............ posts.store › PostController@store
GET|HEAD   posts/create ..... posts.create › PostController@create
GET|HEAD   posts/{post} ..... posts.show › PostController@show
PUT|PATCH  posts/{post} ..... posts.update › PostController@update
DELETE     posts/{post} ..... posts.destroy › PostController@destroy
GET|HEAD    posts/{post}/edit. posts.edit › PostController@edit
\end{mycode}

Tal como se puede ver, por haber indicado la ruta anterior, automáticamente nos ha creado las rutas para listar, crear, visualizar, actualizar, editar y borrar el recurso. Con una única línea nos evita tener que escribir todas las líneas que supondrían de configuración.


\chapter{Crear registro}

A la hora de crear un registro, Laravel por defecto hace uso de la ruta “create”, por lo que deberemos crear un botón en la vista principal que nos mande a la URL “/posts/create”, por lo que la vista de creación será \configfile{create.blade.php}. Sin entrar en detalles, ya que es un formulario simple, tendrá la siguiente forma:

\begin{mycode}{Vista del formulario}{html+smarty}{{\footnotesize }}
@section('content')
<div class="container">
  <form class="mt-2" name="create_platform" action="{{route('posts.store')}}"
    method="POST" enctype="multipart/form-data">
    @csrf
    <div class="form-group mb-3">
      <label for="titulo" class="form-label">Titulo</label>
      <input type="text" class="form-control" id="titulo" name="titulo" required/>
    </div>
    <div class="form-group mb-3">
      <label for="texto" class="form-label">Texto</label>
      <textarea type="textarea" rows="5" class="form-control" id="texto" name="texto">
      </textarea>
    </div>
    <div class="form-check">
      <input class="form-check-input" type="checkbox" id="publicado" name="publicado">
      <label class="form-check-label" for="publicado">¿Publicar?
      </label>
    </div>
  <button type="submit" class="btn btn-primary" name="">Crear</button>
</form>
</div>
\end{mycode}

Al pulsar el botón “Crear” se realizará una petición “\textbf{POST}” a la ruta “\textbf{posts.store}”, por lo tanto es la función que debemos modificar ahora en el controlador, que junto con la función “create”, tendrá la siguiente forma:

\begin{mycode}{Añadiendo funcionalidad al controlador}{php}{}
<?php
//...
public function create(){
    return view('posts.create');
}

public function store(Request $request){
    $post = new Post();
    $post->titulo = $request->titulo;
    $post->texto = $request->texto;
    $post->publicado = $request->has('publicado');
    $post->save();
    return redirect()->route('posts.index');
}
\end{mycode}


\chapter{Editar registro}

Una vez creados los registros nos va a interesar poder editarlos. Para ello, tendremos que añadir a las vistas correspondientes (el listado general y/o desde la vista del post) un botón que nos lleve a la ruta para editar, que es: \configlink{/posts/{id}/edit}.

Para poder visualizar los datos, deberemos obtener desde el controlador los datos referentes a ese “id” para poder visualizarlo en el formulario que crearemos en la vista \configfile{posts/edit.blade.php}. Después, en la función update deberemos realizar el guardado de las modificaciones.

\begin{mycode}{Añadiendo funcionalidad al controlador}{php}{}
<?php
//...
public function edit(Post $post){
    return view('posts.edit',['post'=>$post]);
}

public function update(Request $request, Post $post){
    $post->titulo = $request->titulo;
    $post->texto = $request->texto;
    $post->publicado = $request->has('publicado');
    $post->save();
    return view('posts.show',['post'=>$post]);
}
\end{mycode}

Tal como se puede ver, la función de actualizar lo que hace es recibir los datos del formulario y el registro a actualizar. Debemos sustituir los campos y para finalizar guardar los cambios del registro. Después, volvemos a la vista para visualizar los cambios.

La vista para editar el registro quedaría:

\begin{mycode}{Vista del formulario}{html+smarty}{{\footnotesize }}
@extends('layouts.app')
@section('content')
<div class="container">
  <form class="mt-2" name="create_platform" action="{{route('posts.update',$post)}}"
    method="POST" enctype="multipart/form-data">
    @csrf
    @method('PUT')
    <div class="form-group mb-3">
      <label for="titulo" class="form-label">Titulo</label>
      <input type="text" class="form-control" id="titulo" name="titulo" required
        value="{{$post->titulo}}"/>
    </div>
    <div class="form-group mb-3">
      <label for="texto" class="form-label">Texto</label>
      <textarea type="textarea" rows="5" class="form-control" id="texto" name="texto">
        {{$post->texto}}
      </textarea>
    </div>
    <div class="form-check">
      <input class="form-check-input" type="checkbox" id="publicado" name="publicado"
        @checked($post->publicado)>
      <label class="form-check-label" for="publicado">
      ¿Publicar?
      </label>
    </div>

    <button type="submit" class="btn btn-primary" name="">Actualizar</button>
  </form>
</div>
@endsection
\end{mycode}

Dado que es el formulario de edición, deben existir datos, de ahí que para cada apartado haya que añadir el parámetro “value” en los \textit{inputs}, el valor dentro del \textit{textarea}, o darle el valor correspondiente al \textit{checkbox}.

También hay que tener en cuenta que debido a cómo funciona el protocolo HTTP con los formularios, \href{https://developer.mozilla.org/en-US/docs/Web/HTTP/Methods/PUT}{que no se puede utilizar en formularios}, debemos añadir \inlineconsole{@mehotd('PUT')} para que genere un \href{https://laravel.com/docs/10.x/blade#method-field}{método oculto} en el formulario.

\exercisebox{Dado que el formulario de crear y actualizar es prácticamente igual, es interesante }


\chapter{Borrar registro}

Por último, tenemos que poder eliminar registros, por lo que deberemos añadir un botón que ejecute la acción de borrado que se recibirá en el controlador.
Este botón lo vamos a añadir a la lista de posts, que junto con el botón editar del apartado anterior, quedaría:

\begin{mycode}{Vista del formulario}{html+smarty}{{\footnotesize }}
@foreach ($posts as $post)
  {{-- visualizamos los atributos del objeto --}}
  <li class="pt-1">
    <div class="d-flex flex-row">
      <a href="posts/{{$post->id}}"> {{$post->titulo}}</a>.
      Escrito el {{$post->created_at}}
      <a class="btn btn-warning btn-sm" href="{{route('posts.edit',$post)}}"
        role="button">Editar</a>

      <form action="{{route('posts.destroy',$post)}}" method="POST">
        @csrf
        @method('DELETE')
        <button class="btn btn-sm btn-danger" type="submit"
          onclick="return confirm('Are you sure?')">Delete
        </button>
      </form>
    </div>
  </li>
@endforeach
\end{mycode}

Y por último el controlador debe borrar el objeto cuando se llama a la función \textbf{destroy}:

\begin{mycode}{Añadiendo funcionalidad al controlador}{php}{}
<?php
//....
public function destroy(Post $post) {
    $post->delete();
    return redirect()->route('posts.index');
}
\end{mycode}

    \part{\textit{Middlewares} y autenticación}
    \chapter{\textit{Middlewares}}

Un \textit{middleware} en Laravel es un mecanismo que inspecciona y filtra las peticiones HTTP que llegan a la aplicación. El ejemplo más claro, y que veremos después, es comprobar si un usuario está autenticado mientras usa la aplicación. En caso de no estar autenticado, le mandará a la página de login/registro.

Se pueden crear otros \textit{middlewares} que nuestra aplicación necesite, como por ejemplo registrar todas las peticiones que llegan a la aplicación, validaciones \href{https://es.wikipedia.org/wiki/Cross-site_request_forgery}{CSRF} de formularios, validación de cabeceras...

Los \textit{middlewares} se sitúan en la ruta \configdir{app/Http/Middleware/}, donde ya existen varios tras realizar la instalación del \textit{framework} Laravel. Para que entren en funcionamiento, se debe realizar la configuración en el fichero de rutas, ya que se activarán dependiendo de las rutas en las que lo indiquemos.

\chapter{Configurando el \textit{middleware} de autenticación}

El sistema de autenticación de Laravel es el ejemplo más claro de \textit{middleware} que podemos utilizar, ya que por defecto viene instalado, pero no está configurado. En pasos anteriores hemos creado el interfaz para poder registrar usuarios y realizar el login en la aplicación.

Ahora es el momento de realizar la activación del sistema de autenticación, y que si no se ha hecho el login, no se pueda ver la aplicación y nos envíe a la página de registro.

Para ello, debemos realizar la modificación de rutas, en la que debemos indicar qué rutas queremos que estén dentro del \textit{middleware} de autenticación. En este caso vamos a elegir que para toda la aplicación sea necesario estar autenticado:


\begin{mycode}{Rutas bajo el \textit{middleware} de autenticación}{php}{}
<?php
//...
Route::middleware(['auth'])->group(function () {
    Route::resources([
    'posts' => PostController::class,
    ]);
});
\end{mycode}


\section{Comprobar rutas bajo \textit{middlewares}}
Si queremos comprobar qué rutas están bajo la influencia de un \textit{middleware}, necesitaremos mirar las rutas en modo \textit{verbose}:


\begin{mycode}{Vista de las rutas en modo \textit{verbose}}{console}{}
root@1b29e46c10ae:/var/www/html# php artisan route:list -v
...
GET|HEAD  posts ............. posts.index › PostController@index
 → web
 → App\Http\Middleware\Authenticate
POST      posts ............. posts.store › PostController@store
 → web
 → App\Http\Middleware\Authenticate
DELETE    posts/{post} ...... posts.destroy › PostController@destroy
 → web
 → App\Http\Middleware\Authenticate
...
\end{mycode}

Se puede comprobar que estas rutas se aplican para la parte “web” de nuestra aplicación, y que antes de ser ejecutadas pasarán por el \textit{middleware} “\textbf{Authenticate}”.

\chapter{Realizar excepciones}

No siempre vamos a querer que toda la aplicación esté bajo el sistema de autenticación, ya que lo habitual es que sólo sea necesario para las acciones que puedan suponer un riesgo de seguridad (edición de datos, borrado de datos, apartados sensibles,...), por lo tanto es interesante que haya rutas que no requieran de estar autenticado.

Para ver cómo funciona, vamos a añadir excepciones al listado de todos los posts y a la visualización de cada post por separado. Para ello, al fichero de rutas añadiremos:

\begin{mycode}{Rutas que están exentas del \textit{middleware} de autenticación}{php}{}
<?php
//...
Route::controller(PostController::class)->group(function () {
    Route::get('/posts', 'index')->name('posts.index');
    Route::get('/posts/{post}', 'show')->name('posts.show');
})->withoutMiddleware([Auth::class]);
\end{mycode}

Tal como se puede ver, hemos creado dos rutas del controlador “PostController” que se les indica “\texttt{->withoutMiddleware}”, para que no se aplique, en este caso, la comprobación de autenticación.


\chapter{Comprobar si el usuario está autenticado}

Por último, debemos asegurar que los botones de edición o borrado sólo aparezcan cuando el usuario esté logueado. Para ello tenemos el sistema \inlineconsole{@auth ... @endauth}. Si modificamos el fichero \configfile{posts/index.blade.php} para evitar que aparezcan los botones de edición y borrado de un post, quedaría:

\begin{mycode}{Comprobar si se está autenticado}{html+smarty}{{\small }}
@auth
  <a class="btn btn-warning btn-sm" href="{{route('posts.edit',$post)}}"
   role="button">Editar</a>

  <form action="{{route('posts.destroy',$post)}}" method="POST"
    enctype="multipart/form-data">
    @csrf
    @method('DELETE')
    <button class="btn btn-sm btn-danger" type="submit"
      onclick="return confirm('Are you sure?')">Delete
    </button>
  </form>
@endauth
\end{mycode}

    \part{Relacionar modelos}
    \chapter{Crear modelo relacionado}

Siguiendo con nuestro blog, donde ya tenemos una aplicación donde crear posts sólo si estamos logueado, es buen momento de añadir nuevas características. Vamos a incluir la opción de tener comentarios, al menos a nivel relacional.

Sin entrar en los atributos que tiene cada entidad/modelo, la relación que tienen los comentarios respecto a un \textit{post} sería la siguiente:

\begin{center}
    \includegraphics[width=0.5\linewidth]{e-r.png}
\end{center}

Es decir, un \textit{post} puede tener muchos comentarios. Un comentario sólo puede pertenecer a un \textit{post}. En principio el único atributo que vamos a permitir es el propio comentario, aparte de la fecha de creación. Para crear el modelo haríamos:

\begin{mycode}{Crear Modelo}{console}{{\small }}
root@1b29e46c10ae:/var/www/html# php artisan make:model Comentario -crms
\end{mycode}


\chapter{Crear migración}
Al igual que vimos al inicio, este comando nos ha creado el modelo, el controlador de \textit{resource}, el sistema de migración y el fichero para añadir la semilla a la base de datos. A la hora de generar la tabla, tenemos que hacer referencia a qué \textit{post} pertenece el comentario, por lo tanto el \textit{migration} queda:

\begin{mycode}{Crear migration}{php}{}
<?php
//...
public function up(): void{
    Schema::create('comentarios', function (Blueprint $table) {
        $table->id();
        $table->string('texto');
        $table->unsignedBigInteger('post_id');
        $table->foreign('post_id')->references('id')->on('posts');
        $table->timestamps();
    });
}
\end{mycode}

Tal como se puede ver, a la hora de crear la tabla en el \textit{migration} se ha creado un campo llamado “\textbf{post\_id}” que después se le ha indicado que es de tipo “clave foránea”. En la \href{https://laravel.com/docs/10.x/migrations#foreign-key-constraints}{documentación} se explican distintas opciones para este tipo de casos.

\exercisebox{\textbf{Crea un “seed” para añadir un comentario al primer \textit{post}}}

\chapter{Crear relación de modelos}

Hasta ahora la relación se ha creado a nivel de base de datos, pero es necesario que Laravel a nivel de \textit{framework}, mientras programamos, sea consciente de que los modelos están relacionados entre sí. Para ello, una vez más en la \href{https://laravel.com/docs/10.x/eloquent-relationships#one-to-many}{documentación} se puede ver cómo Eloquent hace uso de los distintos tipos de relaciones.

Para ello, deberemos modificar ambos ficheros de los modelos que entran en juego en esta relación:
\begin{itemize}
    \item Relación “\textbf{uno a muchos}”, donde un \textit{post} puede tener muchos comentarios. Modificaremos el modelo \configfile{App/Models/Post.php} para que contenga:
\begin{mycode}{Añadir relación “uno a muchos” en Post}{php}{}
<?php
//...
use Illuminate\Database\Eloquent\Relations\HasMany;
class Post extends Model{
    use HasFactory;
    public function comentarios(): HasMany {
        return $this->hasMany(Comentario::class);
    }
}
\end{mycode}

    \item Relación inversa “\textit{\textbf{BelongsTo}}”, donde un comentario pertenece a un \textit{post}. En este caso, modificaremos el modelo \configfile{App/Models/Comentario.php}.

\begin{mycode}{Añadir relación inversa en Comentario}{php}{}
<?php
//...
use Illuminate\Database\Eloquent\Relations\BelongsTo;

class Comentario extends Model{
    use HasFactory;
    public function post(): BelongsTo{
        return $this->belongsTo(Post::class);
    }
}
\end{mycode}
\end{itemize}

Tras esto, ya sea a través de una acción o desde Tinker, podremos obtener los comentarios de un \textit{post} específico, perfecto para dibujarlos en la vista donde se visualiza el \textit{post}. Y al revés, dado un comentario, obtener a qué \textit{post} pertenece.

    \part{Cómo crear una API}
    \chapter{Introducción}

Las \textbf{API} (en inglés “\textit{application programming interface}”) son hoy en día una parte fundamental de servicios y aplicaciones. Nos permite obtener datos, comunicar aplicaciones entre sí, y realizar una separación entre la parte lógica de la aplicación y la parte visual, pudiendo ser esta última una aplicación web, una de móvil, de televisión...

Es por eso que aprender y entender cómo crear una API es una parte fundamental que todo programador debe conocer, ya que de esta manera vamos a entender de mejor manera cómo funcionan. Esto nos será muy útil también, incluso, para utilizarlas.

Por todo ello, a continuación se va a explicar cómo hacer que nuestra aplicación cuente con API, para poder ser utilizada desde otra aplicación o para ser utilizada para obtener datos desde otro tipo de interfaz. Para ello, cabe recordar que las peticiones y los resultados deben ir en formato \href{https://es.wikipedia.org/wiki/JSON}{JSON}.


\chapter{Rutas para la API}

Hasta ahora hemos hecho uso del fichero \configfile{routes/web.php} para añadir rutas a nuestra aplicación. En el mismo directorio podemos ver que existe otro fichero llamado \configfile{routes/api.php}, que, como su propio nombre indica, vamos a utilizar para añadir las rutas que utilizaremos en nuestra API.

Si observamos el fichero, vemos ya existe una ruta creada:

\begin{mycode}{Contenido del fichero api.php}{php}{{\small }}
<?php
use Illuminate\Http\Request;
use Illuminate\Support\Facades\Route;

Route::middleware('auth:sanctum')->get('/user', function (Request $request) {
    return $request->user();
});
\end{mycode}

Si obtenemos el listado de rutas, veremos que ya existe una ruta para conocer el estado del usuario. A continuación vamos a añadir las rutas correspondientes para toda la gestión de los “posts” de nuestra aplicación.

Debemos recordar que para poder añadir las nuevas rutas, hay que incluir el controlador correspondiente, en este caso el \textbf{PostController}.

\begin{mycode}{Añadir nuevas rutas al fichero api.php}{php}{}
<?php
...
use App\Http\Controllers\PostController;
Route::resources([
    'posts' => PostController::class,
]);
\end{mycode}

Si ahora visualizamos el listado de rutas completo veremos las nuevas rutas. Si sólo nos interesan las rutas específicas a la API, podemos añadir un parámetro indicando sólo parte de la ruta, como se muestra a continuación:


\begin{mycode}{Añadir nuevas rutas al fichero web.php}{console}{{\small }}
root@5cff1feaf785:/var/www/html# php artisan route:list --path=api

GET|HEAD        api/posts ............ posts.index › PostController@index
POST            api/posts ............ posts.store › PostController@store
GET|HEAD        api/posts/create ..... posts.create › PostController@create
GET|HEAD        api/posts/{post} ..... posts.show › PostController@show
PUT|PATCH       api/posts/{post} ..... posts.update › PostController@update
DELETE          api/posts/{post} ..... posts.destroy › PostController@destroy
GET|HEAD        api/posts/{post}/edit  posts.edit › PostController@edit
GET|HEAD        api/user ....................................................
\end{mycode}

Por otro lado, dado que la API es un sistema en el que no vamos a guardar un estado (ya que haremos uso de \textit{\textbf{tokens}}), debemos hacer dos pequeños cambios a ficheros de configuración general.


En \configfile{app\Http\kernel.php} hay que modificar la sección de la API dejando:

\begin{mycode}{Modificar kernel.php}{php}{\small}
<?php
//...
'api' => [
    \Laravel\Sanctum\Http\Middleware\EnsureFrontendRequestsAreStateful::class,
    \Illuminate\Routing\Middleware\ThrottleRequests::class.':api',
    \Illuminate\Routing\Middleware\SubstituteBindings::class,
],
\end{mycode}

También hay que deshabilitar el \href{https://en.wikipedia.org/wiki/Cross-site\_request\_forgery}{CSRF} para estas rutas. Para ello modificamos el fichero \configfile{app\Http\Middleware\VerifyCsrfToken.php} dejando:

\begin{mycode}{Quitamos la protección CSRF a las rutas de la API}{php}{}
<?php
//...
    protected $except = [
        'api/*'
    ];
\end{mycode}

\chapter{Uso de controladores para la API}

Para que la modificación previa funcione es necesario modificar el controlador, ya que actualmente sólo devuelve la vista en formato código HTML. Por tanto, si queremos utilizar el mismo controlador, deberemos modificar las funciones. En el caso de la función “index” del PostController queda:


\begin{mycode}{Modificar la función de PostController}{php}{}
<?php
...
public function index(Request $request) {
    $posts = Post::orderBy('created_at')->get();
    if ($request->expectsJson()) {
        return response()->json($posts);
    } else {
        return view('posts.index',['posts' => $posts]);
    }
}
\end{mycode}

Tal como se puede ver, la función recibe dos modificaciones:

\begin{itemize}
    \item \textbf{Añadir parámetro “Request”}: De esta manera, podremos conocer si la petición viene desde la web, o si por el contrario se espera la respuesta en formato JSON.

    \item \textbf{Comprobar qué se espera}: Tal como se puede ver, se ha añadido un “if” donde se mira si la petición se espera en formato JSON (“expectsJson()”). En caso afirmativo, se devuelve la respuesta correspondiente en formato JSON.
\end{itemize}

\errorbox{\textbf{Es recomendable hacer uso de controladores específicos para la API}}



\section{Crear controladores específicos}

Debido a que las API se suelen versionar, es recomendable mantener los controladores de la web y de la API separados. Esto permite seguir el principio \textbf{\textit{KISS}} (\textit{Keep It Simple, Stupid!}). De esta manera se va a poder realizar modificaciones en un apartado de nuestra aplicación sin temer que podemos romper otra parte.

Es por ello, que lo ideal es crear controladores específicos para las funcionalidades que va a tener la API, y que se encuentren separados. Para ello realizaremos lo siguiente:

\begin{itemize}
    \item \textbf{Deshacer los cambios} de la función “index” vistos en el apartado anterior.
    \item \textbf{Crear nuevo controlador} que será específico para la API:

\begin{mycode}{Crear nuevo PostController exclusivo para la API}{console}{{\small}}
# php artisan make:controller API/PostController --api --model=Post
\end{mycode}

    Es necesario explicar lo siguiente:
    \begin{itemize}
        \item \textbf{“API/PostController”}: Esto indica cuál es la ruta donde se creará el fichero, que en este caso es \configfile{app/Http/Controllers/API/PostController.php}.

       \item \textbf{\texttt{--api}}: Este parámetro va a generar un controlador que carece de las funciones “create” y “edit”, ya que no son necesarias en una API, dado que son exclusivas a visualizar los formularios en un interfaz web.

       \item \textbf{\texttt{--model=Post}}: Para que el nuevo fichero del controlador ya tenga el include del modelo necesario.
    \end{itemize}

    \item \textbf{Modificar la ruta para la API} y que de esta manera haga uso del nuevo controlador exclusivo. El cambio es el siguiente:

\begin{mycode}{Modificar api.php para el nuevo controlador}{php}{}
<?php
...
use App\Http\Controllers\API\PostController;
\end{mycode}

    \item Modificar nuevo controlador, para que devuelva los datos correspondientes:

\begin{mycode}{Modificar el nuevo controlador PostController}{php}{}
<?php
...
use Illuminate\Http\Response;
public function index(){
    $posts = Post::orderBy('created_at')->get();
    return response()->json(['posts'=>$posts])
        ->setStatusCode(Response::HTTP_OK);
}
\end{mycode}

En este caso no hemos realizado ninguna comprobación, pero en una API de verdad debemos comprobar si se han encontrado resultados, y dependiendo de ellos devolver un \textbf{estado de respuesta distinto}.

La \href{https://github.com/symfony/symfony/blob/6.3/src/Symfony/Component/HttpFoundation/Response.php}{librería} contiene una gran cantidad de variables para hacer referencia a los posibles códigos que podemos devolver, así como el texto que acompañan al código.

\end{itemize}


\chapter{Comprobar funcionamiento}
Es momento de comprobar que todo funciona de manera correcta, y para ello debemos realizar una petición a la URL \href{http://localhost/api/posts}{http://localhost/api/posts} teniendo en cuenta el ejemplo que hemos estado realizando.

Para realizar la prueba podemos hacerlo de distintas formas, cada una de ellas dependiendo de la motivación que tengamos:

\begin{itemize}
    \item Utilizando un interfaz gráfico como \href{https://www.postman.com/}{Postman}, que nos va a facilitar hacer peticiones GET y peticiones más complejas.

    \begin{center}
        \includegraphics[width=0.7\linewidth]{postman.png}
    \end{center}

    \item Desde el propio \textbf{navegador web}. Veremos los datos JSON devueltos en formato texto directamente. Para comprobar que funciona, puede ser más que suficiente.




    \item Desde una \textbf{consola Linux}, haciendo uso del comando \textbf{curl}, podemos también comprobar de manera rápida si el \textit{endpoint} está funcionando:

\begin{mycode}{Modificar api.php para el nuevo controlador}{console}{}
ruben@vega:~$ curl -s  http://localhost/api/posts
{"posts":[{"id":1,"titulo":"Primer post111","texto":"Este es...”}]}
\end{mycode}

    Si queremos tener un resultado más visual, podremos hacer uso del comando “\textbf{jq}”, que deberemos instalarlo. De esta manera, podremos hacer:

\begin{mycode}{Modificar api.php para el nuevo controlador}{console}{}
ruben@vega:~$ curl -s  http://localhost/api/posts | jq
{
    "posts": [
      {
        "id": 1,
        "titulo": "Primer post111",
        "texto": "Este es el texto del primer post",
        "publicado": 1,
        "deleted_at": null,
        "created_at": "2023-10-04T08:20:29.000000Z",
        "updated_at": "2023-11-16T08:08:27.000000Z"
      }
    ]
}
\end{mycode}


\end{itemize}

\chapter{Autenticación}

Para poder realizar ciertas acciones a través de la API es lógico pensar que también deberemos estar autenticados, y para eso es necesario asegurar que al acceder a las rutas lo estemos. Para todo ello, vamos a crear un controlador propio donde tener en cuenta los datos que se envían al acceder a la API.

La idea general es que una aplicación al hacer uso de una API debe tener en cuenta:

\begin{itemize}
    \item Si el usuario no está registrado, poder registrarse.
    \item Si el usuario está registrado, se podrá loguear con lo que recibirá un \textbf{\textit{token}}.
    \item A partir de este momento, cada acción que quiera realizar, deberá ir acompañado del token para demostrar que está autenticado.
    \item Los \textit{tokens} pueden tener una vida útil. Por lo tanto, si el token expira, deberá volver a loguearse.
\end{itemize}

\section{Crear controlador de autenticación}

Se va a crear un controlador propio para tener el control de las acciones que se pueden realizar a través de la API, y así asegurar cuál es el estado de los tokens y/o del usuario que pide realizar una acción.

Para crear un controlador propio que sólo será usado para la API:

\begin{mycode}{Crear nuevo controlador}{console}{{\footnotesize }}
root@5cff1feaf785:/var/www/html# php artisan make:controller API/AuthController
INFO  Controller [app/Http/Controllers/API/AuthController.php] created successfully.
\end{mycode}

En este controlador vamos a tener tres funciones:

\begin{itemize}
    \item \textbf{register}: para que el usuario se pueda registrar.
    \item \textbf{login}: para que el usuario se pueda loguear y recibir un token.
    \item \textbf{logout}: para que el usuario se pueda desloguear y de esta manera el token se revoque.
\end{itemize}

El controlador, quedaría de la siguiente manera.

\errorbox{\textbf{Es importante entender qué hace cada una de las funciones}}

\begin{mycode}{Nuevo AuthController para la API}{php}{{\small }}
<?php
namespace App\Http\Controllers\API;
use App\Http\Controllers\Controller;
use Illuminate\Http\Request;

use App\Models\User;
use Illuminate\Support\Facades\Hash;
use Illuminate\Validation\ValidationException;
use Illuminate\Http\Response;

class AuthController extends Controller {

    public function register(Request $request){
        $validatedData = $request->validate([
        'name' => 'required|string|max:255',
        'email' => 'required|string|email|max:255|unique:users',
        'password' => 'required|string|min:8',
        ]);

        $user = User::create([
        'name' => $validatedData['name'],
        'email' => $validatedData['email'],
        'password' => Hash::make($validatedData['password']),
        ]);
        return response()->json([
        'name' => $user->name,
        'email' => $user->email,
        ])->setStatusCode(Response::HTTP_CREATED);
    }

    public function login(Request $request){
        $request->validate([
        'email' => 'required|email',
        'password' => 'required',
        'device_name' => 'required',
        ]);
        $user = User::where('email',  $request->email)->first();

        if (! $user || ! Hash::check($request->password, $user->password)){
            return response()->json([
            'message' => ['Username or password incorrect'],
            ])->setStatusCode(Response::HTTP_UNAUTHORIZED);
        }
        // FIXME: queremos dejar más dispositivos?
        // $user->tokens()->delete();

        return response()->json([
        'status' => 'success',
        'message' => 'User logged in successfully',
        'name' => $user->name,
        'token' => $user->createToken($request->device_name)
        ->plainTextToken,
        ]);
    }

    public function logout(Request $request){
        $request->user()->currentAccessToken()->delete();
        return response()->json([
        'status' => 'success',
        'message' => 'User logged out successfully'
        ])->setStatusCode(Response::HTTP_OK);
    }
}
\end{mycode}

Es importante notar un comentario que se ha dejado en la función “\textbf{login}”. Dependiendo de si queremos que la API permita tener varios tokens para un mismo usuario o no (posibles logins desde distintos dispositivos), deberemos dejar comentado o descomentar la línea indicada.

Para que estas funciones entren en juego, debemos modificar el fichero de rutas \configfile{api.php}

\begin{mycode}{Rutas de autenticación para la API}{php}{}
<?php
//...
Route::post('/register', [AuthController::class, 'register']);
Route::post('/login', [AuthController::class, 'login']);
Route::post('/logout', [AuthController::class, 'logout'])
    ->middleware('auth:sanctum');
\end{mycode}




\section{Modificar rutas}

Tal como hemos hecho anteriormente, para que la aplicación funcione bajo el sistema de autenticación, y que automáticamente nos indique que no estamos autenticados, debemos realizar la modificación del fichero de rutas \configfile{api.php}


De manera similar a la aplicación web, deberemos indicar qué rutas queremos que se puedan obtener sin estar autenticado y cuáles no.


\begin{mycode}{Rutas autenticadas para la API}{php}{}
<?php
//...
Route::middleware(['auth:sanctum'])->group(function () {
    Route::resources([
        'posts' => PostController::class,
    ]);
});

Route::controller(PostController::class)->group(function () {
    Route::get('/posts', 'index')->name('posts.index');
    Route::get('/posts/{post}', 'show')->name('posts.show');
})->withoutMiddleware(['auth:sanctum']);
\end{mycode}

\section{Pruebas de funcionamiento}


Para que la autenticación funcione, podemos realizar pruebas a través de Postman, de manera similar a como hemos hecho anteriormente. Ahora hay que tener en cuenta que la petición va a ser de tipo \textbf{POST}, y dado que queremos crear un \textit{post}, tendremos que realizar el paso de parámetros.

Por lo tanto, los pasos que debemos realizar son:

\begin{itemize}
    \item Realizar prueba de login. Debemos pasar los parámetros y obtendremos el token y una serie de datos extra, que podremos utilizar en nuestra aplicación.
    \begin{itemize}
        \item email
        \item password
        \item device\_name
    \end{itemize}

    \begin{center}
        \includegraphics[width=0.8\linewidth]{postman_login.png}
    \end{center}

    \item Con ese token podremos realizar la prueba de añadir un nuevo \textit{post} a través de la API.

    \begin{center}
        \includegraphics[width=0.8\linewidth]{postman_create.png}
    \end{center}

    Deberemos indicar:

    \begin{itemize}
        \item Los parámetros necesarios para la creación del \textit{post}: “título”, “texto” y “publicado”.
        \item El tipo de token de autenticación en la pestaña “Auth” de Postman tiene que ser de tipo “\textbf{Bearer Token}”.
    \end{itemize}

\end{itemize}


\chapter{Visualizar API}

Suele ser habitual tener un interfaz donde se muestra el funcionamiento de nuestra API, o cuáles son los \textit{endpoints} de la misma. Es decir, qué URL se pueden consultar, qué método hay que utilizar, si es necesario el paso de parámetros, ...

Hoy en día existe la especificación \href{https://www.openapis.org/}{OpenAPI} para la generación de la API, y sobre ella existen distintos interfaces web. Uno de los interfaces más utilizados es \href{https://swagger.io/tools/swagger-ui/}{Swagger UI} que nos muestra un bonito interfaz y a la vez es posible utilizarlo para realizar consultas a la API.

Para poder instalarlo en nuestro proyecto Laravel, necesitamos realizar la instalación de unas dependencias y la posterior instalación en el proyecto.

\begin{mycode}{Instalación de dependencias}{console}{{\small}}
root@5cff1feaf785:/var/www/html# composer require "darkaonline/l5-swagger"
root@5cff1feaf785:/var/www/html# php artisan vendor:publish \
    --provider "L5Swagger\L5SwaggerServiceProvider"
\end{mycode}


Para poder generar el interfaz de manera correcta añadimos comentarios a las funciones. En uno de los controladores añadiremos la siguiente cabecera, que nos va a servir para definir el tipo de autenticación:

\begin{mycode}{Cabecera para la API}{php}{}
<?php
//...
/**
* @OA\Info(title="API", version="1.0"),
* @OA\SecurityScheme(
*     in="header",
*     scheme="bearer",
*     bearerFormat="JWT",
*     securityScheme="bearerAuth",
*     type="http",
* ),
*/
\end{mycode}


Para documentar la función \textbf{index}, encima de ella añadiremos lo siguiente. Hay que darse cuenta que en este caso sólo hemos documentado la respuesta “200”.

\begin{mycode}{Comentario para /api/posts}{php}{}
<?php
//...
/**
* @OA\Get(
*     path="/api/posts",
*     summary="Mostrar posts",
*     @OA\Response(
*         response=200,
*         description="Mostrar todos los posts."
*     ),
*     @OA\Response(
*         response="default",
*         description="Ha ocurrido un error."
*     )
* )
*/
public function index(){
//...
\end{mycode}

Para documentar la función de guardar un \textit{post} desde la API, usaremos los siguientes comentarios:

\begin{mycode}{Comentario para función POST /api/posts}{php}{}
<?php
//...
    /**
* @OA\Post(
*     path="/api/posts",
*     summary="Create a post",
*     @OA\Parameter(
*         name="titulo",
*         in="query",
*         description="The title of the post",
*         required=true,
*         @OA\Schema(
*             type="string"
*         )
*     ),
*     @OA\Response(
*         response=200,
*         description="successful operation",
*         @OA\JsonContent(
*             type="string"
*         ),
*     ),
*     @OA\Response(
*         response=401,
*         description="Unauthenticated"
*     ),
*     security={
    *         {"bearerAuth": {}}
    *     }
* )
*/
public function store(Request $request){
\end{mycode}


Por último, vamos a poner otro ejemplo, para documentar el obtener un \textit{post} concreto. Para ello los comentarios serán:

\begin{mycode}{Comentario para /api/posts/{id}}{php}{}
<?php
//...
/**
* @OA\Get(
*     path="/api/posts/{id}",
*     summary="Mostrar un post concreto",
*     @OA\Parameter(
*          name="id",
*          description="Project id",
*          required=true,
*          in="path",
*          @OA\Schema(
*              type="integer"
*          )
*     ),
*     @OA\Response(
*         response=200,
*         description="Mostrar el post especificado."
*     ),
*     @OA\Response(
*         response="default",
*         description="Ha ocurrido un error."
*     )
* )
*/
public function show(Post $post){
\end{mycode}


Y para la función \textbf{show}, que sólo nos muestra un único post, añadiremos los siguientes comentarios:

\begin{mycode}{Comentario para /api/posts/ID}{php}{}
<?php
...
/**
* @OA\Get(
*     path="/api/posts/{id}",
*     summary="Mostrar un post concreto",
*     @OA\Parameter(
*          name="id",
*          description="Project id",
*          required=true,
*          in="path",
*          @OA\Schema(
*              type="integer"
*          )
*     ),
*     @OA\Response(
*         response=200,
*         description="Mostrar el post especificado."
*     ),
*     @OA\Response(
*         response="default",
*         description="Ha ocurrido un error."
*     )
* )
*/
public function show(Post $post) {
    //...
\end{mycode}


No se han añadido todos los parámetros en todos los casos, ya que resulta redundante y es fácil añadir los que faltan.

Para conocer todas las funcionalidades de los comentarios, es recomendable mirar la \href{https://github.com/zircote/swagger-php#usage}{documentación}. Desde aquí nos mostrará el enlace para la \href{https://zircote.github.io/swagger-php/#links}{documentación oficial de Swagger PHP}, o lo que nos puede resultar más interesante, que es un conjunto de \href{https://github.com/zircote/swagger-php/tree/master/Examples}{ejemplos} junto con el \href{https://petstore.swagger.io/#/}{resultado en forma de web}.


Tras esto, ejecutaremos el comando que recorrerá los controladores para generar el fichero \configfile{storage/api-docs/api-docs.json}. Este es el fichero que el interfaz web tendrá en cuenta a la hora de generar la web que podemos ver a continuación.

El comando es el siguiente:

\begin{mycode}{Generamos el fichero json para Swagger}{console}{}
root@5cff1feaf785:/var/www/html# php artisan l5-swagger:generate
Regenerating docs default
\end{mycode}



Tras ejecutar el comando anterior, si vamos a la url \href{http://localhost/api/documentation}{http://localhost/api/documentation} tendremos acceso y veremos el interfaz para nuestra API:

\begin{center}
    \includegraphics[frame,width=0.8\linewidth]{swagger.png}
\end{center}



\exercisebox{Completar los comentarios añadiendo los parámetros que faltan a la función de crear y los necesarios para la función de borrado de posts.}

% Mostrar SWAGGER.io
% https://styde.net/como-documentar-una-api-en-laravel-usando-swagger/
% https://github.com/DarkaOnLine/L5-Swagger
% https://github.com/zircote/swagger-php
% https://zircote.github.io/swagger-php/







%\chapter{Versionar API}
% https://laravel.io/articles/api-versioning-in-laravel
% https://dev.to/dalelantowork/laravel-8-api-versioning-4e8

    \part{Puesta en producción}
    \chapter{Directorios ignorados del proyecto}

Al crear el proyecto, se nos ha creado un fichero \configfile{.gitignore} en el que aparecen distintos directorios que están ignorados a la hora de añadir el proyecto a un repositorio Git.

Estos ficheros y directorios no es necesario que estén subidos al repositorio, ya sea porque son ficheros que deben ser generados o porque contienen configuración y contraseñas.

A continuación un listado de algunos ficheros y directorios y la explicación de por qué están ignorados.

\begin{itemize}
    \item \configdir{node_modules} : Es el directorio donde se guardan los paquetes, y sus dependencias, que se han descargado a través del gestor de paquetes \href{https://www.npmjs.com/}{NPM}. Esta configuración aparece en el fichero de configuración \configfile{package.json} .

    \item \configdir{public/build} : En este directorio se guardan los ficheros (javascript y css, entre otros) que se generan a través del comando \commandbox{npm run build} . Este comando debe ejecutarse antes de la puesta en producción

    \item \configdir{vendor} : En este caso es el directorio de las librerías que son necesarias a través del gestor de paquetes \href{https://getcomposer.org/}{Composer}. Estas librerías, y sus dependencias, son las que el proyecto Laravel necesita y aparecen en el fichero \configfile{composer.json} .

    \item \configfile{.env} : Es el fichero de configuración general de la aplicación y donde aparecen los servicios y sus contraseñas. Nunca debería subirse un fichero de configuración con contraseñas al repositorio. Para asegurar que no faltan configuraciones, se podría subir un fichero igual al \configfile{.env}, pero sin que aparezcan las contraseñas.

    \item \configdir{.vscode} : No es recomendable subir ficheros de configuración de IDEs, ya que cada desarrollador puede querer configuraciones propias.
\end{itemize}


\chapter{Poner proyecto en producción}

La puesta en producción de un proyecto es un punto crítico, ya que cualquier fallo o problema puede ocasionar que nuestra aplicación no funcione de manera correcta. Es por eso, que siempre debería haber un “guión” para indicar los pasos a realizar y que no se nos olvide ninguno durante la puesta en producción.

Los pasos a seguir dependerán de la aplicación o servicio que vayamos a poner en producción. Para el caso de Laravel, y tal como hemos estado realizando el desarrollo, serán los siguientes:

\begin{itemize}
    \item Clonar el proyecto.
    \item Crear contenedores temporales.
    \item Realizar la instalación de las dependencias necesarias.
    \item Ejecutar la construcción de los ficheros \textit{assets} necesarios (javascripts y css).
    \item Crear contenedores finales.
    \item Ejecutar \textit{migrations} y/o \textit{seeds} necesarios.
    \item Testear que todo funciona de manera correcta.
\end{itemize}

Para esta explicación se partirá de una instalación de Ubuntu LTS con Docker y Docker Compose instalado. Hay que tener en cuenta que la puesta en producción utilizando otro sistema de instalación puede variar en alguno de los pasos.

\warnbox{Los pasos de la puesta en producción pueden variar dependiendo del sistema de instalación utilizado, pero serán parecidos a los explicados aquí.}


\section{Clonar el proyecto}

Este paso no tiene mayor dificultad, ya que se presupone que tenemos nuestro proyecto en un repositorio Git en algún tipo de plataforma centralizada (como puede ser Github o GitLab).

El clonado del repositorio lo realizaremos como cualquier otro proyecto, por lo que no se explicará cómo realizarlo. Hay que recordar que en este clonado faltarán ficheros y directorios tal como se ha explicado previamente.


\section{Crear contenedores temporales}

Para poder crear y levantar los contenedores necesarios necesitamos del fichero de configuración donde están las contraseñas. Para este ejemplo, copiaremos el fichero \configfile{.env} del directorio de desarrollo.

\warnbox{En producción suele haber un SGBD en otro servidor configurado para varios proyectos, por eso, entre otras cosas, el fichero de configuración será distinto.}

Ahora, levantaremos los contenedores a través del comando \commandbox{docker compose up -d}. Debido a que nos siguen faltando directorios de las dependencias, el servidor MySQL no se levantará, pero más adelante se solucionará.

Por lo tanto, deberemos ejecutar lo siguiente:

\begin{mycode}{Usamos el instalador de Laravel}{console}{{\small }}
ruben@vega:~/proyecto_produccion$ docker compose up -d
WARN[0000] The "WWWUSER" variable is not set. Defaulting to a blank string.
WARN[0000] The "WWWGROUP" variable is not set. Defaulting to a blank string.
[+] Running 3/2
 ✓ Network proyecto_produccion_sail              Created    0.1s
 ✓ Container proyecto_produccion-mysql-1         Created    0.0s
 ✓ Container proyecto_produccion-laravel.test-1  Created    0.0s
Attaching to proyecto_produccion-laravel.test-1, proyecto_produccion-mysql-1
\end{mycode}

Tal como se ha comentado, el contenedor de MySQL no se va a levantar debido a que en el fichero de configuración \configfile{docker-compose.yml} aparecen configuraciones de ficheros dentro del directorio \configdir{vendor} que todavía no existen.


\section{Instalación de las dependencias necesarias}

Las dependencias que se necesitan instalar son componentes del \textit{framework} Laravel, y por tanto para ello necesitamos entrar al contenedor y ejecutar ciertos comandos, tal como se va a mostrar a continuación. Más adelante se explicarán los comandos realizados.

\begin{mycode}{Usamos el instalador de Laravel}{console}{{\footnotesize}}
ruben@vega:~/proyecto_produccion$ docker compose exec -it laravel.test /bin/bash

root@469e75d4a713:/var/www/html# composer install
Installing dependencies from lock file (including require-dev)
Verifying lock file contents can be installed on current platform.
Package operations: 111 installs, 0 updates, 0 removals
...
root@469e75d4a713:/var/www/html# npm install
added 37 packages, and audited 38 packages in 1s
...
Run `npm audit` for details.
root@469e75d4a713:/var/www/html# npm run build
> build
> vite build
...
✓ built in 3.85s
root@469e75d4a713:/var/www/html# chmod 777 -R storage/
\end{mycode}

A continuación la explicación de los comandos ejecutados, ya que es importante entender qué se ha realizado:

\begin{itemize}
    \item \commandbox{docker compose exec -it laravel.test /bin/bash} : con este comando vamos a entrar dentro del contenedor de Laravel, donde debemos realizar la instalación de las dependencias.

    \item \commandbox{composer install} : instalamos las depedencias necesarias a través del gestor de dependencias \href{https://getcomposer.org/}{Composer}.

    \item \commandbox{npm install} : instalamos las depedencias necesarias a través del gestor de dependencias \href{https://www.npmjs.com/}{NPM}.

    \item \commandbox{npm run build} : es necesario generar los assets de javascript y CSS para que la aplicación funcione.

    \item \commandbox{chmod 777 -R storage/} : Modificamos los permisos de ciertos ficheros para que el servidor web pueda escribir datos.
\end{itemize}


\section{Crear contenedores finales}

Ahora que ya tenemos las dependencias instaladas, es momento de poder parar los contenedores temporales, eliminarlos y levantar los contenedores finales. Para ello, realizaremos los siguientes pasos, desde nuestra consola de Ubuntu:

\begin{mycode}{Usamos el instalador de Laravel}{console}{{\footnotesize}}
ruben@vega:~/proyecto_produccion$ docker compose down -v
✓ Container proyecto_produccion-laravel.test-1  Removed    0.9s
✓ Container proyecto_produccion-mysql-1         Removed    0.0s
✓ Volume proyecto_produccion-mysql              Removed    0.0s
✓ Network proyecto_produccion_sail              Removed

ruben@vega:~/proyecto_produccion$ ./vendor/bin/sail up -d
\end{mycode}

Los comandos realizados son:
\begin{itemize}
    \item \commandbox{docker compose down -v} : tira abajo los contenedores temporales y borra los volúmenes que utilizan los contenedores. El parámetro \inlineconsole{-v} es necesario para borrar el volumen de MySQL ya que durante el arranque del contenedor anterior ha tenido errores y por tanto

    \item \commandbox{./vendor/bin/sail up -d} : este comando ya lo conocemos, y es para levantar los contenedores necesarios de la aplicación.
\end{itemize}

A partir de este momento la aplicación estará funcionando y habrá que ejecutar las migraciones y/o los \textit{seeds} necesarios para ajustar la base de datos a la realidad de nuestra aplicación.

    \part{Introducción a Docker}
    \graphicspath{{../../otros/Docker/}}
    \chapter{Introducción}

Hoy en día es muy habitual hacer uso de los sistemas de contenedores, el más conocido es \href{https://docs.docker.com/}{Docker}, en el mundo del desarrollo de \textit{software}. Este sistema trae consigo una serie de ventajas que veremos más adelante, que nos permite asegurar, entre otras cosas, que las versiones utilizadas en el entorno de producción son las mismas que durante las etapas de desarrollo.

En este documento se va a explicar cómo realizar la instalación y configuración de un sistema basado en contenedores Docker para poder arrancar servicios, y ciertas configuraciones que son necesarias conocer.

\chapter{Sistemas de contenedores}

Los sistemas de contenedores son un método de virtualización (conocido como “virtualización a nivel de sistema operativo”), en el que se permite ejecutar sobre una capa virtualizadora del núcleo del sistema operativo distintas instancias de “espacio de usuario”.

Este “espacio de usuario” (donde se ejecutarán aplicaciones, servicios...) se les denomina \textbf{contenedores}, y aunque pueden ser como un servidor real, están bajo un mecanismo de aislamiento proporcionado por el \textit{kernel} del sistema operativo, y sobre el que se pueden aplicar límites de espacio, recursos de memoria, de acceso a disco...

\infobox{\textbf{Un contenedor es un espacio de ejecución de servicios al que se les puede aplicar límites de recursos (como la memoria, el acceso a disco...)}}

Desde el punto de vista del usuario, que un servicio se ejecute en una máquina virtual o en un contenedor es indistinguible. En cambio, desde el punto de vista de un administrador de sistemas o de un desarrollador, el uso de contenedores trae consigo una serie de ventajas que veremos en apartados posteriores.


\section{Un poco de historia}
Aunque está muy en boga el despliegue de aplicaciones haciendo uso de contenedores, no es un concepto nuevo, ya que lleva existiendo desde la década de los 80 en sistemas UNIX con el concepto de \href{https://es.wikipedia.org/wiki/Chroot}{chroot}.

\textbf{Chroot}, también conocido como “jaulas chroot”, permitían ejecutar comandos dentro de un directorio sin que, en principio, se pudiese salir de dicha ruta. Tenía muy pocas restricciones de seguridad, pero era un primer paso al sistema de contenedores.

\href{https://es.wikipedia.org/wiki/LXC}{LXC} nace en 2008 utilizando distintas funcionalidades del kernel Linux para proveer un entorno virtual donde poder ejecutar distintos procesos y tener su propio espacio de red. Con LXC nacen distintas herramientas para controlar estos contenedores, así como para crear plantillas y una \textbf{API que permite interaccionar con LXC} desde distintos lenguajes de programación.

Ha habido otras tecnologías en Linux, como \href{https://es.wikipedia.org/wiki/OpenVZ}{OpenVz}, pero luego nos centraremos en Docker, ya que es lo más conocido actualmente.


\section{Qué es un contenedor y cómo se crea}

Para entender qué es un contenedor dentro de la infraestructura Docker y cómo se crea, tenemos que diferenciar distintos conceptos:
\begin{itemize}
    \item \textbf{Imagen Docker}
    \item \textbf{Contenedor Docker}
\end{itemize}

A continuación se van a detallar en profundidad.

\subsection{Imágenes Docker}

Para crear un contenedor necesitamos hacer uso de una \textbf{“imagen”, que es un archivo inmutable (no modificable) que contiene el código de la aplicación que queremos ejecutar y todas sus dependencias necesarias}, para que pueda ser ejecutada de manera rápida y confiable independientemente del entorno en el que se encuentre.

Las imágenes, debido a su origen \textbf{sólo-lectura}, se pueden considerar como \textbf{“plantillas”}, que son la representación de una aplicación y el entorno necesario para ser ejecutada en un momento específico en el tiempo. \textbf{Esta consistencia es una de las grandes características de Docker}.

\infobox{\textbf{Una imagen contiene el código y las dependencias que se necesita al crear un contenedor para ser ejecutado, independiente del entorno donde se ejecuta.}}

Una imagen puede ser creada utilizando otras imágenes como base. Por ejemplo, la imagen de \href{https://hub.docker.com/_/phpmyadmin}{PHPMyAdmin} empaqueta la aplicación PHPMyAdmin sobre la imagen \textbf{PHP} (versión 8.1-apache), que a su vez hace uso de la imagen \textbf{Debian} (versión 11-slim).


\begin{center}
    \includegraphics[width=0.9\linewidth]{img/docker/imagen1.png}
    \captionof{figure}{Jerarquía de imágenes usadas por PHPMyAdmin. \href{https://hub.docker.com/layers/library/phpmyadmin/latest/images/sha256-79e38dd8b2ab0e92505aa92040fa49dce4fa921a977b6ce4d030a63b4f120009?context=explore}{Fuente}}
\end{center}

A las imágenes creadas se les suele añadir etiquetas (\textbf{tags}) para diferenciar versiones o características internas. Cada creador determina las etiquetas que le interesa crear. Por ejemplo:
\begin{itemize}
    \item \textbf{latest}: Se le denomina a la última imagen creada.
    \item php:\textbf{8.1-apache}: Indica que en esta imagen PHP la versión es la 8.1 y además cuenta con Apache.
\end{itemize}

Podemos utilizar imágenes públicas descargadas a través de un \textbf{\textit{registry}} público, que no es otra cosa que un repositorio de imágenes subidas por la comunidad. El \textit{registry} principal más utilizado es \href{https://hub.docker.com/}{Docker Hub}.

\infobox{\textbf{Las imágenes Docker  pueden ser públicas o privadas y se almacenan en un repositorio llamado \underline{registry}, siendo el más conocido Docker Hub}}

\textbf{Se pueden crear nuestras propias imágenes privadas}, que pueden ser almacenadas en nuestros equipos o a través de un \textbf{registry privado} que podemos crear (también existen servicios de pago).


\subsection{Contenedores Docker}
Un contenedor Docker es un \textbf{entorno de tiempo de ejecución virtualizado donde los usuarios pueden aislar aplicaciones}. Estos contenedores son unidades compactas y portátiles a las que se les puede aplicar un sistema de limitación de recursos.

\infobox{\textbf{Un contenedor se crea a través de una imagen, es la versión ejecutable de la misma que se crea en un entorno virtualizado}}

Un contenedor se crea a través de una imagen y es la versión ejecutable de la misma. Lo que se hace es crear una capa de escritura sobre la imagen inmutable, donde se podrán escribir datos. Se pueden crear un número ilimitados de contenedores haciendo uso de la misma imagen base.

\vspace{-20pt}
\begin{center}
    \includegraphics[width=0.6\linewidth]{img/docker/contenedor.png}
    \captionof{figure}{Imagen “interna” de un contenedor}
\end{center}
\vspace{-15pt}

\textbf{La capa de escritura no es persistente} y se pierde al eliminar el contenedor, es decir, \textbf{los datos de un contenedor se eliminan al borrar el contendor}. Para evitar este comportamiento se puede hacer uso de un \hyperlink{volumen_persistente_datos}{volumen persistente de datos}, de esta manera esos datos no se pierden.

\errorbox{\textbf{Los datos creados dentro de un contenedor se borran al eliminar el contenedor}}


\section{Contenedores vs. Máquinas virtuales}

El uso de máquinas virtuales está muy extendido gracias a que cada vez es más sencillo crearlas. Esto no quiere decir que siempre sea la mejor opción, por lo que  se va a realizar una comparativa teniendo en cuenta distintos aspectos a la hora de realizar un desarrollo con máquinas virtuales y con sistemas de contenedores.


\subsection{Infraestructura}

La creación de máquinas virtuales nos permite crear entornos aislados en los que poder instalar el Sistema Operativo que más nos interese y con ello poder instalar el software y los servicios que necesitemos.

Las máquinas virtuales se virtualizan a nivel de hardware, donde debe existir un Sistema Operativo con Hypervisor que permita dicha virtualización. Por otro lado, los contenedores se virtualizan en la capa de aplicación, haciendo que este sistema sea mucho más ligero, permitiendo utilizar esos recursos en los servicios que necesitamos hacer funcionar dentro de los contenedores.

\vspace{-15pt}
\begin{center}
    \includegraphics[width=0.85\linewidth]{img/docker/docker\_vs\_vm.png}
    \captionof{figure}{Infraestructura Máquinas Virtuales vs Docker}
\end{center}
\vspace{-15pt}


En la imagen se puede apreciar una comparativa diferenciando cómo quedaría una infraestructura de 3 aplicaciones levantadas en distintas máquinas virtuales o en distintos contenedores.

Tal como se puede ver en la imagen, \textbf{al tener cada servicio en una máquina virtual separada}, se va a tener que virtualizar todo el Sistema Operativo en el que se encuentre, con el consiguiente \textbf{coste de recursos (memoria RAM y disco duro) y con el coste en tiempo de tener que realizar la configuración y securización del mismo}.

\infobox{\textbf{Usando contenedores la infraestructura se simplifica notáblemente}}

Por otro lado, en un sistema de contenedores, cada contenedor es un servicio aislado, en el que sólo tendremos que preocuparnos (en principio) de configurar sus parámetros.


\subsection{Ventajas durante el desarrollo}

A la hora de desarrollar una aplicación es habitual hacer pruebas utilizando distintas versiones de librerías, \textit{frameworks} o versiones de un mismo lenguaje de programación. De esta manera, podremos ver si nuestra aplicación es compatible.

Cuando se hace uso de una máquina virtual dependemos de las versiones que tiene nuestra distribución y es posible que no podamos instalar nuevas versiones u otras versiones en paralelo.

Por ejemplo, la última versión de PHP actualmente es la 8.2.4 y de Apache la 2.4.56:

\begin{itemize}
    \item En\textbf{ Debian 11} sólo se puede instalar PHP 7.4 y Apache 2.4.54.
    \item En \textbf{Ubuntu 22.04} la versión de PHP es la 8.1 y la de Apache la 2.4.52.
\end{itemize}

Con Docker, podremos levantar contenedores con distintas versiones del servicio que nos interese en paralelo para comprobar si nuestra aplicación/servicio es compatible.

\infobox{\textbf{Con Docker es posible levantar servicios con distintas versiones en paralelo}}

Por otro lado, si un desarrollador quiere utilizar un sistema operativo distinto, no se tendrá que preocupar de si su distribución tiene las mismas versiones. O en el caso de usar Windows/Mac, no tener que estar realizando instalaciones de las versiones concretas.


\subsection{Ventajas durante la puesta en producción}
Ligado al apartado anterior, durante la puesta en producción es obligatorio hacer uso de las mismas versiones utilizadas durante el desarrollo para asegurar la compatibilidad.

\errorbox{\textbf{Para asegurar la compatibilidad en producción, siempre se debe usar la misma versión de los servicios que en desarrollo}}

Si tenemos un servidor que no está actualizado, o en el mismo servidor tenemos distintas aplicaciones que requieren utilizar distintas versiones de software, en un entorno de máquinas virtuales se hace muy complejo, ya que lo habitual será tener que instalar nuevas máquinas virtuales.

\warnbox{\textbf{No siempre es posible tener distintas versiones del mismo software en un mismo servidor}}

En un entorno con contenedores, al igual que se ha comentado antes, esto no es problema.

\subsection{Rapidez en el despliegue}

Ligado a todo lo anterior, realizar el despliegue de un entorno de desarrollo/producción es más rápido utilizando contenedores, sin importar el sistema operativo en el que nos encontremos.

\infobox{\textbf{El despliegue con contenedores es más rápido.}}

Más adelante se verá cómo realizar el despliegue de distintos servicios haciendo uso de un único comando.


\chapter{Docker}

\href{https://www.docker.com/}{Docker} es un proyecto de Software Libre nacido en 2013 que permite realizar el despliegue de aplicaciones y servicios a través de contenedores de manera rápida y sencilla, tal como veremos más adelante.

Estos contenedores proporcionan una capa de abstracción y permiten aislar las aplicaciones del resto del sistema operativo a través del uso de ciertas características del kernel Linux.

Dentro del contenedor, se puede destacar el aislamiento a nivel:

\begin{itemize}
    \item Árbol de procesos
    \item Sistemas de ficheros montados
    \item ID de usuario
    \item Aislamiento de recursos (CPU, memoria, bloques de E/S...)
    \item Red aislada
\end{itemize}

Al igual que sucede con otro tipo de \textit{software},  para que Docker haga uso de todas estas características, está construido haciendo uso de otras aplicaciones y servicios.


\begin{center}
    \includegraphics[width=0.75\linewidth]{img/docker/docker_interfaces.png}
    \captionof{figure}{Tecnologías usadas por Docker. Fuente: \href{https://en.wikipedia.org/wiki/File:Docker-linux-interfaces.svg}{Wikipedia}}
\end{center}

En el 2015 la empresa Docker creó la \textbf{\textit{\href{https://en.wikipedia.org/wiki/Open_Container_Initiative}{Open Container Initiative}}}, proyecto actualemente bajo la Linux Foundation, con la intención de diseñar un estándar abierto para la virtualización a nivel de sistema operativo.

\section{Instalación}

Dependiendo del sistema operativo en el que nos encontremos, Docker tiene la opción de instalarse de distintas maneras. En sistemas operativos GNU/Linux cada distribución tiene un paquete para poder realizar la instalación del mismo.

\begin{mycode}{Instalación de Docker en Ubuntu}{console}{}
root@vega:~# apt install docker.io
\end{mycode}

\errorbox{El nombre del paquete en Ubuntu y Debian es \textbf{docker.io}}

En sistemas Windows y MacOS existe la opción de instalar \href{https://docs.docker.com/get-docker/}{Docker Desktop}, una versión que utiliza una máquina virtual para simplificar la instalación en estos sistemas.

\section{Configuración y primeros pasos}

Tras realizar la instalación veremos cómo el servicio Docker ha levantado un interfaz nuevo en nuestra máquina, cuya IP es \textbf{172.17.0.1/16}, siendo el direccionamiento por defecto.

\begin{mycode}{Nueva IP en el equipo}{console}{}
root@vega:~# ip a
...
3: docker0: <BROADCAST,MULTICAST,UP,LOWER_UP> mtu 1500 qdisc noqueue
    link/ether 02:42:9c:1f:e2:90 brd ff:ff:ff:ff:ff:ff
    inet 172.17.0.1/16 brd 172.17.255.255 scope global docker0
      valid_lft forever preferred_lft forever
\end{mycode}

Esta IP hará de \textbf{puente} (similar a lo que sucede con las máquinas virtuales) cuando levantemos contenedores nuevos. Los contenedores estarán dentro de ese direccionamiento 172.17.0.0/16, por lo tanto, aislados de la red principal del equipo.

\infobox{\textbf{Los contenedores que levantemos estarán en la red 172.17.0.0/16}}

El comando \commandbox{docker} tiene muchas opciones, por lo que es recomendable ejecutarlo sin parámetros. De esta manera se pueden ver todas las opciones y una ayuda simplificada para cada una de ellas.

\begin{mycode}{Algunas de las opciones del comando docker}{console}{}
root@vega:~# docker
Usage:  docker [OPTIONS] COMMAND

Management Commands:
builder     Manage builds
completion  Generate the autocompletion script for the specified shell
config      Manage Docker configs
container   Manage containers
context     Manage contexts
image       Manage images
manifest    Manage Docker image manifests and manifest lists
network     Manage networks
node        Manage Swarm nodes
plugin      Manage plugins
secret      Manage Docker secrets
service     Manage services
stack       Manage Docker stacks
swarm       Manage Swarm
system      Manage Docker
trust       Manage trust on Docker images
volume      Manage volumes

Commands:
...
\end{mycode}

Para cada una de estas opciones, se le puede añadir el parámetro \inlineconsole{--help} para mostrar la ayuda. Hay un segundo apartado que se ha cortado, en el que se incluyen más comandos.

Para asegurar que el servicio Docker está funcionando, podemos hacer uso de \commandbox{docker info}, que nos mostrará mucha información acerca del servicio. Pero si lo que queremos es comprobar si tenemos algún contenedor corriendo, es más sencillo hacer \commandbox{docker ps} (que es la versión simplificada de \commandbox{docker container ls} ):

\begin{mycode}{Comprobar estado de Docker y contenedores levantados}{console}{}
root@vega:~# docker ps
CONTAINER ID   IMAGE     COMMAND   CREATED   STATUS    PORTS     NAMES

root@vega:~# docker container ls
CONTAINER ID   IMAGE     COMMAND   CREATED   STATUS    PORTS     NAMES
\end{mycode}

En este caso, como no hay ningún contenedor levantado, sólo muestra las cabeceras de las columnas del listado.

\section{Levantando nuestro primer contenedor}
Es momento de crear nuestro primer contenedor. Para ello, dado que se está usando la consola, hay que hacer uso del comando \commandbox{docker} con una serie de parámetros. En este caso se ha optado por levantar el servicio \textbf{Apache HTTPD}:

\begin{mycode}{Levantando el primer contenedor}{console}{}
root@vega:~# docker run -p 80:80 httpd
AH00558: httpd: Could not reliably determine the server's ...
AH00558: httpd: Could not reliably determine the server's ...
[Fri Mar 24 18:25:14.194246 2023] [mpm_event:notice] ...
[Fri Mar 24 18:25:14.194347 2023] [core:notice] [pid  ...
172.17.0.1 - - [24/Mar/2023:18:25:41 +0000] "GET / HTTP/1.1" 304 -
\end{mycode}

Vemos los logs del servicio Apache al arrancar y si vamos al navegador a la dirección \href{http://localhost}{http://localhost} muestra lo siguiente:

\begin{center}
    \includegraphics[width=0.6\linewidth]{img/docker/apache.png}
\end{center}

Y para entender lo que hace el comando, los parámetros son:
\begin{itemize}
    \item \textbf{docker}: Cliente de consola para hacer uso de Docker.
    \item \textbf{run}: Ejecuta un comando en un nuevo contenedor (y si no existe lo crea).
    \item \textbf{-p 80:80}: Publica en el puerto 80 del servidor el puerto 80 utilizado en el contenedor. Se puede pensar que es como hacer un \textbf{port-forward} en un firewall.
    \item \textbf{httpd}: Es la \textbf{imagen} del contenedor que se va a arrancar. En este caso, la imagen del servidor \href{https://hub.docker.com/_/httpd}{Apache HTTPD}.
\end{itemize}

Y si vemos qué muestra el estado de docker, vemos cómo aparece el contenedor levantado.

\begin{mycode}{Comprobar estado de Docker y contenedores levantados}{console}{{\scriptsize }}
root@vega:~# docker ps
CONTAINER ID   IMAGE     COMMAND       CREATED         STATUS         PORTS               NAMES
a1c3362b0d6c   httpd     "httpd-..."   3 seconds ago   Up 2 seconds   0.0.0.0:80->80/tcp  great
\end{mycode}

En la columna PORTS se puede apreciar cómo aparece que se ha levantado el puerto  \textbf{ \texttt{0.0.0.0:80} } (escucha en el puerto 80 para cualquier IP del sistema operativo) que es una redirección al puerto \textbf{\texttt{80/TCP}} interno del contenedor.


\section{Contenedores en \textit{background} y más opciones}
Tal como se puede ver en el ejemplo anterior, el contenedor se queda en primer plano, viendo los logs del Apache. Esto para ver qué es lo que está sucendiendo durante el desarrollo puede ser útil, pero lo ideal es que el contenedor arranque en modo \textbf{\textit{background}}, y cuando necesitemos vayamos a ver los logs.


A continuación se va a arrancar un nuevo contenedor de Apache con nuevos parámetros:
\begin{mycode}{Crear un contenedor Web en el puerto 8080}{console}{}
root@vega:~# docker run --name mi-apache -d -p 8080:80 httpd
\end{mycode}

Los nuevos parámetros son:
\begin{itemize}
    \item \textbf{\texttt{--name mi-apache}}: De esta manera se le da un nombre al contenedor, para poder identificarlo de manera rápida entre todos los contenedores creados.
    \item \textbf{\texttt{-d}}: Este parámetro es para hacer el \textbf{\textit{detach}} del comando, y de esta manera mandar a \textbf{\textit{background}} la ejecución del contenedor.
    \item \textbf{\texttt{-p 8080:80}}: Publica en el puerto 8080 del servidor el puerto 80 utilizado en el contenedor. Se puede pensar que es como hacer un \textbf{port-forward} en un firewall.
\end{itemize}

\section{Parar, arrancar y borrar contenedores}

Hasta ahora hemos aprendido a crear contenedores, pero en ciertos momentos nos puede interesar parar un contenedor que no estemos utilizando, o una vez haya cumplido su función, borrarlo.

\subsection{Parar contenedores}

Para parar un contenedor, debemos conocer el nombre del mismo o su ID (que es único). Estos datos los podemos conocer a través del comando  \commandbox{docker ps}.

Con esto, podemos ejecutar:

\begin{mycode}{Parar un contenedor}{console}{{\small }}
root@vega:~# docker stop mi-apache
\end{mycode}

\subsection{Arrancar un contenedor parado}

Una vez parado un contenedor, o al reiniciar el servidor, si queremos arrancar un contenedor parado, debemos conocer también su ID o nombre.

Para visualizar todos los contenedors (tanto los arrancados como los parados), lo podemos hacer a través del comando \commandbox{docker ps -a}.

Gracias a ese listado, podemos volver a arrancar un contenedor que esté parado con \commandbox{docker start mi-apache}, siendo “mi-apache” el contenedor que queremos arrancar.

\subsection{Borrar un contenedor}
Si queremos borrar un contenedor, éste debe estar parado, ya que Docker no nos va a dejar borrar un contenedor que está en ejecución.

Es interesante borrar contenedores que hayamos creado de pruebas o contenedores que ya no se vayan a utilizar más, para de esta manera liberar recursos.

Para borrarlo, similar a los casos anteriores, se hará con \commandbox{docker rm mi-apache}.

\section{Variables de entorno}
Algunos contenedores tienen la opción de recibir variables de entorno al ser creados. Estas variables pueden afectar al comportamiento del contenedor, o para ser inicializado de alguna manera distinta a las opciones por defecto.

El creador de la imagen Docker puede crear las variables de entorno que necesite para después utilizarlas en su aplicación. A modo de ejemplo, se va a utilizar la imagen de la aplicación \href{https://hub.docker.com/_/phpmyadmin}{PHPMyAdmin}.

A continuación se van a crear 2 contenedores de PHPMyAdmin, diferenciados por el puerto, el nombre, y la variable de entorno \textbf{PMA\_ARBITRARY}:
\begin{itemize}
    \item El primer contenedor va a estar en el puerto 8081, se le va a dar el nombre “myadmin-1” y no va a tener la variable de entorno incializada.
    \item El segundo contenedor va a estar en el puerto 8082, se va a llamar “myadmin-2” y va a tener la variable \textbf{PMA\_ARBITRARY} inicializada a “1”, tal como aparece en la \href{https://hub.docker.com/_/phpmyadmin}{documentación de la imagen de PHPMyAdmin}.
\end{itemize}

Para ello, se han ejecutado los siguientes comandos:

\begin{mycode}{Creación de dos contenedores PHPMyAdmin}{console}{{\footnotesize }}
root@vega:~# docker run --name myadmin-1 -d -p 8081:80 phpmyadmin

root@vega:~# docker run --name myadmin-2 -e PMA_ARBITRARY=1 -d -p 8082:80 phpmyadmin
\end{mycode}

Tal como se puede ver, al segundo contenedor se le ha pasado un nuevo parámetro \textbf{\texttt{-e}}, que significa que lo que viene a continuación es una variable de entorno (en inglés \textbf{environment}). En este caso, la variable de entorno es \textbf{PMA\_ARBITRARY} que se ha inicializado a \textbf{1}.

Si ahora en nuestro navegador web apuntamos al puerto 8081 y al puerto 8082 de la IP de nuestro servidor, veremos cómo existe una ligera diferencia en el formulario que nos muestra la web.

En el formulario del puerto 8081 (donde no hemos inicializado la variable) sólo podemos indicar el usuario y la contraseña. Por el contrario, en el formulario del puerto 8082, al inicializar la variable \textbf{PMA\_ARBITRARY}, y tal como nos dice la documentación de la imagen, nos permite indicar la IP del servidor MySQL al que nos queremos conectar.

{
    \begin{minipage}{0.43\linewidth}
        \vspace{-11pt}
        \includegraphics[width=\linewidth]{img/docker/phpmyadmin1.png}
    \end{minipage}
    \hfill
    \begin{minipage}{0.43\linewidth}
        \vspace{-11pt}
        \includegraphics[width=\linewidth]{img/docker/phpmyadmin2.png}
    \end{minipage}

    \begin{center}
        \vspace{-18pt}
        {\footnotesize A la izquierda formulario del puerto 8081, sin variable inicializada. A la derecha, puerto 8082 con variable inicializada.}
    \end{center}
}

Dado que una variable puede afectar al comportamiento (o la creación) del servicio que levantemos a través de un contenedor, es importante leer la documentación e identificar las variables que tiene por si nos son de utilidad.

\infobox{\textbf{Es recomendable leer la documentación de las imágenes Docker para identificar las posibles variables de entorno que existen y ver si nos son útiles.}}




\hypertarget{volumen_persistente_datos}{}
\section{Volumen persistente de datos}
Hasta ahora hemos levantado un contenedor a través de una imagen que levanta el servicio Apache, mostrando su página por defecto. Podríamos escribir en el contenedor la página HTML que nos interesase, pero hay que entender que \textbf{los datos de un contenedor desaparecen cuando el contenedor se elimina}.

Para que los cambios realizados dentro de un contenedor se mantengan, tenemos que hacer uso de los denominados \textbf{volúmenes de datos}. Esto no es más que \textbf{hacer un montaje de una ruta del disco duro del sistema operativo dentro de una ruta del contenedor}.


Estos volúmenes que le asignamos al contenedor pueden ser de dos tipos:
\begin{itemize}
    \item \textbf{Sólo lectura}: Nos puede interesar asignar un volumen de sólo lectura cuando le pasamos ficheros de configuración o la propia web que queremos visualizar.
    \item \textbf{Lectura-Escritura}: En este caso se podrá escribir en el volumen. Por ejemplo, el directorio donde una base de datos guarda la información o una web donde deja imágenes subidas por usuarios.
\end{itemize}

De esta manera, tendremos que asignar el número de volúmenes necesarios a cada contenedor dependiendo de la imagen utilizada, el servicio que se levanta y lo que queremos hacer con los datos que le asignemos o generemos en el contenedor.

En la siguiente imagen se puede ver una infraestructura con dos contenedores y dos volúmenes:
\begin{itemize}
    \item \textbf{Contenedor Web}: Se le asigna un volumen en modo \textbf{sólo lectura} cuya ruta original está en \configdir{/opt/www-data}, que dentro del contenedor está en \configdir{/var/www/html}.
    \item \textbf{Contenedor MySQL}: Dado que los datos de la base de datos deben ser guardados, en este caso se le asigna un volumen que permite escritura. Por lo tanto, lo que se crea dentro del contenedor en \configdir{/var/lib/mysql} realmente se estará guardando en el sistema operativo anfitrión en \configdir{/opt/mysql-data}
\end{itemize}
\begin{center}
    \includegraphics[width=0.65\linewidth]{img/docker/volumes.png}
    \captionof{figure}{Ejemplo de dos volúmenes asignados a distintos contenedores}
\end{center}


\subsection{Añadir volumen de escritura al crear un contenedor}

Al añadir un volumen cuando creamos un contenedor hace que por defecto sea en modo lectura-escritura. Cualquier escritura realizada dentro del contenedor en la ruta especificada va a resultar en que el fichero se creará en la ruta indicada del sistema operativo anfitrión.

\begin{mycode}{Añadir volumen sólo lectura al crear un contenedor}{console}{{\small }}
root@vega:~# ls /opt/mysql-data
root@vega:~# docker run -d -p 3306:3306 --name mi-db \
    -v /opt/mysql-data:/var/lib/mysql \
    -e MYSQL_ROOT_PASSWORD=my-secret-pw \
    mysql:latest
root@vega:~# ls /opt/mysql-data
auto.cnf      client-key.pem      '#innodb_temp'      server-cert.pem   ...
\end{mycode}

Para este ejemplo se ha creado un contenedor usando la imagen de \href{https://hub.docker.com/_/mysql}{MySQL}, al que se le ha asignado un volumen (\textbf{por defecto se asigna permitiendo la escritura}) y un parámetro necesario para realizar la posterior conexión con contraseña.

\begin{itemize}
    \item \textbf{-v /opt/mysql-data:/var/lib/mysql}: A través del parámetro \textbf{-v} se le indica al contenedor que se le va a pasar un volumen. Posteriormente se le indica la ruta del sistema operativo anfitrión \configdir{/opt/mysql-data} que se montará dentro del contenedor en \configdir{/var/lib/mysql}.
    \item \textbf{-e MYSQL\_ROOT\_PASSWORD=my-secret-pw}: El parámetro “\textbf{-e}” sirve para pasarle al contenedor \textbf{variables de entorno}. En este caso, y tal como dice la \href{https://hub.docker.com/_/mysql}{web de la imagen MySQL}, esta es la manera de asignar la contraseña del usuario \textbf{root} durante la inicialización de la base de datos.
\end{itemize}

Tras crear el contenedor, y asegurarnos que está levantado haciendo uso del comando \commandbox{docker ps}, podemos realizar la conexión desde el sistema operativo anfitrión o desde cualquier otro lugar usando la contraseña indicada previamente.

\begin{mycode}{Añadir volumen sólo lectura al crear un contenedor}{console}{{\small }}
root@vega:~# mysql -h127.0.0.1 -uroot -P3306 -p
    Enter password:
    ...
    MySQL [(none)]> show databases;
    +--------------------+
    | Database           |
    +--------------------+
    | information_schema |
    | mysql              |
    | performance_schema |
    | sys                |
    +--------------------+
    4 rows in set (0,007 sec)
\end{mycode}

\subsection{Añadir volumen en modo sólo-lectura}
A continuación se van a explicar los pasos para levantar un contenedor que contiene una web simple creada en PHP, que está alojada en la ruta \configdir{/opt/www-data} del sistema operativo anfitrión.


\begin{mycode}{Añadir volumen sólo lectura al crear un contenedor}{console}{}
root@vega:~# ls /opt/www-data
index.php
root@vega:~# docker run -d -p 80:80 --name mi-web \
    -v /opt/www-data:/var/www/html:ro \
    php:8.2.4-apache
\end{mycode}

El parámetro nuevo asignado en la creación de este contenedor es:
\begin{itemize}
    \item \textbf{-v /opt/www-data:/var/www/html:ro}: Para indicarle que le vamos a asignar un volumen siendo la ruta real en el sistema de ficheros del sistema operativo anfitrión \configdir{/opt/www-data} y la ruta destino \textbf{dentro del contenedor} y que va a ser en modo \textbf{read-only} \configdir{/var/www/html}.
\end{itemize}

Más adelante, cuando veamos \hyperlink{entrar_en_contenedor}{cómo entrar dentro de un contenedor Docker}, se podría usar el comando para ir a la ruta dentro del contenedor y comprobar que efectivamente está en modo sólo-lectura.

\hypertarget{entrar_en_contenedor}{}
\section{Entrar dentro de un contenedor Docker}
Normalmente no suele ser necesario entrar dentro de un contenedor, ya que, tal como se ha dicho antes, cualquier modificación realizada dentro de él se perderá (salvo que sea dentro de un volumen persistente).

Aún así, para realizar pruebas o comprobaciones del correcto funcionamiento de una imagen puede ser interesante entrar dentro de un contenedor. Para ello, el comando a ejecutar es el siguiente:

\begin{mycode}{Acceder a un contenedor}{console}{}
root@vega:~# docker exec -it mi-db /bin/bash
\end{mycode}

Los parámetros utilizados son:
\begin{itemize}
    \item \textbf{exec}: Indicamos que queremos ejecutar un comando dentro de un contenedor que está corriendo.
    \item \textbf{-it}: Son dos parámetros unidos, que sirven para mantener la entrada abierta (modo interactivo) y crear una TTY (consola)
    \item \textbf{mi-db}: Es el nombre del contenedor al que se quiere entrar. También se puede indicar el \textbf{ID} del contenedor.
    \item \textbf{/bin/bash}: el comando que queremos ejecutar. En este caso, una shell \textbf{bash}. En algunos casos esta shell no está instalada y debemos usar \textbf{/bin/sh}
\end{itemize}

Hay que tener en cuenta que dentro de un contenedor está el mínimo software posible para que la aplicación/servicio funcione, por lo que habrá muchos comandos que no existan.


\section{Ciclo de vida de un contenedor Docker}
Un contenedor tiene un ciclo de vida que puede pasar por distintos estados. Para pasar entre estados se debe realizar a través de distintos comandos de Docker.

\begin{center}
    \includegraphics[width=\linewidth]{img/docker/lifecycle.png}
    \captionof{figure}{Estados de un contenedor}
\end{center}

En la imagen se representa los estados más básicos junto con los comandos para pasar entre ellos.


\section{Otros comandos útiles}

Para obtener toda la información de un contenedor, incluido su estado, volúmenes utilizados, puertos, ...

\begin{mycode}{Obtener toda la información de un contenedor}{console}{}
root@vega:~# docker inspect mi-db
\end{mycode}

Listar las imágenes descargadas en local. Al tener las imágenes en local, no hará falta volver a descargarlas, por lo que crear un nuevo contenedor que haga uso de una de ellas será mucho más rápido.

\begin{mycode}{Listado de imágenes en local}{console}{}
root@vega:~# docker image ls
REPOSITORY       TAG            IMAGE ID       CREATED        SIZE
php              8.2.4-apache   de23bf333100   3 days ago     460MB
httpd            latest         192d41583429   3 days ago     145MB
mysql            latest         483a8bc460a9   3 days ago     530MB
\end{mycode}

Borrar una de las imágenes que no se esté utilizando en ningún contenedor.

\begin{mycode}{Borrar una imagen concreta}{console}{}
root@vega:~# docker image rm httpd
\end{mycode}


Cuando un contenedor está en modo \textit{\textbf{detached}} no aparecen los logs, por lo que para poder visualizarlos tenemos un comando especial para ello.
\begin{mycode}{Ver los logs de un contenedor}{console}{}
root@vega:~# docker logs mi-web -f
172.17.0.1 - - [26/Mar/2023:18:05:29 +0000] "GET / HTTP/1.1" 200 248
172.17.0.1 - - [26/Mar/2023:18:05:29 +0000] "GET / HTTP/1.1" 200 248
172.17.0.1 - - [26/Mar/2023:18:05:29 +0000] "GET / HTTP/1.1" 200 248
\end{mycode}


Listar los volúmenes existentes en el sistema. El listado muestra los que se están utilizando en contenedores (activos o parados) o los que se han utilizado en contenedores que ya no existen.

\begin{mycode}{Listar volúmenes}{console}{{\small}}
root@vega:~# docker volume ls
DRIVER    VOLUME NAME
local     0d6c400a6407f5cdea81a2f0158222fdd87d7f3b3e2b5969ca466d743fc71f5c
local     1d2f52018e17af0689e070a55337154c1dd68517c54435ecc24d597f7509d43c
local     6b72797227ef4708ca23ee1dfcb4b651b42eeacefd4166b898407ad4aadda10c
\end{mycode}


Si queremos realizar una limpieza de todos los recursos (contenedores, imágenes,  volúmenes) que no se estén utilizando, se puede utilizar el siguiente comando.
\begin{mycode}{Borrar recursos que no estén activos}{console}{}
root@vega:~# docker system prune -a
WARNING! This will remove:
- all stopped containers
- all networks not used by at least one container
- all images without at least one container associated to them
- all build cache

Are you sure you want to continue? [y/N] y
\end{mycode}

\errorbox{\textbf{El comando anterior hace que se borren contenedores parados}}


Para conocer las estadísticas de uso de cada contenedor.
\begin{mycode}{Ver las estadísticas de los contenedores}{console}{{\scriptsize }}
root@vega:~# docker stats
CONTAINER ID  NAME    CPU %   MEM USAGE / LIMIT     MEM %   NET I/O          BLOCK I/O   PIDS
2fcf97530766  mi-web  0.00%   54.62MiB / 15.47GiB   0.34%   9.44MB / 172kB   0B / 0B     6
413d6e9f590f  mi-db   0.55%   357.9MiB / 15.47GiB   2.26%   319kB / 280B     0B / 0B     38
\end{mycode}




    \part{Instalar Ubuntu  Server 22.04}
    \graphicspath{{../../anexos/instalar_ubuntu_lts/img/}}
    \hypertarget{instalar_ubuntu_lts}{}

\chapter{Instalar Ubuntu 20.04 LTS}
En este anexo realizaremos la instalación de la distribución Ubuntu 20.04 LTS en su versión para servidores. En este anexo no se va a explicar cómo realizar la creación de una máquina virtual donde se aloja el sistema operativo, ya que existen distintos tipos de virtualizadores.

No se realizará una guía “paso a paso”, sino que se centrará en las partes más importantes de la instalación y en las que más dudas puedan surgir.

\section{Descargar Ubuntu 20.04}
La ISO la obtendremos de la \href{https://ubuntu.com/#download}{web oficial} y seleccionaremos la versión 20.04 LTS de Ubuntu Server. Esta ISO contendrá el sistema base de Ubuntu y nos guiará para realizar la instalación del sistema operativo.

Una vez descargada la ISO tendremos que cargarla en el sistema de virtualización elegido y arrancar la máquina virtual.


\section{Instalar Ubuntu 20.04}
Tras arrancar la máquina virtual nos aparecerá un menú para seleccionar el idioma durante la instalación y le daremos a “Instalar Ubuntu Server”.

\begin{center}
    \vspace{-10pt}
    \includegraphics[width=15cm]{ubuntu_1.png}
    \vspace{-20pt}
\end{center}

A partir de aquí comenzará el instalador y los pasos que nos aparecerán serán los siguientes (algunos de estos pasos puede que no estén 100\% traducidos al castellano):

\begin{enumerate}
    \item Elegir el idioma del sistema
    \item Actualización del instalador:
    \begin{itemize}
        \item Si la máquina virtual se puede conectar a internet, comprobará si existe una actualización del propio instalador de Ubuntu.
        \item Podemos darle a “Continuar sin actualizar”
    \end{itemize}
    \item Configuración del idioma del teclado
    \item Configuración de la red
    \item Configuración del proxy de red
    \item Configuración del “mirror” o servidor espejo desde donde descargarse los \hyperlink{paquete_de_software}{paquetes de software} para las actualizaciones posteriores.
    \item Selección del disco duro donde realizar la instalación
    \item Elegir el particionado de disco.
    \item Configuración del perfil. Introduciremos el nombre de usuario, el nombre del servidor y la contraseña del usuario que vamos a crear.
    \item Configuración de SSH Server. Aceptaremos que se instale el servidor SSH durante la instalación. En caso de no seleccionar esta opción, posteriormente podremos realizar la instalación.
    \item “Featured Server Snaps”. En esta pantalla nos permite instalar software muy popular en servidores.
\end{enumerate}


Una vez le demos a continuar, comenzará la instalación en el disco duro. Debido a que durante la instalación tenemos conexión a internet, el propio instalador se descarga las últimas versiones de los paquetes de software desde los repositorios oficiales.


Al terminar la instalación, tendremos que reiniciar la máquina virtual.

\section{Post-instalación}
Tras realizar el reinicio de la máquina virtual nos encontraremos con que el sistema arranca en el sistema recién instalado, y que tendremos que loguearnos introduciendo el usuario y la contraseña utilizadas en la instalación.

\subsection{Actualización del sistema}
Por si acaso, realizaremos la actualización del índice del repositorio, actualizaremos el sistema y en caso necesario realizaremos un nuevo reinicio:

\begin{mycode}{Actualizar Ubuntu}{console}{}
mikeldi@ubuntu:~$ sudo su
[sudo] password for mikeldi:
root@ubuntu:~# apt update
...
root@ubuntu:~# apt upgrade
...
\end{mycode}

Con estos comandos nos aseguramos que el sistema está actualizado a los últimos paquetes que están en el repositorio.


\hypertarget{configurar_ip_estatica_ubuntu}{}
\subsection{Poner IP estática}
Debido a la configuración de red de nuestro servidor, la IP está puesta en modo dinámica, esto quiere decir que nuestro equipo ha cogido la IP por configuración de DHCP de nuestra red. Debido a que un servidor debe de tener IP estática, tenemos que realizar la modificación adecuada para ponerle la IP estática que mejor nos convenga. Para ello editaremos el fichero de configuración situado en la siguiente ruta: \configfile{ /etc/netplan/00-installer-config.yaml }

Lo modificaremos para que sea parecido a (siempre teniendo en cuenta la IP y gateway de nuestra red):


\begin{mycode}{Configurando IP estática en Ubuntu}{yaml}{}
network:
  ethernets:
    enp1s0:
      dhcp4: no
      addresses:
      - 192.168.200.10/24
      gateway4: 192.168.200.1
      nameservers:
        addresses: [8.8.8.8]
  version: 2
\end{mycode}

El fichero de configuración que hemos modificado es de tipo \href{https://es.wikipedia.org/wiki/YAML}{YAML}, que es un formato de texto que suele ser utilizado en programación o en ficheros de configuración. Este tipo de ficheros tiene en cuenta los espacios para el uso de la identación, y no suele permitir el uso de tabuladores.

Para aplicar los cambios realizados en el fichero de configuración deberemos ejecutar el siguiente comando que aplicará los cambios:

\begin{mycode}{Aplicar configuración de IP}{console}{}
root@ubuntu:~# netplan apply
\end{mycode}

\clearpage

    \graphicspath{{../../temas_comunes/administracion_remota/img/}}
    \chapter{Administración remota}

En informática no siempre tenemos los equipos que administramos en nuestra oficina. Pueden estar en otro edificio, la oficina de un cliente, en internet … por lo tanto no siempre es posible acceder de manera física a ellos, y por tanto entra en juego la \textbf{administración remota}.

Podemos definir la administración remota como el sistema que nos permite realizar ciertas acciones “lanzadas” desde nuestro equipo local pero que serán ejecutadas en un equipo remoto.

Se pueden diferenciar varios tipos de sistemas dentro de la administración remota, pero nos vamos a centrar en los siguientes:

\begin{itemize}
    \item \textbf{Cliente remoto}: Lanzamos una acción a ejecutar desde un equipo remoto a través de algún tipo de interfaz o comando (que viajará a través de un protocolo securizado) y esperaremos a la respuesta.

    \item \textbf{Acceso remoto}: En este caso lo que hacemos es conectarnos al equipo a través de un protocolo que nos va a permitir administrarlo como si estuviésemos delante de él.
\end{itemize}

Todos estos sistemas pueden ser complementarios, y puede que podamos administrar un mismo servicio de todas estas maneras, por lo que queda a nuestra disposición elegir el mejor método en cada momento.

Por otro lado, dependiendo de qué tipo de administración vayamos a llevar a cabo, o el protocolo que utilice, tendremos que tener acceso al servidor de alguna manera (ya sea conexión directa o mediante VPN).

\infobox{\textbf{Dependiendo de la administración remota que realicemos, necesitaremos conexión directa o mediante VPN al equipo que nos queremos conectar.}}

Por último, también debemos de conocer el tipo de protocolo que vamos a utilizar al realizar la conexión remota y por dónde va a pasar esa comunicación. Siempre hay que premiar la seguridad de la comunicación, y más cuando esta puede pasar por redes no controladas. Por lo tanto, deberemos asegurar que el protocolo utilizado es seguro, o en caso contrario, securizarlo de alguna manera.

\errorbox{\textbf{Siempre debemos confirmar que la comunicación que se realiza para la administración remota viaja cifrada.}}

Más adelante veremos cómo securizar una comunicación no segura realizando un túnel mediante el protocolo SSH en entornos GNU/Linux.


\section{Cliente remoto}

Este sistema de administración permite enviar acciones al equipo remoto a través de un protocolo establecido, y  dependiendo de la acción ejecutada se esperará una respuesta o no.

\begin{center}
    \includegraphics[width=0.7\linewidth]{ejecucion_remota.png}
\end{center}

Hoy en día es muy habitual este tipo de sistemas a través de disintos \textbf{CLI} (\textit{client line interface})  o \textbf{GUI} (\textit{graphic user interface}) que nos permiten administrar servicios remotos. Por ejemplo:

\begin{itemize}
    \item \textbf{{https://www.mysql.com/}{MySQL}}: El sistema gestor de base de datos \href{https://www.mysql.com/}{MySQL} cuenta con un cliente para realizar la  conexión, ya sea desde el propio equipo o desde un equipo remoto.

    Este cliente se puede ejecutar desde línea de comandos, aunque también viene integrado en distintos interfaces gráficos como \href{https://dev.mysql.com/downloads/benchmarks.html}{MySQL Benchmark}, \href{https://dbeaver.io/}{Dbeaver}, ...

    \item \textbf{\href{https://aws.amazon.com/es/cli/}{AWS CLI}}: Es el interfaz de línea de comandos para poder administrar de manera remota los servicios contratados en la nube AWS de Amazon.

    \item \textbf{\href{https://cloud.google.com/cli}{Gcloud CLI}}: Similar al caso anterior pero esta vez para Google Cloud.

    \item \textbf{\href{https://www.microsoft.com/en-us/download/details.aspx?id=45520}{Remote Server Administration Tools for Windows 10}}: En este caso se trata de un interfaz gráfico (\textbf{GUI}) que nos permite administrar un Windows Server desde un equipo Windows 10.
\end{itemize}


Antes de poder realizar la conexión remota con el cliente \textbf{debemos configurar un sistema de autenticación} para que el servicio remoto acepte las peticiones enviadas. En algunos casos será usando unos sistemas de certificados y en otros introduciendo un usuario y contraseña que establecerá una sesión temporal.

En el caso de AWScli y GCloud no nos estamos conectando directamente a nuestros servidores alojados en esas nubes, si no que lanzamos la orden a un “proxy” que verificará nuestros credenciales, verá los permisos que tenemos y después realizará la acción solicitada.


%TODO: mirar algo de esto
%COSAS A MIRAR:
%
%
%ADmin center
%
%https://learn.microsoft.com/en-us/windows-server/manage/windows-admin-center/plan/installation-options
%
%sconfig en windows    https://learn.microsoft.com/en-us/windows-server/administration/server-core/server-core-sconfig
%
%
%diferencias entre windows server core and escritorio https://learn.microsoft.com/en-us/windows-server/get-started/install-options-server-core-desktop-experience


\section{Acceso remoto}
Este sistema permite acceder al sistema y podremos administrarlo como si nos encontrásemos delante. Dependiendo del sistema la conexión nos permitirá interactuar de alguna de las siguientes maneras:

\begin{itemize}
    \item \textbf{CLI}: Mediante una conexión de línea de comandos. Es el caso más habitual en servidores GNU/Linux y la conexión se hace a través del protocolo seguro \textbf{SSH}.

    \item \textbf{GUI}: Podremos obtener un interfaz gráfico con el que veremos lo que está ocurriendo en pantalla en ese momento. En este caso, dependiendo del sistema, existirán distintas opciones, pero vamos a nombrar dos de ellas:

    \begin{itemize}
        \item \textbf{RDP}: Es el protocolo de escritorio remoto de Microsoft que transmite la información gráfica que el usuario debería ver por la pantalla, la transforma en el formato propio del protocolo, y la envía al cliente conectado. El problema es que este sistema desconecta al usuario que está logueado para poder hacer uso del escritorio remoto.

        \item \textbf{VNC}: En inglés \textit{Virtual Network Computing}, es un servicio con estructura \textbf{cliente-servidor} que permite visualizar el escritorio del servidor desde un programa cliente. En este caso, no existe desconexión del usuario que está logueado y por tanto podrá ver lo que le están realizando de manera remota.

        Es muy habitual que los equipos de usuarios ya tengan la instalación del servidor hecha, para que de esta manera, en caso de incidencia, poder realizar la conexión remota sin que el usuario tenga que realizar ninguna acción.
    \end{itemize}
\end{itemize}

A continuación se va a detallar algunos de los métodos mencionados.



\section{SSH}
SSH es un protocolo de comunicación segura mediante cifrado cuya función principal es el acceso remoto a un servidor. La arquitectura que utiliza es la de cliente-servidor.

Aunque el uso más habitual de SSH es el acceso remoto, también se puede utilizar para:
\begin{itemize}
    \item Securizar protocolos no seguros mediante la realización de túneles.
    \item Acceder a un equipo saltando a través de otro.
\end{itemize}

Estas funcionalidades las veremos más adelante.


\subsection{Servidor SSH}
En el servidor al que nos queramos conectar deberá estar instalado el demonio/servicio SSH, conocido como \textbf{sshd}. Es habitual que ya esté instalado en sistemas GNU/Linux, pero de no ser así deberemos usar el sistema de paquetes de nuestra distribución para hacer la instalación. El nombre suele ser \textbf{openssh-server}.

Este servicio por defecto se pondrá a la escucha en el puerto 22/TCP:

\begin{mycode}{SSHd escuchando en puerto 22}{console}{{\scriptsize }}
ruben@vega:~$ sudo ss -pntaln
State   Recv-Q   Send-Q   Local Address:Port   Peer Address:Port   Process
LISTEN  0        128      0.0.0.0:22           0.0.0.0:*           users:(("sshd",pid=1122,fd=3))
LISTEN  0        128      [::]:22              [::]:*              users:(("sshd",pid=1122,fd=4))
\end{mycode}

La configuración del servicio se realiza a través de un fichero de configuración que está situado en la ruta \configfile{/etc/ssh/sshd_config}. Las distribuciones de GNU/Linux ya traen una configuración predeterminada que suele constar de las siguientes directivas (aunque hay muchas más):

\begin{itemize}
    \item \textbf{Port}: Normalmente viene comentada, ya que el puerto por defecto es el 22. En caso de querer cambiar el puerto, podremos modificar esta línea, asegurando que no esté comentada.

    \item \textbf{ListenAddress}: Por defecto SSH se pondrá a la escucha en todos los interfaces que tengamos configurados. Si sólo nos interesa escuchar en alguna de las IPs que tengamos configuradas, deberemos modificar esta configuración.

    \item \textbf{PermitRootLogin}: Para evitar problemas de seguridad, esta directiva suele estar configurada a “\textbf{No}”, para evitar que se puedan usar los credenciales de root para hacer el login.

    \item \textbf{PubkeyAuthentication}: Esta directiva permite realizar la conexión a través de unas claves públicas/privadas que podemos crear. Se explicará más adelante.
\end{itemize}

Hoy día también se puede instalar en Windows 10 y posteriores a través de un comando, siendo administrador de PowerShell:

\begin{mycode}{Instalando OpenSSH Server en Windows 10}{powershell}{{\footnotesize }}
PS C:\Windows\System32> Add-WindowsCapability -Online -Name OpenSSH.Server~~~~0.0.1.0

PS C:\Windows\System32> Start-Service sshd

PS C:\Windows\System32> Set-Service -Name sshd -StartupType 'Automatic'
\end{mycode}

O desde el interfaz gráfico a través de las “\textbf{Características opcionales}”, buscando por ssh
\begin{center}
    \includegraphics[frame,width=0.6\linewidth]{ssh_server_windows.png}
\end{center}


\subsection{Cliente SSH}
El cliente SSH es aquel programa que a través del protocolo SSH se puede conectar a un servidor SSH. Existen distintos tipos de clientes que podemos utilizar:

\begin{itemize}
    \item \textbf{CLI}: El cliente de consola es el más habitual. Está instalado de forma habitual en todas las distribuciones de GNU/Linux (normalmente el paquete se llama \textbf{openssh-client}). También lo encontramos instalado por defecto en MacOS.

    Hoy día también está instalado en Windows 10, y de no estar, se puede instalar a través de las “\textbf{Características Opcionales}”.

    \item \textbf{GUI}: Existen distintos interfaces gráficos que nos puede interesar utilizar:
    \begin{itemize}
        \item \textbf{\href{https://putty.org/}{Putty}}: Un cliente muy habitual en entornos Windows.
        \item \textbf{\href{https://github.com/cyd01/KiTTY/}{Kitty}}: Una versión mejorada del anterior.
        \item \textbf{\href{https://www.termius.com/}{Termius}}: Cuenta con versión de escritorio y móvil.
    \end{itemize}
\end{itemize}


Para realizar la conexión al servidor SSH debemos conocer:

\begin{itemize}
    \item \textbf{Dirección del servidor}: Ya sea mediante IP o nombre FQDN (\textit{fully qualified domain name}) que se resuelva.
    \item \textbf{Puerto}: Ya hemos comentado que por defecto el puerto es 22.
    \item \textbf{Usuario}: Para realizar el sistema de autenticación, necesitamos un usuario que exista en el sistema.
    \item \textbf{Contraseña}: Los credenciales de acceso del usuario.
\end{itemize}

Para realizar la conexión desde un cliente de consola ejecutaremos:

\begin{mycode}{Conexión SSH }{console}{ }
ruben@vega:~$  ssh usuario@192.168.1.200 -p 22
\end{mycode}

En el comando anterior podemos identificar:
\begin{itemize}
    \item \textbf{ssh}: el cliente de consola
    \item \textbf{usuario}: el nombre del usuario con el que nos queremos conectar al servidor remoto.
    \item \textbf{@}: la arroba en inglés significa “at”, que indica “usuario en el servidor X”.
    \item \textbf{192.168.1.200}: La IP del servidor al que nos queremos conectar.
    \item \textbf{-p 22}: Estos dos parámetros van juntos,  “-p” indica que vamos a indicar el puerto de conexión y “22” que nos queremos conectar a ese puerto. Debido a que 22 es el puerto por defecto, podríamos no poner estas opciones si sabemos que el servidor escucha en el puerto 22.
\end{itemize}

Si realizamos la conexión a través de un cliente de interfaz, como es putty, el aspecto será el siguiente, donde sólo podremos introducir la IP del servidor. Cuando se comience con la conexión nos pedirá los credenciales de acceso.

\begin{center}
    \includegraphics[width=0.5\linewidth]{putty.png}
\end{center}


Si es la primera vez que nos conectamos a un servidor mediante SSH nos saldrá un mensaje como el siguiente:

\begin{mycode}{Conexión SSH }{console}{{\small }}
ruben@vega:~$  ssh usuario@192.168.1.200 -p 22

The authenticity of host '192.168.1.200 (192.168.1.200)' can't be established.
ECDSA key fingerprint is SHA256:uK9MOl0gLDhTtCrlcafc1zVObVA/vnOMn6TWFsQb23o.
Are you sure you want to continue connecting (yes/no/[fingerprint])?
\end{mycode}

Este “key fingerprint” es un identificador que está relacionado con el fichero de “\textbf{clave pública}” del servidor. Es como el DNI del servidor. La primera vez se nos guarda ese \textit{fingerprint}, y en caso de que en una próxima conexión varíe, nos avisará. No suele ser habitual que este identificador cambie.


\hypertarget{ssh_clave_publica_privada}{}
\subsection{Conexión mediante certificados de clave pública/clave privada}

Existe una alternativa a la hora de realizar una conexión SSH para que no nos pida la contraseña del usuario, y es \textbf{hacer uso de los certificados de clave pública y clave privada}. Este concepto de “clave pública y clave privada” viene de la \href{https://es.wikipedia.org/wiki/Criptograf%C3%ADa_asim%C3%A9trica}{\textbf{criptografía asimétrica}}.

Este sistema de criptografía asimétrica hace uso de dos claves que están asociadas entre sí:
\begin{itemize}
    \item \textbf{Clave privada}: Es la base del sistema criptográfico, y como su nombre indica, se debe de mantener en privado. \textbf{Nunca se debe de compartir}, ya que entonces se podrían hacer pasar por nosotros.
    \item \textbf{Clave pública}: Asociada a la clave privada, la clave pública puede ser compartida y enviada a otros ordenadores para poder realizar la conexión.
\end{itemize}

Para generar el par de claves se realiza con el siguiente comando:

\begin{mycode}{Crear par de claves pública/privada}{console}{}
ruben@vega:~$  ssh-keygen
Generating public/private rsa key pair.
Enter file in which to save the key (/home/ruben/.ssh/id_rsa):

Enter passphrase (empty for no passphrase):
Enter same passphrase again:

Your identification has been saved in /home/ruben/.ssh/id_rsa
Your public key has been saved in /home/ruben/.ssh/id_rsa.pub
The key fingerprint is:
SHA256:SPqPOYBmPb8PCFhcZgqcWZPZzaL5RNfMeKmHqebvC7E ruben@vega
The key's randomart image is:
+---[RSA 3072]----+
|o +oB o = .      |
| * B.+ = *       |
|  + + + =        |
| o o + = .       |
|. .o+.o S        |
|  +.+*o          |
| o  +Eo          |
|     .+=         |
|      *B+        |
+----[SHA256]-----+
\end{mycode}

El comando muestra los siguientes pasos:
\begin{enumerate}
    \item Creación de la pareja de claves pública/privada haciendo uso del sistema criptográfico \href{https://en.wikipedia.org/wiki/RSA_(cryptosystem)}{\textbf{RSA}}.
    \item Lugar donde se va a guardar la clave privada. Por defecto en \configdir{~/.ssh/id_rsa}.
    \item Contraseña para securizar la clave privada. De esta manera, para poder usarla habrá que introducir dicha contraseña. Dado que nosotros queremos evitar introducir contraseñas, lo dejaremos en blanco.
    \item Lugar donde se va a guardar la clave pública. Por defecto en \configdir{~/.ssh/id_rsa.pub}
\end{enumerate}

Una vez tenemos nuestro par de claves, podemos copiar la clave pública al usuario del servidor que nos interese mediante el siguiente comando:

\begin{mycode}{Crear par de claves pública/privada}{console}{}
ruben@vega:~$  ssh-copy-id user@servidor_remoto
\end{mycode}

Para ello es imprescindible conocer previamente la contraseña del usuario en el servidor. El comando \commandbox{ssh-copy-id} realizará una conexión SSH y copiará el contenido de la \textbf{clave pública}, \configdir{~/.ssh/id_rsa.pub}, dentro del fichero \configfile{~/.ssh/authorized_keys} del usuario en el servidor remoto. Este paso se puede realizar a mano (con un editor de texto).

\errorbox{\textbf{Windows no tiene el comando “ssh-copy-id”, por lo que deberemos hacer el paso a mano, tal como se ha explicado.}}

Al realizar la siguiente conexión, ya no necesitaremos introducir la contraseña del usuario, ya que el sistema remoto podrá autenticarnos a través del sistema clave pública/privada.


\subsection{Crear túneles SSH}
Una de las funcionalidades extra de SSH es la posibilidad de crear “túneles” para securizar protocolos no seguros, o poder acceder a servicios que sólo escuchan en \textit{localhost}.

Pongamos como ejemplo el siguiente escenario:

\vspace{-15pt}
\begin{center}
    \includegraphics[width=0.5\linewidth]{tunel1.png}
\end{center}
\vspace{-15pt}

Tenemos un servidor web con Apache y MySQL al que queremos acceder. Por seguridad MySQL \textbf{sólo está escuchando en localhost (127.0.0.1)}, por lo que el acceso al servicio MySQL no es posible. Para administrarlo nos tenemos que conectar al servidor, y realizarlo de manera local.

En este punto es donde entra en juego la creación de un túnel seguro al servidor, para poder acceder al servicio remoto. \textbf{Para ello es imprescindible poder realizar una conexión SSH} (ya sea mediante usuario o clave pública/privada).

Para crear un túnel, desde el equipo de escritorio, lanzaremos el siguiente comando:

\begin{mycode}{Crear par de claves pública/privada}{console}{}
ruben@vega:~$  ssh usuario@192.168.1.200 -L 6306:127.0.0.1:3306 -N
\end{mycode}

Al ejecutar este comando, habremos creado un túnel que enlaza el puerto remoto 3306 (que sólo se escucha en el “127.0.0.1” del servidor), con el puerto local 6306 del equipo de escritorio. A continuación la explicación del comando y sus parámetros:


\begin{itemize}
    \item “\textbf{ssh usuario@192.168.1.200}”: es como una conexión SSH normal. Lo que estamos indicando es que queremos conectarnos con  “usuario” al servidor remoto 192.168.1.200 a través de SSH.
    \item \textbf{-L 6306:127.0.0.1:3306}: Especifica que el puerto local especificado se va redirigir al puerto e IP remota. Para entender esto hay que separar dos partes de los parámetros:
    \begin{itemize}
        \item \textbf{6306}: Especifica la IP y el puerto local. En este caso, antes del puerto no hemos especificado IP, por lo que se creará un puerto 6306 que sólo se pone a la escucha en \textbf{\textit{localhost}} en el equipo de escritorio
        \item \textbf{127.0.0.1:3306}: Esta es la dirección y puerto remoto al que nos queremos conectar. En este caso, es el puerto 3306 que está escuchando en la IP 127.0.0.1 del servidor.
    \end{itemize}
    \item \textbf{-N}: Sirve para que no ejecute ningún comando en el servidor remoto, y por tanto no nos abrirá conexión de terminal.
\end{itemize}

A nivel visual, y para entender de mejor manera lo realizado, sirva la siguiente imagen y los pasos que se pueden dar en un escenario real:

\vspace{-10pt}
\begin{center}
    \includegraphics[width=\linewidth]{tunel2.png}
\end{center}
\vspace{-10pt}

Con una aplicación en el equipo de escritorio queremos conectarnos al servidor MySQL que sólo escucha en el servidor remoto. Ejecutamos el túnel visto anteriormente, y los pasos que podremos realizar son los siguientes:

\begin{enumerate}
    \item[0.] Mediante una aplicación nos podemos conectar al puerto local 6306, que ha sido creado mediante el comando anterior. Este puerto local está redirigido al puerto del servidor remoto. Por lo tanto la conexión  se securiza a través del túnel
    \item Como el túnel está establecido, la conexión viaja de manera segura a través de él.
    \item Al llegar al servidor remoto, sabe que la conexión debe ir al puerto 3306 de la IP 127.0.0.1, que es lo establecido en el comando.
    \item La conexión vuelve al túnel.
    \item Viaja por el túnel hasta llegar a la comunicación que se había establecido previamente.
\end{enumerate}

De esta manera, hemos podido realizar una conexión a un servicio remoto a través de SSH y completamente seguro.


\subsubsection{Acceder a un equipo saltando a través de otro.}
Este es un caso especial de túnel, similar a lo explicado previamente. En lugar de querer realizar una conexión a un servicio del equipo al que nos conectamos, en este caso lo usaremos de salto para acceder a otro servidor.

\vspace{-10pt}
\begin{center}
    \includegraphics[width=\linewidth]{tunel3.png}
\end{center}
\vspace{-10pt}

En este caso estando en casa nos queremos conectar a un equipo de la oficina. No tenemos VPN, y no hay redirección de puertos directa, pero podemos acceder al firewall perimetral. Por lo tanto, lo podemos utilizar para saltar al servidor que nos interesa.

En este caso, el comando a ejecutar sería:

\begin{mycode}{Crear par de claves pública/privada}{console}{{\small}}
ruben@vega:~$  ssh usuario_firewall@84.85.86.87 -L 2222:172.17.1.200:22 -N
\end{mycode}

De esta manera, ahora desde nuestro equipo podremos realizar una conexión SSH al puerto 2222 que realmente será una redirección que viaja a través del túnel hasta el firewall, y que a su vez redirige al puerto 22 del servidor 172.17.1.200.

\warnbox{\textbf{El servicio remoto al que nos queremos conectar debe escuchar en la IP del equipo al que nos queremos conectar. El equipo que usamos para saltar debe poder conectarse a él.}}


%\section{VNC}

% TODO: comentar VNC




\clearpage

\end{document}