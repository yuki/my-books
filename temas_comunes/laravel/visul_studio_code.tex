\chapter{Usar Visual Studio Code con Laravel}

\href{https://code.visualstudio.com/}{Visual Studio Code} es un entorno de desarrollo integrado (IDE) desarrollado por Microsoft y con licencia MIT, lo que lo hace Software Libre y que cualquiera pueda ver el código fuente, así como realizar modificaciones.

El problema es que Microsoft no ha liberado todo el código fuente, y los binarios que ofrece para descargar hacen uso de ese software, así como la inclusión de sistemas de telemetría. Es por eso que existe un proyecto llamado \href{https://vscodium.com/}{VSCodium} que ofrece los binarios libres de ese código.

Entre las ventajas que ofrece este IDE podemos destacar:

\begin{itemize}
    \item Se puede programar para muchos lenguajes de programación, no está especializado en uno sólo.

    \item Es extensible mediante \textit{plugins}. Hoy en día existen infinidad de extensiones para todo tipo de desarrollos.

    \item Es multiplataforma.

    \item Altamente configurable.

    \item Configurando la cuenta de GitHub, se puede sincronizar las configuraciones entre distintos dispositivos.

    \item Existe una versión \href{https://vscode.dev/}{online}.
\end{itemize}

\section{Extensiones recomendadas}

Para desarrollar con Laravel, aunque se puede extender a cualquier proyecto que haga uso de un entorno Docker, es recomendable utilizar una serie de extensiones para facilitar el desarrollo con ellos. De todas maneras, Visual Studio Code nos va a recomendar extensiones a medida que lo usemos, ya que observará el tipo de desarrollo que estamos realizando.

Entre las extensiones que se recomiendan están:
\begin{itemize}
    \item \href{https://marketplace.visualstudio.com/items?itemName=ms-vscode-remote.remote-containers}{Dev Containers}: Nos permite abrir un directorio o un repositorio que está dentro de un contenedor como si fuera local de nuestro equipo.

    \item \href{https://marketplace.visualstudio.com/items?itemName=onecentlin.laravel-extension-pack}{Laravel Extension Pack}: Es una “meta-extensión”, ya que incluye a otras extensiones creadas especialmente para ayudar durante el desarrollo de Laravel.
\end{itemize}