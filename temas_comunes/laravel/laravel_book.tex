\newcommand{\ClassPath}{../../yukibook.cls}
\documentclass{\ClassPath/yukibook}


\begin{document}

    \yukibook{Introducción a Laravel} % Title
    {Rubén Gómez Olivencia}  % Author
    {2023}    % Year
    {} % Name of degree
    {} % catch phrase
    {} % the phrase's author
    {img/logo.png} %cover
    {ff5449}
    {} %mini-title

    \coverpage
    \graphicspath{{../../yukibook.cls/}}
    \licensepage
    %
    \tableofcontents

    %--------------------------------------------------------------------------
    % Start your parts, chapters and sections here
    %--------------------------------------------------------------------------
    \graphicspath{{img/}}

    \part{Introducción}
    \chapter{Introducción}

La \textbf{informática} es un área de la ciencia que abarca distintas disciplinas teóricas (como la creación de algoritmos, teoría de computación, teoría de la información, ...) y disciplinas prácticas (diseño de hardware, implementación de software).

A la hora de crear programas (o \textit{software}), podemos identificarlos de distintos tipos:
\begin{itemize}
    \item \textbf{Software de sistema}: Programas o aplicaciones que pertenecen al sistema y nos ayudan a mejorarlo, administrarlo ... Pueden ser aplicaciones de monitorización, de auditoría de logs, \textit{drivers}, ...

    \item \textbf{Software de desarrollo}: En este caso serán aplicaciones que nos ayudarán a crear otras aplicaciones. Por ejemplo: librerías de funciones, compiladores, \textit{debuggers}, IDEs...

    \item \textbf{Aplicaciones de usuario}: Son aplicaciones que los usuarios finales utilizarán en su día a día. Podríamos diferenciarlas como:
    \begin{itemize}
        \item \textbf{Aplicaciones generalistas}: Son aquellas que cualquier tipo de usuario utilizará en cualquier momento. Son creadas con un propósito específico, pero que no hay que tener grandes conocimientos para usarlas. Por ejemplo: navegadores web, clientes de correo, aplicaciones de ofimática simple, calculadora, calendario, ...

        \item \textbf{Aplicaciones de uso específico}: En este caso son aplicaciones creadas para un usuario específico, con una utilidad muy concreta y que normalmente deben existir conocimientos para utilizarla.

        Pueden ser aplicaciones no muy complejas, pero cuya utilidad, o lo que hagan, tenga importancia y conste de procesos complejos. Por ejemplo: aplicaciones CAD, sistemas de virtualización, aplicaciones científicas (R, JupyterLab), aplicaciones empresariales, ...
    \end{itemize}
\end{itemize}

En esta asignatura veremos distintos tipos de software especializado dentro de la gestión empresarial como son los ERP y los CRM, que podríamos englobar como \textbf{sistemas de información}.

\chapter{Sistemas de información}

Un sistema de información, de manera generalizada, es aquel que ayuda a administrar, recolectar, recuperar, procesar, almacenar y distribuir información relevante para ser usados dentro de los procesos fundamentales de una organización.

Normalmente estos sistemas de información son fáciles de usar, tienen cierto grado de flexibilidad (se pueden adaptar a las empresas), permiten guardar y recuperar información de manera rápida y sencilla.

De esta manera, la información resultante será más valiosa para la propia organización, ya que tendrá una “imagen” más amplia y habiendo podido relacionar más información que de no haber utilizado este tipo de software.

\section{Componentes}
Un sistema de información debe contar con los siguientes componentes básicos, que deben interactuar entre sí de manera adecuada para un buen funcionamiento global:
\begin{itemize}
    \item El \textbf{hardware}, equipo físico utilizado para procesar y almacenar datos.
    \item El \textbf{software} y los procedimientos utilizados para transformar y extraer información.
    \item Los \textbf{datos} que representan las actividades de la empresa.
    \item La \textbf{red} que permite compartir recursos entre computadoras y dispositivos.
    \item Las \textbf{personas} que desarrollan, mantienen y utilizan el sistema.
\end{itemize}

El último punto es muy importante, ya que de nada sirve tener la mejor herramienta, en el mejor hardware, si luego las personas que van a hacer uso de ella no tienen los conocimientos suficientes.

\errorbox{\textbf{Las personas que utilizan los sistemas de información deben tener los conocimientos adecuados para su correcta utilización.}}

Es por eso que las personas que hagan uso del sistema de información deberán ser entrenadas y/o tener manuales para su correcto uso, así como \textbf{también tener en cuenta sus opiniones para mejorar los procesos internos de la empresa}.

\section{Datos vs información}

Los datos reflejan hechos recogidos en la organización y que están todavía sin procesar (reflejan valores o resultados de mediciones). Estos datos serán hechos o cifras sobre algún tema específico concreto, que a simple vista no tienen por qué decir nada.

Por otro lado, la información se obtiene una vez se han procesado, agregado y/o presentado de manera adecuada esos datos para que puedan ser útiles y de esta manera obtener un valor que de otra manera no se podría obtener.

El ejemplo más claro entre datos e información se puede obtener en cualquier \textbf{estudio científico}, en el que a tras la obtención de unos datos, a través del método científico se llega una conclusión y con ello información.

Por ejemplo, mediciones del dióxido de carbono (CO$_{2}$) en la atmósfera, se obtienen  datos y se llega a la siguiente imagen que es la información:

\begin{center}
    \includegraphics[width=0.7\linewidth]{co2.png}
    \captionof{figure}{Mediciones de CO$_{2}$ en los últimos miles de años. Fuente: \href{https://climate.nasa.gov/en-espanol/signos-vitales/dioxido-de-carbono/}{NASA}.}
\end{center}



\section{Objetivo}

El objetivo de los sistemas de información, y en este caso, los utilizados para la gestión empresarial, es el de realizar acciones de manera más rápida y eficiente, por lo que también debería ser más económico para la empresa.

El uso de las tecnologías de la información y la comunicación en las empresas se ha convertido en un \textbf{elemento esencial como motor vertebrador y fuente de ventajas competitivas}.

Hoy en día una empresa que no haga uso de la informática está condicionando su estrategia empresarial, y es bastante probable que esté perdiendo oportunidades de negocio, así como la posibilidad de desarrollar sus productos y servicios.

Es por eso que el uso de la informática y de \textit{software} especializado de gestión empresarial puede ayudar a las empresas en:

\begin{itemize}
    \item Obtener ventajas competitivas.
    \item Mejorar la eficiencia interna de la empresa: reducir costes, mejorar la productividad, mejorar la organización de la información, ...
    \item Mejorar y facilitar la toma de decisiones a través de la recopilación de la información.
    \item Para desarrollar nuevas estrategias de negocios.
\end{itemize}

\section{Requisitos}

Para que la información sea útil en la toma de decisiones dentro de una organización, debe cumplir una serie de requisitos:

\begin{itemize}
    \item \textbf{Exactitud}: debe ser precisa y libre de errores.
    \item \textbf{Comprensión}: inteligible por el usuario.
    \item \textbf{Completitud}: debe contener todos aquellos hechos que pudieran ser importantes.
    \item \textbf{Economicidad}: el coste para obtener la información debe ser menor que el beneficio.
    \item \textbf{Confianza}: garantizar tanto la calidad de los datos utilizados, como la de las fuentes de información.
    \item \textbf{Relevancia}: ha de ser útil para la toma de decisiones.
    \item \textbf{Nivel de detalle}: se debe proporcionar con la presentación y el formato adecuados, para que resulte sencilla y fácil de manejar.
    \item \textbf{Oportunidad}: se debe entregar la información a la persona que corresponde y en el momento adecuado.
    \item \textbf{Verificabilidad}: la información ha de poder ser contrastada y comprobada en todo momento.
\end{itemize}

\warnbox{A tener en cuenta: \textbf{el exceso de información también puede ser contraproducente}.}

\section{Actividades}

A la hora de hacer uso de un sistema de información, las actividades que se pueden realizar con él se pueden resumir en:

\begin{itemize}
    \item \textbf{Recopilación}: Es la recogida de datos en bruto. Estos datos pueden ser de dentro de la organización, del exterior, recopilados de manera automática o de manera manual.
    \item \textbf{Almacenamiento}: Los datos deben ser guardados de manera estructurada para su posterior uso. Por otro lado, \textbf{nos debemos asegurar que su persistencia no corra peligro}, por lo que deberemos contar con un sistema de almacenamiento que sea capaz de asegurar posibles problemas. Para ello deberemos tener un sistema en \textbf{alta disponiblidad}, y con un buen sistema de \textbf{backups} configurado.

    También hay que asegurar que \textbf{el acceso a los datos estará limitado y asegurado mediante sistemas de control de acceso y de autenticación}.

    \item \textbf{Procesamiento}: Es el punto clave en el que los datos se convierten en información, de esta manera cumpliendo la labor de ayudar a la organización en la toma de decisiones.

    \item \textbf{Distribución}: El sistema permitirá distribuir la información entre las personas que la necesiten.
\end{itemize}


\section{Tipos de sistemas de información}

Aunque existen distintos tipos de sistemas de información, y su clasificación se puede realizar teniendo en cuenta distintas funcionalidades y/o objetivos, nos vamos a centrar en dos tipos:

\begin{itemize}
    \item \textbf{ERP}: \textit{Enterprise Resource Planning} o planificación de recursos en la empresa. Se trata de los sistemas de gestión integrados que permiten dar soporte a la totalidad de los procesos de una empresa: control económico financiero, logística, producción, mantenimiento, recursos humanos, ...

    \item \textbf{CRM}: \textit{Customer Relationship Management}, sistemas para gestionar las relaciones con los clientes y el soporte a todos los contactos comerciales.
\end{itemize}
    \chapter{Modelo-Vista-Controlador}

La arquitectura \textbf{Modelo-Vista-Controlador} es un patrón de diseño de software que separa las funciones que el software realiza en tres capas principales:

\begin{itemize}
    \item \textbf{Modelo de datos}: Es la representación de la información que la que la aplicación interactúa, tanto para obtener la información como para ser actualizada.

    El modelo de datos normalmente equivale al diseño de base de datos, donde cada modelo representa a una entidad en un diseño de base de datos relacional.

    El controlador es el encargado de realizar las peticiones al modelo, ya se actualizaciones o la obtención de información.

    \item \textbf{Controlador}: Responde a acciones del usuario (o eventos), que normalmente desencadenan en una acción al modelo de datos (ya sea obtención de datos, actualización, borrado...).

    El controlador hace de intermediario entre la vista y el modelo.

    \item \textbf{Vista}: Es la parte que muestra al usuario los datos obtenidos y con la que este interactúa. Esta interacción generará posibles acciones que irán al controlador para volver a empezar el ciclo.
\end{itemize}

\section{Interacción de los componentes}

Aunque existen distintas implementaciones de la arquitectura Modelo-Vista-Controlador, el flujo de acciones suele ser similar al siguiente:

{
\begin{minipage}{0.56\linewidth}
\begin{enumerate}
    \item El usuario interactúa con la interfaz de usuario de alguna forma (por ejemplo, el usuario pulsa un botón, enlace, etc.).

    \item El controlador recibe (por parte de los objetos de la interfaz-vista) la notificación de la acción solicitada por el usuario. El controlador gestiona el evento que llega, frecuentemente a través de un gestor de eventos (handler) o callback.

    \item El controlador realiza una petición al modelo, ya sea para solicitar información o para actualizarla. El modelo debe confirmar si la acción se ha realizado de manera correcta o no.

    \item El controlador delega en la vista la información obtenida para que sea visualizada.

    \item La interfaz se mantiene a la espera de una nueva interacción para comenzar de nuevo el ciclo.
\end{enumerate}
\end{minipage}
\hfill
\begin{minipage}{0.4\linewidth}
    \includegraphics[width=\linewidth]{mvc.png}
\end{minipage}
}

\vspace{10pt}
Tal como se ha dicho, pueden existir distintas implementaciones, pero de manera generalizada y simplificada este sería el esquema básico de interacción.

    \part{Crear entorno Laravel}
    \chapter{Crear primer proyecto en Laravel}

A la hora de crear un proyecto en Laravel lo primero que deberíamos hacer es visitar la \href{https://laravel.com/docs/10.x/installation}{documentación}, ya que nos dará distintas opciones dependiendo del sistema operativo en el que nos encontremos. Aparte, podremos ver si ha habido cambios desde la última vez que hayamos creado un proyecto.

\section{Servicios a utilizar}

Antes de crear el proyecto, debemos tomar una serie de decisiones para nuestro \textit{stack} de aplicación. Laravel cuenta con distintos servicios, algunos de ellos necesarios y otros optativos, por lo que deberemos tenerlos en cuenta.

Los servicios entre los que deberemos decidir son:

\begin{itemize}
    \item \textbf{Sistema Gestor de Base de Datos a utilizar}: Laravel permite el uso de distintos sistemas de bases de datos relacionales como son \href{https://dev.mysql.com/downloads/mysql/}{MySQL}, \href{https://www.postgresql.org/}{PostgreSQL} y \href{https://mariadb.org/}{MariaDB}. Por defecto hace uso de \textbf{MySQL}.
    \item \textbf{Sistema de caché}: Podemos hacer uso de distintos sistemas para cachear desde la sesión a información obtenida de la base de datos y también HTML. Por defecto, \textbf{Laravel cachea la sesión en el sistema de ficheros}, pero eso puede ser lento, por lo que se permite hacer uso de sistemas \textbf{clave-valor} para el almacenamiento de información para acelerar el rendimiento de la aplicación web. Se puede elegir \href{https://www.memcached.org/}{Memcached} o \href{https://redis.io/}{Redis} entre otros.
\end{itemize}

Otros servicios que podemos instalar y que nos darán ciertas funcionalidades son:

\begin{itemize}
    \item \textbf{\href{https://github.com/axllent/mailpit}{Mailpit}}: Es un sistema para controlar los emails que envía nuestra aplicación durante el desarrollo. En lugar de enviarlos a las cuentas finales, se quedan almacenados y se pueden visualizar a través de una web que a modo de buzón de correo. También ofrece una API.

    \item Uso de  \href{https://min.io/}{MinIO} para simular el \textbf{almacenamiento en la nube S3}. De esta manera no tendremos que crear un Bucket de pruebas.

    \item Sistema de \textbf{búsqueda \textit{full-text}} en la base de datos gracias a \href{https://laravel.com/docs/10.x/scout#introduction}{Scout} y haciendo uso del backend \href{https://www.meilisearch.com/}{MeiliSearch}.

    \item Creación y automatización de \textbf{tests} utilizando \href{https://www.selenium.dev/}{Selenium}.
\end{itemize}

Estas son algunas de los servicios que podríamos configurar antes de comenzar a crear nuestra aplicación. Para comenzar de manera sencilla nos centraremos únicamente en la elección de la base de datos, dejando el resto de servicios para más adelante.


\section{Instalación mediante Sail y Docker}

En la \href{https://laravel.com/docs/10.x/installation}{documentación} de Laravel nos explica cómo realizar la instalación de distintos modos teniendo en cuenta el sistema operativo, los servicios iniciales que nos interesan y el sistema de instalación que mejor se adapte a nuestro entorno.


El sistema es similar utilizando GNU/Linux, Windows y MacOS, con la salvedad de que en Windows deberíamos instalar Docker Desktop y \textit{Windows Subsystem for Linux} (WSL). De manera generalizada, es necesario tener instalado:

\begin{itemize}
    \item Entorno GNU/Linux
    \item Docker
    \item Docker Compose
    \begin{mycode}{Instalar Docker Compose}{console}{}
ruben@vega:~$ sudo apt install docker-compose
\end{mycode}
\end{itemize}

Para realizar la instalación sólo vamos a elegir tener el servicio de MySQL, para simplificarlo, tal como se ha comentado previamente. Para ello, deberemos ejecutar lo siguiente en el directorio donde nos interese crear el directorio del proyecto.

\begin{mycode}{Usamos el instalador de Laravel}{console}{{\small }}
ruben@vega:~$ curl -s "https://laravel.build/example-app?with=mysql" | bash
\end{mycode}

Este comando lo que va a hacer es descargarse un script que va a ejecutar lo siguiente:
\begin{enumerate}
    \item Se va a asegurar que Docker está corriendo
    \item Va a levantar un contenedor con la imagen “laravelsail/php82-composer” que nos va a crear un directorio llamado \configdir{example-app} con un proyecto limpio de Laravel usando MySQL como SGBD.
    \item Si no tenemos la imagen de MySQL la descarga.
\end{enumerate}


\subsection{Ventajas}

Este sistema de instalación permite realizar un despliegue sencillo en un equipo donde tengamos instalado Docker, con todas las ventajas que ello ofrece, aparte de la posibilidad de elegir los servicios que necesitemos inicialmente.

Podríamos resumir las ventajas en la siguiente lista:

\begin{itemize}
    \item Instalación rápida con un único comando.
    \item Ventajas de usar Docker: todos los desarrolladores usan el mismo contenedor/entorno de desarrollo.
    \item No es necesario tener nada más que Docker instalado en el equipo anfitrión (ni PHP, composer, servicios web, ...).
\end{itemize}

\subsection{Desventajas}
Aunque las ventajas durante el desarrollo de aplicaciones son notables, también pueden existir algunas desventajas:

\begin{itemize}
    \item El servidor web que se arranca por defecto no es el más recomendado para despliegues en producción, obteniendo mejor rendimiento con el servidor web \href{https://nginx.org/en/}{Nginx}.

    \item En un principio puede resultar “raro” desarrollar dentro de un contenedor.
\end{itemize}


\section{Iniciar servicios}

Una vez terminada la descarga del código y tras realizar las acciones que necesita, el propio asistente nos avisa de qué tenemos que realizar para que nuestro entorno arranque.

\begin{mycode}{Arrancamos los servicios}{console}{}
ruben@vega:~$ cd example-app && ./vendor/bin/sail up -d
[+] Running 2/2
  Container example-app-mysql-1         Started    0.4s
  Container example-app-laravel.test-1  Started    0.7s
\end{mycode}

Y con ello podemos ir al puerto 80 a través de nuestro navegador y veremos la página principal para comprobar que todo ha ido bien.

\begin{center}
    \includegraphics[frame,width=0.7\linewidth]{intro.png}
\end{center}


\chapter{Variables de entorno}
Todo proyecto de Laravel cuenta con unas variables de entorno del proyecto. Es un fichero de configuración situado en la raíz del proyecto que se llama \configfile{.env} y en él se encuentran las credenciales para acceder a la base de datos, servidor SMTP, ...

Dado que hemos generado el entorno a través del asistente, se ha rellenado con la configuración por defecto, entre las que nos encontramos que la aplicación tiene \textbf{el modo debug activado}. Durante el desarrollo nos va a ayudar para poder hacer \textit{debugging} mientras realizamos acciones, pero esta opción debería estar deshabilitada al poner la aplicación en producción.

Por último, es conveniente recordar que este fichero \textbf{nunca debería estar en un repositorio público}, ya que contiene información sensible como lo son los credenciales de acceso a bases de datos o servicios externos.

\errorbox{\textbf{Cuidado con versionar el fichero “\texttt{.env}”, ya que contiene información sensible}}
    \chapter{Usar Visual Studio Code con Laravel}

\href{https://code.visualstudio.com/}{Visual Studio Code} es un entorno de desarrollo integrado (IDE) desarrollado por Microsoft y con licencia MIT, lo que lo hace Software Libre y que cualquiera pueda ver el código fuente, así como realizar modificaciones.

El problema es que Microsoft no ha liberado todo el código fuente, y los binarios que ofrece para descargar hacen uso de ese software, así como la inclusión de sistemas de telemetría. Es por eso que existe un proyecto llamado \href{https://vscodium.com/}{VSCodium} que ofrece los binarios libres de ese código.

Entre las ventajas que ofrece este IDE podemos destacar:

\begin{itemize}
    \item Se puede programar para muchos lenguajes de programación, no está especializado en uno sólo.

    \item Es extensible mediante \textit{plugins}. Hoy en día existen infinidad de extensiones para todo tipo de desarrollos.

    \item Es multiplataforma.

    \item Altamente configurable.

    \item Configurando la cuenta de GitHub, se puede sincronizar las configuraciones entre distintos dispositivos.

    \item Existe una versión \href{https://vscode.dev/}{online}.
\end{itemize}

\section{Extensiones recomendadas}

Para desarrollar con Laravel, aunque se puede extender a cualquier proyecto que haga uso de un entorno Docker, es recomendable utilizar una serie de extensiones para facilitar el desarrollo con ellos. De todas maneras, Visual Studio Code nos va a recomendar extensiones a medida que lo usemos, ya que observará el tipo de desarrollo que estamos realizando.

Entre las extensiones que se recomiendan están:
\begin{itemize}
    \item \href{https://marketplace.visualstudio.com/items?itemName=ms-vscode-remote.vscode-remote-extensionpack}{Remote Development}: Nos instala un grupo de extensiones para poder trabajar contra un servidor remoto.

    \item \href{https://marketplace.visualstudio.com/items?itemName=onecentlin.laravel-extension-pack}{Laravel Extension Pack}: Es una “meta-extensión”, ya que incluye a otras extensiones creadas especialmente para ayudar durante el desarrollo de Laravel.

    \item \href{https://marketplace.visualstudio.com/items?itemName=xdebug.php-pack}{PHP Extension Pack}: Es un conjunto de extensiones que nos va a permitir trabajar de manera más cómoda durante el desarrollo de código PHP.

    \item \href{https://marketplace.visualstudio.com/items?itemName=formulahendry.auto-close-tag}{Auto Close Tag}: Muy útil durante el desarrollo de HTML, ya que cuando creamos una etiqueta, automáticamente nos crea la etiqueta de cerrado.
\end{itemize}

Existe una infinidad de extensiones que nos pueden ayudar durante el desarrollo,

\section{Conexión al servidor}
Si hacemos uso de una máquina virtual para el desarrollo, por no usar GNU/Linux en la máquina anfitriona donde hemos instalado el contenedor de Laravel, Visual Studio Code nos permite conectarnos por SSH a un servidor donde vayamos a realizar el desarrollo.

Para conectarnos usaremos la extensión recién instalada “Remote explorer”, nos aseguramos que estamos en la opción “Remotes (Tunnels/SSH)” y crearemos una nueva conexión SSH al servidor a través del icono “+”, en el que nos pedirá realizar la conexión SSH:

\begin{center}
    \includegraphics[frame,width=0.5\linewidth]{visual_studio_code_ssh.png}
\end{center}

\warnbox{Es recomendable hacer uso del \hyperlink{ssh_clave_publica_privada}{sistema de certificados de clave pública/privada} de SSH para realizar la conexión}

\begin{center}
    \includegraphics[width=0.9\linewidth]{visual_studio_code_ssh2.png}
\end{center}

Nos pedirá dónde queremos guardar la configuración, dejando la ruta por defecto, que es el fichero \configfile{.ssh/config} dentro de la “home” de nuestro usuario. Si hemos realizado la configuración de los certificados de clave pública/privada, no nos pedirá la contraseña.


\section{Conexión al contenedor}

Gracias a la extensión “remote development” instalada previamente, vamos a poder trabajar \textbf{dentro del contenedor de Laravel}. De esta manera Visual Studio Code va a tener acceso al intérprete de PHP para poder ayudarnos durante el desarrollo.

Hacer uso de esta funcionalidad es muy útil ya que al estar dentro del contenedor, estamos dentro del entorno de desarrollo de manera “inmersiva”, pudiendo instalar componentes o ejecutar órdenes dentro del contenedor.

Para realizar la conexión, deberemos ver los contenedores en la extensión “Remote Explorer”, en el apartado “\textbf{Dev Containers}”:

\begin{center}
    \includegraphics[frame,width=0.5\linewidth]{visual_studio_code_container.png}
\end{center}

Al acceder al contenedor, en este caso “example-app\_laravel.test\_1”, Visual Studio nos debería abrir el directorio principal donde está situada la aplicación Laravel, \configdir{/var/www/html}.

    \part{Funcionalidad básica}
    \chapter{Introducción}

Ahora que ya tenemos el entorno creado, es momento de empezar a añadir funcionalidad básica a nuestra aplicación y comenzar a crear nuestra aplicación. Para estos ejemplos se ha decidido crear una pequeña aplicación a modo de blog, con posts y comentarios.

\chapter{Artisan}
\href{https://laravel.com/docs/10.x/artisan}{Artisan} es la interfaz de línea de comandos que vamos a utilizar para realizar todo tipo de interacción entre el proyecto y el propio \textit{framework} Laravel. Esta interfaz nos va a permitir, entre otras cosas:

\begin{itemize}
    \item Crear modelos y controladores.
    \item Crear una sesión a la base de datos.
    \item Controlar el estado de los “migrations”.
    \item Hacer uso de los “seeds” en la base de datos.
    \item Limpiar la caché de objetos.
\end{itemize}

Cada comando contará con su ayuda, por lo que es recomendable ir mirando la ayuda y así conocer las distintas opciones para cada uno de ellos.


\chapter{Crear modelo}
Un blog tiene una serie de “Posts”, que son las entradas que los usuarios introducen en el blog. De momento vamos a ignorar el apartado de usuarios, para simplificarlo. Una entrada del blog contará con los siguientes atributos:

\begin{itemize}
    \item Título
    \item Texto
    \item Si está publicada o no
\end{itemize}

Para crear el modelo, ejecutaremos el siguiente comando. Este comando lo debemos ejecutar dentro del contenedor Docker y dentro de la ruta donde se encuentra el proyecto Laravel, que es \configdir{/var/www/html}:

\begin{mycode}{Crear Modelo}{console}{}
root@1b29e46c10ae:/var/www/html# php artisan make:model Post -crms
\end{mycode}

Este comando nos va a crear el modelo Post junto con:
\begin{itemize}
    \item \textbf{Controlador} de tipo “resource”, lo que va a permitir realizar acciones “\textbf{CRUD}” (\textit{create}, \textit{read}, \textit{update} y \textit{delete}), necesarias en cualquier aplicación web.
    \item \textbf{\textit{Migration}}: Un fichero para realizar la migración del modelo en la base de datos.
    \item \textbf{\textit{Seed}}: Un fichero de tipo “semilla” para introducir datos en la base de datos.
\end{itemize}

\chapter{Entendiendo las “\textit{migrations}” de base datos}

Hoy en día son muchos los \textit{frameworks} que hacen uso de sistemas de tipo \textbf{\textit{migration}} a la hora de interactuar en el tiempo con la base de datos. Podríamos definirlo como un \textbf{sistema de control de versiones para el esquema de base de datos}.

Este sistema permite ir evolucionando el esquema de base de datos (tablas, columnas de las tablas, funciones...) a medida que el propio código fuente de la aplicación va evolucionando. De esta manera, si tenemos el código en un punto concreto, con el sistema \textbf{migrations} nos va a crear la base de datos tal como se necesita en ese punto.

Al crear nuestro proyecto Laravel, ya contamos con una serie de ficheros de migraciones para la base de datos. Estos ficheros se encuentran en \configdir{app/database/migrations/}, teniendo cada fichero un formato similar a \configfile{YYYY_mm_dd_HHMMSS_comentario.php} siendo:

\begin{itemize}
    \item \textbf{YYYY}: el año que se ha creado el fichero.
    \item \textbf{mm}: el mes que se ha creado el fichero.
    \item \textbf{dd}: el día que se ha creado el fichero.
    \item \textbf{HHMMSS}: la hora, minuto y segundo.
    \item \textbf{comentario}: un pequeño comentario sobre el contenido del fichero.
\end{itemize}

De esta manera, los migrations se van a poder ejecutar en orden de fecha de creación, donde normalmente suele ser:
\begin{itemize}
    \item \textbf{De más antiguo a más nuevo}: Se van creando las tablas, y se van añadiendo modificaciones. Es el ciclo normal de de desarrollo, y este es el sistema de uso habitual.
    \item \textbf{De más nuevo a más antiguo}: Se vuelve atrás en el proyecto, eliminando modificaciones. Utilizado para ir a una versión antigua del proyecto.
\end{itemize}

Vamos a utilizar como ejemplo el primer fichero que existe en el directorio, que es para hacer uso de la tabla de usuarios del sistema de autenticación de Laravel. El fichero tiene una clase que extiende de la clase \textbf{Migration} con dos funciones:

\begin{mycode}{Fichero Migration}{PHP}{}
<?php
use Illuminate\Database\Migrations\Migration;
use Illuminate\Database\Schema\Blueprint;
use Illuminate\Support\Facades\Schema;

return new class extends Migration {
    public function up(): void {
        Schema::create('users', function (Blueprint $table) {
            $table->id();
            $table->string('name');
            $table->string('email')->unique();
            $table->timestamp('email_verified_at')->nullable();
            $table->string('password');
            $table->rememberToken();
            $table->timestamps();
        });
    }

    public function down(): void {
        Schema::dropIfExists('users');
    }
};
\end{mycode}

La función \inlineconsole{up()} se ejecutará cuando realizamos la migración, mientra que la función \inlineconsole{down()} se usará cuando realicemos un “\textbf{\textit{rollback}}” (echar para atrás una migración).


\warnbox{\textbf{Por convenio, el nombre de los modelos suelen ser en singular, mientras que las tablas se deben crear en plural. \href{https://laravel.com/docs/10.x/eloquent\#table-names}{Pero se puede cambiar el nombre de la tabla}.}}

\section{Opciones de las migraciones}

En la \href{https://laravel.com/docs/10.x/migrations#tables}{documentación oficial} se explican cómo funcionan los \textit{migrations} y las funcionalidades básicas y avanzadas que tienen.

Teniendo en cuenta lo visto en el punto anterior, podemos visualizar que las acciones del \textit{migration} contiene varias líneas, y vamos a destacar lo siguiente para el fichero \configfile{2014_10_12_000000_create_users_table.php}:

\begin{itemize}
    \item Crea una tabla llamada “\textbf{users}”, que contiene varias columnas
    \item \textbf{id}: es un alias al método \textbf{bigIncrements}. Va a generar una columna de tipo “\textit{big integer}” sin signo, que se va a ir incrementando y que va a ser \textbf{clave primaria}.

    \item \textbf{string}: existen varias columnas de tipo “string”, que son “name”, “email” y “password”. Es lo equivalente a “varchar”, sin indicar en este caso el número de longitud. Se le puede indicar como segundo parámetro.
    \item \textbf{unique()}: el contenido de este campo (en el ejemplo el \textbf{email}) debe ser único en la tabla.
    \item \textbf{timestamp}: crea un campo de tipo TIMESTAMP.
    \item \textbf{nullable}: permite que este campo sea \textbf{null}.

    \item \textbf{timestamps()}: Este es un método especial que crea dos campos en la base de datos: “\textbf{created\_at”} y “\textbf{updated\_at}”. De esta manera sabemos cuándo se ha creado y modificado el registro en la base de datos.
\end{itemize}

\exercisebox{Añade a la migración del modelo Post, la generación de los campos: “título”, “texto” y “publicado”. Recuerda mirar la documentación oficial.}

\section{Uso de las migraciones}

Una vez tenemos distintos ficheros de migraciones, hay que saber cómo aplicarlos y qué sucede con ellos. De nuevo, en la \href{https://laravel.com/docs/10.x/migrations#running-migrations}{documentación} aparecen distintos ejemplos, de los cuales se van a destacar sólo unos a continuación:

\subsection{Desplegar migraciones}
Para realizar el despliegue de todas las migraciones debemos ejecutar el siguiente comando:

\begin{mycode}{Ejecutar migraciones}{console}{}
root@1b29e46c10ae:/var/www/html# php artisan migrate
   INFO  Preparing database.
Creating migration table ............................. 52ms DONE

   INFO  Running migrations.
2014_10_12_000000_create_users_table ..............   108ms DONE
2014_10_12_100000_create_password_reset_tokens_table  127ms DONE
2019_08_19_000000_create_failed_jobs_table .........   88ms DONE
2019_12_14_000001_create_personal_access_tokens_table 140ms DONE
2023_09_26_094514_create_posts_table ...............   74ms DONE
\end{mycode}

\subsection{Comprobar estado de las migraciones}

Para comprobar el estado de las migraciones podemos realizarlo de la siguiente manera:
\begin{mycode}{Estado de las migraciones}{console}{}
root@1b29e46c10ae:/var/www/html# php artisan migrate:status

Migration name ................................ Batch / Status
2014_10_12_000000_create_users_table ................. [1] Ran
2014_10_12_100000_create_password_reset_tokens_table . [1] Ran
2019_08_19_000000_create_failed_jobs_table ........... [1] Ran
2019_12_14_000001_create_personal_access_tokens_table  [1] Ran
2023_09_26_094514_create_posts_table ................. [1] Ran
\end{mycode}


Si queremos ver a nivel de base de datos qué ha pasado, podemos ejecutar una sesión y visualizar la propia base de datos. Veremos cómo se ha creado la base de datos, las tablas, y una tabla especial llamada \textbf{migrations}, que contiene qué ficheros se han desplegado.

\begin{mycode}{Ejecutar migraciones}{mysql}{}
root@1b29e46c10ae:/var/www/html# php artisan db

mysql> use example_app;
Database changed

mysql> show tables;
+------------------------+
| Tables_in_example_app  |
+------------------------+
| failed_jobs            |
| migrations             |
| password_reset_tokens  |
| personal_access_tokens |
| posts                  |
| users                  |
+------------------------+
6 rows in set (0.00 sec)

mysql> select * from migrations;
+----+-------------------------------------------------------+-------+
| id | migration                                             | batch |
+----+-------------------------------------------------------+-------+
|  1 | 2014_10_12_000000_create_users_table                  |     1 |
|  2 | 2014_10_12_100000_create_password_reset_tokens_table  |     1 |
|  3 | 2019_08_19_000000_create_failed_jobs_table            |     1 |
|  4 | 2019_12_14_000001_create_personal_access_tokens_table |     1 |
|  5 | 2023_09_26_094514_create_posts_table                  |     1 |
+----+-------------------------------------------------------+-------+
5 rows in set (0.00 sec)
\end{mycode}


\subsection{\textit{Rollback} la última migración}

En un momento dado nos puede interesar echar atrás la última migración, y para ello contamos con la opción \textbf{\textit{rollback}}. Este sistema puede que deshaga las migraciones de varios ficheros.

\begin{mycode}{Ejecutar migraciones}{mysql}{}
root@1b29e46c10ae:/var/www/html# php artisan migrate:rollback

INFO  Rolling back migrations.
2023_09_26_094514_create_posts_table ................  27ms DONE
2019_12_14_000001_create_personal_access_tokens_table  26ms DONE
2019_08_19_000000_create_failed_jobs_table ..........  25ms DONE
2014_10_12_100000_create_password_reset_tokens_table   24ms DONE
2014_10_12_000000_create_users_table ................  27ms DONE
\end{mycode}

En este caso, como el migrate hizo todos los ficheros, el \textit{rollback} se ha ejecutado de todos los ficheros pero \textbf{en orden inverso al de creación}.


\subsection{Limpiar, \textit{reset} y recarga de migraciones}

Para asegurar que el sistema de migraciones está funcionando bien, para hacer pruebas, o para realizar despliegues limpios quizá nos interese borrar todas las migraciones de la aplicación o realizar una recarga de las mismas.

\begin{itemize}
    \item \textbf{db:wipe}: borra todas las tablas, vistas y tipos.
    \item \textbf{migrate:fresh}: borra todas las tablas de la base de datos y aplica de nuevo todas migraciones.
    \item \textbf{migrate:reset}: deshace todas las migraciones de la aplicación. Básicamente es dejar la base de datos limpia. \textbf{En este caso no se borra la tabla “\textit{migration}”}.
    \item \textbf{migrate:refresh}: deshace todas las migraciones de la aplicación y las vuelve a aplicar en orden.
\end{itemize}


\section{Uso de las semillas}

A la hora de crear una aplicación es posible que nos interese que tras realizar un primer despliegue existan datos en la base de datos. Ya sea porque estos datos son necesarios para el correcto funcionamiento de la aplicación o para darle una funcionalidad básica.

Para poblar de datos la base de datos existe el sistema de semillas, o \textbf{\textit{seeds}}. Este sistema funciona a través de sus propios ficheros, que se pueden crear por modelo (tal como hemos hecho en este capítulo), o de manera general en una semilla propia.

Con la generación del modelo se ha creado también el fichero  al que vamos a añadirle el código necesario para que cree un primer post de pruebas: \configfile{app/database/seeders/PostSeeder.php}.

\begin{mycode}{\textit{Seed} del PostSeeder.php}{php}{}
<?php
use Illuminate\Database\Console\Seeds\WithoutModelEvents;
use Illuminate\Database\Seeder;
use Illuminate\Support\Facades\DB;

class PostSeeder extends Seeder {
    public function run(): void {
        DB::table('posts')->insert([
            "titulo"=>"Primer post",
            "texto"=>"Este es el texto del primer post",
            "publicado"=>true,
            "created_at"=>now(),
        ]);
    }
}
\end{mycode}

Para poder hacer uso del modelo “\textbf{DB}” es necesario hacer uso de la librería \configfile{Illuminate\Support\Facades\DB}. Ahora sólo queda ejecutar el \textit{seed} tal como se explica en la \href{https://laravel.com/docs/10.x/seeding}{documentación}:

\begin{mycode}{Ejecutar el seed}{console}{}
root@1b29e46c10ae:/var/www/html# php artisan db:seed PostSeeder
INFO  Seeding database.
\end{mycode}

Si se comprueba la base de datos, se verá cómo en la tabla aparecen los datos del \textit{seed}.


\chapter{Rutas de la aplicación}

Aunque ya tenemos un controlador y datos en la aplicación, hasta ahora son inaccesibles. Lo único que vemos en la aplicación es la página de bienvenida al proyecto y si ponemos cualquier ruta en la URL nos aparece un error “404 Not Found”.

Esto es debido al sistema de enrutado de la aplicación, que sólo permite acceder al \textit{path} “/” que nos muestra la plantilla de bienvenida. Esta configuración se puede ver en el fichero \configfile{routes/web.php}.

\begin{mycode}{Rutas de la aplicación web de Laravel}{php}{}
<?php
use Illuminate\Support\Facades\Route;

Route::get('/', function () {
    return view('welcome');
});
\end{mycode}

Cualquier intento de acceso a algo que no sea esa ruta dará un error 404. Este es un sistema de seguridad para controlar a qué se tiene acceso en la aplicación, y por eso que debemos modificar este fichero para poder acceder a nuestro nuevo controlador.

\begin{mycode}{Añadiendo rutas para el nuevo controlador}{php}{}
<?php
// ...
use App\Http\Controllers\PostController;
Route::controller(PostController::class)->group(function () {
    Route::get('/posts', 'index')->name('posts.index');
    Route::get('/posts/{post}', 'show')->name('posts.show');
});
\end{mycode}

Este código indica que se va a utilizar la clase “PostController” para el grupo de las rutas que aparecen en ese trozo de código. Si vamos al fichero \configfile{App\Http\Controllers\PostController.php} veremos que por defecto todas las funciones están vacías, y es por eso que no nos devuelve ningún dato.

Por lo tanto, la idea es:

\begin{itemize}
    \item \textbf{/posts}: irá a la función “index” del controlador. Esta función normalmente lista el contenido de la tabla de base de datos que hace referencia al modelo. En nuestro caso, mostrará todos los posts del blog (normalmente en formato paginado).
    \item \textbf{/posts/\{post\}}: esta ruta será la utilizada cuando queramos ir a ver un registro del modelo concreto. En este caso “\{post\}” indicará el “id” dentro de la base de datos que se le pasará a la función “show”.
\end{itemize}

En el siguiente apartado, cuando modifiquemos el controlador quedará más claro.

\section{Tipos de rutas}

Hay que entender que las rutas funcionan en base al protocolo HTTP. Esto quiere decir que existen distintas maneras de acceder a la misma URL dependiendo del tipo de petición que se realice en base a lo que realicemos con el navegador.

Normalmente, cuando navegamos por internet, las peticiones que se realizan son de tipo \textbf{GET}, ya que estamos pidiendo información al servidor web. En cambio, cuando rellenamos un formulario y le damos a enviar, se hace uso del “verbo” \textbf{POST}, ya que se envían datos al servidor.

Las peticiones HTTP que se pueden utilizar son:

\begin{itemize}
    \item \textbf{GET}: Se realiza una petición a la ruta especificada. Estas peticiones sólo obtienen información.
    \item \textbf{POST}: Se envían datos al servidor, que van incluidos dentro del cuerpo de la petición. Lo habitual cuando utilizamos formularios. Se utiliza para crear nuevos recursos.
    \item \textbf{PUT}: Similar a POST, pero en este caso suele estar orientado a modificar datos previamente creados.
    \item \textbf{PATCH}: Como PUT, sobreescribe completamente un recurso existente.
    \item \textbf{DELETE}: Borra el recurso especificado.
\end{itemize}

Es conveniente mirar la \href{https://laravel.com/docs/10.x/routing}{documentación} cuando queramos realizar algún tipo de petición distinto de GET, ya que nos ayudará a comprender mejor qué es lo que está sucediendo.

\chapter{Controladores y Vistas}
Ahora que ya tenemos las rutas creadas, es momento de que los datos se visualicen en la aplicación. Para ello es necesario entender cómo funciona el sistema de plantillas utilizado por Laravel, llamado \textbf{\href{https://laravel.com/docs/10.x/blade}{Blade}}, que junto con el sistema de \textbf{enrutado} visto previamente, relaciona la URL a la que se llama con el controlador y la vista correspondientes.

\section{Obtener datos en el controlador}

El ejemplo va a consistir en obtener todos los posts de la base de datos y hacer un listado con ellos. Por otro lado, al seleccionar un post concreto, se mostrará dicho post. Para ello vamos a modificar el controlador para modificar las dos funciones que se utilizan en las rutas:

\begin{mycode}{Funciones modificadas en el controlador Post}{php}{}
<?php
// ...
use App\Models\Post;
// ...
class PostController extends Controller{
    public function index(){
        $posts = Post::orderBy('created_at')->get();
        return view('posts.index',['posts' => $posts]);
    }
    //...
    public function show(Post $post){
        return view('posts.show',['post'=>$post]);
    }
\end{mycode}

El problema de este código es que estamos llamando a unas vistas que todavía no hemos creado, y les estamos pasando como variables a la vista los datos obtenidos dentro de un array. Podremos pasar tantas variables como queramos.


\section{Generar vista}

El sistema de plantillas y vistas Blade se guardan en la ruta \configdir{resources/views}, y en el primer caso lo que estamos diciendo es que haga uso de “posts.index”, que quiere decir el fichero “index.blade.php” del directorio “posts”. Por lo tanto, deberemos crear un fichero en la ruta \configfile{resources/views/posts/index.blade.php}, que corresponde a la vista que estamos llamando.

\infobox{\textbf{Es recomendable para cada Modelo/Controlador crear un directorio de vistas}}

Ahora es momento de visualizar los datos en la vista. Para ello, recorreremos el listado obtenido y lo visualizaremos, todo ello en la vista. El sistema de plantillas \href{https://laravel.com/docs/10.x/blade}{Blade} permite introducir funcionalidad similar a PHP en la vista mezclado con HTML. También permite incrustar código PHP directamente, pero intentaremos evitarlo.

El sistema de plantillas tiene una serie de palabras reservadas similar a la de los lenguajes de programación más habituales. En este ejemplo se va a recorrer con un bucle for la lista, se crea una variable de indexación, y así poder visualizar los atributos:

\begin{mycode}{Vista “index.blade.php”}{html+smarty}{}
<ul>
  {{--esto es un comentario: recorremos el listado de posts--}}
  @foreach ($posts as $post)
    {{-- visualizamos los atributos del objeto --}}
    <li>
      <a href="{{route('posts.show',$post)}}"> {{$post->titulo}}</a>.
      Escrito el {{$post->created_at}}
    </li>
  @endforeach
</ul>
\end{mycode}

Si ahora visualizamos la ruta “/posts” obtendremos el listado. Es importante destacar que para el enlace que nos lleva a visualizar un post concreto \textbf{se ha hecho uso del sistema ed rutas} al que se le pasa como parámetro el post.

\exercisebox{Crea la vista para visualizar toda la información del post en “show.blade.php”.}


\chapter{\textit{Soft Deleting}}

Laravel, a través de su ORM Eloquent, nos permite hacer uso del sistema “\textit{\href{https://laravel.com/docs/10.x/eloquent\#soft-deleting}{soft deleting}}”, que en lugar de borrar los registros de la base de datos, lo que hace es marcarlo como borrado. Esto lo hace a través de una columna en la base de datos, indicando con una fecha cuándo se ha borrado.

Es habitual hacer uso de estos sistemas, por si el borrado ha sido erróneo, y de esta manera poder recuperar registros (ya que realmente no se han borrado).

Para hacer uso de este sistema debemos indicarlo en el modelo, para ello le indicaremos que se va a usar “\textbf{SoftDeletes}”:

\begin{mycode}{Indicar en el modelo el uso de Softdeletes}{php}{}
<?php
//...
use Illuminate\Database\Eloquent\SoftDeletes;
class Post extends Model{
    use SoftDeletes;
    //...
}
\end{mycode}

Y también debemos indicarlo en la generación de la base de datos (o en un nuevo \textit{migration}). De esta manera, se creará la columna correspondiente que es necesaria.

\begin{mycode}{Indicar en el “migration” el uso de Softdeletes}{php}{}
<?php
//...
public function up(): void {
    Schema::create('posts', function (Blueprint $table) {
        $table->id();
        $table->string("titulo",128);
        $table->string("texto",5000);
        $table->boolean("publicado");
        $table->softDeletes();
        $table->timestamps();
    });
}
\end{mycode}

Si ejecutamos los \textit{migrations}, veremos que la tabla tiene un campo “\textbf{deleted\_at}”, que por defecto estará a NULL. Si ahora borramos un registro, se actualizará esa columna con la fecha del momento en el que se ha realizado la acción de borrado. \textbf{Estos registros pueden ser recuperados}.



\chapter{Debug}
Durante el desarrollo es habitual hacer uso de sistemas de \textit{debug}, por ejemplo para poder ver el contenido de variables y parar la ejecución del algoritmo que estamos programando.

Laravel cuenta con una función llamada \inlineconsole{dd()} que podemos utilizar en cualquier momento. Por ejemplo, si lo usamos en el controlador creado previamente:

\begin{mycode}{Llamar a Tinker con Artisan}{php}{}
<?php
 public function index(){
    $posts = Post::all();
    dd($posts);
    return view('posts.index',['posts' => $posts]);
}
\end{mycode}

En este caso, se ejecutará la petición de obtener todos los \textit{posts}, y acto seguido la función \inlineconsole{dd($posts)} lo que hará será mostrar por pantalla el contenido de la variable y terminará la ejecución del código.


\chapter{Consola Tinker}

Hoy en día muchos \textit{frameworks} tienen algún sistema de consola interactiva con la que poder utilizar las funcionalidades del mismo. De esta manera, podemos realizar comprobaciones, interactuar con los modelos, objetos... pero sin tener que hacerlo desde el código de la web.

En el caso de Laravel la consola se llama \href{https://laravel.com/docs/10.x/artisan#tinker}{Tinker}, y se puede llamar de dos maneras, dependiendo desde dónde lo hagamos:
\begin{itemize}
    \item Si lo realizamos desde dentro del contenedor, usaremos Artisan de la siguiente manera:
\begin{mycode}{Llamar a Tinker con Artisan}{console}{}
root@1b29e46c10ae:/var/www/html# php artisan tinker
Psy Shell v0.11.21 (PHP 8.2.10 — cli) by Justin Hileman
\end{mycode}

    \item Si lo hacemos desde nuestro servidor anfitrión, usaremos Sail:
\begin{mycode}{Arrancamos los servicios}{console}{}
ruben@vega:~$ cd example-app && ./vendor/bin/sail tinker
Psy Shell v0.11.21 (PHP 8.2.10 — cli) by Justin Hileman
\end{mycode}
\end{itemize}

Una vez dentro, podremos hacer uso de los modelos, por ejemplo, para ver los datos que tenemos en la base de datos.

\begin{mycode}{Arrancamos los servicios}{psysh}{}
> Post::all();
[!] Aliasing 'Post' to 'App\Models\Post' for this Tinker session.
= Illuminate\Database\Eloquent\Collection {#7247
    all: [
    App\Models\Post {#7249
        id: 1,
        titulo: "Primer post",
        texto: "Este es el texto del primer post",
        publicado: 1,
        created_at: "2023-10-01 16:57:30",
        updated_at: null,
    },
    ],
}
\end{mycode}

    \part{Usar Bootstrap en Laravel}
    \chapter{Instalar dependencias}

Laravel tenía soporte nativo de Bootstrap, pero decidió sustituirlo por \href{https://tailwindcss.com/}{Tailwind}. Eso no quita que podamos usar Bootstrap, pero necesitaremos realizar la instalación de dependencias.

\begin{mycode}{Rutas de la aplicación web de Laravel}{console}{}
root@1b29e46c10ae:/var/www/html# composer require laravel/ui --dev
root@1b29e46c10ae:/var/www/html# php artisan ui bootstrap --auth
root@1b29e46c10ae:/var/www/html# npm install
root@1b29e46c10ae:/var/www/html# npm run build
\end{mycode}

De esta manera no sólo hemos instalado las dependencias necesarias para hacer uso de Bootstrap, si no que también nos ha generado unas vistas para el sistema de autenticación en \configdir{resources/views/auth} y una plantilla general para la aplicación.


\chapter{Plantilla general}

Anteriormente se ha mencionado que Blade es un sistema de plantillas para Laravel. Esto significa que es capaz de generar unos componentes de vistas que a su vez incorporan otras vistas, de esta manera generando plantillas que se pueden reutilizar ahorrando código y simplificando la aplicación.

Con lo realizado previamente se ha generado una plantilla general en el fichero \configfile{resources/views/layouts/app.blade.php}, que se puede dividir en dos apartados:

\begin{itemize}
    \item \textbf{Cabecera “nav”}: es la cabecera de la aplicación. Aparece el nombre de la aplicación y a la derecha tiene enlaces para hacer login o registrarse en la aplicación con el sistema de autenticación.

    \item \inlineconsole{@yield('content')}: este apartado será sustituido por la vista desde la que se llame a esta plantilla.
\end{itemize}

De esta manera, en todas las vistas de la aplicación que llamemos a la plantilla, no tendremos que escribir el código de la cabecera. Lógicamente, \textbf{es posible añadir nuevos apartados a esta vista} para cumplir con el objetivo final de la aplicación.

\section{Cómo usar la plantilla}

Para poder hacer uso de la plantilla, debemos indicarlo en las correspondientes vistas. Como ejemplo, se va a utilizar la vista creada en el capítulo anterior, la que muestra por pantalla los \textit{posts} del blog.

La vista modificada quedaría:

\begin{mycode}{Vista modificada usando la plantilla}{html+smarty}{}
@extends('layouts.app')

@section('content')
  <div class="container">
    <ul>
      {{--esto es un comentario: recorremos el listado de posts--}}
      @foreach ($posts as $post)
        {{-- visualizamos los atributos del objeto --}}
        <li>{{$post->titulo}}. Escrito el {{$post->created_at}}</li>
      @endforeach
    </ul>
  </div>
@endsection
\end{mycode}

Tal como se puede ver, lo primero que se indica es que esta vista “extiende” de la plantilla correspondiente. Posteriormente, lo que se hace es indicar una sección de la plantilla que va a ser sustituida por el contenido que aparece entre “@section” y “@endsection”.

    \part{Métodos \textit{create}, \textit{update}, \textit{delete}}
    \chapter{Crear rutas necesarias}

Una aplicación web normalmente nos va a permitir crear datos, no sólo visualizarlos. Por lo tanto vamos a tener que crear la vista de un formulario que el usuario pueda utilizar para crear datos a través del controlador.

Tal como hemos dicho, las funcionalidades de la aplicación empiezan por crear una ruta a la que el usuario puede acceder. En este caso, se podrían crear las rutas necesarias para visualizar el formulario de creación, obtener los datos para la creación, edición y actualización de datos...

En lugar de eso el sistema de rutas de Laravel nos permite simplificarlo, y si tenemos un modelo que sabemos que es de tipo “resource”, nos permite crear todas las rutas necesarias para la gestión de los datos. Por lo tanto, las rutas quedarían de la siguiente manera:

\begin{mycode}{Rutas simplificadas para un modelo de tipo “resource”}{php}{}
<?php
//...
Route::resources([
    'posts' => PostController::class,
]);
\end{mycode}

Si miramos las rutas generadas, veremos que nos ha creado todas las rutas necesarias para interactuar con los posts:

\begin{mycode}{Mirando todas las rutas creadas}{console}{}
root@1b29e46c10ae:/var/www/html# php artisan route:list
GET|HEAD   posts ............ posts.index › PostController@index
POST       posts ............ posts.store › PostController@store
GET|HEAD   posts/create ..... posts.create › PostController@create
GET|HEAD   posts/{post} ..... posts.show › PostController@show
PUT|PATCH  posts/{post} ..... posts.update › PostController@update
DELETE     posts/{post} ..... posts.destroy › PostController@destroy
GET|HEAD    posts/{post}/edit. posts.edit › PostController@edit
\end{mycode}

Tal como se puede ver, por haber indicado la ruta anterior, automáticamente nos ha creado las rutas para listar, crear, visualizar, actualizar, editar y borrar el recurso. Con una única línea nos evita tener que escribir todas las líneas que supondrían de configuración.


\chapter{Crear registro}

A la hora de crear un registro, Laravel por defecto hace uso de la ruta “create”, por lo que deberemos crear un botón en la vista principal que nos mande a la URL “/posts/create”, por lo que la vista de creación será \configfile{create.blade.php}. Sin entrar en detalles, ya que es un formulario simple, tendrá la siguiente forma:

\begin{mycode}{Vista del formulario}{html+smarty}{{\footnotesize }}
@section('content')
<div class="container">
  <form class="mt-2" name="create_platform" action="{{route('posts.store')}}"
    method="POST" enctype="multipart/form-data">
    @csrf
    <div class="form-group mb-3">
      <label for="titulo" class="form-label">Titulo</label>
      <input type="text" class="form-control" id="titulo" name="titulo" required/>
    </div>
    <div class="form-group mb-3">
      <label for="texto" class="form-label">Texto</label>
      <textarea type="textarea" rows="5" class="form-control" id="texto" name="texto">
      </textarea>
    </div>
    <div class="form-check">
      <input class="form-check-input" type="checkbox" id="publicado" name="publicado">
      <label class="form-check-label" for="publicado">¿Publicar?
      </label>
    </div>
  <button type="submit" class="btn btn-primary" name="">Crear</button>
</form>
</div>
\end{mycode}

Al pulsar el botón “Crear” se realizará una petición “\textbf{POST}” a la ruta “\textbf{posts.store}”, por lo tanto es la función que debemos modificar ahora en el controlador, que junto con la función “create”, tendrá la siguiente forma:

\begin{mycode}{Añadiendo funcionalidad al controlador}{php}{}
<?php
//...
public function create(){
    return view('posts.create');
}

public function store(Request $request){
    $post = new Post();
    $post->titulo = $request->titulo;
    $post->texto = $request->texto;
    $post->publicado = $request->has('publicado');
    $post->save();
    return redirect()->route('posts.index');
}
\end{mycode}


\chapter{Editar registro}

Una vez creados los registros nos va a interesar poder editarlos. Para ello, tendremos que añadir a las vistas correspondientes (el listado general y/o desde la vista del post) un botón que nos lleve a la ruta para editar, que es: \configlink{/posts/{id}/edit}.

Para poder visualizar los datos, deberemos obtener desde el controlador los datos referentes a ese “id” para poder visualizarlo en el formulario que crearemos en la vista \configfile{posts/edit.blade.php}. Después, en la función update deberemos realizar el guardado de las modificaciones.

\begin{mycode}{Añadiendo funcionalidad al controlador}{php}{}
<?php
//...
public function edit(Post $post){
    return view('posts.edit',['post'=>$post]);
}

public function update(Request $request, Post $post){
    $post->titulo = $request->titulo;
    $post->texto = $request->texto;
    $post->publicado = $request->has('publicado');
    $post->save();
    return view('posts.show',['post'=>$post]);
}
\end{mycode}

Tal como se puede ver, la función de actualizar lo que hace es recibir los datos del formulario y el registro a actualizar. Debemos sustituir los campos y para finalizar guardar los cambios del registro. Después, volvemos a la vista para visualizar los cambios.

La vista para editar el registro quedaría:

\begin{mycode}{Vista del formulario}{html+smarty}{{\footnotesize }}
@extends('layouts.app')
@section('content')
<div class="container">
  <form class="mt-2" name="create_platform" action="{{route('posts.update',$post)}}"
    method="POST" enctype="multipart/form-data">
    @csrf
    @method('PUT')
    <div class="form-group mb-3">
      <label for="titulo" class="form-label">Titulo</label>
      <input type="text" class="form-control" id="titulo" name="titulo" required
        value="{{$post->titulo}}"/>
    </div>
    <div class="form-group mb-3">
      <label for="texto" class="form-label">Texto</label>
      <textarea type="textarea" rows="5" class="form-control" id="texto" name="texto">
        {{$post->texto}}
      </textarea>
    </div>
    <div class="form-check">
      <input class="form-check-input" type="checkbox" id="publicado" name="publicado"
        @checked($post->publicado)>
      <label class="form-check-label" for="publicado">
      ¿Publicar?
      </label>
    </div>

    <button type="submit" class="btn btn-primary" name="">Actualizar</button>
  </form>
</div>
@endsection
\end{mycode}

Dado que es el formulario de edición, deben existir datos, de ahí que para cada apartado haya que añadir el parámetro “value” en los \textit{inputs}, el valor dentro del \textit{textarea}, o darle el valor correspondiente al \textit{checkbox}.

También hay que tener en cuenta que debido a cómo funciona el protocolo HTTP con los formularios, \href{https://developer.mozilla.org/en-US/docs/Web/HTTP/Methods/PUT}{que no se puede utilizar en formularios}, debemos añadir \inlineconsole{@mehotd('PUT')} para que genere un \href{https://laravel.com/docs/10.x/blade#method-field}{método oculto} en el formulario.

\exercisebox{Dado que el formulario de crear y actualizar es prácticamente igual, es interesante }


\chapter{Borrar registro}

Por último, tenemos que poder eliminar registros, por lo que deberemos añadir un botón que ejecute la acción de borrado que se recibirá en el controlador.
Este botón lo vamos a añadir a la lista de posts, que junto con el botón editar del apartado anterior, quedaría:

\begin{mycode}{Vista del formulario}{html+smarty}{{\footnotesize }}
@foreach ($posts as $post)
  {{-- visualizamos los atributos del objeto --}}
  <li class="pt-1">
    <div class="d-flex flex-row">
      <a href="posts/{{$post->id}}"> {{$post->titulo}}</a>.
      Escrito el {{$post->created_at}}
      <a class="btn btn-warning btn-sm" href="{{route('posts.edit',$post)}}"
        role="button">Editar</a>

      <form action="{{route('posts.destroy',$post)}}" method="POST">
        @csrf
        @method('DELETE')
        <button class="btn btn-sm btn-danger" type="submit"
          onclick="return confirm('Are you sure?')">Delete
        </button>
      </form>
    </div>
  </li>
@endforeach
\end{mycode}

Y por último el controlador debe borrar el objeto cuando se llama a la función \textbf{destroy}:

\begin{mycode}{Añadiendo funcionalidad al controlador}{php}{}
<?php
//....
public function destroy(Post $post) {
    $post->delete();
    return redirect()->route('posts.index');
}
\end{mycode}

    \part{\textit{Middlewares} y autenticación}
    \chapter{\textit{Middlewares}}

Un \textit{middleware} en Laravel es un mecanismo que inspecciona y filtra las peticiones HTTP que llegan a la aplicación. El ejemplo más claro, y que veremos después, es comprobar si un usuario está autenticado mientras usa la aplicación. En caso de no estar autenticado, le mandará a la página de login/registro.

Se pueden crear otros \textit{middlewares} que nuestra aplicación necesite, como por ejemplo registrar todas las peticiones que llegan a la aplicación, validaciones \href{https://es.wikipedia.org/wiki/Cross-site_request_forgery}{CSRF} de formularios, validación de cabeceras...

Los \textit{middlewares} se sitúan en la ruta \configdir{app/Http/Middleware/}, donde ya existen varios tras realizar la instalación del \textit{framework} Laravel. Para que entren en funcionamiento, se debe realizar la configuración en el fichero de rutas, ya que se activarán dependiendo de las rutas en las que lo indiquemos.

\chapter{Configurando el \textit{middleware} de autenticación}

El sistema de autenticación de Laravel es el ejemplo más claro de \textit{middleware} que podemos utilizar, ya que por defecto viene instalado, pero no está configurado. En pasos anteriores hemos creado el interfaz para poder registrar usuarios y realizar el login en la aplicación.

Ahora es el momento de realizar la activación del sistema de autenticación, y que si no se ha hecho el login, no se pueda ver la aplicación y nos envíe a la página de registro.

Para ello, debemos realizar la modificación de rutas, en la que debemos indicar qué rutas queremos que estén dentro del \textit{middleware} de autenticación. En este caso vamos a elegir que para toda la aplicación sea necesario estar autenticado:


\begin{mycode}{Rutas bajo el \textit{middleware} de autenticación}{php}{}
<?php
//...
Route::middleware(['auth'])->group(function () {
    Route::resources([
    'posts' => PostController::class,
    ]);
});
\end{mycode}


\section{Comprobar rutas bajo \textit{middlewares}}
Si queremos comprobar qué rutas están bajo la influencia de un \textit{middleware}, necesitaremos mirar las rutas en modo \textit{verbose}:


\begin{mycode}{Vista de las rutas en modo \textit{verbose}}{console}{}
root@1b29e46c10ae:/var/www/html# php artisan route:list -v
...
GET|HEAD  posts ............. posts.index › PostController@index
 → web
 → App\Http\Middleware\Authenticate
POST      posts ............. posts.store › PostController@store
 → web
 → App\Http\Middleware\Authenticate
DELETE    posts/{post} ...... posts.destroy › PostController@destroy
 → web
 → App\Http\Middleware\Authenticate
...
\end{mycode}

Se puede comprobar que estas rutas se aplican para la parte “web” de nuestra aplicación, y que antes de ser ejecutadas pasarán por el \textit{middleware} “\textbf{Authenticate}”.

\chapter{Realizar excepciones}

No siempre vamos a querer que toda la aplicación esté bajo el sistema de autenticación, ya que lo habitual es que sólo sea necesario para las acciones que puedan suponer un riesgo de seguridad (edición de datos, borrado de datos, apartados sensibles,...), por lo tanto es interesante que haya rutas que no requieran de estar autenticado.

Para ver cómo funciona, vamos a añadir excepciones al listado de todos los posts y a la visualización de cada post por separado. Para ello, al fichero de rutas añadiremos:

\begin{mycode}{Rutas que están exentas del \textit{middleware} de autenticación}{php}{}
<?php
//...
Route::controller(PostController::class)->group(function () {
    Route::get('/posts', 'index')->name('posts.index');
    Route::get('/posts/{post}', 'show')->name('posts.show');
})->withoutMiddleware([Auth::class]);
\end{mycode}

Tal como se puede ver, hemos creado dos rutas del controlador “PostController” que se les indica “\texttt{->withoutMiddleware}”, para que no se aplique, en este caso, la comprobación de autenticación.


\chapter{Comprobar si el usuario está autenticado}

Por último, debemos asegurar que los botones de edición o borrado sólo aparezcan cuando el usuario esté logueado. Para ello tenemos el sistema \inlineconsole{@auth ... @endauth}. Si modificamos el fichero \configfile{posts/index.blade.php} para evitar que aparezcan los botones de edición y borrado de un post, quedaría:

\begin{mycode}{Comprobar si se está autenticado}{html+smarty}{{\small }}
@auth
  <a class="btn btn-warning btn-sm" href="{{route('posts.edit',$post)}}"
   role="button">Editar</a>

  <form action="{{route('posts.destroy',$post)}}" method="POST"
    enctype="multipart/form-data">
    @csrf
    @method('DELETE')
    <button class="btn btn-sm btn-danger" type="submit"
      onclick="return confirm('Are you sure?')">Delete
    </button>
  </form>
@endauth
\end{mycode}

    \part{Relacionar modelos}
    \chapter{Crear modelo relacionado}

Siguiendo con nuestro blog, donde ya tenemos una aplicación donde crear posts sólo si estamos logueado, es buen momento de añadir nuevas características. Vamos a incluir la opción de tener comentarios, al menos a nivel relacional.

Sin entrar en los atributos que tiene cada entidad/modelo, la relación que tienen los comentarios respecto a un \textit{post} sería la siguiente:

\begin{center}
    \includegraphics[width=0.5\linewidth]{e-r.png}
\end{center}

Es decir, un \textit{post} puede tener muchos comentarios. Un comentario sólo puede pertenecer a un \textit{post}. En principio el único atributo que vamos a permitir es el propio comentario, aparte de la fecha de creación. Para crear el modelo haríamos:

\begin{mycode}{Crear Modelo}{console}{{\small }}
root@1b29e46c10ae:/var/www/html# php artisan make:model Comentario -crms
\end{mycode}


\chapter{Crear migración}
Al igual que vimos al inicio, este comando nos ha creado el modelo, el controlador de \textit{resource}, el sistema de migración y el fichero para añadir la semilla a la base de datos. A la hora de generar la tabla, tenemos que hacer referencia a qué \textit{post} pertenece el comentario, por lo tanto el \textit{migration} queda:

\begin{mycode}{Crear migration}{php}{}
<?php
//...
public function up(): void{
    Schema::create('comentarios', function (Blueprint $table) {
        $table->id();
        $table->string('texto');
        $table->unsignedBigInteger('post_id');
        $table->foreign('post_id')->references('id')->on('posts');
        $table->timestamps();
    });
}
\end{mycode}

Tal como se puede ver, a la hora de crear la tabla en el \textit{migration} se ha creado un campo llamado “\textbf{post\_id}” que después se le ha indicado que es de tipo “clave foránea”. En la \href{https://laravel.com/docs/10.x/migrations#foreign-key-constraints}{documentación} se explican distintas opciones para este tipo de casos.

\exercisebox{\textbf{Crea un “seed” para añadir un comentario al primer \textit{post}}}

\chapter{Crear relación de modelos}

Hasta ahora la relación se ha creado a nivel de base de datos, pero es necesario que Laravel a nivel de \textit{framework}, mientras programamos, sea consciente de que los modelos están relacionados entre sí. Para ello, una vez más en la \href{https://laravel.com/docs/10.x/eloquent-relationships#one-to-many}{documentación} se puede ver cómo Eloquent hace uso de los distintos tipos de relaciones.

Para ello, deberemos modificar ambos ficheros de los modelos que entran en juego en esta relación:
\begin{itemize}
    \item Relación “\textbf{uno a muchos}”, donde un \textit{post} puede tener muchos comentarios. Modificaremos el modelo \configfile{App/Models/Post.php} para que contenga:
\begin{mycode}{Añadir relación “uno a muchos” en Post}{php}{}
<?php
//...
use Illuminate\Database\Eloquent\Relations\HasMany;
class Post extends Model{
    use HasFactory;
    public function comentarios(): HasMany {
        return $this->hasMany(Comentario::class);
    }
}
\end{mycode}

    \item Relación inversa “\textit{\textbf{BelongsTo}}”, donde un comentario pertenece a un \textit{post}. En este caso, modificaremos el modelo \configfile{App/Models/Comentario.php}.

\begin{mycode}{Añadir relación inversa en Comentario}{php}{}
<?php
//...
use Illuminate\Database\Eloquent\Relations\BelongsTo;

class Comentario extends Model{
    use HasFactory;
    public function post(): BelongsTo{
        return $this->belongsTo(Post::class);
    }
}
\end{mycode}
\end{itemize}

Tras esto, ya sea a través de una acción o desde Tinker, podremos obtener los comentarios de un \textit{post} específico, perfecto para dibujarlos en la vista donde se visualiza el \textit{post}. Y al revés, dado un comentario, obtener a qué \textit{post} pertenece.

    %Cómo crear una API con Laravel
    %    - JSON web token
    %    - crear esquema E/R
    %    - crear controlador
    %    - crear funciones

\end{document}