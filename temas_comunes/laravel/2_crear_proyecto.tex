\chapter{Crear primer proyecto en Laravel}

A la hora de crear un proyecto en Laravel lo primero que deberíamos hacer es visitar la \href{https://laravel.com/docs/10.x/installation}{documentación}, ya que nos dará distintas opciones dependiendo del sistema operativo en el que nos encontremos. Aparte, podremos ver si ha habido cambios desde la última vez que hayamos creado un proyecto.

Antes de crear el proyecto, debemos tomar una serie de decisiones para nuestro \textit{stack} de aplicación. Las decisiones son:

\begin{itemize}
    \item \textbf{Sistema Gestor de Base de Datos a utilizar}: Laravel permite el uso de distintos sistemas de bases de datos relacionales como son MySQL, PostgreSQL y MariaDB. Por defecto hace uso de MySQL.
    \item \textbf{Sistema de caché}: Podemos hacer uso de sistemas \textbf{clave-valor} para el almacenamiento de información para acelerar el rendimiento de la aplicación web. Dependiendo de la aplicación se suele cachear información obtenida de la base de datos y también HTML generado de ciertas partes de la web.

\end{itemize}