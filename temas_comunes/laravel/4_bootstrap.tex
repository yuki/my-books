\chapter{Instalar dependencias}

Laravel tenía soporte nativo de Bootstrap, pero decidió sustituirlo por \href{https://tailwindcss.com/}{Tailwind}. Eso no quita que podamos usar Bootstrap, pero necesitaremos realizar la instalación de dependencias.

\begin{mycode}{Rutas de la aplicación web de Laravel}{console}{}
root@1b29e46c10ae:/var/www/html# composer require laravel/ui --dev
root@1b29e46c10ae:/var/www/html# php artisan ui bootstrap --auth
root@1b29e46c10ae:/var/www/html# npm install
root@1b29e46c10ae:/var/www/html# npm run build
\end{mycode}

De esta manera no sólo hemos instalado las dependencias necesarias para hacer uso de Bootstrap, si no que también nos ha generado unas vistas para el sistema de autenticación en \configdir{resources/views/auth} y una plantilla general para la vista.


\chapter{Plantilla de vista}

\section{cómo usar la plantilla}

modificar la vista previa para usar la plantilla