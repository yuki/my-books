\chapter{Instalar dependencias}

Laravel tenía soporte nativo de Bootstrap, pero decidió sustituirlo por \href{https://tailwindcss.com/}{Tailwind}. Eso no quita que podamos usar Bootstrap, pero necesitaremos realizar la instalación de dependencias.

\begin{mycode}{Para usar Bootstrap con Laravel}{console}{}
root@1b29e46c10ae:/var/www/html# composer require laravel/ui --dev
root@1b29e46c10ae:/var/www/html# php artisan ui bootstrap --auth
root@1b29e46c10ae:/var/www/html# npm install
root@1b29e46c10ae:/var/www/html# npm run build
\end{mycode}

A continuación se detalla qué hace cada comando:
\begin{itemize}
    \item \commandbox{composer require laravel/ui --dev}: \href{https://getcomposer.org/}{Composer} es el gestor de dependencias utilizado por PHP. Lo que se está indicando es que se necesita como dependencia el paquete “laravel/ui” durante el desarrollo.
    \item \commandbox{php artisan ui bootstrap --auth}: Se indica qué \textit{framework} para el interfaz se va a utilizar. Aparte, con el parámetro “--auth” se le indica que genera las plantillas para la autenticación.
    \item \commandbox{npm install}: instala las dependencias indicadas en el primer comando.
    \item \commandbox{npm run build}: ejecuta la acción “build” indicada en el fichero \configfile{package.json}. En este caso “compila” los javascripts y los css que se van a utilizar.
\end{itemize}

De esta manera no sólo hemos instalado las dependencias necesarias para hacer uso de Bootstrap, si no que también nos ha generado unas vistas para el sistema de autenticación en \configdir{resources/views/auth} y una plantilla general para la aplicación.


\chapter{Plantilla general}

Anteriormente se ha mencionado que Blade es un sistema de plantillas para Laravel. Esto significa que es capaz de generar unos componentes de vistas que a su vez incorporan otras vistas, de esta manera generando plantillas que se pueden reutilizar ahorrando código y simplificando la aplicación.

Con lo realizado previamente se ha generado una plantilla general en el fichero \configfile{resources/views/layouts/app.blade.php}, que se puede dividir en dos apartados:

\begin{itemize}
    \item \textbf{Cabecera “nav”}: es la cabecera de la aplicación. Aparece el nombre de la aplicación y a la derecha tiene enlaces para hacer login o registrarse en la aplicación con el sistema de autenticación.

    \item \inlineconsole{@yield('content')}: este apartado será sustituido por la vista desde la que se llame a esta plantilla.
\end{itemize}

De esta manera, en todas las vistas de la aplicación que llamemos a la plantilla, no tendremos que escribir el código de la cabecera. Lógicamente, \textbf{es posible añadir nuevos apartados a esta vista} para cumplir con el objetivo final de la aplicación.

\section{Cómo usar la plantilla}

Para poder hacer uso de la plantilla, debemos indicarlo en las correspondientes vistas. Como ejemplo, se va a utilizar la vista creada en el capítulo anterior, la que muestra por pantalla los \textit{posts} del blog.

La vista modificada quedaría:

\begin{mycode}{Vista “show.blade.php” modificada usando la plantilla}{html+smarty}{}
@extends('layouts.app')

@section('content')
<div class="container">
  <h1>{{$post->titulo}}</h1>
  <p>Creado el {{$post->created_at}}</p>
  <p>{{$post->texto}}</p>
</div>
@endsection
\end{mycode}

Tal como se puede ver, lo primero que se indica es que esta vista “extiende” de una plantilla concreta. Después se indica la sección de la plantilla que va a ser sustituida por el contenido que aparece entre “@section” y “@endsection”, que en este caso es la que corresponde al \inlineconsole{@yield('content')} que hemos visto previamente.


\chapter{Modificación de rutas}
Aparte de lo visto hasta ahora, por haber activado el sistema de autenticación, se nos han generado nuevas rutas. Estas rutas las podemos ver a través del siguiente comando, o mirando el fichero de rutas:

\begin{mycode}{Mirando todas las rutas creadas}{console}{}
root@1b29e46c10ae:/var/www/html# php artisan route:list
\end{mycode}
