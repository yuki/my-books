\chapter{Introducción a Laravel}

\href{https://laravel.com/}{Laravel} es un \textit{framework} para crear aplicaciones y servicios web haciendo uso del lenguaje de programación \href{https://es.wikipedia.org/wiki/PHP}{PHP} y buscando la simplicidad y evitar el “\textit{spaghetti code}”. Hace uso de la arquitectura “modelo-vista-controlador” (MVS) y es un proyecto de código abierto.

\section{Características}
Entre las características que tiene Laravel, se pueden destacar:

\begin{itemize}
    \item Sistema de enrutamiento, también RESTful.
    \item Motor de plantillas web llamado \href{https://laravel.com/docs/10.x/blade}{Blade}.
    \item Creador de queries a la base de datos llamada \href{https://laravel.com/docs/10.x/queries}{Fluent}.
    \item \href{https://laravel.com/docs/10.x/eloquent}{Eloquent} como ORM (\textit{object-relational mapper}).
    \item Uso de “\textit{migrations}” para crear la base de datos a modo de sistema de control de versiones.
    \item Posibilidad de usar “semillas” (en inglés “\textit{seeds}”) en la base de datos para importar datos, ya se de test o datos iniciales necesarios.
    \item Permite hacer uso de paquetes de \href{https://getcomposer.org/}{Composer}.
    \item Soporte para el caché.
    \item Soporte para MVC.
    \item Posibilidad de paginación automática.
\end{itemize}

Estas características las iremos utilizando para crear nuestro primer proyecto y para posteriormente aprender a crear una API.