La alta disponibilidad en servidores se puede definir como el diseño de infraestructura (y su implantación) que asegura la continuidad del servicio y que no tiene un único punto de fallo.

Es lógico entender que un servicio debe de ser contínuo en el tiempo, ya que debe de dar servicio de manera continuada para que los usuarios puedan acceder a él. Pero para que esta premisa sea efectiva, y para asegurarnos que así sea, \textbf{la infraestructura debe de estar redundada y carecer de puntos de fallo únicos en su diseño}.

Esto quiere decir, que de cada servicio y para cada posible punto de fallo deberá haber al menos dos de ellos, para que en caso de que uno deje de funcionar el servicio siga funcionando (dos tomas eléctricas separadas, dos servidores que otorguen el servicio, dos conexiones a internet, ... ).

Es habitual que un sistema en Alta Disponibilidad deba de estar pensado desde el diseño. Algunos tipos  de servicios pueden empezar como un único servidor y posteriormente realizar un \hyperlink{escalado_horizontal}{escalado horizontal}, formando la alta disponibilidad, mientras que para otros \textbf{el diseño en alta disponibilidad debe de estar pensado desde el comienzo} (habitualmente en algunos tipos de \hyperlink{cluster}{clusters}).


\section{Importancia de un sistema en Alta Disponibilidad}

Como se ha citado previamente, la alta disponibilidad nos va a asegurar al menos dos grandes ventajas:

\begin{itemize}
    \item Una continuidad en el servicio
    \item Un diseño libre de puntos de fallos únicos, gracias a la redundancia.
\end{itemize}

La redundancia permitirá que en caso de fallo de algún equipamiento/servicio, al estar redundando, no afecte al servicio. Gracias al \hyperlink{monitorizacion_de_sgbds}{sistema de monitorización} seremos capaces de ver el problema y solventarlo lo antes posible. De estar el diseño correcto, el servicio mantendrá su actividad, mientras que por el contrario, si ha habido algún fallo en el diseño de infraestructura (o el problema es más grave de lo esperado) el servicio se verá afectado.


\section{Tipos de Alta Disponibilidad}
Existen muchos tipos de alta disponibilidad dependiendo de en qué capa de infraestructura estemos hablando. Por poner unos ejemplos:

\begin{itemize}
    \item \textbf{Redundancia eléctrica}: Los servidores normalmente cuentan con doble fuente de alimentación, por lo que cada fuente de alimentación debe de estar conectada a una toma eléctrica completamente separada de la otra.
    \item \textbf{Redundancia de conectividad física}: El acceso a internet debe de ser redundado.
    \item \textbf{Redundancia de conectividad LAN}: El acceso a la LAN/DMZ/red de servicio debe de estar redundado (stacks de switches, LACPs configurados en switches y servidores, … ).
    \item \textbf{Redundancia de servidores}: Debe de existir una redundancia de servidores para asegurar que el servicio funcione en más de un servidor físico.
    \item \textbf{Redundancia de servicio}: El servicio que se ofrece debe de estar redundado entre los distintos servidores.
\end{itemize}


La alta disponibilidad también se puede diferenciar como:


\begin{itemize}
    \item \textbf{Alta disponibilidad real}: En caso de que haya algún problema el servicio continúa como si no hubiese pasado nada, gracias a la redundancia completa de servicios/dispositivos.
    \item \textbf{Alta disponibilidad pasiva}: En caso de error, los servidores activos serían los que reciben el impacto del problema y hay que escalar los servidores pasivos a modo activo para que comiencen a funcionar otorgando el servicio. Como se puede presuponer, esta modificación puede ser realizada de manera automática o de manera manual (lo que llevaría algo de tiempo, y por tanto el servicio se vería afectado).
\end{itemize}