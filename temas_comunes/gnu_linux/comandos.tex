\chapter{Comandos de administración básica en GNU/Linux}
A lo largo de este documento hemos visto distintos comandos para realizar la administración de usuarios y grupos locales o para crear un directorio activo en Samba. En este documento vamos a añadir otros comandos que nos pueden ser útiles a la hora de usar un sistema GNU/Linux y realizar su administración.

\section{Comandos de red}
Para ver los interfaces de red y las direcciones IP que tienen

\begin{mycode}{Obtener los interfaces y las IPs}{console}{}
ruben@vega:~$ ip a
1: lo: <LOOPBACK,UP,LOWER_UP> mtu 65536
link/loopback 00:00:00:00:00:00 brd 00:00:00:00:00:00
inet 127.0.0.1/8 scope host lo

2: enp4s0: <BROADCAST,MULTICAST,UP,LOWER_UP> mtu 1500
link/ether 1a:8a:1c:ff:25:15 brd ff:ff:ff:ff:ff:ff
inet 192.168.1.99/24 brd 192.168.1.255 scope global enp4s0
\end{mycode}

Para ver la ruta por defecto (el gateway o puerta de enlace).

\begin{mycode}{Obtener la puerta de enlace}{console}{}
ruben@vega:~$ ip route show default
default via 192.168.1.1 dev enp4s0 onlink
\end{mycode}

Ver los puertos TCP y servicios que están a la escucha en nuestro servidor
\begin{mycode}{Listar los puertos TCP a la escucha}{console}{}
root@vega:~# ss -pntal
\end{mycode}

\section{Comandos sobre procesos}
Listar todos los procesos
\begin{mycode}{Listar todos los procesos}{console}{}
root@vega:~# ps aux
\end{mycode}

Listar todos los procesos en forma de árbol (para saber de quién son hijos)
\begin{mycode}{Listar todos los procesos en forma de árbol}{console}{}
root@vega:~# pstree -p
\end{mycode}

Matar un proceso (donde PID es el identificador del proceso).
\begin{mycode}{Matar un proceso}{console}{}
root@vega:~# kill -9 PID
\end{mycode}

\section{Estado de la carga y memoria del servidor}
Para ver los procesos y su estado por consumo de CPU, RAM…
\begin{mycode}{Ver el estado del servidor}{console}{}
root@vega:~# top
\end{mycode}

Para ver los procesos y su estado por consumo de CPU, RAM… es necesario instalar este paquete
\begin{mycode}{Ver el estado del servidor}{console}{}
root@vega:~# htop
\end{mycode}

\section{Comandos sobre servicios (systemd/systemctl)}
GNU/Linux cuenta con un sistema unificado (\textbf{systemd}) para administrar el sistema y los servicios que tenemos en nuestro servidor. Dado que es una pieza fundamental en el sistema operativo, debemos de conocer ciertos comandos para poder desempeñar tareas con él.

Listar todos los servicios/unidades
\begin{mycode}{Listar todos los servicios}{console}{}
root@vega:~# systemctl
\end{mycode}

Comprobar si algún servicio ha fallado
\begin{mycode}{Comprobar servicios que han fallado}{console}{}
root@vega:~# systemctl --failed
\end{mycode}

Comprobar el estado de un servicio concreto (en este caso, ssh)
\begin{mycode}{Comprobar servicios que han fallado}{console}{}
root@vega:~# systemctl status ssh
\end{mycode}

Parar un servicio concreto
\begin{mycode}{Parar un servicio concreto}{console}{}
root@vega:~# systemctl stop ssh
\end{mycode}

Arrancar un servicio concreto
\begin{mycode}{Arrancar un servicio concreto}{console}{}
root@vega:~# systemctl start ssh
\end{mycode}

Ver los logs de todo el sistema
\begin{mycode}{Ver los logs del sistema}{console}{}
root@vega:~# journalctl
\end{mycode}

Ver los logs de un servicio concreto (en este caso, ssh)
\begin{mycode}{Ver los logs del sistema}{console}{}
root@vega:~# journalctl -u ssh
\end{mycode}

Ver los logs del kernel
\begin{mycode}{Ver los logs del kernel}{console}{}
root@vega:~# journalctl -k
\end{mycode}
