\documentclass{../../../yukibook.cls/yukibook}

\begin{document}

\yukibook{Sistemas Operativos \linebreak en red} 	% Title
  {Rubén Gómez}  % Author
  {2021-2022}    % Year
  {Técnico en Sistemas microinformáticos y redes} % Name of degree
  {\textquote{A very good phrase for a very good book}}	% catch phrase
  {The phrase's author}	% the phrase's author
  {img/cover.png}

%--------------------------------------------------------------------------
% Start your parts, chapters and sections here
%--------------------------------------------------------------------------

%\part{Part 1}

\chapter{Introducción a GNU/Linux}
\section{Un poco de historia}
Para conocer cómo nació el movimiento GNU y el kernel Linux debemos conocer un poco de historia de la informática y cómo evolucionó en los primeros años.

\subsection{El nacimiento de Unix}

\begin{description}
\item[1964-1969]Los laboratorios \textbf{Bell} empiezan un proyecto con el \textbf{MIT} (Instituto Tecnológico de Massachusetts) y \textbf{General Electric} para desarrollar un sistema de \textbf{tiempo compartido} (“time-sharing computing”): se llamaría \textbf{Multics} (Multiplexed Information and Computing Service).

Hasta este momento, los sistemas utilizados eran de un único proceso, la CPU no era compartida por múltiples procesos sino que se ejecutaba por lotes (se les mandaba los procesos a ejecutar y se ejecutaban en orden).

Multics obtuvo licencia libre en el 2007. En Diciembre del 2016 salió la última versión 12.6f.

\itemimage{1969}{r}{0.33}
  {img/Ken_Thompson_and_Dennis_Ritchie--1973.jpg}
  {\href{https://en.wikipedia.org/wiki/Ken_Thompson}{Ken y Dennis . Origen: Wikipedia}}
  {
  Uno de los desarrolladores de Multics, \href{https://en.wikipedia.org/wiki/Ken_Thompson}{Ken Thompson}, decidió escribir su propio sistema operativo. Ken Thompson es conocido también por crear el lenguaje de programación \textbf{B}, el sistema de codificación de caracteres UTF-8 y el lenguaje de programación Go, entre otras cosas.

A Ken Thompson se le une \href{https://en.wikipedia.org/wiki/Dennis_Ritchie}{Dennis Ritchie} y otros, y empiezan a programar un sistema de ficheros jerárquico, el concepto de procesos de computación, ficheros de dispositivos, un intérprete de comandos, … El resultado de lo programado era más pequeño y simple que Multics, lo que se convertiría en Unix. En Agosto ya tendrían el sistema operativo, se auto-gestiona,  tenía un assembler, un editor y una shell de comandos.

Dennis Ritchie es conocido también por crear junto con Ken el lenguaje de programación \textbf{C} (aparece por primera vez en 1972).
}


\item[1970]En ese momento el nuevo sistema operativo se llamaba \textbf{Unics} (\textit{Uniplexed Information and Computing Service}, un juego de palabras en contraposición a  Multics). No tenían todavía dinero de la organización en el desarrollo (era desarrollado por los programadores) y tampoco era multitarea todavía.

A finales de año el sistema ya era conocido como \textbf{UNIX}, y se había portado a la máquina PDP-11.

\textbf{Las primeras versiones de Unix incluían el código fuente} para que las universidades lo pudiesen modificar y así poder extenderlo a sus necesida des.


\item[1971]El sistema se empieza a hacer complejo y como querían que más usuarios lo usasen, crean el sistema de manuales que es utilizado hoy en día (mediante el comando \textbf{"man"}).

\begin{center}
  \includegraphics[width=0.8\linewidth]{img/Ken_Thompson_(sitting)_and_Dennis_Ritchie_at_PDP-11_(2876612463).jpg}
  \vspace{-10pt}\captionof{figure}{\href{https://en.wikipedia.org/wiki/Ken_Thompson}{Dennis Ritchie y Ken Thompson. Origen: Wikipedia}}\vspace{-13pt}
\end{center}


\item[1973]La versión 4 del sistema es reescrita completamente en C. Hasta este momento el sistema había estado escrito en ensamblador, por lo que no era portable entre distintos tipos de máquinas, aunque la primera versión portada a otra plataforma fue en 1978. Se cree que había “más de 20” instalaciones del sistema.

\item[1974]La versión 5 se licencia para ser utilizada en \textbf{instituciones educativas}.

\item[1975]La versión 6 se licencia para poder ser utilizadas por empresas por \$20.000 de la época.

\item[1977]La universidad de Berkeley lanza su primera versión de Unix bajo la Berkeley Software Distribution (BSD).

\item[1979]Con la salida de Unix v7, se comienza a portar a los distintos ``microordenadores'' de la época y a los distintos microprocesadores (Motorola 68000, Intel 8086, … ).

\item[1980]Microsoft anuncia su primer Unix para microcomputadoras de 16 bits (Xenix).
\end{description}

\subsection{El nacimiento de GNU (GNU's Not Unix}
\begin{description}

\item[1971]\href{https://en.wikipedia.org/wiki/Richard_Stallman}{Richard Stallman} comienza su carrera en el MIT en el laboratorio de inteligencia artificial.

Es conocido no sólo por el movimiento GNU, si no también por crear GCC y Emacs entre otra gran cantidad de software.

En esa época el software se distribuía de manera abierta para poder ser modificado. Lo habitual era realizar modificaciones para mejorar el software y distribuirlo entre compañeros y universidades.

\itemimage{1982}{r}{0.25}
  {img/Richard_Stallman_2016_Talk_in_Madrid_06.jpg}
  {\href{https://commons.wikimedia.org/wiki/File:Richard_Stallman_2016_Talk_in_Madrid_06.jpg}{Richard Stallman: Wikimedia}}
  {
Richard Stallman quiere modificar el firmware de unas impresoras y el fabricante le pide que firme un acuerdo de no divulgación si le enseñan el código. Esto hace que Stallman se enfurezca y es cuando decide que la situación actual debe cambiar y volver al sistema de intercambio de software anterior.

\item[1983] Se anuncia el nacimiento del proyecto \textbf{GNU}, cuya finalidad es la de construir un sistema operativo completamente libre, compatible con Unix. La idea es dar a los usuarios la libertad y el control de sus ordenadores.

\item[1985] Se lanza el \href{https://www.gnu.org/gnu/manifesto.es.html}{manifiesto GNU}, y ya cuenta con un editor de texto (Emacs), compilador de C, una shell, varias utilidades … El núcleo inicial todavía no es funcional.
}


\item[1986]
Richard Stallman escribe y publica la definición de lo que es Free Software (Software Libre) a través de la \href{https://es.wikipedia.org/wiki/Free_Software_Foundation}{Free Software Foundation}.

\begin{tcolorbox}[title=Aclarando la palabra “free”:,sidebyside,righthand width=0.12\linewidth]

\textbf{The word “free” in our name does not refer to price; it refers to freedom.}

La palabra “free” no se refiere a gratis, si no que se refiere a libertad.

\tcblower
\includegraphics[width=\linewidth]{img/gnu.png}
\end{tcolorbox}

Más adelante veremos a qué se refiere sobre libertad en el software.

\end{description}

\subsection{El nacimiento de Minix}
\begin{description}
\item[1987]Andrew S. Tanenbaum crea  Minix como propósito educativo y para enseñar cómo funciona un sistema operativo.

\item[1991]Sale la versión 1.5 de Minix y es portada a distintas arquitecturas (IBM, Motorola 68000, Amiga, Apple Macintosh, …).

\item[1992]Debate con Linus Torvalds sobre la arquitectura del kernel Linux (núcleo monolítico) en lugar de usar un micronúcleo.

\end{description}


\subsection{El nacimiento de Linux}
\begin{description}

\item[1991] Un estudiante en la universidad de Helsinki, \href{https://en.wikipedia.org/wiki/Linus_Torvalds}{Linus Torvalds}, comienza un proyecto personal escrito para su nuevo ordenador, un PC con procesador 80386.

El desarrollo comienza bajo \textbf{Minix}, usando el compilador \textbf{GCC} del movimiento GNU (GCC = GNU Compiler Collection).

El proyecto termina convirtiéndose en un kernel de un sistema operativo y escribió al grupo de noticias de Minix diciendo:

\begin{tcolorbox}[title=Email de Linus Torvalds presentando Linux,sidebyside,righthand width=0.30\linewidth]
  “Hola a todos los que estáis ahí fuera usando minix.\\


  Estoy haciendo un sistema operativo (libre), (solamente por aficion, no será grande ni profesional como el GNU) para clones 386(486) AT.

  ...

  PD. Sí – está libre de cualquier código de minix, y tiene un sistema de ficheros multi-hilo. NO es portable (usa el cambio de tareas del 386 etc), y probablemente nunca soporte otra cosa que no sean los discos duros AT, porque es todo lo que tengo :-(. ”
  \tcblower
  \includegraphics[width=\linewidth]{img/Linus_Torvalds.jpeg}
  \vspace{-30pt}\captionof{figure}{\href{https://en.wikipedia.org/wiki/Linus_Torvalds}{Linus torvalds. Origen: Wikipedia}}
\end{tcolorbox}


\item[1992] Originalmente la licencia de Linux era propia e impedía el uso comercial de Linux. En la versión 0.99 esto cambia y se cambia a la licencia GNU Public License (\textbf{GPL}).

\item[1993] El proyecto cuenta con más de 100 desarrolladores. El kernel se adapta al entorno del proyecto GNU. Nace la distribución \textbf{Debian} (una de las más importantes a día de hoy)

\begin{center}
  \includegraphics[width=0.5\linewidth]{img/debian-logo.jpg}
  \vspace{-10pt}\captionof{figure}{\href{https://www.debian.org}{Debian}}
\end{center}

\item[1994] Se libera la versión 1.0. El proyecto XFree86 se une y Linux consigue interfaz gráfico. Nacen las primeras distribuciones comerciales \textbf{Red Hat} y \textbf{Suse}.

\item[1998] Empresas como \textbf{IBM}, \textbf{Compaq} y \textbf{Oracle} anuncian que apoyan a Linux. Nace el interfaz gráfico \textbf{KDE}.

\item[1999] Nace el interfaz gráfico \textbf{GNOME} como reemplazo a KDE, ya que KDE hacía uso de una librería propietaria en aquel momento (QT).

\item[2001] Steve Ballmer (CEO de Microsoft) dice: \textbf{“Linux es un cáncer”}.

\item[2002] Se libera OpenOffice (originalmente suite ofimática de Sun Microsystems). Nace Mozilla (hoy día:  Firefox).

\item[2003] IBM lanza un anuncio para la Linux Foundation: \href{https://www.youtube.com/watch?v=x7ozaFbqg00}{https://www.youtube.com/watch?v=x7ozaFbqg00}

\item[2004] Nace \textbf{Ubuntu} (basándose en Debian) y Steve Ballmer (CEO de Microsoft) dice que Linux infringe muchas de sus patentes.

\item[2008] Nace \textbf{\href{https://es.wikipedia.org/wiki/Android}{Android}}, sistema operativo con kernel Linux. Actualmente es el sistema operativo de móviles que más terminales tiene.

\item[2009] Red Hat iguala a Sun Microsystem en capitalización bursátil (un gran logro simbólico).

\item[2014] Satya Nadella (CEO de Microsoft) muestra en una presentación la siguiente transparencia:

\begin{center}
  \includegraphics[width=0.5\linewidth]{img/Microsoft_Linux.jpg}
  \vspace{-10pt}\captionof{figure}{\href{https://commons.wikimedia.org/wiki/File:Microsoft_Linux.jpg}{Origen: Wikipedia}}
\end{center}


\item[2016]
Microsoft anuncia \href{https://es.wikipedia.org/wiki/Windows_Subsystem_for_Linux}{WSL} (\textit{Windows Subsystem for Linux}) y se puede instalar en Windows 10 y Windows Server 2019. Permite correr ejecutables de Linux nativamente.

\end{description}

\subsection{Cronograma de sistemas Unix}
En el siguiente cronograma se puede ver la línea temporal de los sistemas Unix:

\begin{center}
  \includegraphics[width=0.7\linewidth]{img/Evolución_UNIX.png}
  \vspace{-10pt}\captionof{figure}{\href{https://commons.wikimedia.org/wiki/File:Evolución_UNIX.png}{Origen: Wikipedia}}
\end{center}

\section{Resumen}
Linux es conocido como un sistema operativo libre pero el nombre de Linux se  centra única y exclusivamente en el \textbf{kernel} (o \textbf{núcleo}) del sistema operativo.

El sistema operativo completo debería llamarse \textbf{GNU/Linux}, ya que el kernel es una “pequeña” parte (aunque muy importante) dentro de todo el sistema operativo. El resto de herramientas utilizadas en el sistema operativo pertenecen al proyecto GNU.


\chapter{Licencias Libres}
\section{Software Libre}

En 1986 Richard Stallman saca a la luz la definición de lo que es Free Software (Software Libre) a través de la \href{https://es.wikipedia.org/wiki/Free_Software_Foundation}{Free Software Foundation}:

\begin{tcolorbox}[title=Aclarando la palabra “free”:,sidebyside,righthand width=0.12\linewidth]

    \textbf{The word “free” in our name does not refer to price; it refers to freedom.}

    La palabra “free” no se refiere a gratis, si no que se refiere a libertad.

    \tcblower
    \includegraphics[width=\linewidth]{img/gnu.png}
\end{tcolorbox}


Las libertad en el software se refiere a:
\begin{tcolorbox}[title=Libertades del Software Libre:]
    \begin{enumerate}
        \setcounter{enumi}{-1}
        \item La libertad de ejecutar el programa, para cualquier propósito .

        \item La libertad de estudiar cómo trabaja el programa, y cambiarlo para que haga lo que usted quiera. El acceso al código fuente es una condición necesaria para ello.

        \item La libertad de redistribuir copias para que pueda ayudar al prójimo.

        \item La libertad de mejorar el programa y publicar sus mejoras, y versiones modificadas en general, para que se beneficie toda la comunidad. El acceso al código fuente es una condición necesaria.
    \end{enumerate}
\end{tcolorbox}

El movimiento del Free Software es un movimiento que tiene que ver más con la filosofía y la ética que con la tecnología en sí misma.


\subsection{Copyleft y GNU Public License (GPL)}
Es una práctica legal que consiste en el ejercicio del derecho de autor (copyright en inglés) con el objetivo de propiciar el libre uso y distribución de una obra, exigiendo que los concesionarios preserven las mismas libertades al distribuir sus copias y derivados (\href{https://es.wikipedia.org/wiki/Copyleft}{Wikipedia}).

\begin{center}
  \includegraphics[width=\linewidth]{img/Mapa_conceptual_del_software_libre.png}
  \vspace{-30pt}\captionof{figure}{\href{https://commons.wikimedia.org/wiki/File:Mapa_conceptual_del_software_libre.png}{Mapa conceptual del Software Libre: Wikipedia}}\vspace{-20pt}
\end{center}

Con esto nació la licencia GNU GPL, la cual permite al usuario final la libertad de usar, estudiar, compartir y modificar el software recibido. Tiene que quedar claro que un programa comercial puede ser Software Libre.

\subsection{Diferencias con el Open Source}
Los programas Open Source son aquellos que podemos ver el código fuente pero esto no quiere decir que podamos modificarlo o adaptarlo a nuestras necesidades.

El Open Source es menos restrictivo que el Software Libre y se puede decir que todo Software Libre es Open Source, pero no todo Open Source tiene por qué ser libre.


\section{Licencias libres más conocidas}
Un listado de las licencias libres más utilizadas:

\begin{itemize}
    \item \href{https://es.wikipedia.org/wiki/GNU_General_Public_License}{GNU GPL}
    \item \href{https://es.wikipedia.org/wiki/Licencia_BSD}{BSD}
    \item \href{https://es.wikipedia.org/wiki/Licencia_MIT}{MIT}
    \item \href{https://es.wikipedia.org/wiki/Apache_License}{Licencia Apache}
    \item \href{https://es.wikipedia.org/wiki/Licencia_PHP}{Licencia PHP}
    \item \href{https://es.wikipedia.org/wiki/Licencias_Creative_Commons}{Creative Commons} (no todas las versiones). Más utilizadas en contenido multimedia.
\end{itemize}


\chapter{Sistema de ficheros en GNU/Linux}
El sistema de ficheros en GNU/Linux, al igual que en Unix, es jerárquico, comenzando en la raíz denominada “/”. Partiendo de esta raíz, el resto del sistema de ficheros nace en forma de ramificaciones generando lo que se denominan “rutas de ficheros”, que es el camino completo para llegar al mismo.

\section{Filesystem Hierarchy Standard}
Debido a que en GNU/Linux todo se representa como ficheros (discos, dispositivos, programas, … ) es necesario que exista un orden a la hora de ser almacenados. Con esa intención nace en 1993 el estándar de la jerarquía de ficheros de Linux, enfocado a reestructurar los archivos. Posteriormente se unieron otros derivados de UNIX (la comunidad de desarrollo de BSD) por lo que terminó adoptando el nombre FHS.

Aún siendo un estándar, no todas las distribuciones lo siguen al pie de la letra, y otros Unix, como MacOS, tienen sus propias rutas especiales.


\section{Directorios importantes}
A continuación se exponen los directorios más importantes del sistema junto con la descripción del contenido que deben de tener:
\begin{itemize}

    \item \textbf{/boot/}: archivos de arranque del kernel, normalmente junto con la configuración utilizada para compilarlos.
    \item \textbf{/dev/}: contiene archivos especiales de bloque que representan los dispositivos del hardware que está corriendo el sistema operativo
    \item \textbf{/etc/}: contiene los archivos de configuración del servidor y de los servicios que corren en él. Está subdividido en directorios por servicios o configuraciones.
    \item \textbf{/home/}: los directorios de trabajo de los usuarios normales del sistema
    \item \textbf{/lib/}: librerías que hacen funcionar a los programas
    \item \textbf{/root/}: es la home del usuario root
    \item \textbf{/var/}: archivos variables del sistema
    \begin{itemize}
      \item \textbf{/var/lib/}: aquí se suelen guardar los ficheros de los programas que “crecen”: bases de datos, ficheros caché…
      \item \textbf{/var/log/}: los logs del sistema
    \end{itemize}
\end{itemize}

Junto a todos estos directorios, se ha separado los lugares en los que van los binarios, o ejecutables de los programas. Lo habitual es que se encuentren en estas rutas:

\begin{itemize}
    \item \textbf{/bin/}: aplicaciones esenciales del sistema
    \item \textbf{/sbin/}: aplicaciones que en principio sólo debería ejecutar el usuario root o programas de administración del sistema
    \item \textbf{/usr/bin/}: ejecutables de usuario
    \item \textbf{/usr/sbin/}: ejecutables de superusuario
\end{itemize}
Aunque las rutas de los ejecutables denotan quién debería ejecutar el programa, en la vida real no tiene por qué ser una limitación.

\section{Dispositivos de almacenamiento y discos duros}
En sistemas operativos Windows es habitual que cada partición cuente con una letra para acceder a ella, al igual que ocurre cuando introducimos un dispositivo de almacenamiento externo (un pendrive).

Tal como se ha comentado, en sistemas Unix el sistema de ficheros es una jerarquía, y por tanto todo dispositivo de almacenamiento nuevo deberá estar montado bajo la raíz “/”. Hoy día, en distribuciones con escritorio, al introducir un pendrive éste es auto-montado (es accesible) desde la ruta \textbf{/media/}, donde aparecerán tantos directorios como discos hayamos conectado.

\subsection{Almacenamiento permanente}
Si queremos que un disco duro nuevo sea permanente en nuestro sistema, podremos montarlo en cualquier lugar de la estructura jerárquica. Debido a este sistema, el usuario final no se tendrá que preocupar en almacenar los ficheros en una ruta distinta, si no que será el administrador el que haya hecho que esa ruta ahora pertenezca a un disco duro nuevo.

Imaginemos que el sistema operativo se ha instalado en un disco duro pequeño de 32Gb de espacio y se está llenando, y el directorio que más ocupa es el directorio de los usuarios. Podremos añadir al servidor un nuevo disco duro montado en /home y por tanto a partir de ahora los datos guardados en /home estarán en un nuevo disco duro más grande.

\begin{mycode}{Ejemplo de discos en un sistema con ``lsblk''`}{console}{}
root@vega:~# lsblk
NAME                       MAJ:MIN RM   SIZE RO TYPE MOUNTPOINTS
sda                          8:0    0   1,8T  0 disk
└─sda1                       8:1    0   1,8T  0 part /home/backup

sdb                          8:16   0   3,6T  0 disk
└─sdb1                       8:17   0   3,6T  0 part /home/disco4tb
sdc                          8:32   0 447,1G  0 disk
├─sdc1                       8:33   0   529M  0 part
├─sdc2                       8:34   0   100M  0 part
├─sdc3                       8:35   0    16M  0 part
└─sdc4                       8:36   0 446,5G  0 part
nvme0n1                    259:0    0 931,5G  0 disk
├─nvme0n1p1                259:1    0   512M  0 part
└─nvme0n1p2                259:2    0   800G  0 part /home
nvme1n1                    259:3    0 931,5G  0 disk
├─nvme1n1p1                259:4    0   512M  0 part /boot/efi
├─nvme1n1p2                259:5    0    90G  0 part /
├─nvme1n1p3                259:6    0   300G  0 part
│ ├─VMs-ubuntu--20.04--so1 254:0    0    10G  0 lvm
│ ├─VMs-manjaro            254:2    0    20G  0 lvm
│ └─VMs-win10              254:3    0    35G  0 lvm
└─nvme1n1p4                259:7    0 156,2G  0 part
\end{mycode}


\chapter{Gestión de usuarios locales en GNU/Linux}
En las distribuciones GNU/Linux lo habitual suele ser que existan al menos dos usuarios tras una instalación:

\begin{itemize}
    \item \textbf{root}: usuario administrador o súper usuario.
    \item \textbf{usuario no-privilegiado}: durante la instalación de la distribución nos suele preguntar para crear un usuario del sistema, que no tendrá privilegios.
\end{itemize}


El usuario root, como se ha dicho previamente, es el administrador del sistema, tiene permisos para realizar cualquier tarea dentro de nuestro sistema: instalar paquetes, desinstalarlos, modificar cualquier fichero, realizar formateos... Por lo tanto, el \textbf{realizar tareas como usuario root puede ser peligroso si cometemos algún fallo}.

Las buenas prácticas nos dicen que las tareas cotidianas del sistema deberíamos realizarlas como usuario normal y \textbf{sólo convertirnos en root cuando sea estrictamente necesario}.

\section{Creación de usuarios locales}

Tras instalar el sistema, veremos que se nos han creado varios usuarios en el sistema, aparte del usuario \textbf{root} y el usuario \textbf{no-privilegiado}. Para poder ver los usuarios que existen en nuestro sistema podemos verlo en el fichero  \configfile{ /etc/passwd }  o podríamos obtener un listado ejecutando el siguiente comando:

\begin{mycode}{Listar usuarios del sistema}{console}{}
root@vega:~# cut -d: -f1 /etc/passwd
\end{mycode}

Para crear un usuario:

\begin{mycode}{Crear usuarios del sistema}{console}{\small}
root@vega:~# adduser mikeldi

Añadiendo el usuario `mikeldi' ...
Añadiendo el nuevo grupo `mikeldi' (1001) ...
Añadiendo el nuevo usuario `mikeldi' (1001) con grupo `mikeldi' ...
Creando el directorio personal `/home/mikeldi' ...
Copiando los ficheros desde `/etc/skel' ...
Nueva contraseña:
Vuelva a escribir la nueva contraseña:
passwd: contraseña actualizada correctamente
Cambiando la información de usuario para mikeldi
Introduzca el nuevo valor, o pulse INTRO para usar el valor predeterminado
    Nombre completo []:
    Número de habitación []:
    Teléfono del trabajo []:
    Teléfono de casa []:
    Otro []:
¿Es correcta la información? [S/n]
\end{mycode}

Y la línea que nos creará en el fichero  \configfile{ /etc/passwd }   es:
\begin{tcolorbox}[colback=white,title=Ejemplo de usaurio en “/etc/passwd”]
 \mintinline{console}{ mikeldi:x:1001:1001:mikeldi,,,:/home/mikeldi:/bin/bash }
\end{tcolorbox}

El fichero \configfile{ /etc/passwd }  nos muestra los datos de los usuarios, siendo un fichero que tiene distintos datos separados por “:”, siendo cada apartado:

\begin{center}
  \includegraphics[width=0.7\linewidth]{img/usuario_tabla.png}
\end{center}


En las primeras versiones GNU/Linux la contraseña de los usuarios aparecía en el propio fichero /etc/passwd, lo que suponía un problema en la seguridad, ya que no estaban cifradas. Actualmente, las contraseñas de los usuarios se almacenan cifradas en el fichero \configfile{ /etc/shadow }. El fichero es similar al passwd, estando separados los apartados por “:”


\begin{center}
  \includegraphics[width=0.7\linewidth]{img/shadow_tabla.png}
\end{center}


En el apartado de la contraseña podemos saber cierta información acerca de la misma ya que tiene el siguiente formato: \textbf{“\$id\$salt\$hashed”}
\begin{itemize}
    \item \textbf{id}: el algoritmo utilizado para cifrar la contraseña
    \begin{itemize}
        \item \$1\$ – MD5
        \item \$2a\$ – Blowfish
        \item \$2y\$ – Eksblowfish
        \item \$5\$ – SHA-256
        \item \$6\$ – SHA-512
    \end{itemize}
\end{itemize}

Aparte, también podemos encontrarnos con:
\begin{itemize}
    \item Contraseña vacía:  Si no hay contraseña, al pedirnos la contraseña a la hora de hacer login será suficiente con pulsar “intro”.
    \item \textbf{!}, \textbf{*}: la cuenta está bloqueada para la contraseña. El usuario no podrá loguearse utilizando la contraseña. Resulta útil si queremos bloquear el acceso con contraseña pero no con otros métodos (clave pública SSH).
    \item \textbf{*LK*}: cuenta bloqueda. El usuario no podrá loguearse.
    \item \textbf{*NP*}, \textbf{!!}: Nunca se ha puesto una contraseña
\end{itemize}


\section{Gestión de grupos}
En algunas distribuciones GNU/Linux, al crear un usuario directamente nos crea un grupo para el nuevo usuario. En otras, el usuario pertenece al grupo “users”.

Para saber los grupos a los que pertenece un usuario podemos ejecutar el comando \commandbox{ groups }. Los grupos del sistema aparecen en el fichero \configfile{ /etc/group }, y al igual que los ficheros vistos previamente, están separados por “\textbf{:}”.

\begin{center}
  \includegraphics[width=0.6\linewidth]{img/grupo_tabla.png}
\end{center}

\section{Permisos de ficheros}
En GNU/Linux los ficheros cuentan con 3 tipos de permisos:
\begin{itemize}
    \item lectura (\textbf{r}ead): el usuario puede leer el fichero
    \item escritura (\textbf{w}rite): el usuario puede escribir en el fichero
    \item ejecución (e\textbf{x}ecute): el usuario puede el fichero o puede ver el contenido de un directorio
\end{itemize}


Todos ello para los distintos usuarios que pueden existir en el sistema:
\begin{itemize}
    \item \textbf{dueño del fichero}: la persona que ha creado el fichero
    \item \textbf{grupo}: los usuarios pertenecientes al grupo al que pertenece el fichero tendrán ciertos privilegios
    \item \textbf{el resto de usuarios}: los permisos que tendrán el resto de usuarios que no son ni el dueño ni pertenecen al grupo
\end{itemize}

Todo ello se puede visualizar en el sistema de ficheros si listamos los permisos del fichero:

\begin{mycode}{Ver los permisos de un fichero}{console}{}
mikeldi@vega:~$ ls -lh fichero.txt
-rw-r--r-- 1 mikeldi mikeldi 0 dic  8 19:17 fichero.txt
\end{mycode}

Los permisos se pueden ver en los primeros 10 caracteres:

\begin{center}
  \includegraphics[width=0.7\linewidth]{img/permisos_fichero.png}
\end{center}

Existen los distintos tipos de ficheros:
\begin{itemize}
    \item \textbf{-} : fichero normal
    \item \textbf{d} : directorio
    \item \textbf{b} : dispositivo de bloque (ejemplo: /dev/sda*)
    \item \textbf{c} : dispositivo de carácter (las consolas. ejemplo: /dev/tty*)
    \item \textbf{s} : socket local
    \item \textbf{p} : tubería (pipe)
    \item \textbf{l} : enlace simbólico (link)
\end{itemize}

\subsection{Permisos especiales}

Existen otros permisos especiales:
\begin{itemize}
    \item \textbf{SUID}: permiso especial que permite que el fichero sea ejecutado con los permisos del dueño del fichero (aunque lo ejecute otro usuario). Se visualiza con una “S” en el permiso de ejecución del dueño  \texttt{-rwSrw-r- -} .
%% TODO: modificar los fondos de los permisos
    \item \textbf{SGID}: permiso especial que permite que el fichero sea ejecutado como el grupo. Aparece una “S” en el permiso de ejecución del grupo: \texttt{-rwx- -S- - -} .
    \item \textbf{STICKY}: si el bit sticky está activado en un directorio sólo el usuario root, el dueño del directorio o el dueño del fichero puede borrar ficheros de dicho directorio. Aparece una “t” en el permiso de ejecución del resto de usuarios:   \texttt{d-rwx-rx-r-t} .

\end{itemize}

\subsection{Cambiando permisos y dueños a los ficheros y a los directorios}

Para cambiar los permisos a los ficheros y a los directorios se hace con el comando \textbf{chmod}.

Para cambiar permisos de dueño a los ficheros y a los directorios se hace con el comando \textbf{chown}.

\section{La importancia de “sudo”}
En muchas distribuciones GNU/Linux el usuario no-privilegiado que se crea tiene permiso de “sudo” para poder ejecutar comandos como si se tratara del administrador/\textbf{root} (u otro usuario) para poder realizar tareas de administración. Es habitual que en estas distribuciones \textbf{el usuario root no suela tener contraseña}. 

Por lo tanto, cuando un usuario necesita realizar una tarea como administrador, deberá ejecutar el siguiente comando:

\begin{mycode}{Editar un fichero con permisos de root}{console}{}
mikeldi@vega:~$ sudo nano /etc/passwd
\end{mycode}

Tras realizar este comando, el sistema nos pedirá la contraseña del usuario con el que lo estemos ejecutando y comprobará que el usuario tiene permisos de “sudo” para poder ejecutar el comando (en este caso: nano). 

El comando “\textbf{sudo}” viene de “\textbf{su}per user \textbf{do}” (que en inglés sería: super usuario haz”), y aunque su uso habitual es el de permitir realizar cualquier comando de administración, la configuración permite mucho más, pudiendo permitir a ciertos usuarios sólo realizar ciertas tareas. Por ejemplo:

\begin{itemize}
    \item Usuario \textbf{mikeldi}: tendría permisos para poder realizar cualquier comando del sistema.
    \item Usuario \textbf{dba}: sólo tendría permisos para poder realizar el reinicio del sistema de base de datos.
    \item Usuario \textbf{adminweb}: sólo tendría permisos para poder realizar el reinicio del servidor web.
    \item ....
\end{itemize}

De esta manera, la gestión de nuestro servidor estaría basada en múltiples usuarios y cada usuario sólo sería capaz de realizar pequeñas tareas, por lo que la seguridad del servidor sería mayor y limitaría lo que los usuarios puedan realizar.

\subsection{Configurando “sudoers”}
Los permisos de sudo se realizan en el fichero  \configfile{/etc/sudoers} , y para su edición se hace uso del comando \textbf{visudo}, el cual abre el fichero y se asegura que a la hora de guardar la sintaxis es correcta.

Si realizamos cualquier modificación sobre el fichero, éste será tenido en cuenta la próxima vez que se realice la ejecución del comando “sudo”, por lo tanto, no hay que realizar ningún reinicio de servicio.

El fichero \configfile{/etc/sudoers}  tiene permisos de sólo lectura para el usuario root y el grupo root:

\begin{mycode}{Permisos del fichero \faFile \hspace{1pt} /etc/sudoers}{console}{}
root@vega:# ls -lh /etc/sudoers
-r--r----- 1 root root 669 jun  5  2017 /etc/sudoers
\end{mycode}

Un fichero sudoers suele tener el siguiente aspecto:

\begin{mycode}{Contenido del fichero \faFile \hspace{1pt} /etc/sudoers}{bash}{\footnotesize}
Defaults    env_reset
Defaults    mail_badpass
Defaults    secure_path="/usr/local/sbin:/usr/local/bin:/usr/sbin:/usr/bin:/sbin:/bin"

# User privilege specification
root    ALL=(ALL:ALL) ALL

# Allow members of group sudo to execute any command
%sudo   ALL=(ALL:ALL) ALL

# See sudoers(5) for more information on "#include" directives:

#includedir /etc/sudoers.d
\end{mycode}

La línea que más importa en este fichero es la que indica “\textbf{\%sudo   ALL=(ALL:ALL) ALL}” y es explicada en su comentario anterior. Lo que quiere decir es que cualquier usuario que pertenezca al grupo “sudo” podrá realizar cualquier comando del sistema como superusuario. La sintaxis de la línea es:

\begin{itemize}
    \item \textbf{\%sudo}:  cualquier usuario que pertenezca al grupo “sudo”
    \item \textbf{ALL}= : desde cualquier host o IP
    \item \textbf{(ALL:ALL)}: el usuario que ejecuta puede ejecutar el comando como cualquier usuario y cualquier grupo
    \item \textbf{ALL}: puede ejecutar cualquier comando
\end{itemize}

Un ejemplo limitando el uso de sudo a un único comando a un usuario:

\begin{mycode}{Añadimos configuración al fichero \faFile \hspace{1pt} /etc/sudoers}{bash}{}
ruben    ALL=(ALL:ALL) NOPASSWD:/bin/systemctl suspend
\end{mycode}

Con esta línea lo que estamos permitiendo es que el usuario “ruben” puede ejecutar el comando “/bin/systemctl suspend” (suspender el equipo) y sin necesidad de meter contraseña al hacer sudo, gracias a la opción “NOPASSWD”).

\section{Diferencias entre “sudo”, “su” y “su -”}
Como ya se ha comentado en el apartado anterior, “sudo” permite la ejecución de comandos como cualquier usuario, siendo lo habitual ejecutarlo como root. Ahora bien, en entornos donde el usuario root tiene contraseña, nos puede interesar convertirnos en él para realizar tareas sin tener que estar ejecutando “sudo” a cada comando. Al ser root, tendremos que tener especial cuidado.

\section{Variables de entorno}
En cualquier sistema operativo existen las denominadas “variables de entorno”. Son variables que cada usuario tiene y sirven para indicar ciertos parámetros que se están utilizando (la SHELL que se está usando), o parámetros que se van a usar a la hora de ejecutar comandos o realizar tareas, ya que se consultan a ellas. En GNU/Linux las variables de entorno se pueden consultar ejecutando:

\begin{mycode}{Vemos las variables de entorno del usuario ruben}{console}{}
ruben@vega:~$ printenv
LANG=es_ES.utf8
LOGNAME=ruben
XDG_VTNR=2
COLORTERM=truecolor
PWD=/home/ruben
DESKTOP_SESSION=gnome
USERNAME=ruben
SHELL=/usr/bin/zsh
PATH=/usr/local/bin:/usr/bin:/bin:/usr/local/games:/usr/games
...
\end{mycode}

Una variable de entorno puede consultarse haciendo:

\begin{mycode}{Consultamos el contenido de la variable \$PATH}{bash}{}
ruben@vega:~$ echo $PATH
/usr/local/bin:/usr/bin:/bin:/usr/local/games:/usr/games

\end{mycode}

Como se puede ver, es con un “\textbf{\$}” y el nombre de la variable en mayúsculas. Existen muchas variables de entorno, y podríamos crear las nuestras propias si así lo necesitáramos.

\section{La importancia de “su -”}
Con el comando “\textbf{su}” nos podemos convertir en cualquier otro usuario del sistema siempre y cuando \textbf{conozcamos su contraseña}. Hay que notar la diferencia respecto a “\textbf{sudo}” que cuando lo ejecutamos nos pide \textbf{nuestra contraseña}.

\textbf{Al ejecutar “su” nos convertimos en el usuario root} (o ejecutando “su usuario2”, nos convertimos en el usuario2), \textbf{pero no hacemos uso de sus variables de entorno}, si no que seguimos  con las variables de entorno del usuario que éramos previamente.
Para convertirnos en el usuario y que obtengamos sus variables de entorno es necesario ejecutar “\textbf{su -}”, o lo que es lo mismo: “\textbf{su -l}”, que el manual nos dice: “\textit{Start the shell as a login shell with an environment similar to a real login}”. Por ejemplo: 

\begin{mycode}{Consultamos el contenido de la variable \$PATH en distintas situaciones}{bash}{}
ruben@vega:~$ echo $PATH
/usr/local/bin:/usr/bin:/bin:/usr/local/games:/usr/games

ruben@vega:~$ su
Contraseña:

root@vega:/home/ruben# echo $PATH
/usr/local/bin:/usr/bin:/bin:/usr/local/games:/usr/games

root@vega:/home/ruben# exit

ruben@vega:~$ su -
Contraseña:

root@vega:~# echo $PATH
/usr/local/sbin:/usr/local/bin:/usr/sbin:/usr/bin:/sbin:/bin

\end{mycode}

El usuario “ruben” tiene unos valores en la variable de entorno PATH (es la variable que se encarga de tener las rutas de los ejecutables de los programas). Al convertirse en root haciendo uso de “su”, y mirar la variable PATH, podemos observar que es igual que el usuario prueba.

Ahora bien, si a la hora de convertirse en root hace uso de “su -”, se puede ver cómo la variable PATH obtiene otros valores, siendo lo más significativo que aparecen las rutas “/usr/local/sbin” y “/usr/sbin” que son las rutas donde se almacenan los ejecutables que (en principio) sólo deberían ejecutarse como administrador del sistema.


\chapter{SAMBA}
\section{Introducción}
Samba es una implementación libre del protocolo de archivos compartidos que es ampliamente utilizado por los sistemas Microsoft Windows para sistemas de tipo UNIX. De esta forma, es posible que computadoras con GNU/Linux, Mac OS X o Unix en general se vean como servidores o actúen como clientes en redes de Windows.

Samba también permite validar usuarios haciendo de Controlador Principal de Dominio (PDC), como miembro de dominio e incluso como un dominio Active Directory para redes basadas en Windows.

Samba fue desarrollado originalmente para Unix por Andrew Tridgell utilizando un sniffer o capturador de tráfico para entender el protocolo usando \textbf{ingeniería inversa}.

\subsection{SMB/CIFS}
Server Message Block (\textbf{SMB}) y Common Internet File System (\textbf{CIFS}) son protocolos de red desarrollados para compartir archivos e impresoras entre nodos de una red. El protocolo SMB fue desarrollado originalmente por IBM y posteriormente ampliado por Microsoft y renombrado como CIFS.

Los términos SMB y CIFS son a menudo intercambiables pero hay características en la implementación de SMB de Microsoft que no son parte del protocolo SMB original. Sin embargo, desde una perspectiva funcional, ambos son protocolos utilizados por Samba.

La versión 1 del protocolo SMB no debería usarse y en Samba está deshabilitado por defecto debido a los ataques que hubo por el ransomware \href{https://es.wikipedia.org/wiki/Ataques_ransomware_WannaCry}{Wannacry}. Estos ataques utilizaban una vulnerabilidad en servidores que no estaban parcheados.

\section{Instalación}
Para instalar Samba haremos uso de los repositorios oficiales de nuestra distribución, y por tanto, podremos instalarlo haciendo uso del siguiente comando:

\begin{mycode}{Instalamos SAMBA y parte de las dependencias}{console}{}
root@vega:~# apt install samba cifs-utils smbclient winbind
\end{mycode}

Tras la instalación, nos habrá creado un directorio de configuración  \configdir{/etc/samba} , cuyo fichero de configuración principal es \configfile{/etc/samba/smb.conf} . Tras la instalación, tendremos un fichero de configuración estándar que no podrá realizar demasiado, ya que la configuración es escasa.

\section{Comprobar configuración}
Como ya se ha comentado, la configuración principal está en  \configfile{/etc/samba/smb.conf}  , con una configuración estándar, por lo que tendremos que adecuarla a nuestras necesidades.

Para asegurar que las modificaciones que hemos realizado son correctas, podremos hacer uso del comando:



\part{Anexos}

\graphicspath{{../../../anexos/instalar_ubuntu_lts/}}
\hypertarget{instalar_ubuntu_lts}{}

\chapter{Instalar Ubuntu 20.04 LTS}
En este anexo realizaremos la instalación de la distribución Ubuntu 20.04 LTS en su versión para servidores. En este anexo no se va a explicar cómo realizar la creación de una máquina virtual donde se aloja el sistema operativo, ya que existen distintos tipos de virtualizadores.

No se realizará una guía “paso a paso”, sino que se centrará en las partes más importantes de la instalación y en las que más dudas puedan surgir.

\section{Descargar Ubuntu 20.04}
La ISO la obtendremos de la \href{https://ubuntu.com/#download}{web oficial} y seleccionaremos la versión 20.04 LTS de Ubuntu Server. Esta ISO contendrá el sistema base de Ubuntu y nos guiará para realizar la instalación del sistema operativo.

Una vez descargada la ISO tendremos que cargarla en el sistema de virtualización elegido y arrancar la máquina virtual.


\section{Instalar Ubuntu 20.04}
Tras arrancar la máquina virtual nos aparecerá un menú para seleccionar el idioma durante la instalación y le daremos a “Instalar Ubuntu Server”.

\begin{center}
    \vspace{-10pt}
    \includegraphics[width=15cm]{ubuntu_1.png}
    \vspace{-20pt}
\end{center}

A partir de aquí comenzará el instalador y los pasos que nos aparecerán serán los siguientes (algunos de estos pasos puede que no estén 100\% traducidos al castellano):

\begin{enumerate}
    \item Elegir el idioma del sistema
    \item Actualización del instalador:
    \begin{itemize}
        \item Si la máquina virtual se puede conectar a internet, comprobará si existe una actualización del propio instalador de Ubuntu.
        \item Podemos darle a “Continuar sin actualizar”
    \end{itemize}
    \item Configuración del idioma del teclado
    \item Configuración de la red
    \item Configuración del proxy de red
    \item Configuración del “mirror” o servidor espejo desde donde descargarse los \hyperlink{paquete_de_software}{paquetes de software} para las actualizaciones posteriores.
    \item Selección del disco duro donde realizar la instalación
    \item Elegir el particionado de disco.
    \item Configuración del perfil. Introduciremos el nombre de usuario, el nombre del servidor y la contraseña del usuario que vamos a crear.
    \item Configuración de SSH Server. Aceptaremos que se instale el servidor SSH durante la instalación. En caso de no seleccionar esta opción, posteriormente podremos realizar la instalación.
    \item “Featured Server Snaps”. En esta pantalla nos permite instalar software muy popular en servidores.
\end{enumerate}


Una vez le demos a continuar, comenzará la instalación en el disco duro. Debido a que durante la instalación tenemos conexión a internet, el propio instalador se descarga las últimas versiones de los paquetes de software desde los repositorios oficiales.


Al terminar la instalación, tendremos que reiniciar la máquina virtual.

\section{Post-instalación}
Tras realizar el reinicio de la máquina virtual nos encontraremos con que el sistema arranca en el sistema recién instalado, y que tendremos que loguearnos introduciendo el usuario y la contraseña utilizadas en la instalación.

\subsection{Actualización del sistema}
Por si acaso, realizaremos la actualización del índice del repositorio, actualizaremos el sistema y en caso necesario realizaremos un nuevo reinicio:

\begin{mycode}{Actualizar Ubuntu}{console}{}
mikeldi@ubuntu:~$ sudo su
[sudo] password for mikeldi:
root@ubuntu:~# apt update
...
root@ubuntu:~# apt upgrade
...
\end{mycode}

Con estos comandos nos aseguramos que el sistema está actualizado a los últimos paquetes que están en el repositorio.


\hypertarget{configurar_ip_estatica_ubuntu}{}
\subsection{Poner IP estática}
Debido a la configuración de red de nuestro servidor, la IP está puesta en modo dinámica, esto quiere decir que nuestro equipo ha cogido la IP por configuración de DHCP de nuestra red. Debido a que un servidor debe de tener IP estática, tenemos que realizar la modificación adecuada para ponerle la IP estática que mejor nos convenga. Para ello editaremos el fichero de configuración situado en la siguiente ruta: \configfile{ /etc/netplan/00-installer-config.yaml }

Lo modificaremos para que sea parecido a (siempre teniendo en cuenta la IP y gateway de nuestra red):


\begin{mycode}{Configurando IP estática en Ubuntu}{yaml}{}
network:
  ethernets:
    enp1s0:
      dhcp4: no
      addresses:
      - 192.168.200.10/24
      gateway4: 192.168.200.1
      nameservers:
        addresses: [8.8.8.8]
  version: 2
\end{mycode}

El fichero de configuración que hemos modificado es de tipo \href{https://es.wikipedia.org/wiki/YAML}{YAML}, que es un formato de texto que suele ser utilizado en programación o en ficheros de configuración. Este tipo de ficheros tiene en cuenta los espacios para el uso de la identación, y no suele permitir el uso de tabuladores.

Para aplicar los cambios realizados en el fichero de configuración deberemos ejecutar el siguiente comando que aplicará los cambios:

\begin{mycode}{Aplicar configuración de IP}{console}{}
root@ubuntu:~# netplan apply
\end{mycode}

\clearpage

\graphicspath{{../../../anexos/ubuntu_raid1/}}
\chapter{Configurar RAID 1 durante la instalación de Ubuntu}

Tal como hemos podido ver anteriormente, durante la \hyperlink{instalar_ubuntu_lts}{instalación de Ubuntu 20.04}, en el paso 7 podemos realizar la instalación en el disco duro que tengamos instalado en el servidor físico o en la máquina virtual.

En este paso podemos realizar distintas configuraciones:

\begin{center}
    \includegraphics[width=10cm]{01_storage_layout.png}
\end{center}

\begin{itemize}
    \item Usar el disco entero.
    \begin{itemize}
        \item Nos permitirá crear un sistema con LVM (por defecto activado) y con posibilidad de cifrar la partición creada.
    \end{itemize}
    \item Crear un diseño de almacenamiento personalizado.
\end{itemize}

    En la segunda opción podremos:
\begin{itemize}
    \item  Crear particiones a nuestro gusto.
    \item  Elegir el sistema de ficheros de las particiones.
    \item  Crear sistema RAID por software.
\end{itemize}


\section{Pasos previos}
Dado que vamos a crear un sistema RAID 1 durante la instalación de Ubuntu, necesitaremos al menos \textbf{dos discos duros} en nuestro servidor antes de comenzar con la instalación.

En nuestro sistema virtualizado hemos añadido dos discos duros de igual tamaño (15GB), en los cuales crearemos particiones para posteriormente sobre ellas realizar el RAID 1.

\section{Entendiendo las particiones a realizar}
En este apartado vamos a explicar la teoría que está detrás del sistema de particionado que vamos a necesitar crear y que posteriormente realizaremos en el sistema de instalación de Ubuntu.

\subsection{Situación inicial: discos duros sin particionar}
Como ya se ha comentado, en nuestro servidor vamos a contar con dos discos duros de igual tamaño. Esto suele ser lo habitual, pero lo importante es que las particiones que vayamos a crear sean del tamaño exacto, aunque un disco duro sea de mayor tamaño (aunque lógicamente, ese espacio quedará desaprovechado).

En la siguiente imagen vemos que tenemos dos discos duros de 15GB de tamaño cada uno:

\begin{center}
    \includegraphics[width=8cm]{raid1_01.png}
\end{center}


\subsection{Particionado inicial}
A continuación vamos a tener que pensar cómo van a ser las particiones que vamos a necesitar en nuestro servidor. En nuestro caso vamos a crear dos:
\begin{itemize}
    \item  \textbf{14GB}: Sistema operativo.
    \item  \textbf{1GB}: (o hasta completar) SWAP.
\end{itemize}

Como se puede entender, al tener una única partición, todo el sistema raiz “/” va a ir en ella, mientras que la otra partición será la usada para el área de intercambio.

Es importante entender que en este paso sólo \textbf{vamos a crear las particiones pero sin darles formato}. Por lo tanto, nuestros discos duros ahora tendrían este aspecto:

\begin{center}
    \includegraphics[width=8cm]{raid1_02.png}
\end{center}



\subsection{Crear particiones RAID}
El siguiente paso es crear las particiones “virtuales” RAID. Vamos a crear una primera partición RAID que va a incluir las particiones de 14GB de ambos discos duros, y la segunda partición virtual incluirá las particiones de 1GB.

\begin{center}
    \includegraphics[width=8cm]{raid1_03.png}
\end{center}

De esta manera, tendremos unas particiones MD0 y MD1 que son particiones virtuales.

\subsection{Formatear particiones RAID con el formato adecuado}
El último paso de la instalación es hacer uso de las particiones RAID creadas y formatearlas con el sistema de ficheros acorde a las necesidades que tengamos, y elegir el punto de montaje adecuado.

\begin{center}
    \includegraphics[width=8cm]{raid1_04.png}
\end{center}

En nuestro caso va a ser:
\begin{itemize}
    \item  \textbf{MD0}: sistema de ficheros ext4 y lo vamos a utilizar como sistema de ficheros raíz “/”.
    \item  \textbf{MD1}: formateado como SWAP y actuará como área de intercambio.
\end{itemize}

Tras este paso, la instalación del sistema operativo puede continuar de la manera habitual.

\section{Realizando el particionado en el instalador de Ubuntu}
Tras haber entendido las particiones que vamos a realizar, ahora es el momento de proceder en el instalador de Ubuntu. Vamos a seguir los mismos pasos que hemos explicado en el apartado anterior, de esta manera aplicaremos lo aprendido a nivel teórico.

\subsection{Situación inicial: discos duros sin particionar}
Tal como hemos comentado previamente, en el paso 7 de la instalación, elegiremos la opción de crear un “diseño de almacenamiento personalizado”. Tras entrar en esta opción, el instalador tendrá el siguiente aspecto:

\begin{center}
    \includegraphics[width=8cm]{ubuntu_raid1_01.png}
\end{center}

Tal como se puede ver, tenemos dos discos duros en el sistema: \textbf{vda} y \textbf{vdb}. El nombre de los discos viene de \textbf{V}irtual \textbf{D}isk, dado que la instalación la estamos realizando en una máquina virtual.

\subsection{Particionado inicial}
En este paso vamos a crear en cada uno de ellos la partición de 14GB y  el resto del espacio la usaremos para la segunda partición.

\subsubsection{Marcar discos como dispositivos de arranque}
Por cómo funciona el sistema de arranque de Linux, antes de realizar las particiones vamos a marcar que ambos discos duros sean dispositivos de arranque (“Boot Device”). Para ello pulsaremos “Intro” en cada uno de los discos y elegiremos la opción correspondiente (imágenes de cada disco):

\begin{center}
    \includegraphics[width=4cm]{ubuntu_raid1_02-1.png}
    \hspace{3cm}
    \includegraphics[width=4.90cm]{ubuntu_raid1_02-2.png}
\end{center}

De esta manera, el instalador de Ubuntu creará una pequeña partición al inicio del disco donde al terminar se realizará la instalación del sistema de arranque GRUB en ambos discos duros.


\subsubsection{Crear particiones}
Ahora es el momento de crear las particiones, y los pasos serán seleccionar el disco duro, pulsar “Intro”, se nos desplegará un pequeño menú y vamos a elegir la opción  \textit{“Add GPT Partition”} y rellenaremos el tamaño de la partición que nos interese en el momento y el formato lo dejaremos en \textit{“Leave unformatted”} (dejar sin formatear).

\begin{center}
    \includegraphics[width=8cm]{ubuntu_raid1_03.png}
    \hfill
    \includegraphics[width=8cm]{ubuntu_raid1_04.png}
\end{center}

Estos pasos lo realizaremos en cada disco duro con las particiones que vamos a necesitar, quedando al finalizar el sistema así:

\begin{center}
    \includegraphics[width=8cm]{ubuntu_raid1_05.png}
\end{center}

Tal como se puede ver, cada disco duro tiene dos particiones con el tamaño deseado que no están siendo utilizadas, y en la parte de abajo aparecen las particiones denominadas “BIOS grub spacer”.


\subsection{Crear particiones RAID}
El siguiente paso es crear las particiones RAID en las que haremos que el sistema cree un RAID 1 haciendo uso de las particiones de los discos duros físicos. Seleccionaremos la opción “\textit{Create software RAID (md)}” y nos aparecerá una ventana en la que podremos elegir:

{
    \begin{minipage}{9cm}
        \begin{itemize}
            \item \textbf{Nombre}: de la partición que vamos a crear. Es habitual que estas particiones empiecen por “\textbf{md}”, ya que viene de “\textit{multiple device}”.
            \item \textbf{Nivel RAID}: Podremos elegir entre las versiones 0, 1, 5, 6 y 10 de RAID. Por defecto está seleccionada la opción RAID 1.
            \item  \textbf{Dispositivos}: sobre el que aplicaremos el RAID.
        \end{itemize}
    \end{minipage}
    \hfill
    \begin{minipage}{7cm}
        \includegraphics[width=7cm]{ubuntu_raid1_06.png}
    \end{minipage}
}


Crearemos primero \textbf{md0} seleccionando las particiones de 14GB tal como aparece en la siguiente imagen:

Y a continuación crearemos \textbf{md1} con las particiones restantes. Tal como se puede ver a continuación, las particiones de 14GB ya no aparecen, porque están siendo usadas en el otro RAID.

\begin{center}
    \includegraphics[width=9cm]{ubuntu_raid1_07.png}
\end{center}


Tras la creación de los dispositivos “md”, nos aparecerán como dispositivos disponibles para usarlas en el siguiente paso:

\begin{center}
    \includegraphics[width=9cm]{ubuntu_raid1_08.png}
\end{center}

\subsection{Formatear particiones RAID con el formato adecuado}
Aunque este paso lo vamos a realizar sobre las particiones RAID creadas previamente, es un paso que es habitual realizar cuando queremos que nuestra instalación tenga un sistema de particiones propio.

Tenemos que pensar que las particiones RAID ahora son como particiones normales, a las que les vamos a querer dar un formato y utilizarlas para realizar la instalación.

Vamos a seleccionar la primera, \textbf{md0}, posicionándonos encima de ella y dándole a “Intro” y posteriormente dándole a “\textit{Add GPT partition}”:

\begin{center}
    \includegraphics[width=8cm]{ubuntu_raid1_09.png}
\end{center}

Dejaremos el tamaño en blanco, indicando que usaremos todo el espacio libre, lo vamos a formatear con el sistema de ficheros \textbf{ext4} y se va a montar como el sistema de ficheros “/”.

En \textbf{md1} también le daremos a crear nueva partición, y haremos lo mismo pero usando el formato \textbf{swap}. Quedando una vez terminado el sistema de particiones de la siguiente manera:

\begin{center}
    \includegraphics[width=8cm]{ubuntu_raid1_10.png}
\end{center}

Y tras esto podremos continuar la instalación de manera normal.
\clearpage

\graphicspath{{../../../anexos/monitorizacion_munin/}}
\hypertarget{instalar_munin}{}
\chapter{Instalar sistema de monitorización Munin}
A continuación se detalla cómo realizar la instalación y configuración básica de Munin como sistema de monitorización.

\section{Introducción}

\href{https://munin-monitoring.org/}{Munin} es un sistema de monitorización que se instala en cada servidor que queramos monitorizar. Es un sistema de monitorización sencillo que automáticamente monitoriza el propio servidor. Munin puede utilizarse para:

\begin{itemize}
    \item Monitorizar el propio servidor
    \item Crear gráficas con los resultados de monitorización
    \item Usar plugins para monitorizar distintos servicios
    \item Realizar avisos sencillos por mail por situaciones anómalas
\end{itemize}

Munin necesita de un servidor web, como puede ser \href{https://httpd.apache.org/}{Apache}, para poder visualizar las gráficas que crea con los datos obtenidos de la monitorización. Estas gráficas no son más que datos que se obtienen de la monitorización, que son almacenados en pequeñas bases de datos y que mediante un proceso CRON se generan las imágenes con dichos datos.

Aunque Munin puede ser suficiente para la monitorización de un servidor, no se suele utilizar en casos en los que se tenga más de 10-15 servidores, ya que hay que ir servidor a servidor para realizar la configuración, por lo que dificulta la administración.

Munin posee un sistema básico de monitorización centralizada, pero dista mucho de lo que se puede considerar un sistema de monitorización centralizada real. Tal como se ha dicho, para pocos servidores puede ser suficiente, pero no para infraestructuras grandes.

La ventaja que se obtiene al utilizar Munin es que es una instalación muy sencilla, tal como veremos a continuación, y que por tanto, para infraestructuras pequeñas puede ser útil.

\section{Instalación}
Para que Munin funcione se va a necesitar instalar el propio Munin y \href{https://httpd.apache.org/}{Apache} como servidor web para mostrar la web en la que se visualizan los datos obtenidos de la monitorización.

Para realizar la instalación, ejecutaremos:

\begin{mycode}{Instalar Munin y Apache2}{console}{}
root@vega:~# apt install munin apache2
\end{mycode}

Tras la ejecución de este comando, ya se habrá instalado los servicios necesarios para que Munin funcione y el servidor web Apache.

\subsection{Configuración de Apache}
La configuración para que Munin se pueda ver a través del servidor web Apache se encuentra en el fichero de configuración \configfile{ /etc/munin/apache24.conf }  . Normalmente esta configuración ya está aplicada en el Apache tras realizar la instalación, y se puede ver que existe un enlace simbólico, que si seguimos desde  \configfile{ /etc/apache2/conf-enabled/munin.conf }  nos llevará al fichero mencionado previamente.

Esta configuración permite ver la web de Munin pero sólo si estamos conectados desde el propio servidor, a modo de seguridad sólo permite conexiones desde “localhost”. Esto no suele ser lo habitual en servidores, por lo que vamos a modificar la configuración para poder acceder a esta web desde cualquier IP.


\errorbox{
    Permitir el acceso desde cualquier IP es un fallo de seguridad. Habría que modificar la configuración y poner autenticación con “auth-basic”.
}

Para ello, modificaremos el fichero arriba mencionado y modificaremos las líneas donde aparece “Require” y pondremos “\textbf{Require all granted}”. Tras realizar la modificación de la configuración de Apache, deberemos reiniciar el servicio:

\begin{mycode}{Reiniciamos Apache para que coja la configuración}{console}{}
root@vega:~# systemctl restart apache2
\end{mycode}

Y tras esto, ya deberíamos poder ir al navegador web y poniendo \textbf{http://IPSERVIDOR/munin} (donde IPSERVIDOR es la IP del servidor que acabamos de configurar) y ver el interfaz de Munin:

\begin{tcolorbox}[title=Interfaz de Munin]
    \centering
    \includegraphics[frame,width=0.9\linewidth]{munin-01.png}
\end{tcolorbox}


\section{Configuración}

Tras la instalación, Munin por defecto tiene una configuración sencilla que hace una monitorización básica del servidor. La configuración se puede encontrar en  \configdir{/etc/munin/}   y podemos diferenciar 3 ficheros/directorios:

\begin{itemize}
    \item \textbf{munin.conf}: fichero de configuración principal. En él se puede configurar todo lo relacionado con Munin, avisos, procesos...
    \item \textbf{munin-node.conf}: es el fichero de configuración que se utiliza si se utiliza el sistema básico de monitorización centralizada. En este fichero especificaremos desde qué servidores permitiremos que puedan monitorizar el propio servidor.
    \item \textbf{plugins}: directorio con los plugins activados que se van a utilizar durante la monitorización. En este directorio están los enlaces simbólicos de los plugins que van a ser utilizados durante la monitorización del servidor. Los plugins están en  \configdir{/usr/share/munin/plugins/}. Normalmente suelen ser scripts en los lenguajes Perl, Python o Shell.
\end{itemize}

\section{CRON}

La monitorización del servidor se realiza a través de un proceso CRON que instala Munin y que por defecto se hace cada 5 minutos. El fichero CRON está situado en \configfile{ /etc/cron.d/munin }, y cuenta con 4 procesos automáticos que se ejecutan en el servidor:

\begin{itemize}
    \item Proceso de monitorización
    \item 3 procesos de limpieza de caché y ficheros temporales que son creados durante la monitorización
\end{itemize}


\clearpage

\graphicspath{{../../../anexos/}}
\chapter{Glosario}

A continuación se expone un glosario de términos con sus correspondientes definiciones:

\begin{description}
    \hypertarget{altadisponibilidad}{}
    \item[Alta Disponibilidad:] Es un diseño de arquitectura de sistemas y la implementación que asegura que el servicio instalado y otorgado sea funcional sin que haya parada en el mismo. Esta arquitectura trata de que no haya ningún \hyperlink{spf}{\textit{single point of failure} (punto único de fallo)} en la misma.

    \hypertarget{cluster}{}
    \item[Clúster:] Se denomina clúster a un conjunto de ordenadores unidos entre sí mediante conectividad de red que actúan como si de un único servidor se tratara. Dependiendo del tipo de clúster que se va a crear, debe de ser pensado desde el diseño del servicio, ya que es la aplicación o servicio quién se encarga de crear el clúster (como ocurre con MySQL Cluster).

    \hypertarget{dependencia_software}{}
    \item [Dependencia de software:] Cuando se crea cualquier tipo de software lo habitual es hacer uso de otro software (librerías de seguridad, acceso a disco, codecs de vídeo, librerías 3D…) que son necesarias para el correcto funcionamiento de nuestro programa. Este otro software (que puede ser propio o ajeno) se denomina \textbf{dependencia}, ya que sin él, nuestro programa no funcionará y es necesario que exista en el sistema para hacer funciona nuestro programa.

    En las \hyperlink{distribucion_gnu_linux}{distribuciones GNU/Linux} se hace uso de los denominados \hyperlink{paquete_de_software}{paquetes de software} en los cuales se indican las dependencias que necesitan para funcionar y que por tanto se instalarán a la par que el programa elegido, por lo que nos aseguramos que el software instalado funcionará en cuanto termine la instalación.

    En caso de descargar un software ajeno de los \hyperlink{repositorio_de_software}{repositorios} oficiales de la distribución, es posible que necesitemos completar esas dependencias por nuestra cuenta, pero hoy en día es habitual que los creadores de software lo tengan en cuenta y esas dependencias estén en los repositorios oficiales.

    \hypertarget{distribucion_gnu_linux}{}
    \item [Distribución GNU/Linux:] Es una distribución de software basada en el núcleo Linux que incluye software \hyperlink{gnu}{GNU} para componer un Sistema Operativo que pueda ser utilizado por los usuarios. Cada distribución suele \hyperlink{paquete_de_software}{empaquetar el software} en un formato propio que aparte del propio software indica las \hyperlink{dependencia_de_software}{dependencias} de software que necesita para funcionar, por lo que hace que la instalación del software se realice de manera sencilla. El software de la distribución está almacenado en los \hyperlink{repositorio_de_software}{repositorios de software} oficiales de la distribución.

    Las distribuciones suelen estar orientadas para un uso generalizado, pero es cierto que algunas, por su historia o por su manera de entender el empaquetado de software, se necesitan más conocimientos, pero hoy en día no es lo habitual.

    Existen muchas distribuciones GNU/Linux, pero las que podríamos destacar son \hyperlink{ubuntu}{Ubuntu}, Debian, Red Hat y CentOS, que son las de mayor uso hoy en día a nivel profesional.

    \hypertarget{escalado_horizontal}{}
    \item[Escalado Horizontal:] Se llama escalado horizontal a la infraestructura que crece de manera horizontal añadiendo más servidores del mismo servicio. Estos servidores serán accesibles mediante un proxy o de manera directa, y todos contarán con el mismo servicio (web, base de datos, …). No confundir con un clúster, ya que la relación de los servidores en el escalado horizontal no tienen por qué ir en clúster.

    \hypertarget{escalado_vertical}{}
    \item[Escalado Vertical:] Es el incremento de hardware de un servidor. Supongamos que un servidor empieza a tener problemas de carga, pues con el escalado vertical se le añadiría más RAM, más procesador y/o discos duros más rápidos (en caso de ser una máquina virtual sería sencillo, en caso contrario habría que realizar la migración a un servidor nuevo).

    \hypertarget{gnu}{}
    \item[GNU:] Del acrónimo \textbf{GNU’s Not Unix} (GNU no es Unix) es un sistema operativo y un conjunto de programas libres cuyo origen surgió de la idea de crear un sistema operativo Unix basado en \hyperlink{software_libre}{Software Libre}.

    El desarrollo de GNU nació en 1983 por Richard Stallman comenzando por el compilador GCC, al que se fueron uniendo todo tipo de software y creando la Free Software Foundation (o FSF, fundación por el software libre) la cual creó la \hyperlink{licencias_libres}{licencia libre} más conocida actualmente: la \textbf{GPL} (GNU General Public License).

    El proyecto GNU avanzó en el tiempo y creó el kernel Hurd, pero bien es cierto que nunca llegó a ser funcional del todo y actualmente el kernel más utilizado es Linux, pero no es el único, ya que el software GNU también es usado en conjunto con otros kernels como son los \textbf{*BSD}, de ahí la importancia que cuando hacemos referencia al sistema operativo se haga uso de \hyperlink{gnu_linux}{GNU/Linux}.

    \hypertarget{gnu_linux}{}

    \itemimage{GNU/Linux:}{r}{0.21}
    {img/Gnulinux.svg.png}
    {\href{https://es.wikipedia.org/wiki/GNU/Linux\#/media/Archivo:Gnulinux.svg}{GNU/Linux: Wikipedia}}
    {
        Aunque comúnmente solemos llamar a las \hyperlink{distribucion_gnu_linux}{distribuciones} como “Linux” esto no suele ser correcto ya que en la distribución aparte del kernel va un conjunto enorme de software del proyecto GNU. Por lo tanto, lo ideal siempre es hacer uso del nombre completo GNU/Linux.

        El proyecto \hyperlink{gnu}{GNU} y sus herramientas y software son usados con otros kernels como son los *BSD en distribuciones como FreeBSD u OpenBSD. También existen versiones con kernel BSD para la distribución Debian, por lo que en ese caso sería “Debian GNU/BSD”.
    }


    \hypertarget{json}{}
    \item[JSON:] Es un formato de texto sencillo para el intercambio de datos. Aunque originalmente fue creado como notación de objetos para Javascript, su amplia utilización ha hecho que sea utilizado como alternativa a XML.


    \hypertarget{licencias_libres}{}
    \item[Licencias libres:] Una licencia de software es un contrato entre el creador (o el titular de los derechos de autor) del software y el usuario. Todo software que usamos suele exigir la lectura de esta licencia y es por ello muy importante conocer qué se puede y no se puede hacer con dicho software.

    Las licencias libres son aquellas que nos permiten hacer con el software lo que las cuatro libertades del \hyperlink{software_libre}{Software Libre} exige.

    Entre las licencias libres más utilizadas hoy en día están la GPL (General Public License del proyecto \hyperlink{gnu}{GNU}), la Apache License, algunas de las versiones de las licencias Creative Commons, …


    \hypertarget{linux}{}
    \item[Linux:] Creado originalmente por Linus Torvalds en 1991 y actualmente desarrollado por cientos de desarrolladores de todo el mundo, Linux es el núcleo (o kernel) gratuito y libre similar al núcleo de los sistemas operativos Unix.

    Comenzó como un proyecto personal de Linus (siendo estudiante universitario) para su ordenador 386 y actualmente está portado a \href{https://es.wikipedia.org/wiki/Portabilidad\_del\_n\%C3\%BAcleo\_Linux\_y\_arquitecturas\_soportadas}{decenas de plataformas hardware}. Es el proyecto más grande y ambicioso del \hyperlink{software_libre}{Software Libre}, aunque originalmente no se permitía el uso comercial del mismo (hasta la versión 0.12).

    Al poco tiempo de comenzar su desarrollo el proyecto \hyperlink{gnu}{GNU} lo adoptó como su kernel naciendo lo que actualmente conocemos como \hyperlink{gnu_linux}{GNU/Linux} y con ello cientos de \hyperlink{distribucion_gnu_linux}{distribuciones}.

    Es un núcleo de tipo monolítico que permite la carga de módulos en tiempo de ejecución


    \hypertarget{lts}{}
    \item[LTS:] Del inglés \textit{\textbf{L}ong \textbf{T}erm \textbf{S}upport} (en castellano “soporte a largo plazo”), es una característica en informática que hace referencia a versiones especiales de software que contarán con un soporte más largo del habitual, por lo que serán las versiones idóneas para usar en servidores.

    Estas versiones suelen contar con actualizaciones de seguridad, pero no con cambios notorios en la forma del software para fomentar la fiabilidad del mismo. Lo habitual es utilizar este tipo de versiones en servidores, que aunque puedan no tener las últimas modificaciones de las versiones más recientes del software, nos aseguramos la fiabilidad. Esto hace que tengamos que decidir si es necesario contar con las características de las últimas versiones (ya sea nuevos servicios, opciones nuevas, velocidad, … ) o si preferimos contar con una versión que tendrá un ciclo de vida más longevo pero con actualizaciones de seguridad.

    Es habitual verlo en proyectos de \hyperlink{software_libre}{Software Libre}, como ejemplos podemos tomar el kernel \hyperlink{linux}{Linux} (actualmente la versión 5.4.58 es la denominada LTS) y la distribución \hyperlink{ubuntu}{Ubuntu} (en este caso la versión 20.04).


    \hypertarget{paquete_de_software}{}
    \item[Paquetes de Software:] Un paquete de software no es más que una manera de poder distribuir el software creado. En \hyperlink{distribucion_gnu_linux}{distribuciones GNU/Linux} estos paquetes determinan las \hyperlink{dependencia_software}{dependencias} que necesitan para que su instalación sea lo más sencilla posible.

    Lo habitual es que estos paquetes estén gestionados mediante un sistema de gestión propio para conocer cuáles están instalados, sus dependencias, desinstalarlos de manera sencila...

    No sólo se usa en distribuciones GNU/Linux, ya que varios lenguajes de programación hacen lo propio para distribuir software en forma de paquetes. Como ejemplos:
    \begin{itemize}
        \item En distribuciones GNU/Linux tenemos APT, Yum, Zypper, Portage, ...
        \item En lenguajes de programación tenemos Gem para Ruby, Eggs para Python, CPAN en Perl, ...
    \end{itemize}


    \hypertarget{repositorio_de_software}{}
    \item[Repositorio de Software:] Se podría denominar repositorio como el almacén donde se guardan los \hyperlink{paquete_de_software}{paquetes de software}. Las \hyperlink{distribucion_gnu_linux}{distribuciones GNU/Linux} cuentan con sus repositorios oficiales, donde se almacena el software para cada versión que tiene la distribución.

    Aparte del software que podemos instalar, también cuentan con un índice para saber los paquetes y las versiones que se almacena en ellos. Este índice es necesario que lo actualicemos de manera periódica (en Ubuntu ejecutando: “apt update”) ya que gracias a él sabremos si tenemos que realizar actualizaciones de los paquetes instalados.

    También podemos utilizar repositorios externos al de la distribución, repositorios oficiales de un software por ejemplo, que nos permiten instalar la última versión de ese software sobre nuestra distribución. Cuando un paquete con el mismo nombre existe en distintos repositorios, siempre se instalará del repositorio que tenga la versión más nueva.

    No es buena práctica, y \textbf{está completamente desaconsejado}, mezclar repositorios de distribuciones distintas aunque el gestor de paquetes sea el mismo (usar repositorios de Debian en Ubuntu o viceversa).


    \hypertarget{spf}{}
    \item[Single Point of Failure:] O punto único de fallo, es un componente de un sistema que tras un fallo en su funcionamiento ocasiona un fallo global en el sistema completo, dejándolo inoperante. Un SPOF puede ser un componente de hardware, software o eléctrico.


    \hypertarget{software_libre}{}
    \item[Software Libre:] El movimiento del Software Libre fue creado por Richard Stallman a la par que creaba el proyecto \hyperlink{gnu}{GNU}. Para que un software sea considerado como Software Libre debe contener una \hyperlink{licencias_libres}{licencia libre} que debe otorgar las cuatro libertades siguientes:
    \begin{itemize}
        \item Libertad de usar el software para cualquier propósito.
        \item Libertad de estudiar el software y su funcionamiento interno (es por ello necesario poder acceder al código fuente).
        \item Libertad de distribuir el software con quien queramos.
        \item Libertad de poder modificar y mejorar el software según nos interese.
    \end{itemize}

    Es muy importante tener en cuenta que Software Libre no significa gratis, ya que en inglés el término viene de Free Software donde “Free” puede significar libre y gratis. Es cierto que la gran mayoría del Software Libre puede ser gratis, pero no todo el software gratis es Software Libre.


    \hypertarget{ssh_server}{}
    \item [SSH Server]: De \textbf{S}ecure \textbf{SH}ell, es el nombre de un protocolo y del programa (tanto servidor, como cliente) cuya función principal es la de acceder de manera remota a través de un canal seguro a un servidor.

    SSH permite no sólo la conexión a un servidor sino también la transferencia de ficheros y creación de túneles cifrados por los que pueden viajar otros protocolos. El puerto habitual de uso para este protocolo es el \textbf{22}.


    \hypertarget{systemd}{}
    \item [Systemd]: es un conjunto de demonios de administración de sistema, bibliotecas y herramientas diseñados como una plataforma de administración y configuración central para interactuar con el núcleo del Sistema operativo \hyperlink{gnu_linux}{GNU/Linux}.


    \hypertarget{ubuntu}{}

    \itemimage{Ubuntu:}{r}{0.21}
    {img/ubuntu.svg.png}
    {\href{https://es.wikipedia.org/wiki/Ubuntu}{Ubuntu: Wikipedia}}
    {
        Es una \hyperlink{distribucion_gnu_linux}{distribución de GNU/Linux} originalmente basada en Debian y creada por la compañía Canonical en el 2004. En su momento fue una de las distribuciones que apostaron por un sistema de instalación sencillo y con la intención de detectar el máximo hardware posible para acercarse a la gran cantidad de usuarios posibles.

        Hoy en día es una de las distribuciones más utilizadas tanto a nivel de escritorio como a nivel de servidores ya que cuenta con dos versiones separadas a la hora de realizar la instalación (aunque realmente es la misma distribución).

        Una de sus ventajas es la creación de versiones \hyperlink{lts}{LTS} cada dos años, que son versiones que garantizan su soporte técnico durante más tiempo por lo que supone una ventaja a la hora de realizar la instalación en servidores. Con ellos nos aseguramos que el software va a ser actualizado ante fallos de seguridad durante más tiempo que las versiones que no son LTS.
    }


\end{description}

\clearpage

\end{document}
