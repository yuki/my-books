\chapter{Introducción}
\href{https://es.wikipedia.org/wiki/Odoo}{Odoo}, antes conocido como \textit{OpenERP} es un software de planificación de recursos empresariales con licencia dual. Existe una versión de código abierto y una versión con licencia comercial con características y servicios exclusivos.

Odoo también cuenta con un apartado de CRM (en inglés \textit{customer relationship management}, o gestión de relaciones con el cliente), pudiendo también crear una web de comercio electrónico, facturación, ...


\chapter{Instalación}

Tal como hemos visto en el tema anterior, la instalación de un sistema se puede realizar de distintas maneras, por lo que deberemos atender a las necesidades del proyecto para realizar una instalación adecuada.

En nuestro caso, se va a optar por realizar una instalación a través de servicios Docker, de esta manera podemos realizar pruebas con distintas versiones (tanto de Odoo como de la base de datos).

\infobox{La alternativa sería realizar la instalación en una máquina virtual o haciendo uso de los distintos instaladores que existen en la \href{https://www.odoo.com/es_ES/page/download}{web oficial}}

Para realizar la instalación seguiremos las indicaciones que aparecen en la web de \href{https://hub.docker.com/_/odoo}{Docker Hub} haciendo pequeñas modificaciones.

\section{Servicios independientes}

Es el sistema más básico, que requiere de levantar dos contenedores Docker:
\begin{itemize}
    \item Contenedor de base de datos \textbf{PostgreSQL}. Podremos elegir entre las distintas versiones del gestor de base de datos, pero haremos caso a las recomendaciones de la web. Habría que tener especial cuidado con el usuario y la contraseña que utilizamos.

    En este caso también se ha expuesto el puerto 5432 que es el puerto por defecto de PostgreSQL:

\begin{mycode}{Crear y arrancar el contenedor de la base de datos}{console}{}
ruben@vega:~$ docker run -d -e POSTGRES_USER=odoo \
-e POSTGRES_PASSWORD=odoo -e POSTGRES_DB=postgres \
-p 5432:5432 --name odoo_db postgres:15
\end{mycode}

    \item Contenedor con el propio \textbf{Odoo}. Este contenedor tendrá los servicios y librerías necesarias para poder hacer funcionar la aplicación web.

\begin{mycode}{Crear y arrancar el contenedor de la base de datos}{console}{}
ruben@vega:~$ docker run -p 8069:8069 --name odoo \
--link odoo_db:db -t odoo
\end{mycode}
\end{itemize}


\section{Docker Compose}

Docker Compose es una herramienta para correr servicios multi-contenedor y se crea a través de un fichero en formato YAML. Es una manera de crear,parar,reconstruir una arquitectura de servicios de manera rápida y sencilla.

Se debe crear un fichero \configfile{compose.yaml} y lo ideal es que esté dentro de un directorio con el nombre del “stack de servicios”, ya que coge el directorio como parte del nombre a la hora de crear los contenedores.

\begin{mycode}{Contenido de fichero compose.yaml}{yaml}{}
version: '3.1'
services:
  web:
    image: odoo:16.0
    depends_on:
      - db
    ports:
      - "8070:8069"
  db:
    image: postgres:15
    environment:
      - POSTGRES_DB=postgres
      - POSTGRES_PASSWORD=odoo
      - POSTGRES_USER=odoo
    ports:
      - "5433:5432"
\end{mycode}

Para arrancar los servicios se realiza, desde el mismo directorio donde se encuentra el fichero, con el siguiente comando

\begin{mycode}{Levantar docker compose}{console}{}
ruben@vega:~$ docker compose up
\end{mycode}

\errorbox{En Julio del 2023 se migró a Compose v2, tal como se indica en la \href{https://docs.docker.com/compose/}{web oficial}. Dependiendo de la versión que tengamos instalada será “docker compose up” o “docker-compose up”}


\section{Herramientas extra}
Para poder acceder a la base de datos podemos hacer uso de un cliente externo. De esta manera no tendremos que entrar al contenedor y tendremos un interfaz gráfico con el que poder administrarla.

Podemos utilizar:
\begin{itemize}
    \item \textbf{DBeaver}: Es una aplicación de escritorio que permite conectarnos a distintos SGBDs. Existe versión \href{https://dbeaver.io/}{community} y otra con \href{https://dbeaver.com/buy/}{licencia} que permite también conectarse a bases de datos NO-SQL.

    \item \textbf{\href{https://www.pgadmin.org/}{pgAdmin}}: Es una aplicación que permite administrar PostgreSQL a través del servidor web.
\begin{mycode}{Crear y arrancar el contenedor pgAdmin}{console}{}
ruben@vega:~$ docker run -p 8090:80 \
-e 'PGADMIN_DEFAULT_EMAIL=user@domain.com' \
-e 'PGADMIN_DEFAULT_PASSWORD=SuperSecret' \
--name pgadmin4 -d dpage/pgadmin4
\end{mycode}
\end{itemize}