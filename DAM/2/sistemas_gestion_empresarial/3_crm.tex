\chapter{Introducción}

Tal como nos dice \href{https://es.wikipedia.org/wiki/Gesti%C3%B3n_de_Relaciones_con_el_Cliente}{Wikipedia}, la gestión o administración de relaciones con el cliente (\textit{customer relationship management}), más conocida por sus siglas en inglés \textbf{CRM}, puede tener varios significados:

\begin{itemize}
    \item \textbf{Administración o gestión basada en la relación con los clientes}: un modelo de gestión de toda la organización, basada en la satisfacción del cliente (u orientación al mercado según otros autores). El concepto más cercano es marketing tradicional.

    \item \textbf{Software para la administración o gestión de la relación con los clientes}: Sistemas informáticos de apoyo a la gestión de las relaciones con los clientes, a la venta y al marketing, y que se integran en los llamados Sistemas de Gestión Empresarial (SGE), y que incluyen CRM, ERP, PLM, SCM y SRM.

    El software de CRM puede comprender varias funcionalidades para gestionar las ventas y los clientes de la empresa: automatización y promoción de ventas, tecnologías data warehouse («almacén de datos») para agregar la información transaccional y proporcionar capa de reporting, dashboards e indicadores claves de negocio, funcionalidades para seguimiento de campañas de marketing y gestión de oportunidades de negocio, capacidades predictivas y de proyección de ventas.
\end{itemize}

El CRM es un enfoque para gestionar la interacción de una empresa con sus clientes actuales y potenciales, es una forma de pensar y de actuar de una empresa hacia los clientes/consumidores. \textbf{Utiliza el análisis de datos de la historia de los clientes con la empresa y para mejorar las relaciones comerciales} con dichos clientes, centrándose específicamente en la retención de los mismos y, en última instancia, impulsando el crecimiento de las ventas.


\chapter{Objetivos}

