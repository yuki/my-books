\chapter{Introducción}

Tal como nos dice \href{https://es.wikipedia.org/wiki/Gesti%C3%B3n_de_Relaciones_con_el_Cliente}{Wikipedia}, la gestión o administración de relaciones con el cliente (\textit{customer relationship management}), más conocida por sus siglas en inglés \textbf{CRM}, puede tener varios significados:

\begin{itemize}
    \item \textbf{Administración o gestión basada en la relación con los clientes}: un modelo de gestión de toda la organización, basada en la satisfacción del cliente (u orientación al mercado según otros autores). El concepto más cercano es marketing tradicional.

    \item \textbf{Software para la administración o gestión de la relación con los clientes}: Sistemas informáticos de apoyo a la gestión de las relaciones con los clientes, a la venta y al marketing, y que se integran en los llamados Sistemas de Gestión Empresarial (SGE), y que incluyen CRM, ERP, PLM, SCM y SRM.

    El software de CRM puede comprender varias funcionalidades para gestionar las ventas y los clientes de la empresa: automatización y promoción de ventas, tecnologías data warehouse («almacén de datos») para agregar la información transaccional y proporcionar capa de reporting, dashboards e indicadores claves de negocio, funcionalidades para seguimiento de campañas de marketing y gestión de oportunidades de negocio, capacidades predictivas y de proyección de ventas.
\end{itemize}

El CRM es un enfoque para gestionar la interacción de una empresa con sus clientes actuales y potenciales, es una forma de pensar y de actuar de una empresa hacia los clientes/consumidores. \textbf{Utiliza el análisis de datos de la historia de los clientes con la empresa y para mejorar las relaciones comerciales} con dichos clientes, centrándose específicamente en la retención de los mismos y, en última instancia, impulsando el crecimiento de las ventas.


\chapter{Objetivos}

Teniendo en cuenta lo dicho anteriormente, si el CRM está separado del programa que gestiona las ventas y recursos, puede resultar complejo el realizar una buena gestión de todo ello. Al final, la relación con el cliente es debido a las ventas o proyectos que se le realizan.

Los sistemas de software CRM aportan distintas funciones para a la gestión de las relaciones con el cliente:

\begin{itemize}
    \item Almacenamiento de los datos de los clientes a nivel comercial.
    \item Creación de segmentos personalizados para distintos objetivos.
    \item Seguimiento tanto de clientes como de ventas.
    \item Automatización de los procesos de venta (leads, alertas, tareas…).
    \item Generación de promociones específicas.
    \item Gestión del servicio postventa de los productos.
\end{itemize}


\chapter{Componentes}
Los componentes principales de un CRM están construidos y manejan la relación con el cliente en base al marketing, observando la relación que existe a lo largo del tiempo a través de las distintas etapas de proyectos, ya que estas no son homogéneas.

\warnbox{No es lo mismo la relación con un cliente al inicio del primer proyecto que durante la ejecución de un tercer proyecto.}


\section{Tipos de comunicación}
Podríamos identificar como buenos componentes de relación con el cliente los siguientes:

\begin{itemize}
    \item \textbf{Comunicación verbal}: Parte de la comunicación que realizaremos con el cliente será de forma verbal. De esta manera obtendremos una cercanía y un inicio de relación donde asentar las bases de la relación comercial.

    No tiene por qué ser el primer acercamiento, pero será, normalmente, el que más estreche la relación. También suele ser la manera más rápida para compartir cierta comunicación y para comenzar las bases de inicio de proyectos.

    \item \textbf{Internet}: Hoy en día el marketing a través de Internet es el más habitual. Ya sea a través de la página web corporativa, redes sociales, o en algunos casos incluso a través de campañas de publicidad incrustadas en otras webs.

    Hoy en día para la captación de nuevos clientes, para dar a conocer las novedades y los avances en productos/proyectos, es una manera clave de mantener o iniciar la relación con clientes.

    \item \textbf{Campañas por e-mail}: Aunque un poco relegadas por las redes sociales, las campañas por e-mail siguen siendo una manera en la que una empresa se debe relacionar con los clientes. Pueden ser de dos tipos:
    \begin{itemize}
        \item \textbf{Dirigidas a clientes concretos}: Dependiendo de lo que queramos ofrecer, y teniendo en cuenta lo que los clientes han contratado previamente, se podrá dirigir las campañas a clientes concretos, tratando de buscar el mayor impacto posible.

        \item \textbf{Generalizadas}: Por ser campañas más generalistas, o simplemente para ofrecer información general (posibles cambios dentro de la empresa, o mejoras en los servicios), estas campañas pueden ser dirigidas a todos los clientes en general.
    \end{itemize}

    \item \textbf{Marketing telefónico}: Algunas empresas, dirigidas a servicios generalistas, pueden hacer uso de campañas de marketing a través del teléfono en busca de nuevos clientes.

    \item \textbf{Soporte}: Una vez terminado un proyecto, o con la contratación de un servicio, es habitual tener un sistema de soporte al cliente.

    A través de este sistema, el cliente podrá pedir “ayuda” y dependiendo del tipo de empresa un técnico especializado le resolverá las dudas o le ayudará.
\end{itemize}