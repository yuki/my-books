\chapter{Introducción}

Los sistemas de planificación de recursos empresariales (\textbf{ERP}, por sus siglas en inglés, \textit{enterprise resource planning}) son los sistemas de información gerenciales que integran y manejan muchos de los negocios asociados con las operaciones de producción y de los aspectos de distribución de una compañía en la producción de bienes o servicios.

Los sistemas ERP son llamados ocasionalmente \textit{\textbf{back office}} (trastienda) ya que el cliente y el público general no tienen acceso a él. Sólo los usuarios internos de la empresa (y no tienen por qué ser todos) accederán a distintos apartados para realizar modificaciones. Estas modificaciones se visualizarán, o tendrán efecto, sobre lo que el usuario final verá.

\chapter{Objetivos}

Los sistemas ERP son sistemas de gestión de información que automatizan muchas de las prácticas de negocio asociadas con los aspectos operativos o productivos de una empresa.

Los sistemas ERP suelen estar compuestos por distintos módulos para realizar diferentes acciones dentro de la empresa. En caso de necesitar cualquiera de ellos, se realizará la instalación o configuración del mismo, ya que lo habitual suele ser que estén desactivados por defecto.

Entre los módulos más habituales que nos podemos encontrar se pueden destacar: producción, ventas, compras, logística, contabilidad (de varios tipos), gestión de proyectos, GIS, inventarios y control de almacenes, pedidos, nóminas, ...

Los objetivos principales de los sistemas ERP son:

\begin{itemize}
    \item Unificación y trazabilidad de todos los procesos en un mismo sistema.
    \item Optimización de los procesos empresariales.
    \item Planificación de los recursos.
    \item Automatización de los procesos entre las áreas de la empresa.
    \item Acceso a los datos y creación de información estructurada.
    \item Posibilidad de compartir información entre todos los componentes de la organización.
    \item Eliminación de datos y operaciones innecesarias de reingeniería.
\end{itemize}


\chapter{Características}

Tal como nos dice \href{https://es.wikipedia.org/wiki/Sistema_de_planificaci%C3%B3n_de_recursos_empresariales#Definición}{Wikipedia}, las características que distinguen a un ERP de cualquier otro software empresarial son que deben ser modulares, configurables y especializados:

\begin{itemize}
    \item \textbf{Modulares}. Los ERP entienden que una empresa es un conjunto de departamentos que se encuentran interrelacionados por la información que comparten y que se genera a partir de sus procesos. Una ventaja de los ERP, tanto económica como técnica, es que la funcionalidad se encuentra dividida en módulos, los cuales pueden instalarse de acuerdo con los requerimientos del cliente. Ejemplo: ventas, materiales, finanzas, control de almacén, recursos humanos, etc.

    \item \textbf{Configurables}. Los ERP pueden ser configurados mediante desarrollos en el código del software. Por ejemplo, para controlar inventarios, es posible que una empresa necesite manejar la partición de lotes, pero otra empresa no. Los ERP más avanzados suelen incorporar herramientas de programación de cuarta generación para el desarrollo rápido de nuevos procesos.

    \item \textbf{Especializados}. Un ERP especializado, brinda soluciones existentes en áreas de gran complejidad y bajo una estructura de constante evolución. Estas áreas suelen ser, el verdadero problema de las empresas, además de contener todas las áreas transversales. Trabajar bajo ERP especializados es el paso lógico de las empresas que requieren soluciones reales a sus verdaderas necesidades.Un ERP genérico solo ofrece un bajo porcentaje de efectividad basado en respuestas generalistas, que requieren ampliaciones funcionales.
\end{itemize}

