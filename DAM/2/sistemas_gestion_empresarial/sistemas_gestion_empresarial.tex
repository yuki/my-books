\newcommand{\ClassPath}{../../../yukibook.cls}
\documentclass{\ClassPath/yukibook}


\begin{document}

    \yukibook{Sistemas de Gestión Empresarial} % Title
    {Rubén Gómez Olivencia}  % Author
    {2023-2024}    % Year
    {Técnico superior en \linebreak Desarrollo de  Aplicaciones Multiplataforma} % Name of degree
    {}% catch phrase
    {}% the phrase's author
    {img/portada.png} %cover
    {28436c}
    {si} %mini-title

    \coverpage
    \graphicspath{{../../../yukibook.cls/}}
    \licensepage

    \tableofcontents

    %--------------------------------------------------------------------------
    % Start your parts, chapters and sections here
    %--------------------------------------------------------------------------

    \part{Sistemas de Gestión Empresarial}
    \graphicspath{{./img/sge}}
    \chapter{Introducción}

La \textbf{informática} es un área de la ciencia que abarca distintas disciplinas teóricas (como la creación de algoritmos, teoría de computación, teoría de la información, ...) y disciplinas prácticas (diseño de hardware, implementación de software).

A la hora de crear programas (o \textit{software}), podemos identificarlos de distintos tipos:
\begin{itemize}
    \item \textbf{Software de sistema}: Programas o aplicaciones que pertenecen al sistema y nos ayudan a mejorarlo, administrarlo ... Pueden ser aplicaciones de monitorización, de auditoría de logs, \textit{drivers}, ...

    \item \textbf{Software de desarrollo}: En este caso serán aplicaciones que nos ayudarán a crear otras aplicaciones. Por ejemplo: librerías de funciones, compiladores, \textit{debuggers}, IDEs...

    \item \textbf{Aplicaciones de usuario}: Son aplicaciones que los usuarios finales utilizarán en su día a día. Podríamos diferenciarlas como:
    \begin{itemize}
        \item \textbf{Aplicaciones generalistas}: Son aquellas que cualquier tipo de usuario utilizará en cualquier momento. Son creadas con un propósito específico, pero que no hay que tener grandes conocimientos para usarlas. Por ejemplo: navegadores web, clientes de correo, aplicaciones de ofimática simple, calculadora, calendario, ...

        \item \textbf{Aplicaciones de uso específico}: En este caso son aplicaciones creadas para un usuario específico, con una utilidad muy concreta y que normalmente deben existir conocimientos para utilizarla.

        Pueden ser aplicaciones no muy complejas, pero cuya utilidad, o lo que hagan, tenga importancia y conste de procesos complejos. Por ejemplo: aplicaciones CAD, sistemas de virtualización, aplicaciones científicas (R, JupyterLab), aplicaciones empresariales, ...
    \end{itemize}
\end{itemize}

En esta asignatura veremos distintos tipos de software especializado dentro de la gestión empresarial como son los ERP y los CRM, que podríamos englobar como \textbf{sistemas de información}.

\chapter{Sistemas de información}

Un sistema de información, de manera generalizada, es aquel que ayuda a administrar, recolectar, recuperar, procesar, almacenar y distribuir información relevante para ser usados dentro de los procesos fundamentales de una organización.

Normalmente estos sistemas de información son fáciles de usar, tienen cierto grado de flexibilidad (se pueden adaptar a las empresas), permiten guardar y recuperar información de manera rápida y sencilla.

De esta manera, la información resultante será más valiosa para la propia organización, ya que tendrá una “imagen” más amplia y habiendo podido relacionar más información que de no haber utilizado este tipo de software.

\section{Componentes}
Un sistema de información debe contar con los siguientes componentes básicos, que deben interactuar entre sí de manera adecuada para un buen funcionamiento global:
\begin{itemize}
    \item El \textbf{hardware}, equipo físico utilizado para procesar y almacenar datos.
    \item El \textbf{software} y los procedimientos utilizados para transformar y extraer información.
    \item Los \textbf{datos} que representan las actividades de la empresa.
    \item La \textbf{red} que permite compartir recursos entre computadoras y dispositivos.
    \item Las \textbf{personas} que desarrollan, mantienen y utilizan el sistema.
\end{itemize}

El último punto es muy importante, ya que de nada sirve tener la mejor herramienta, en el mejor hardware, si luego las personas que van a hacer uso de ella no tienen los conocimientos suficientes.

\errorbox{\textbf{Las personas que utilizan los sistemas de información deben tener los conocimientos adecuados para su correcta utilización.}}

Es por eso que las personas que hagan uso del sistema de información deberán ser entrenadas y/o tener manuales para su correcto uso, así como \textbf{también tener en cuenta sus opiniones para mejorar los procesos internos de la empresa}.

\section{Datos vs información}

Los datos reflejan hechos recogidos en la organización y que están todavía sin procesar (reflejan valores o resultados de mediciones). Estos datos serán hechos o cifras sobre algún tema específico concreto, que a simple vista no tienen por qué decir nada.

Por otro lado, la información se obtiene una vez se han procesado, agregado y/o presentado de manera adecuada esos datos para que puedan ser útiles y de esta manera obtener un valor que de otra manera no se podría obtener.

El ejemplo más claro entre datos e información se puede obtener en cualquier \textbf{estudio científico}, en el que a tras la obtención de unos datos, a través del método científico se llega una conclusión y con ello información.

Por ejemplo, mediciones del dióxido de carbono (CO$_{2}$) en la atmósfera, se obtienen  datos y se llega a la siguiente imagen que es la información:

\begin{center}
    \includegraphics[width=0.7\linewidth]{co2.png}
    \captionof{figure}{Mediciones de CO$_{2}$ en los últimos miles de años. Fuente: \href{https://climate.nasa.gov/en-espanol/signos-vitales/dioxido-de-carbono/}{NASA}.}
\end{center}



\section{Objetivo}

El objetivo de los sistemas de información, y en este caso, los utilizados para la gestión empresarial, es el de realizar acciones de manera más rápida y eficiente, por lo que también debería ser más económico para la empresa.

El uso de las tecnologías de la información y la comunicación en las empresas se ha convertido en un \textbf{elemento esencial como motor vertebrador y fuente de ventajas competitivas}.

Hoy en día una empresa que no haga uso de la informática está condicionando su estrategia empresarial, y es bastante probable que esté perdiendo oportunidades de negocio, así como la posibilidad de desarrollar sus productos y servicios.

Es por eso que el uso de la informática y de \textit{software} especializado de gestión empresarial puede ayudar a las empresas en:

\begin{itemize}
    \item Obtener ventajas competitivas.
    \item Mejorar la eficiencia interna de la empresa: reducir costes, mejorar la productividad, mejorar la organización de la información, ...
    \item Mejorar y facilitar la toma de decisiones a través de la recopilación de la información.
    \item Para desarrollar nuevas estrategias de negocios.
\end{itemize}

\section{Requisitos}

Para que la información sea útil en la toma de decisiones dentro de una organización, debe cumplir una serie de requisitos:

\begin{itemize}
    \item \textbf{Exactitud}: debe ser precisa y libre de errores.
    \item \textbf{Comprensión}: inteligible por el usuario.
    \item \textbf{Completitud}: debe contener todos aquellos hechos que pudieran ser importantes.
    \item \textbf{Economicidad}: el coste para obtener la información debe ser menor que el beneficio.
    \item \textbf{Confianza}: garantizar tanto la calidad de los datos utilizados, como la de las fuentes de información.
    \item \textbf{Relevancia}: ha de ser útil para la toma de decisiones.
    \item \textbf{Nivel de detalle}: se debe proporcionar con la presentación y el formato adecuados, para que resulte sencilla y fácil de manejar.
    \item \textbf{Oportunidad}: se debe entregar la información a la persona que corresponde y en el momento adecuado.
    \item \textbf{Verificabilidad}: la información ha de poder ser contrastada y comprobada en todo momento.
\end{itemize}

\warnbox{A tener en cuenta: \textbf{el exceso de información también puede ser contraproducente}.}

\section{Actividades}

A la hora de hacer uso de un sistema de información, las actividades que se pueden realizar con él se pueden resumir en:

\begin{itemize}
    \item \textbf{Recopilación}: Es la recogida de datos en bruto. Estos datos pueden ser de dentro de la organización, del exterior, recopilados de manera automática o de manera manual.
    \item \textbf{Almacenamiento}: Los datos deben ser guardados de manera estructurada para su posterior uso. Por otro lado, \textbf{nos debemos asegurar que su persistencia no corra peligro}, por lo que deberemos contar con un sistema de almacenamiento que sea capaz de asegurar posibles problemas. Para ello deberemos tener un sistema en \textbf{alta disponiblidad}, y con un buen sistema de \textbf{backups} configurado.

    También hay que asegurar que \textbf{el acceso a los datos estará limitado y asegurado mediante sistemas de control de acceso y de autenticación}.

    \item \textbf{Procesamiento}: Es el punto clave en el que los datos se convierten en información, de esta manera cumpliendo la labor de ayudar a la organización en la toma de decisiones.

    \item \textbf{Distribución}: El sistema permitirá distribuir la información entre las personas que la necesiten.
\end{itemize}


\section{Tipos de sistemas de información}

Aunque existen distintos tipos de sistemas de información, y su clasificación se puede realizar teniendo en cuenta distintas funcionalidades y/o objetivos, nos vamos a centrar en dos tipos:

\begin{itemize}
    \item \textbf{ERP}: \textit{Enterprise Resource Planning} o planificación de recursos en la empresa. Se trata de los sistemas de gestión integrados que permiten dar soporte a la totalidad de los procesos de una empresa: control económico financiero, logística, producción, mantenimiento, recursos humanos, ...

    \item \textbf{CRM}: \textit{Customer Relationship Management}, sistemas para gestionar las relaciones con los clientes y el soporte a todos los contactos comerciales.
\end{itemize}

    \part{Alta Disponibilidad y Arquitectura de sistemas}
    \chapter{Alta Disponibilidad}
    La alta disponibilidad en servidores se puede definir como el diseño de infraestructura (y su implantación) que asegura la continuidad del servicio y que no tiene un único punto de fallo.

Es lógico entender que un servicio debe de ser contínuo en el tiempo, ya que debe de dar servicio de manera continuada para que los usuarios puedan acceder a él. Pero para que esta premisa sea efectiva, y para asegurarnos que así sea, \textbf{la infraestructura debe de estar redundada y carecer de puntos de fallo únicos en su diseño}.

Esto quiere decir, que de cada servicio y para cada posible punto de fallo deberá haber al menos dos de ellos, para que en caso de que uno deje de funcionar el servicio siga funcionando (dos tomas eléctricas separadas, dos servidores que otorguen el servicio, dos conexiones a internet, ... ).

Es habitual que un sistema en Alta Disponibilidad deba de estar pensado desde el diseño. Algunos tipos  de servicios pueden empezar como un único servidor y posteriormente realizar un \hyperlink{escalado_horizontal}{escalado horizontal}, formando la alta disponibilidad, mientras que para otros \textbf{el diseño en alta disponibilidad debe de estar pensado desde el comienzo} (habitualmente en algunos tipos de \hyperlink{cluster}{clusters}).


\section{Importancia de un sistema en Alta Disponibilidad}

Como se ha citado previamente, la alta disponibilidad nos va a asegurar al menos tres grandes ventajas:

\begin{itemize}
    \item Una continuidad en el servicio
    \item Un diseño libre de puntos de fallos únicos, gracias a la redundancia.
    \item Mejorar el rendimiento global.
\end{itemize}

La redundancia permitirá que en caso de fallo de algún equipamiento/servicio, al estar redundando, no afecte al servicio. Gracias al \hyperlink{monitorizacion_de_sgbds}{sistema de monitorización} seremos capaces de ver el problema y solventarlo lo antes posible. De estar el diseño correcto, el servicio mantendrá su actividad, mientras que por el contrario, si ha habido algún fallo en el diseño de infraestructura (o el problema es más grave de lo esperado) el servicio se verá afectado.


\section{Tipos de Alta Disponibilidad}
Existen muchos tipos de alta disponibilidad dependiendo de en qué capa de infraestructura estemos hablando. Por poner unos ejemplos:

\begin{itemize}
    \item \textbf{Redundancia eléctrica}: Los servidores normalmente cuentan con doble fuente de alimentación, por lo que cada fuente de alimentación debe de estar conectada a una toma eléctrica completamente separada de la otra.
    \item \textbf{Redundancia de conectividad física}: El acceso a internet debe de ser redundado.
    \item \textbf{Redundancia de conectividad LAN}: El acceso a la LAN/DMZ/red de servicio debe de estar redundado (stacks de switches, LACPs configurados en switches y servidores, … ).
    \item \textbf{Redundancia de servidores}: Debe de existir una redundancia de servidores para asegurar que el servicio funcione en más de un servidor físico.
    \item \textbf{Redundancia de servicio}: El servicio que se ofrece debe de estar redundado entre los distintos servidores.
\end{itemize}


La alta disponibilidad también se puede diferenciar como:


\begin{itemize}
    \item \textbf{Alta disponibilidad real}: En caso de que haya algún problema el servicio continúa como si no hubiese pasado nada, gracias a la redundancia completa de servicios/dispositivos.
    \item \textbf{Alta disponibilidad pasiva}: En caso de error, los servidores activos serían los que reciben el impacto del problema y hay que escalar los servidores pasivos a modo activo para que comiencen a funcionar otorgando el servicio. Como se puede presuponer, esta modificación puede ser realizada de manera automática o de manera manual (lo que llevaría algo de tiempo, y por tanto el servicio se vería afectado).
\end{itemize}
    \graphicspath{{../../../temas_comunes/arquitectura_sistemas/img/}}
    


\chapter{Arquitectura de instalación}
A la hora de realizar la instalación de un sistema de información, y teniendo en cuenta que es un pilar fundamental de la empresa, habrá que tomar ciertas decisiones desde el punto de vista de sistemas hardware.

De estas decisiones se encargará el \textbf{\textit{sysadmin}}, o administrador de sistemas, pero tendrá que tener ayuda de los especialistas de la aplicación de sistemas de información, así como de determinar una decisión desde el punto de vista empresarial.

Entre las tareas que hay que tener en cuenta, se podrían destacar las siguientes:

\begin{itemize}
    \item \textbf{Hardware}: Determinar el hardware en donde se va a realizar la instalación. Hoy en día existen distintas alternativas, como son:
    \begin{itemize}
        \item \textbf{Hardware dedicado}: Un servidor propio para el sistema, donde se realizará la instalación sólo para este servicio.
        \item \textbf{Máquina Virtual}: El servicio será instalado en una máquina virtual a través de un sistema de virtualización profesional. El servicio es agnóstico al hardware, por lo que no sabrá si está virtualizado o no. Hoy en día suele ser la opción más común dadas las ventajas que ofrecen.
    \end{itemize}

    \item \textbf{Elección del sistema de información}: Esta es una tarea importante y que no se puede dejar de lado, ya que la decisión de optar por una herramienta u otra puede suponer un problema a futuro.

    Es por eso que se debe realizar un estudio de mercado entre las distintas posibilidades y tener en cuenta, al menos, las siguientes situaciones:

    \begin{itemize}
        \item \textbf{Estado actual de la herramienta}: Es importante saber si la herramienta analizada cuenta con un desarrollo continuado, si existe una empresa o grupo de desarrollo por detrás que la apoye; que no esté abandonada; que sea una herramienta con buena aceptación y críticas...
        \item \textbf{Coste de licencia}: Es una herramienta que cuenta con una licencia a perpetuidad bajo un coste determinado\textbf{,} tiene licencia por el número de usuarios que acceden a ella\textbf{,} es una herramienta de Software Libre ...
        \item \textbf{Seguridad, actualizaciones y parches}: La herramienta cuenta con actualizaciones de seguridad periódicas; no ha habido fallos de seguridad graves en las últimas versiones; cuando se detectan fallos las actualizaciones aparecen de manera rápida y efectiva; las actualizaciones y/o parches son gratuitos o de pago...
        \item \textbf{Coste de mantenimiento}: Existe un coste asociado al mantenimiento de la aplicación, pero este puede ser por parte del sistema (realizar actualizaciones, aumento de recursos...) o por pago de licencias anuales, por versiones...
        \item \textbf{Posibilidades de personalización}: Existe la posibilidad de personalizar la herramienta; parametrizar opciones propias que se ajustan a la empresa; creación de módulos/plugins propios para mejorar/expandir la funcionalidad de la herramienta, ...

        \item \textbf{Conocimientos sobre la herramienta}: Dentro de la organización se cuenta con conocimientos acerca del uso/instalación/administración de la herramienta, debe ser subcontratado o existe la posiblidad de adquirir conocimiento mediante cursos o manuales.
    \end{itemize}

    \item \textbf{Sistema operativo}: Dependiendo del sistema de información elegido, se deberá instalar en un sistema operativo u otro. En este punto se pueden tener en cuenta también los puntos anteriores sobre el conocimiento para la toma de decisiones.

    \item \textbf{Método de instalación}: Hoy en día existen distintas posibilidades a la hora de instalar servicios, por lo que es importante realizar una buena decisión:
    \begin{itemize}
        \item \textbf{Tradicional}: Vamos a llamar sistema tradicional a aquel que se realiza mediante un instalador que realiza la instalación en el sistema operativo, que no suele dar demasiadas opciones de configuración durante el proceso.
        \item \textbf{Contenedores}: Hoy día existen servicios que podemos instalar a través de sistemas de contenedores (como puede ser Docker), los cuales suelen facilitar la instalación, así como la posibilidad de que también sea un sistema multicapa.
        \item \textbf{Por capas}: La instalación multicapa puede resultar un poco más compleja y la aplicación/servicio debe poder permitir realizarlo. Aunque inicialmente pueda suponer un poco más de esfuerzo, pero a la larga puede suponer una gran ventaja como es la \textbf{alta disponibilidad}.

        Pasar de un sistema “monolítico” a un sistema por capas es posible, pero una vez más dependeremos de la aplicación. Por otro lado, si desde el inicio se ha creado un sistema multicapa, escalarlo será más sencillo que realizar la migración cuando ya esté en uso.
    \end{itemize}

\end{itemize}


\section{Arquitectura multicapa}

Un sistema informático multicapa es aquel que hace uso de una arquitectura \textbf{cliente-servidor} en las que existe una separación física entre las distintas funciones que tiene una aplicación o servicio.

Normalmente se suele representar como una arquitectura en tres niveles, siendo estos:

\begin{center}
    \includegraphics[width=0.25\linewidth]{capas.png}
\end{center}

\begin{itemize}
    \item \textbf{Capa de presentación}: Es la que ve el usuario (también se la denomina «capa de usuario»), presenta el sistema al usuario, le comunica la información y captura la información del usuario en un mínimo de proceso (realiza un filtrado previo para comprobar que no hay errores de formato).

    También es conocida como interfaz gráfica y debe tener la característica de ser «amigable» (entendible y fácil de usar) para el usuario. Esta capa se comunica únicamente con la capa de negocio.

    Hoy en día lo habitual es que hagamos uso de servicios web, por lo que la capa de presentación es \textbf{la web que estamos visualizando}. En el caso de aplicaciones móviles, es \textbf{la propia aplicación que tenemos instalada en el dispositivo}.

    \item \textbf{Capa de negocio}: es donde residen los programas que se ejecutan, se reciben las peticiones del usuario y se envían las respuestas tras el proceso. Se denomina capa de negocio (e incluso de lógica del negocio) porque es aquí donde se establecen todas las reglas que deben cumplirse.

    Esta capa se comunica con la capa de presentación, para recibir las solicitudes y presentar los resultados, y con la capa de datos, para solicitar al gestor de base de datos almacenar o recuperar datos de él. También se consideran aquí los programas de aplicación.

    En este tipo de arquitecturas, esta capa es la que se denomina \textbf{\textit{backend}}, y lo habitual es que sea un sistema al que llamamos a través de una \href{https://es.wikipedia.org/wiki/API}{API} (del inglés, \textit{application programming interface}, o interfaz de programación de aplicaciones).

    \item \textbf{Capa de datos}: es donde residen los datos y es la encargada de acceder a los mismos. Está formada por uno o más gestores de bases de datos que realizan todo el almacenamiento de datos, reciben solicitudes de almacenamiento o recuperación de información desde la capa de negocio.
\end{itemize}


\chapter{Escalado vertical vs horizontal}

Teniendo en cuenta todo lo dicho hasta ahora, cuando un sistema empieza a tener problemas de rendimiento deberemos abordar el problema y plantearnos cómo solucionarlo. De no hacerlo, se corre el peligro de que el servicio se vea interrumpido y por tanto perder tiempo de trabajo.

Antes de realizar ninguna modificación habría que analizar qué es lo que está sucediendo (para ello es importante tener un buen sistema de monitorización), y de esta manera saber en qué punto existe el problema y así poder solucionarlo.

Dependiendo de las decisiones tomadas durante la instalación, y tras lo visto previamente, podremos abordarlo de dos maneras diferentes.

\section{Escalado vertical}

Cuando se escala verticalmente un sistema lo que se va a realizar es el \textbf{añadir más recursos al nodo que está teniendo problemas}. Tras el análisis previo realizado se añadirán los recursos necesarios (más RAM, discos duros más rápidos, aumentar el número de procesadores/cores).

Comúnmente también se dice “meter más hierro”, porque antiguamente lo que se hacía era incrementar los recursos hardware del sistema. Hoy en día en sistemas virtualizados, estos recursos se pueden modificar, dependiendo del virtualizador, en caliente, por lo que no sería necesario reiniciar el servicio.

Es el sistema más simple, ya que incrementando los recursos se espera que el problema se apacigüe o desaparezca, aunque esto no tiene por qué ser siempre así.





    \part{ERP}
Los sistemas de planificación de recursos empresariales (\textbf{ERP}, por sus siglas en inglés, \textit{enterprise resource planning}) son los sistemas de información gerenciales que integran y manejan muchos de los negocios asociados con las operaciones de producción y de los aspectos de distribución de una compañía en la producción de bienes o servicios.


\part{CRM}





%    \part{Odoo}
%    \chapter{Introducción}
\href{https://es.wikipedia.org/wiki/Odoo}{Odoo}, antes conocido como \textit{OpenERP} es un software de planificación de recursos empresariales con licencia dual. Existe una versión de código abierto y una versión con licencia comercial con características y servicios exclusivos.

Odoo también cuenta con un apartado de CRM (en inglés \textit{customer relationship management}, o gestión de relaciones con el cliente), pudiendo también crear una web de comercio electrónico, facturación, ...


\chapter{Instalación}

Tal como hemos visto en el tema anterior, la instalación de un sistema se puede realizar de distintas maneras, por lo que deberemos atender a las necesidades del proyecto para realizar una instalación adecuada.

En nuestro caso, se va a optar por realizar una instalación a través de servicios Docker, de esta manera podemos realizar pruebas con distintas versiones (tanto de Odoo como de la base de datos).

\infobox{La alternativa sería realizar la instalación en una máquina virtual o haciendo uso de los distintos instaladores que existen en la \href{https://www.odoo.com/es_ES/page/download}{web oficial}}

Para realizar la instalación seguiremos las indicaciones que aparecen en la web de \href{https://hub.docker.com/_/odoo}{Docker Hub} haciendo pequeñas modificaciones.

\section{Servicios independientes}

Es el sistema más básico, que requiere de levantar dos contenedores Docker:
\begin{itemize}
    \item Contenedor de base de datos \textbf{PostgreSQL}. Podremos elegir entre las distintas versiones del gestor de base de datos, pero haremos caso a las recomendaciones de la web. Habría que tener especial cuidado con el usuario y la contraseña que utilizamos.

    En este caso también se ha expuesto el puerto 5432 que es el puerto por defecto de PostgreSQL:

\begin{mycode}{Crear y arrancar el contenedor de la base de datos}{console}{}
ruben@vega:~$ docker run -d -e POSTGRES_USER=odoo \
-e POSTGRES_PASSWORD=odoo -e POSTGRES_DB=postgres \
-p 5432:5432 --name odoo_db postgres:15
\end{mycode}

    \item Contenedor con el propio \textbf{Odoo}. Este contenedor tendrá los servicios y librerías necesarias para poder hacer funcionar la aplicación web.

\begin{mycode}{Crear y arrancar el contenedor de la base de datos}{console}{}
ruben@vega:~$ docker run -p 8069:8069 --name odoo \
--link odoo_db:db -t odoo
\end{mycode}
\end{itemize}


\section{Docker Compose}

Docker Compose es una herramienta para correr servicios multi-contenedor y se crea a través de un fichero en formato YAML. Es una manera de crear,parar,reconstruir una arquitectura de servicios de manera rápida y sencilla.

Se debe crear un fichero \configfile{compose.yaml} y lo ideal es que esté dentro de un directorio con el nombre del “stack de servicios”, ya que coge el directorio como parte del nombre a la hora de crear los contenedores.

\begin{mycode}{Contenido de fichero compose.yaml}{yaml}{}
version: '3.1'
services:
  web:
    image: odoo:16.0
    depends_on:
      - db
    ports:
      - "8070:8069"
  db:
    image: postgres:15
    environment:
      - POSTGRES_DB=postgres
      - POSTGRES_PASSWORD=odoo
      - POSTGRES_USER=odoo
    ports:
      - "5433:5432"
\end{mycode}

Para arrancar los servicios se realiza, desde el mismo directorio donde se encuentra el fichero, con el siguiente comando

\begin{mycode}{Levantar docker compose}{console}{}
ruben@vega:~$ docker compose up
\end{mycode}

\errorbox{En Julio del 2023 se migró a Compose v2, tal como se indica en la \href{https://docs.docker.com/compose/}{web oficial}. Dependiendo de la versión que tengamos instalada será “docker compose up” o “docker-compose up”}


\section{Herramientas extra}
Para poder acceder a la base de datos podemos hacer uso de un cliente externo. De esta manera no tendremos que entrar al contenedor y tendremos un interfaz gráfico con el que poder administrarla.

Podemos utilizar:
\begin{itemize}
    \item \textbf{DBeaver}: Es una aplicación de escritorio que permite conectarnos a distintos SGBDs. Existe versión \href{https://dbeaver.io/}{community} y otra con \href{https://dbeaver.com/buy/}{licencia} que permite también conectarse a bases de datos NO-SQL.

    \item \textbf{\href{https://www.pgadmin.org/}{pgAdmin}}: Es una aplicación que permite administrar PostgreSQL a través del servidor web.
\begin{mycode}{Crear y arrancar el contenedor pgAdmin}{console}{}
ruben@vega:~$ docker run -p 8090:80 \
-e 'PGADMIN_DEFAULT_EMAIL=user@domain.com' \
-e 'PGADMIN_DEFAULT_PASSWORD=SuperSecret' \
--name pgadmin4 -d dpage/pgadmin4
\end{mycode}
\end{itemize}



%    \part{Cómo crear una API}
    %Cómo crear una API con Laravel
%     - crear proyecto
%    - crear esquema E/R
%    - crear controlador
%    - crear funciones


    % ESTO IGUAL EN DESARROLLO DE INTERFACES?
%    \part{Crear frontend con Angular}
%    - crear proyecto
%    - crear primeras vistas
%    - llamar a la API

%    \part{Introducción a Docker}
%    \graphicspath{{../../../otros/Docker/}}
%    \chapter{Introducción}

Hoy en día es muy habitual hacer uso de los sistemas de contenedores, el más conocido es \href{https://docs.docker.com/}{Docker}, en el mundo del desarrollo de \textit{software}. Este sistema trae consigo una serie de ventajas que veremos más adelante, que nos permite asegurar, entre otras cosas, que las versiones utilizadas en el entorno de producción son las mismas que durante las etapas de desarrollo.

En este documento se va a explicar cómo realizar la instalación y configuración de un sistema basado en contenedores Docker para poder arrancar servicios, y ciertas configuraciones que son necesarias conocer.

\chapter{Sistemas de contenedores}

Los sistemas de contenedores son un método de virtualización (conocido como “virtualización a nivel de sistema operativo”), en el que se permite ejecutar sobre una capa virtualizadora del núcleo del sistema operativo distintas instancias de “espacio de usuario”.

Este “espacio de usuario” (donde se ejecutarán aplicaciones, servicios...) se les denomina \textbf{contenedores}, y aunque pueden ser como un servidor real, están bajo un mecanismo de aislamiento proporcionado por el \textit{kernel} del sistema operativo, y sobre el que se pueden aplicar límites de espacio, recursos de memoria, de acceso a disco...

\infobox{\textbf{Un contenedor es un espacio de ejecución de servicios al que se les puede aplicar límites de recursos (como la memoria, el acceso a disco...)}}

Desde el punto de vista del usuario, que un servicio se ejecute en una máquina virtual o en un contenedor es indistinguible. En cambio, desde el punto de vista de un administrador de sistemas o de un desarrollador, el uso de contenedores trae consigo una serie de ventajas que veremos en apartados posteriores.


\section{Un poco de historia}
Aunque está muy en boga el despliegue de aplicaciones haciendo uso de contenedores, no es un concepto nuevo, ya que lleva existiendo desde la década de los 80 en sistemas UNIX con el concepto de \href{https://es.wikipedia.org/wiki/Chroot}{chroot}.

\textbf{Chroot}, también conocido como “jaulas chroot”, permitían ejecutar comandos dentro de un directorio sin que, en principio, se pudiese salir de dicha ruta. Tenía muy pocas restricciones de seguridad, pero era un primer paso al sistema de contenedores.

\href{https://es.wikipedia.org/wiki/LXC}{LXC} nace en 2008 utilizando distintas funcionalidades del kernel Linux para proveer un entorno virtual donde poder ejecutar distintos procesos y tener su propio espacio de red. Con LXC nacen distintas herramientas para controlar estos contenedores, así como para crear plantillas y una \textbf{API que permite interaccionar con LXC} desde distintos lenguajes de programación.

Ha habido otras tecnologías en Linux, como \href{https://es.wikipedia.org/wiki/OpenVZ}{OpenVz}, pero luego nos centraremos en Docker, ya que es lo más conocido actualmente.


\section{Qué es un contenedor y cómo se crea}

Para entender qué es un contenedor dentro de la infraestructura Docker y cómo se crea, tenemos que diferenciar distintos conceptos:
\begin{itemize}
    \item \textbf{Imagen Docker}
    \item \textbf{Contenedor Docker}
\end{itemize}

A continuación se van a detallar en profundidad.

\subsection{Imágenes Docker}

Para crear un contenedor necesitamos hacer uso de una \textbf{“imagen”, que es un archivo inmutable (no modificable) que contiene el código de la aplicación que queremos ejecutar y todas sus dependencias necesarias}, para que pueda ser ejecutada de manera rápida y confiable independientemente del entorno en el que se encuentre.

Las imágenes, debido a su origen \textbf{sólo-lectura}, se pueden considerar como \textbf{“plantillas”}, que son la representación de una aplicación y el entorno necesario para ser ejecutada en un momento específico en el tiempo. \textbf{Esta consistencia es una de las grandes características de Docker}.

\infobox{\textbf{Una imagen contiene el código y las dependencias que se necesita al crear un contenedor para ser ejecutado, independiente del entorno donde se ejecuta.}}

Una imagen puede ser creada utilizando otras imágenes como base. Por ejemplo, la imagen de \href{https://hub.docker.com/_/phpmyadmin}{PHPMyAdmin} empaqueta la aplicación PHPMyAdmin sobre la imagen \textbf{PHP} (versión 8.1-apache), que a su vez hace uso de la imagen \textbf{Debian} (versión 11-slim).


\begin{center}
    \includegraphics[width=0.9\linewidth]{img/docker/imagen1.png}
    \captionof{figure}{Jerarquía de imágenes usadas por PHPMyAdmin. \href{https://hub.docker.com/layers/library/phpmyadmin/latest/images/sha256-79e38dd8b2ab0e92505aa92040fa49dce4fa921a977b6ce4d030a63b4f120009?context=explore}{Fuente}}
\end{center}

A las imágenes creadas se les suele añadir etiquetas (\textbf{tags}) para diferenciar versiones o características internas. Cada creador determina las etiquetas que le interesa crear. Por ejemplo:
\begin{itemize}
    \item \textbf{latest}: Se le denomina a la última imagen creada.
    \item php:\textbf{8.1-apache}: Indica que en esta imagen PHP la versión es la 8.1 y además cuenta con Apache.
\end{itemize}

Podemos utilizar imágenes públicas descargadas a través de un \textbf{\textit{registry}} público, que no es otra cosa que un repositorio de imágenes subidas por la comunidad. El \textit{registry} principal más utilizado es \href{https://hub.docker.com/}{Docker Hub}.

\infobox{\textbf{Las imágenes Docker  pueden ser públicas o privadas y se almacenan en un repositorio llamado \underline{registry}, siendo el más conocido Docker Hub}}

\textbf{Se pueden crear nuestras propias imágenes privadas}, que pueden ser almacenadas en nuestros equipos o a través de un \textbf{registry privado} que podemos crear (también existen servicios de pago).


\subsection{Contenedores Docker}
Un contenedor Docker es un \textbf{entorno de tiempo de ejecución virtualizado donde los usuarios pueden aislar aplicaciones}. Estos contenedores son unidades compactas y portátiles a las que se les puede aplicar un sistema de limitación de recursos.

\infobox{\textbf{Un contenedor se crea a través de una imagen, es la versión ejecutable de la misma que se crea en un entorno virtualizado}}

Un contenedor se crea a través de una imagen y es la versión ejecutable de la misma. Lo que se hace es crear una capa de escritura sobre la imagen inmutable, donde se podrán escribir datos. Se pueden crear un número ilimitados de contenedores haciendo uso de la misma imagen base.

\vspace{-20pt}
\begin{center}
    \includegraphics[width=0.6\linewidth]{img/docker/contenedor.png}
    \captionof{figure}{Imagen “interna” de un contenedor}
\end{center}
\vspace{-15pt}

\textbf{La capa de escritura no es persistente} y se pierde al eliminar el contenedor, es decir, \textbf{los datos de un contenedor se eliminan al borrar el contendor}. Para evitar este comportamiento se puede hacer uso de un \hyperlink{volumen_persistente_datos}{volumen persistente de datos}, de esta manera esos datos no se pierden.

\errorbox{\textbf{Los datos creados dentro de un contenedor se borran al eliminar el contenedor}}


\section{Contenedores vs. Máquinas virtuales}

El uso de máquinas virtuales está muy extendido gracias a que cada vez es más sencillo crearlas. Esto no quiere decir que siempre sea la mejor opción, por lo que  se va a realizar una comparativa teniendo en cuenta distintos aspectos a la hora de realizar un desarrollo con máquinas virtuales y con sistemas de contenedores.


\subsection{Infraestructura}

La creación de máquinas virtuales nos permite crear entornos aislados en los que poder instalar el Sistema Operativo que más nos interese y con ello poder instalar el software y los servicios que necesitemos.

Las máquinas virtuales se virtualizan a nivel de hardware, donde debe existir un Sistema Operativo con Hypervisor que permita dicha virtualización. Por otro lado, los contenedores se virtualizan en la capa de aplicación, haciendo que este sistema sea mucho más ligero, permitiendo utilizar esos recursos en los servicios que necesitamos hacer funcionar dentro de los contenedores.

\vspace{-15pt}
\begin{center}
    \includegraphics[width=0.85\linewidth]{img/docker/docker\_vs\_vm.png}
    \captionof{figure}{Infraestructura Máquinas Virtuales vs Docker}
\end{center}
\vspace{-15pt}


En la imagen se puede apreciar una comparativa diferenciando cómo quedaría una infraestructura de 3 aplicaciones levantadas en distintas máquinas virtuales o en distintos contenedores.

Tal como se puede ver en la imagen, \textbf{al tener cada servicio en una máquina virtual separada}, se va a tener que virtualizar todo el Sistema Operativo en el que se encuentre, con el consiguiente \textbf{coste de recursos (memoria RAM y disco duro) y con el coste en tiempo de tener que realizar la configuración y securización del mismo}.

\infobox{\textbf{Usando contenedores la infraestructura se simplifica notáblemente}}

Por otro lado, en un sistema de contenedores, cada contenedor es un servicio aislado, en el que sólo tendremos que preocuparnos (en principio) de configurar sus parámetros.


\subsection{Ventajas durante el desarrollo}

A la hora de desarrollar una aplicación es habitual hacer pruebas utilizando distintas versiones de librerías, \textit{frameworks} o versiones de un mismo lenguaje de programación. De esta manera, podremos ver si nuestra aplicación es compatible.

Cuando se hace uso de una máquina virtual dependemos de las versiones que tiene nuestra distribución y es posible que no podamos instalar nuevas versiones u otras versiones en paralelo.

Por ejemplo, la última versión de PHP actualmente es la 8.2.4 y de Apache la 2.4.56:

\begin{itemize}
    \item En\textbf{ Debian 11} sólo se puede instalar PHP 7.4 y Apache 2.4.54.
    \item En \textbf{Ubuntu 22.04} la versión de PHP es la 8.1 y la de Apache la 2.4.52.
\end{itemize}

Con Docker, podremos levantar contenedores con distintas versiones del servicio que nos interese en paralelo para comprobar si nuestra aplicación/servicio es compatible.

\infobox{\textbf{Con Docker es posible levantar servicios con distintas versiones en paralelo}}

Por otro lado, si un desarrollador quiere utilizar un sistema operativo distinto, no se tendrá que preocupar de si su distribución tiene las mismas versiones. O en el caso de usar Windows/Mac, no tener que estar realizando instalaciones de las versiones concretas.


\subsection{Ventajas durante la puesta en producción}
Ligado al apartado anterior, durante la puesta en producción es obligatorio hacer uso de las mismas versiones utilizadas durante el desarrollo para asegurar la compatibilidad.

\errorbox{\textbf{Para asegurar la compatibilidad en producción, siempre se debe usar la misma versión de los servicios que en desarrollo}}

Si tenemos un servidor que no está actualizado, o en el mismo servidor tenemos distintas aplicaciones que requieren utilizar distintas versiones de software, en un entorno de máquinas virtuales se hace muy complejo, ya que lo habitual será tener que instalar nuevas máquinas virtuales.

\warnbox{\textbf{No siempre es posible tener distintas versiones del mismo software en un mismo servidor}}

En un entorno con contenedores, al igual que se ha comentado antes, esto no es problema.

\subsection{Rapidez en el despliegue}

Ligado a todo lo anterior, realizar el despliegue de un entorno de desarrollo/producción es más rápido utilizando contenedores, sin importar el sistema operativo en el que nos encontremos.

\infobox{\textbf{El despliegue con contenedores es más rápido.}}

Más adelante se verá cómo realizar el despliegue de distintos servicios haciendo uso de un único comando.


\chapter{Docker}

\href{https://www.docker.com/}{Docker} es un proyecto de Software Libre nacido en 2013 que permite realizar el despliegue de aplicaciones y servicios a través de contenedores de manera rápida y sencilla, tal como veremos más adelante.

Estos contenedores proporcionan una capa de abstracción y permiten aislar las aplicaciones del resto del sistema operativo a través del uso de ciertas características del kernel Linux.

Dentro del contenedor, se puede destacar el aislamiento a nivel:

\begin{itemize}
    \item Árbol de procesos
    \item Sistemas de ficheros montados
    \item ID de usuario
    \item Aislamiento de recursos (CPU, memoria, bloques de E/S...)
    \item Red aislada
\end{itemize}

Al igual que sucede con otro tipo de \textit{software},  para que Docker haga uso de todas estas características, está construido haciendo uso de otras aplicaciones y servicios.


\begin{center}
    \includegraphics[width=0.75\linewidth]{img/docker/docker_interfaces.png}
    \captionof{figure}{Tecnologías usadas por Docker. Fuente: \href{https://en.wikipedia.org/wiki/File:Docker-linux-interfaces.svg}{Wikipedia}}
\end{center}

En el 2015 la empresa Docker creó la \textbf{\textit{\href{https://en.wikipedia.org/wiki/Open_Container_Initiative}{Open Container Initiative}}}, proyecto actualemente bajo la Linux Foundation, con la intención de diseñar un estándar abierto para la virtualización a nivel de sistema operativo.

\section{Instalación}

Dependiendo del sistema operativo en el que nos encontremos, Docker tiene la opción de instalarse de distintas maneras. En sistemas operativos GNU/Linux cada distribución tiene un paquete para poder realizar la instalación del mismo.

\begin{mycode}{Instalación de Docker en Ubuntu}{console}{}
root@vega:~# apt install docker.io
\end{mycode}

\errorbox{El nombre del paquete en Ubuntu y Debian es \textbf{docker.io}}

En sistemas Windows y MacOS existe la opción de instalar \href{https://docs.docker.com/get-docker/}{Docker Desktop}, una versión que utiliza una máquina virtual para simplificar la instalación en estos sistemas.

\section{Configuración y primeros pasos}

Tras realizar la instalación veremos cómo el servicio Docker ha levantado un interfaz nuevo en nuestra máquina, cuya IP es \textbf{172.17.0.1/16}, siendo el direccionamiento por defecto.

\begin{mycode}{Nueva IP en el equipo}{console}{}
root@vega:~# ip a
...
3: docker0: <BROADCAST,MULTICAST,UP,LOWER_UP> mtu 1500 qdisc noqueue
    link/ether 02:42:9c:1f:e2:90 brd ff:ff:ff:ff:ff:ff
    inet 172.17.0.1/16 brd 172.17.255.255 scope global docker0
      valid_lft forever preferred_lft forever
\end{mycode}

Esta IP hará de \textbf{puente} (similar a lo que sucede con las máquinas virtuales) cuando levantemos contenedores nuevos. Los contenedores estarán dentro de ese direccionamiento 172.17.0.0/16, por lo tanto, aislados de la red principal del equipo.

\infobox{\textbf{Los contenedores que levantemos estarán en la red 172.17.0.0/16}}

El comando \commandbox{docker} tiene muchas opciones, por lo que es recomendable ejecutarlo sin parámetros. De esta manera se pueden ver todas las opciones y una ayuda simplificada para cada una de ellas.

\begin{mycode}{Algunas de las opciones del comando docker}{console}{}
root@vega:~# docker
Usage:  docker [OPTIONS] COMMAND

Management Commands:
builder     Manage builds
completion  Generate the autocompletion script for the specified shell
config      Manage Docker configs
container   Manage containers
context     Manage contexts
image       Manage images
manifest    Manage Docker image manifests and manifest lists
network     Manage networks
node        Manage Swarm nodes
plugin      Manage plugins
secret      Manage Docker secrets
service     Manage services
stack       Manage Docker stacks
swarm       Manage Swarm
system      Manage Docker
trust       Manage trust on Docker images
volume      Manage volumes

Commands:
...
\end{mycode}

Para cada una de estas opciones, se le puede añadir el parámetro \inlineconsole{--help} para mostrar la ayuda. Hay un segundo apartado que se ha cortado, en el que se incluyen más comandos.

Para asegurar que el servicio Docker está funcionando, podemos hacer uso de \commandbox{docker info}, que nos mostrará mucha información acerca del servicio. Pero si lo que queremos es comprobar si tenemos algún contenedor corriendo, es más sencillo hacer \commandbox{docker ps} (que es la versión simplificada de \commandbox{docker container ls} ):

\begin{mycode}{Comprobar estado de Docker y contenedores levantados}{console}{}
root@vega:~# docker ps
CONTAINER ID   IMAGE     COMMAND   CREATED   STATUS    PORTS     NAMES

root@vega:~# docker container ls
CONTAINER ID   IMAGE     COMMAND   CREATED   STATUS    PORTS     NAMES
\end{mycode}

En este caso, como no hay ningún contenedor levantado, sólo muestra las cabeceras de las columnas del listado.

\section{Levantando nuestro primer contenedor}
Es momento de crear nuestro primer contenedor. Para ello, dado que se está usando la consola, hay que hacer uso del comando \commandbox{docker} con una serie de parámetros. En este caso se ha optado por levantar el servicio \textbf{Apache HTTPD}:

\begin{mycode}{Levantando el primer contenedor}{console}{}
root@vega:~# docker run -p 80:80 httpd
AH00558: httpd: Could not reliably determine the server's ...
AH00558: httpd: Could not reliably determine the server's ...
[Fri Mar 24 18:25:14.194246 2023] [mpm_event:notice] ...
[Fri Mar 24 18:25:14.194347 2023] [core:notice] [pid  ...
172.17.0.1 - - [24/Mar/2023:18:25:41 +0000] "GET / HTTP/1.1" 304 -
\end{mycode}

Vemos los logs del servicio Apache al arrancar y si vamos al navegador a la dirección \href{http://localhost}{http://localhost} muestra lo siguiente:

\begin{center}
    \includegraphics[width=0.6\linewidth]{img/docker/apache.png}
\end{center}

Y para entender lo que hace el comando, los parámetros son:
\begin{itemize}
    \item \textbf{docker}: Cliente de consola para hacer uso de Docker.
    \item \textbf{run}: Ejecuta un comando en un nuevo contenedor (y si no existe lo crea).
    \item \textbf{-p 80:80}: Publica en el puerto 80 del servidor el puerto 80 utilizado en el contenedor. Se puede pensar que es como hacer un \textbf{port-forward} en un firewall.
    \item \textbf{httpd}: Es la \textbf{imagen} del contenedor que se va a arrancar. En este caso, la imagen del servidor \href{https://hub.docker.com/_/httpd}{Apache HTTPD}.
\end{itemize}

Y si vemos qué muestra el estado de docker, vemos cómo aparece el contenedor levantado.

\begin{mycode}{Comprobar estado de Docker y contenedores levantados}{console}{{\scriptsize }}
root@vega:~# docker ps
CONTAINER ID   IMAGE     COMMAND       CREATED         STATUS         PORTS               NAMES
a1c3362b0d6c   httpd     "httpd-..."   3 seconds ago   Up 2 seconds   0.0.0.0:80->80/tcp  great
\end{mycode}

En la columna PORTS se puede apreciar cómo aparece que se ha levantado el puerto  \textbf{ \texttt{0.0.0.0:80} } (escucha en el puerto 80 para cualquier IP del sistema operativo) que es una redirección al puerto \textbf{\texttt{80/TCP}} interno del contenedor.


\section{Contenedores en \textit{background} y más opciones}
Tal como se puede ver en el ejemplo anterior, el contenedor se queda en primer plano, viendo los logs del Apache. Esto para ver qué es lo que está sucendiendo durante el desarrollo puede ser útil, pero lo ideal es que el contenedor arranque en modo \textbf{\textit{background}}, y cuando necesitemos vayamos a ver los logs.


A continuación se va a arrancar un nuevo contenedor de Apache con nuevos parámetros:
\begin{mycode}{Crear un contenedor Web en el puerto 8080}{console}{}
root@vega:~# docker run --name mi-apache -d -p 8080:80 httpd
\end{mycode}

Los nuevos parámetros son:
\begin{itemize}
    \item \textbf{\texttt{--name mi-apache}}: De esta manera se le da un nombre al contenedor, para poder identificarlo de manera rápida entre todos los contenedores creados.
    \item \textbf{\texttt{-d}}: Este parámetro es para hacer el \textbf{\textit{detach}} del comando, y de esta manera mandar a \textbf{\textit{background}} la ejecución del contenedor.
    \item \textbf{\texttt{-p 8080:80}}: Publica en el puerto 8080 del servidor el puerto 80 utilizado en el contenedor. Se puede pensar que es como hacer un \textbf{port-forward} en un firewall.
\end{itemize}

\section{Parar, arrancar y borrar contenedores}

Hasta ahora hemos aprendido a crear contenedores, pero en ciertos momentos nos puede interesar parar un contenedor que no estemos utilizando, o una vez haya cumplido su función, borrarlo.

\subsection{Parar contenedores}

Para parar un contenedor, debemos conocer el nombre del mismo o su ID (que es único). Estos datos los podemos conocer a través del comando  \commandbox{docker ps}.

Con esto, podemos ejecutar:

\begin{mycode}{Parar un contenedor}{console}{{\small }}
root@vega:~# docker stop mi-apache
\end{mycode}

\subsection{Arrancar un contenedor parado}

Una vez parado un contenedor, o al reiniciar el servidor, si queremos arrancar un contenedor parado, debemos conocer también su ID o nombre.

Para visualizar todos los contenedors (tanto los arrancados como los parados), lo podemos hacer a través del comando \commandbox{docker ps -a}.

Gracias a ese listado, podemos volver a arrancar un contenedor que esté parado con \commandbox{docker start mi-apache}, siendo “mi-apache” el contenedor que queremos arrancar.

\subsection{Borrar un contenedor}
Si queremos borrar un contenedor, éste debe estar parado, ya que Docker no nos va a dejar borrar un contenedor que está en ejecución.

Es interesante borrar contenedores que hayamos creado de pruebas o contenedores que ya no se vayan a utilizar más, para de esta manera liberar recursos.

Para borrarlo, similar a los casos anteriores, se hará con \commandbox{docker rm mi-apache}.

\section{Variables de entorno}
Algunos contenedores tienen la opción de recibir variables de entorno al ser creados. Estas variables pueden afectar al comportamiento del contenedor, o para ser inicializado de alguna manera distinta a las opciones por defecto.

El creador de la imagen Docker puede crear las variables de entorno que necesite para después utilizarlas en su aplicación. A modo de ejemplo, se va a utilizar la imagen de la aplicación \href{https://hub.docker.com/_/phpmyadmin}{PHPMyAdmin}.

A continuación se van a crear 2 contenedores de PHPMyAdmin, diferenciados por el puerto, el nombre, y la variable de entorno \textbf{PMA\_ARBITRARY}:
\begin{itemize}
    \item El primer contenedor va a estar en el puerto 8081, se le va a dar el nombre “myadmin-1” y no va a tener la variable de entorno incializada.
    \item El segundo contenedor va a estar en el puerto 8082, se va a llamar “myadmin-2” y va a tener la variable \textbf{PMA\_ARBITRARY} inicializada a “1”, tal como aparece en la \href{https://hub.docker.com/_/phpmyadmin}{documentación de la imagen de PHPMyAdmin}.
\end{itemize}

Para ello, se han ejecutado los siguientes comandos:

\begin{mycode}{Creación de dos contenedores PHPMyAdmin}{console}{{\footnotesize }}
root@vega:~# docker run --name myadmin-1 -d -p 8081:80 phpmyadmin

root@vega:~# docker run --name myadmin-2 -e PMA_ARBITRARY=1 -d -p 8082:80 phpmyadmin
\end{mycode}

Tal como se puede ver, al segundo contenedor se le ha pasado un nuevo parámetro \textbf{\texttt{-e}}, que significa que lo que viene a continuación es una variable de entorno (en inglés \textbf{environment}). En este caso, la variable de entorno es \textbf{PMA\_ARBITRARY} que se ha inicializado a \textbf{1}.

Si ahora en nuestro navegador web apuntamos al puerto 8081 y al puerto 8082 de la IP de nuestro servidor, veremos cómo existe una ligera diferencia en el formulario que nos muestra la web.

En el formulario del puerto 8081 (donde no hemos inicializado la variable) sólo podemos indicar el usuario y la contraseña. Por el contrario, en el formulario del puerto 8082, al inicializar la variable \textbf{PMA\_ARBITRARY}, y tal como nos dice la documentación de la imagen, nos permite indicar la IP del servidor MySQL al que nos queremos conectar.

{
    \begin{minipage}{0.43\linewidth}
        \vspace{-11pt}
        \includegraphics[width=\linewidth]{img/docker/phpmyadmin1.png}
    \end{minipage}
    \hfill
    \begin{minipage}{0.43\linewidth}
        \vspace{-11pt}
        \includegraphics[width=\linewidth]{img/docker/phpmyadmin2.png}
    \end{minipage}

    \begin{center}
        \vspace{-18pt}
        {\footnotesize A la izquierda formulario del puerto 8081, sin variable inicializada. A la derecha, puerto 8082 con variable inicializada.}
    \end{center}
}

Dado que una variable puede afectar al comportamiento (o la creación) del servicio que levantemos a través de un contenedor, es importante leer la documentación e identificar las variables que tiene por si nos son de utilidad.

\infobox{\textbf{Es recomendable leer la documentación de las imágenes Docker para identificar las posibles variables de entorno que existen y ver si nos son útiles.}}




\hypertarget{volumen_persistente_datos}{}
\section{Volumen persistente de datos}
Hasta ahora hemos levantado un contenedor a través de una imagen que levanta el servicio Apache, mostrando su página por defecto. Podríamos escribir en el contenedor la página HTML que nos interesase, pero hay que entender que \textbf{los datos de un contenedor desaparecen cuando el contenedor se elimina}.

Para que los cambios realizados dentro de un contenedor se mantengan, tenemos que hacer uso de los denominados \textbf{volúmenes de datos}. Esto no es más que \textbf{hacer un montaje de una ruta del disco duro del sistema operativo dentro de una ruta del contenedor}.


Estos volúmenes que le asignamos al contenedor pueden ser de dos tipos:
\begin{itemize}
    \item \textbf{Sólo lectura}: Nos puede interesar asignar un volumen de sólo lectura cuando le pasamos ficheros de configuración o la propia web que queremos visualizar.
    \item \textbf{Lectura-Escritura}: En este caso se podrá escribir en el volumen. Por ejemplo, el directorio donde una base de datos guarda la información o una web donde deja imágenes subidas por usuarios.
\end{itemize}

De esta manera, tendremos que asignar el número de volúmenes necesarios a cada contenedor dependiendo de la imagen utilizada, el servicio que se levanta y lo que queremos hacer con los datos que le asignemos o generemos en el contenedor.

En la siguiente imagen se puede ver una infraestructura con dos contenedores y dos volúmenes:
\begin{itemize}
    \item \textbf{Contenedor Web}: Se le asigna un volumen en modo \textbf{sólo lectura} cuya ruta original está en \configdir{/opt/www-data}, que dentro del contenedor está en \configdir{/var/www/html}.
    \item \textbf{Contenedor MySQL}: Dado que los datos de la base de datos deben ser guardados, en este caso se le asigna un volumen que permite escritura. Por lo tanto, lo que se crea dentro del contenedor en \configdir{/var/lib/mysql} realmente se estará guardando en el sistema operativo anfitrión en \configdir{/opt/mysql-data}
\end{itemize}
\begin{center}
    \includegraphics[width=0.65\linewidth]{img/docker/volumes.png}
    \captionof{figure}{Ejemplo de dos volúmenes asignados a distintos contenedores}
\end{center}


\subsection{Añadir volumen de escritura al crear un contenedor}

Al añadir un volumen cuando creamos un contenedor hace que por defecto sea en modo lectura-escritura. Cualquier escritura realizada dentro del contenedor en la ruta especificada va a resultar en que el fichero se creará en la ruta indicada del sistema operativo anfitrión.

\begin{mycode}{Añadir volumen sólo lectura al crear un contenedor}{console}{{\small }}
root@vega:~# ls /opt/mysql-data
root@vega:~# docker run -d -p 3306:3306 --name mi-db \
    -v /opt/mysql-data:/var/lib/mysql \
    -e MYSQL_ROOT_PASSWORD=my-secret-pw \
    mysql:latest
root@vega:~# ls /opt/mysql-data
auto.cnf      client-key.pem      '#innodb_temp'      server-cert.pem   ...
\end{mycode}

Para este ejemplo se ha creado un contenedor usando la imagen de \href{https://hub.docker.com/_/mysql}{MySQL}, al que se le ha asignado un volumen (\textbf{por defecto se asigna permitiendo la escritura}) y un parámetro necesario para realizar la posterior conexión con contraseña.

\begin{itemize}
    \item \textbf{-v /opt/mysql-data:/var/lib/mysql}: A través del parámetro \textbf{-v} se le indica al contenedor que se le va a pasar un volumen. Posteriormente se le indica la ruta del sistema operativo anfitrión \configdir{/opt/mysql-data} que se montará dentro del contenedor en \configdir{/var/lib/mysql}.
    \item \textbf{-e MYSQL\_ROOT\_PASSWORD=my-secret-pw}: El parámetro “\textbf{-e}” sirve para pasarle al contenedor \textbf{variables de entorno}. En este caso, y tal como dice la \href{https://hub.docker.com/_/mysql}{web de la imagen MySQL}, esta es la manera de asignar la contraseña del usuario \textbf{root} durante la inicialización de la base de datos.
\end{itemize}

Tras crear el contenedor, y asegurarnos que está levantado haciendo uso del comando \commandbox{docker ps}, podemos realizar la conexión desde el sistema operativo anfitrión o desde cualquier otro lugar usando la contraseña indicada previamente.

\begin{mycode}{Añadir volumen sólo lectura al crear un contenedor}{console}{{\small }}
root@vega:~# mysql -h127.0.0.1 -uroot -P3306 -p
    Enter password:
    ...
    MySQL [(none)]> show databases;
    +--------------------+
    | Database           |
    +--------------------+
    | information_schema |
    | mysql              |
    | performance_schema |
    | sys                |
    +--------------------+
    4 rows in set (0,007 sec)
\end{mycode}

\subsection{Añadir volumen en modo sólo-lectura}
A continuación se van a explicar los pasos para levantar un contenedor que contiene una web simple creada en PHP, que está alojada en la ruta \configdir{/opt/www-data} del sistema operativo anfitrión.


\begin{mycode}{Añadir volumen sólo lectura al crear un contenedor}{console}{}
root@vega:~# ls /opt/www-data
index.php
root@vega:~# docker run -d -p 80:80 --name mi-web \
    -v /opt/www-data:/var/www/html:ro \
    php:8.2.4-apache
\end{mycode}

El parámetro nuevo asignado en la creación de este contenedor es:
\begin{itemize}
    \item \textbf{-v /opt/www-data:/var/www/html:ro}: Para indicarle que le vamos a asignar un volumen siendo la ruta real en el sistema de ficheros del sistema operativo anfitrión \configdir{/opt/www-data} y la ruta destino \textbf{dentro del contenedor} y que va a ser en modo \textbf{read-only} \configdir{/var/www/html}.
\end{itemize}

Más adelante, cuando veamos \hyperlink{entrar_en_contenedor}{cómo entrar dentro de un contenedor Docker}, se podría usar el comando para ir a la ruta dentro del contenedor y comprobar que efectivamente está en modo sólo-lectura.

\hypertarget{entrar_en_contenedor}{}
\section{Entrar dentro de un contenedor Docker}
Normalmente no suele ser necesario entrar dentro de un contenedor, ya que, tal como se ha dicho antes, cualquier modificación realizada dentro de él se perderá (salvo que sea dentro de un volumen persistente).

Aún así, para realizar pruebas o comprobaciones del correcto funcionamiento de una imagen puede ser interesante entrar dentro de un contenedor. Para ello, el comando a ejecutar es el siguiente:

\begin{mycode}{Acceder a un contenedor}{console}{}
root@vega:~# docker exec -it mi-db /bin/bash
\end{mycode}

Los parámetros utilizados son:
\begin{itemize}
    \item \textbf{exec}: Indicamos que queremos ejecutar un comando dentro de un contenedor que está corriendo.
    \item \textbf{-it}: Son dos parámetros unidos, que sirven para mantener la entrada abierta (modo interactivo) y crear una TTY (consola)
    \item \textbf{mi-db}: Es el nombre del contenedor al que se quiere entrar. También se puede indicar el \textbf{ID} del contenedor.
    \item \textbf{/bin/bash}: el comando que queremos ejecutar. En este caso, una shell \textbf{bash}. En algunos casos esta shell no está instalada y debemos usar \textbf{/bin/sh}
\end{itemize}

Hay que tener en cuenta que dentro de un contenedor está el mínimo software posible para que la aplicación/servicio funcione, por lo que habrá muchos comandos que no existan.


\section{Ciclo de vida de un contenedor Docker}
Un contenedor tiene un ciclo de vida que puede pasar por distintos estados. Para pasar entre estados se debe realizar a través de distintos comandos de Docker.

\begin{center}
    \includegraphics[width=\linewidth]{img/docker/lifecycle.png}
    \captionof{figure}{Estados de un contenedor}
\end{center}

En la imagen se representa los estados más básicos junto con los comandos para pasar entre ellos.


\section{Otros comandos útiles}

Para obtener toda la información de un contenedor, incluido su estado, volúmenes utilizados, puertos, ...

\begin{mycode}{Obtener toda la información de un contenedor}{console}{}
root@vega:~# docker inspect mi-db
\end{mycode}

Listar las imágenes descargadas en local. Al tener las imágenes en local, no hará falta volver a descargarlas, por lo que crear un nuevo contenedor que haga uso de una de ellas será mucho más rápido.

\begin{mycode}{Listado de imágenes en local}{console}{}
root@vega:~# docker image ls
REPOSITORY       TAG            IMAGE ID       CREATED        SIZE
php              8.2.4-apache   de23bf333100   3 days ago     460MB
httpd            latest         192d41583429   3 days ago     145MB
mysql            latest         483a8bc460a9   3 days ago     530MB
\end{mycode}

Borrar una de las imágenes que no se esté utilizando en ningún contenedor.

\begin{mycode}{Borrar una imagen concreta}{console}{}
root@vega:~# docker image rm httpd
\end{mycode}


Cuando un contenedor está en modo \textit{\textbf{detached}} no aparecen los logs, por lo que para poder visualizarlos tenemos un comando especial para ello.
\begin{mycode}{Ver los logs de un contenedor}{console}{}
root@vega:~# docker logs mi-web -f
172.17.0.1 - - [26/Mar/2023:18:05:29 +0000] "GET / HTTP/1.1" 200 248
172.17.0.1 - - [26/Mar/2023:18:05:29 +0000] "GET / HTTP/1.1" 200 248
172.17.0.1 - - [26/Mar/2023:18:05:29 +0000] "GET / HTTP/1.1" 200 248
\end{mycode}


Listar los volúmenes existentes en el sistema. El listado muestra los que se están utilizando en contenedores (activos o parados) o los que se han utilizado en contenedores que ya no existen.

\begin{mycode}{Listar volúmenes}{console}{{\small}}
root@vega:~# docker volume ls
DRIVER    VOLUME NAME
local     0d6c400a6407f5cdea81a2f0158222fdd87d7f3b3e2b5969ca466d743fc71f5c
local     1d2f52018e17af0689e070a55337154c1dd68517c54435ecc24d597f7509d43c
local     6b72797227ef4708ca23ee1dfcb4b651b42eeacefd4166b898407ad4aadda10c
\end{mycode}


Si queremos realizar una limpieza de todos los recursos (contenedores, imágenes,  volúmenes) que no se estén utilizando, se puede utilizar el siguiente comando.
\begin{mycode}{Borrar recursos que no estén activos}{console}{}
root@vega:~# docker system prune -a
WARNING! This will remove:
- all stopped containers
- all networks not used by at least one container
- all images without at least one container associated to them
- all build cache

Are you sure you want to continue? [y/N] y
\end{mycode}

\errorbox{\textbf{El comando anterior hace que se borren contenedores parados}}


Para conocer las estadísticas de uso de cada contenedor.
\begin{mycode}{Ver las estadísticas de los contenedores}{console}{{\scriptsize }}
root@vega:~# docker stats
CONTAINER ID  NAME    CPU %   MEM USAGE / LIMIT     MEM %   NET I/O          BLOCK I/O   PIDS
2fcf97530766  mi-web  0.00%   54.62MiB / 15.47GiB   0.34%   9.44MB / 172kB   0B / 0B     6
413d6e9f590f  mi-db   0.55%   357.9MiB / 15.47GiB   2.26%   319kB / 280B     0B / 0B     38
\end{mycode}




%    \graphicspath{{../../../anexos/sistemas_monitorizacion/img/}}
%    \chapter{Sistemas de Monitorización}

El sistema de monitorización se encarga de recopilar información acerca del estado de los servidores de una infraestructura. Entre las métricas y datos que debe recopilar se encuentran:

\begin{itemize}
    \item \textbf{Estado del servidor}: cantidad de RAM utilizada, estado de los discos duros, carga del servidor, sistema de ficheros,  …
    \item Estado de los \textbf{servicios que tiene el servidor}: servidor web, número de procesos que tiene, bases de datos, conexiones existentes, cola de correos electrónicos para enviar si es un servidor de correo, … Dependiendo del servicio habrá que realizar unas comprobaciones u otras.
    \item \textbf{Infraestructura en la que se encuentran}: estado de la red, conexión a otros servidores, …
    \item \textbf{Estado del clúster}: en caso de que el servidor pertenezca a un clúster, hay que comprobar que el clúster se encuentra en perfecto estado.
    \item \textbf{Dependencias externas}: un servidor puede depender a su vez de otros, o de servicios externos, que deben de estar funcionando de manera correcta.
\end{itemize}

Normalmente en la monitorización actúa un \textbf{agente} (o servicio) instalado en el equipo monitorizado que obtiene la información requerida que se envía a un servidor central que recopila la información de toda la infraestructura monitorizada.

Esta información suele ser almacenada durante un periodo de tiempo determinado (un año, por ejemplo) para poder ser usada y comparar la situación de los servidores a lo largo del tiempo. Gracias a esta comparación temporal \textbf{se puede llegar a predecir el estado del servidor} a unos días/semanas vista y evitar problemas antes de que sucedan (no tener espacio en discos duros, mal funcionamiento de servicios por falta de RAM, … ).

\infobox{\centering\textbf{La monitorización de servicios y equipos dentro de una infraestructura debe considerarse parte del proyecto, ya que es una parte muy importante de cara al mantenimiento del mismo.}}

\section{Monitorización de servidores}
Es habitual que los sistemas de monitorización funcionen en base a plantillas, que posteriormente se pueden asociar a los servidores monitorizados. Estas plantillas contendrán los servicios que deben ser monitorizados en cada tipo de servidor, ya que no es lo mismo monitorizar un servidor web o un servidor con un SGBD.

Para monitorizar un servidor lo habitual suele ser realizar las siguientes operaciones:

\begin{itemize}
    \item \textbf{Comprobar conectividad con el servidor}: suele ser habitual que los servidores estén fuera de nuestra red (en un proveedor de Internet, en un cliente, en otra oficina…), por lo que es necesario que exista conectividad de alguna manera para poder realizar la monitorización. En caso de no estar en nuestra red, el uso de una VPN es lo más utilizado.
    \item \textbf{Instalar un agente en el propio servidor}: Será el encargado de recopilar la información necesaria para mandarla al servidor central. Dependiendo del sistema de monitorización utilizado, necesitaremos un tipo de agente u otro. Algunos se encargan de realizar todas las comprobaciones y otros llaman a otros programas para realizar las comprobaciones y después mandar el resultado al servidor central.
    \item \textbf{Dar de alta el servidor en el sistema centralizado}: Tal como se ha dicho previamente, lo habitual es contar con un sistema centralizado en el que se tendrán todos los servidores y el estado de las comprobaciones realizadas. De ser así, habrá que darlo de alta, y para ello se necesitará:

    \begin{itemize}
        \item \textbf{Nombre del servidor}: Un nombre que a simple vista identifique el servidor. Suele ser habitual poner el nombre del cliente también, y/o el tipo de servicio que preste.
        \item \textbf{IP del servidor}: Para poder realizar la conexión al servidor.
        \item \textbf{Plantillas asociadas}: En caso de utilizar un sistema que utilice plantillas, al dar de alta el servidor se le aplicarán las plantillas necesarias para que realicen todos los checks oportunos. Por ejemplo: plantilla de Servidor Linux + plantilla de Servidor web + Plantilla de MySQL.
        \item \textbf{Puerto de conexión}:  Los agentes de monitorización suelen contar con un puerto que queda a la escucha. Si hemos cambiado el puerto, habrá que indicarlo a la hora de dar de alta.
        \item \textbf{Otras opciones}: Dependiendo del sistema de monitorización se podrán añadir muchas más opciones, como por ejemplo:
        \begin{itemize}
            \item \textbf{Servidores de los que se depende}: Imaginemos que el servidor monitorizado depende a su vez de un router que también está monitorizado. Si el router cae, no llegaríamos al servidor, por lo que realmente es una caída por dependencia, aunque el servidor puede estar funcionando de manera correcta. Esto nos puede permitir crear “árboles de dependencias” de servidores.
            \item \textbf{Periodos de monitorización}: Lo habitual es que un servidor esté monitorizado 24x7, pero quizá nos interese realizar cambios y que sólo se monitorice en unas horas determinadas (quizá el resto del tiempo está apagado).
            \item …
        \end{itemize}
    \end{itemize}
\end{itemize}

\section{Funcionamiento de la monitorización}
Para conocer cómo funciona un sistema de monitorización lo mejor es que tomemos como ejemplo un tipo de servicio que queremos monitorizar. Como ejemplo se puede tomar las comprobaciones que queremos realizar a un SGBD (Sistema Gestor de Bases de Datos).

No será lo mismo realizar la monitorización de un servidor MySQL o de un Oracle, pero las comprobaciones que queremos realizar en ellos deberían ser similares. Vamos a querer realizar la monitorización de las mismas comprobaciones: estado de las tablas en memoria, número de hilos en ejecución, número de \textit{slow\_queries}, …; pero los scripts ejecutados serán distintos.

\infobox{\textbf{La monitorización dependerá del propio servicio que vayamos a monitorizar.}}

A continuación se puede ver el estado de un servidor monitorizado a través del sistema de monitorización \href{https://www.centreon.com/}{Centreon}:

\begin{tcolorbox}[colback=white,title=Servidor monitorizado en Centreon]
  \includegraphics[width=\linewidth]{centreon.png}
\end{tcolorbox}



En la imagen anterior se puede comprobar un número de \textbf{\textit{checks}}, o comprobaciones, que se están realizando sobre un servidor concreto. Cada fila es una comprobación y contienen:

\begin{itemize}
    \item \textbf{Nombre del check/servicio}: Un nombre para identificar qué es lo que se está comprobando con el check.
    \item \textbf{Icono para mostrar gráficas}: Algunos checks recibirán información que puede ser graficada para así poder observar patrones en el comportamiento del servidor. Por ejemplo: cantidad de RAM ocupada, número de procesos en el sistema, número de conexiones a un servidor, …
    \item \textbf{Estado del check}: Normalmente, tras realizar la comprobación, el check termina con uno de los siguientes resultados:
    \begin{itemize}
        \item \textbf{OK}: El resultado obtenido es el correcto.
        \item \textbf{Warning}: El resultado obtenido está entre los márgenes de peligro. Es posible que de seguir así pase al estado siguiente:
        \item \textbf{Crítico}: El servicio devuelve un estado que es considerado crítico, lo que puede hacer que llegue a mal funcionamiento del mismo, o incluso que el servidor comience a dejar de funcionar (imaginemos que el servidor está con el 90\% de la RAM ocupada o de disco duro ocupado).
        \item \textbf{Indeterminado}: Por alguna razón, el \textit{check} no se ha realizado, o el valor devuelto es indeterminado o no se puede saber en qué otro estado situarlo.
    \end{itemize}
    \item \textbf{Duración del estado}: Para conocer cuánto tiempo lleva en el estado la comprobación obtenida. Lo ideal es que nunca haya estados que no sean OK y por lo tanto la duración de los mismos sea lo más alta posible.
    \item \textbf{Valor devuelto por la monitorización}: El valor real devuelto por la comprobación realizada. En base a este resultado se puede realizar las gráficas mencionadas previamente.
\end{itemize}

\infobox{\textbf{El estado del servicio dependerá del valor devuelto por la monitorización.}}

Este resultado se cotejará con los valores que hayamos puesto para que sea considerado OK, Warning o Critical. Es decir, \textbf{en algunos casos el estado del servidor depende de los valores devueltos y de la baremación que le hayamos otorgado}.

Pongamos como ejemplo la monitorización de un SGBD:
\begin{itemize}
    \item \textbf{El servicio del SGBD está funcionando}: Ahí no hay baremación posible. Si el servicio no está arrancado, es lógico pensar que el estado es crítico y que por tanto hay que ver qué ha ocurrido.
    \item \textbf{Número de conexiones en el SGBD}:  El resultado devuelto será un número entero (que podremos graficar para obtener patrones). En este caso, podemos decidir los rangos para que el resultado sea OK, Warning o Critical. Es decir, si el resultado obtenido está por debajo del umbral de Warning, el sistema considerará que el estado es OK. Si está en dicho rango, será Warning y si está en el rango de Critical, así lo indicará.
\end{itemize}

Esta baremación y \textbf{estos rangos} se suelen aplicar también en las plantillas de los servicios. Hay que entender que también \textbf{pueden ser modificados y personalizados para un servidor concreto}. No es lo mismo que un SGBD tenga 500 conexiones simultáneas si tiene 8Gb de RAM o si tiene 128Gb (en el primer servidor se puede considerar que es un estado crítico mientras que en el segundo es lo esperado).

Cuando un \textit{check} termina siendo un Warning o un Critical \textbf{es habitual que haya un sistema de alarmas configurado}. Dependiendo del sistema utilizado, notificará a los administradores mediante e-mail, mensajería instantánea, SMS, … para que realicen un análisis lo antes posible y solucionen el estado del servicio.

\infobox{\textbf{Los sistemas de monitorización suelen contar con un sistema de alarmas para que nos avise de los servicios caídos.}}


\subsection{Monitorización básica}
Tal como se ha comentado, en los servidores se suele realizar una monitorización del estado del mismo que suele ser común para todos, por lo que lo habitual suele ser tener una plantilla genérica para todos los servidores con la que se monitorizará:
\begin{itemize}
    \item Cantidad de RAM utilizada
    \item Cantidad de memoria virtual utilizada
    \item Carga de la CPU
    \item Espacio libre en las unidades de disco duro
    \item Estado del sistema RAID del servidor (en caso de tenerlo)
    \item Cantidad de usuarios conectados a la máquina
    \item Estado de puertos de conexión (SSH, por ejemplo)
    \item Latencia hasta llegar al servidor
    \item …
\end{itemize}

Es cierto que no será lo mismo monitorizar un sistema GNU/Linux o un sistema Windows (ya que puede variar alguno de las comprobaciones a realizar), pero el estado general que queremos conocer es el mismo. Por lo tanto, lo habitual es tener dos plantillas, una específica para servidores Windows y otra para GNU/Linux.

\def\test{sgbd}
\ifx\test\@minititle
  %% THIS PART ONLY IN SGBD BOOK
\subsection{Monitorización de SGBDs}
En el caso que nos ocupa, el de los Sistemas Gestores de Bases de Datos, aparte de la monitorización básica comentada previamente, necesitaremos monitorizar el estado del SGBD propiamente dicho. Para ello, de nuevo, se crearía una plantilla específica para cada SGBD que podamos tener en nuestra infraestructura. No será lo mismo monitorizar un servidor basado en MySQL o un Oracle, aunque muchos checks a comprobar deban ser lo mismo, pero la manera en la que se realizará la comprobación en el servidor será distinta.

Entre las comprobaciones que podemos realizar en un SGBD nos podemos encontrar con:
\begin{itemize}
    \item Servicio SGBD arrancado
    \item Cantidad de RAM utilizada por el SGBD
    \item Número de conexiones a las bases de datos
    \item Número de hilos en ejecución del SGBD
    \item Número de queries en ejecución
    \item Número de tablas en memoria
    \item Número de tablas bloqueadas
    \item …

\end{itemize}

\else
  %% THIS PART IN OTHER BOOKS
\subsection{Monitorización de Servicios}
Aparte de la monitorización básica comentada previamente, necesitaremos monitorizar el estado de los servicios que pueda tener el servidor propiamente dicho. Para ello, de nuevo, se crearía una plantilla específica para cada tipo de Servicio que podamos tener en nuestro servidor.

No será lo mismo monitorizar un servidor que tenga un servidor web, un servidor de base de datos, un proxy… O puede que el servidor cuente con todos esos servicios.

Es por eso que a la hora de realizar la monitorización de un servidor \textbf{es muy importante conocer qué funciones desempeña cada servidor en la infraestructura a la que pertenece} y analizar los servicios que tiene arrancados para posteriormente ser monitorizados.

\infobox{\textbf{Es muy importante conocer qué funciones desempeña cada servidor en la infraestructura a la que pertenece.}}
\fi

\section{Tipos de monitorización}
Existen varias maneras de realizar la monitorización de un servidor, y dependerá del gestor de monitorización que usemos (en caso de usar uno).

Es habitual que cuando nos referimos a sistemas de monitorización lo dividamos en dos grandes familias:
\begin{itemize}
    \item Monitorización Activa
    \item Monitorización Pasiva
\end{itemize}

Estas dos maneras de monitorización suelen ser excluyentes, aunque algunos sistemas de monitorización permiten ambas, por lo que nos puede interesar usar una u otra dependiendo de la situación.


\subsection{Monitorización pasiva}
En la monitorización pasiva el servidor (u objeto monitorizado) es el encargado de mandar la información de manera periódica al servidor central. El agente instalado se ejecutará como una tarea programada cada cierto tiempo (habitualmente unos pocos minutos) e informará de la situación cambiante, de haberla, al servidor central.

Esta manera de monitorización es utilizada también cuando no hay un servidor central. En este caso, si la comprobación ha sido incorrecta, podría mandar un mail al administrador del servidor.

\subsection{Monitorización activa}
Suele ser la manera habitual de proceder de los sistemas que cuentan con un servidor centralizado de monitorización. El servidor de monitorización se encarga de preguntar al servidor, a través de la conexión con el agente, por la comprobación de alguno de los checks, y el agente devuelve la información.

A continuación se puede observar las etapas que existen en un sistema de monitorización activa utilizando un servidor de monitorización central:

\begin{tcolorbox}[colback=white,title=Proceso de monitorización activa]
    \centering
    \includegraphics[width=0.6\linewidth]{monitorizacion_activa.png}
\end{tcolorbox}

Las etapas serían:
\begin{enumerate}
    \setcounter{enumi}{-1}
    \item El sistema de monitorización tiene un scheduler (o planificador) que decide cuándo tiene que realizar cada comprobación (normalmente, cada pocos minutos).
    \item El servicio encargado de monitorizar \textbf{establece conexión con el agente remoto} y le pide que compruebe un estado. En este ejemplo se ha optado por la RAM.
    \item El agente en el servidor que se quiere monitorizar recibe la notificación y realiza una \textbf{comprobación local} (normalmente llamando a scripts locales) para obtener la cantidad de RAM ocupada, total y libre que tiene
    \item El agente envía al sistema de monitorización el resultado obtenido en la ejecución de los scripts del paso anterior.
    \begin{enumerate}
        \item El monitorizador al recibir el resultado, lo coteja con los rangos de baremación que tiene y decide si el check está en estado OK, Warning o Critical.
        \item Lo habitual es que si el resultado del servicio no es OK, se ejecute en el servidor de monitorización algún tipo de alarma (ya sea enviar un mail, sistema de mensajería, … ) para notificar a los administradores.
    \end{enumerate}
    \item El sistema de monitorización guarda en una base de datos los resultados obtenidos para así poder realizar posteriores análisis o comprobaciones temporales de los mismos.
    \item Esos datos se suelen visualizar en una interfaz web, tal como hemos visto previamente.
\end{enumerate}

Estos pasos son ejecutados de manera continuada en el servidor de monitorización para cada comprobación que se realiza en cada uno de todos los servidores que se monitorizan. Por lo tanto, se entiende que el propio servidor de monitorización también tiene que ser monitorizado ya que es de vital importancia que su estado sea óptimo.

\subsection{Monitorización centralizada}
Como ya se ha comentado, es el sistema habitual de monitorización. Las ventajas que podemos obtener al hacer uso de este sistema son muchas, pero se pueden destacar las siguientes:
\begin{itemize}
    \item \textbf{Monitorización centralizada}: Aunque parezca obvio, el tener un único sistema en el que concentrar toda la información es muy útil y eficaz.
    \begin{itemize}
        \item La alternativa sería tener una monitorización distinta en cada servidor.
    \end{itemize}
    \item \textbf{Interfaz web}: Hoy en día suele ser habitual que los sistemas de monitorización tengan un servicio web en el que visualizar todos los datos obtenidos.
    \item \textbf{Sistema de plantillas}: De nuevo, es lo habitual, lo que hace que la gestión de monitorización de servidores sea más cómoda.
    \item \textbf{Gestión de usuarios}: Podremos tener usuarios que puedan ver unos servidores u otros, por lo que podemos tener equipos especializados en distintos grupos de monitorización y que sólo se enfoquen en ellos.
    \begin{itemize}
        \item Esto también es útil para dar acceso a los clientes a la monitorización de sus propios servidores.
    \end{itemize}
\end{itemize}

\subsection{Monitorización reactiva}
La monitorización reactiva se puede definir como el sistema de monitorización que no sólo se encarga de comprobar y recibir el estado de los servidores, si no que también reacciona a los mismos para tratar de solucionar los problemas encontrados. Tras esta definición está la idea de que \textbf{existen ciertos fallos recurrentes que no siempre necesitan la intervención humana para solucionarse}, y que por tanto, se puede tratar de ejecutar antes de que sea considerado un problema real.

Como \textbf{ejemplo sencillo} se puede poner \textbf{el espacio libre en disco duro}. Imaginemos que se comprueba que apenas hay espacio en el disco duro de un servidor. En este caso, el sistema de monitorización recibirá que el servidor \textbf{está al 99.95\%} de espacio ocupado, y por tanto, en lugar de notificar a un humano indicando el estado crítico, \textbf{el sistema reacciona de manera automática tratando de liberar espacio}. Se habrá configurado previamente que en la reacción de este error trate de borrar ficheros temporales, vaciar papelera, limpiar ficheros de caché de ciertas rutas … Una vez hecho esto, se volverá a comprobar el estado del servidor. Si el espacio ocupado en disco duro ha bajado y está en modo OK no habrá que hacer nada más, y se habrá evitado que un administrador tenga que realizar dicha tarea. Si por el contrario el estado sigue siendo incorrecto, el sistema notificará el error para que se realice un análisis y se solucione el problema.

Como \textbf{ejemplo extremo} (que no suele ser habitual configurarlo así), imaginemos que \textbf{la RAM consumida por un SGBD es muy alta} y esté poniendo en peligro el estado del servidor, se podría configurar para que \textbf{el sistema reaccione reiniciando el SGBD para que libere la RAM} y vuelva a prestar servicio.


\section{Gestores de monitorización}
Hoy día existen muchos sistemas de monitorización, y dependiendo de nuestras necesidades deberemos optar por uno u otro. A continuación se expondrán varios ejemplos de gestores de monitorización basados en Software Libre, aunque la gran mayoría de ellos cuentan con un sistema dual. Es decir, se puede descargar y montarlo en tu propio servidor o puedes contratar a la empresa para que ellos tengan el servicio central:
\begin{itemize}
    \item \textbf{\href{https://es.wikipedia.org/wiki/Nagios}{Nagios}}: Se puede considerar uno de los sistemas de monitorización más conocidos y del que se han basado otros. Generó mucha comunidad de administradores creando muchos scripts/plugins para hacerlos funcionar con él. Estos mismos scripts suelen ser utilizables en otros sistemas de monitorización.

    \item \textbf{\href{https://www.centreon.com/}{Centreon}}: Originalmente se creó como interfaz web para Nagios, pero poco a poco fue sustituyendo partes de Nagios hasta terminar siendo un sistema de monitorización completo. Existe la posibilidad de realizar la instalación por paquetes, descargar el sistema operativo en una ISO que te instala todo o incluso una máquina virtual con todo ya instalado y con configuración básica. (\href{https://demo.centreon.com/centreon/index.php}{Demo}).

    \item \textbf{\href{https://pandorafms.com/es/}{PandoraFMS}}: Sistema de monitorización creado por el español Sancho Lerena Urrea. Al igual que los anteriores, tiene sistema dual y la instalación se puede realizar por varios métodos.

    \item \textbf{\href{https://www.cacti.net/}{Cacti}}: Sistema más sencillo que los anteriores y habitualmente utilizado sólo en servidores sueltos, es decir, no de manera centralizada.

    \item \textbf{\href{https://munin-monitoring.org/}{Munin}}: Igual que el anterior, ideal para monitorizar unos pocos servidores, ya que no se puede considerar un sistema centralizado como los primeros. Ver \hyperlink{instalar_munin}{anexo de instalación de Munin}.
\end{itemize}

Existen otros sistemas de monitorización basados “en la nube”, cuya funcionalidad es similar a lo expuesto previamente. Para hacer uso de estos sistemas nos descargamos un agente, lo instalamos y se encargará de mandar la información a los servidores de la plataforma contratada. Lógicamente, dependiendo del gasto realizado obtendremos más o menos servicios. Entre este tipo de servicios se pueden destacar:
\begin{itemize}
    \item New Relic
    \item DataDog
\end{itemize}

\def\test{sgbd}
\ifx\test\@minititle
%% THIS PART ONLY IN SGBD BOOK

\section{Monitorizar MySQL}
Todo lo expuesto hasta ahora es referente a los sistemas de monitorización en general, y es similar en todos ellos. Tal como se ha comentado previamente, cuando un servidor cuenta con un servicio éste debe ser monitorizado, y dependiendo del servicio a monitorizar se realizará una serie de comprobaciones u otras.

A la hora de monitorizar un sistema gestor de bases de datos deberíamos tener en cuenta al menos las siguientes comprobaciones, y en el caso de MySQL:
\begin{itemize}
    \item Número de conexiones actuales (usuarios/conexiones tiene el servicio)
    \item Tiempo del servicio activo (uptime)
    \item Número de hilos en ejecución
    \item Tamaño de memoria utilizado
    \item Tamaño de memoria caché utilizado
    \item Número de tablas en memoria
    \item Número de slow queries
    \item Número de conexiones abortadas
    \item Número de queries totales
\end{itemize}

En servidores en alta disponibilidad, deberíamos comprobar, en caso de un clúster:
\begin{itemize}
    \item Número de nodos en el clúster
    \item Estado general del clúster
    \item Latencia entre los nodos del clúster
    \item Si el nodo está conectado al clúster
    \begin{itemize}
        \item  Puede pasar que un nodo “se salga” del clúster (o lo saquemos) para realizar mantenimiento
    \end{itemize}
\end{itemize}

Si el servidor está dentro de una infraestructura de \textbf{Primario → Réplica}:
\begin{itemize}
    \item Estado de la replicación
    \item Retraso de la replicación
    \item Tamaño del binlog
\end{itemize}

Las comprobaciones expuestas son un mero ejemplo, y existen muchas más que podemos realizar a nuestros sistemas. Para poder realizar este tipo de monitorización podemos hacer uso de scripts propios (ya que muchas de las comprobaciones son consultas a variables de estado de MySQL), scripts creados por otras personas o scripts de monitorización hechos de manera exclusiva para MySQL.

\subsection{Scripts propios}
Para realizar gran parte de los checks expuestos previamente se puede hacer uso de scripts propios, ya que pueden realizar consultas de las variables de estado para comprobar y determinar si el estado es correcto.

Un ejemplo de script creado \textit{ad hoc} para nuestro sistema podría ser:

\begin{mycode}{Script propio para monitorizar}{bash}{}
#!/bin/bash
mysql -e "SHOW STATUS LIKE 'Slow_queries';" -N | awk '{print $2}'
\end{mycode}

\subsection{Percona monitoring tools}
La alternativa a hacer uso de scripts propios es usar programas o scripts realizados por terceros, que estén ampliamente probados y que tienen detrás un equipo que quizá sea más amplio que el nuestro, y por tanto esté mejor probado.

La empresa Percona tiene un par de soluciones sobre monitorización de SGBDs (no sólo para MySQL, también para \href{https://www.postgresql.org/}{PostgreSQL} y \href{https://www.mongodb.com/}{MongoDB}), que son:
\begin{itemize}
    \item \href{https://www.percona.com/software/database-tools/percona-monitoring-and-management}{\textbf{Percona Monitoring and Management}}: Un sistema completo de monitorización, interfaz web incluída, que podremos \href{https://www.percona.com/software/pmm/quickstart}{instalar} en nuestros propios servidores. Podemos ver aquí una \href{https://pmmdemo.percona.com/graph/}{demo}.

    \includegraphics[width=\linewidth]{percona_tools.png}

    \item \href{https://www.percona.com/software/database-tools/percona-monitoring-plugins}{\textbf{Percona Monitoring Plugins}}: En este caso son un conjunto de plugins (o scripts) que podremos utilizar en nuestro sistema de monitorización propio (Nagios, Centreon o Cacti). En la \href{https://www.percona.com/doc/percona-monitoring-plugins/LATEST/index.html}{documentación} explican cómo hacer uso de ellos.
\end{itemize}

\fi

\clearpage
%
%    \part{Anexos}
%    \graphicspath{{../../../anexos/}}
%    \chapter{Glosario}

A continuación se expone un glosario de términos con sus correspondientes definiciones:

\begin{description}
    \hypertarget{altadisponibilidad}{}
    \item[Alta Disponibilidad:] Es un diseño de arquitectura de sistemas y la implementación que asegura que el servicio instalado y otorgado sea funcional sin que haya parada en el mismo. Esta arquitectura trata de que no haya ningún \hyperlink{spf}{\textit{single point of failure} (punto único de fallo)} en la misma.

    \hypertarget{cluster}{}
    \item[Clúster:] Se denomina clúster a un conjunto de ordenadores unidos entre sí mediante conectividad de red que actúan como si de un único servidor se tratara. Dependiendo del tipo de clúster que se va a crear, debe de ser pensado desde el diseño del servicio, ya que es la aplicación o servicio quién se encarga de crear el clúster (como ocurre con MySQL Cluster).

    \hypertarget{dependencia_software}{}
    \item [Dependencia de software:] Cuando se crea cualquier tipo de software lo habitual es hacer uso de otro software (librerías de seguridad, acceso a disco, codecs de vídeo, librerías 3D…) que son necesarias para el correcto funcionamiento de nuestro programa. Este otro software (que puede ser propio o ajeno) se denomina \textbf{dependencia}, ya que sin él, nuestro programa no funcionará y es necesario que exista en el sistema para hacer funciona nuestro programa.

    En las \hyperlink{distribucion_gnu_linux}{distribuciones GNU/Linux} se hace uso de los denominados \hyperlink{paquete_de_software}{paquetes de software} en los cuales se indican las dependencias que necesitan para funcionar y que por tanto se instalarán a la par que el programa elegido, por lo que nos aseguramos que el software instalado funcionará en cuanto termine la instalación.

    En caso de descargar un software ajeno de los \hyperlink{repositorio_de_software}{repositorios} oficiales de la distribución, es posible que necesitemos completar esas dependencias por nuestra cuenta, pero hoy en día es habitual que los creadores de software lo tengan en cuenta y esas dependencias estén en los repositorios oficiales.

    \hypertarget{distribucion_gnu_linux}{}
    \item [Distribución GNU/Linux:] Es una distribución de software basada en el núcleo Linux que incluye software \hyperlink{gnu}{GNU} para componer un Sistema Operativo que pueda ser utilizado por los usuarios. Cada distribución suele \hyperlink{paquete_de_software}{empaquetar el software} en un formato propio que aparte del propio software indica las \hyperlink{dependencia_de_software}{dependencias} de software que necesita para funcionar, por lo que hace que la instalación del software se realice de manera sencilla. El software de la distribución está almacenado en los \hyperlink{repositorio_de_software}{repositorios de software} oficiales de la distribución.

    Las distribuciones suelen estar orientadas para un uso generalizado, pero es cierto que algunas, por su historia o por su manera de entender el empaquetado de software, se necesitan más conocimientos, pero hoy en día no es lo habitual.

    Existen muchas distribuciones GNU/Linux, pero las que podríamos destacar son \hyperlink{ubuntu}{Ubuntu}, Debian, Red Hat y CentOS, que son las de mayor uso hoy en día a nivel profesional.

    \hypertarget{escalado_horizontal}{}
    \item[Escalado Horizontal:] Se llama escalado horizontal a la infraestructura que crece de manera horizontal añadiendo más servidores del mismo servicio. Estos servidores serán accesibles mediante un proxy o de manera directa, y todos contarán con el mismo servicio (web, base de datos, …). No confundir con un clúster, ya que la relación de los servidores en el escalado horizontal no tienen por qué ir en clúster.

    \hypertarget{escalado_vertical}{}
    \item[Escalado Vertical:] Es el incremento de hardware de un servidor. Supongamos que un servidor empieza a tener problemas de carga, pues con el escalado vertical se le añadiría más RAM, más procesador y/o discos duros más rápidos (en caso de ser una máquina virtual sería sencillo, en caso contrario habría que realizar la migración a un servidor nuevo).

    \hypertarget{gnu}{}
    \item[GNU:] Del acrónimo \textbf{GNU’s Not Unix} (GNU no es Unix) es un sistema operativo y un conjunto de programas libres cuyo origen surgió de la idea de crear un sistema operativo Unix basado en \hyperlink{software_libre}{Software Libre}.

    El desarrollo de GNU nació en 1983 por Richard Stallman comenzando por el compilador GCC, al que se fueron uniendo todo tipo de software y creando la Free Software Foundation (o FSF, fundación por el software libre) la cual creó la \hyperlink{licencias_libres}{licencia libre} más conocida actualmente: la \textbf{GPL} (GNU General Public License).

    El proyecto GNU avanzó en el tiempo y creó el kernel Hurd, pero bien es cierto que nunca llegó a ser funcional del todo y actualmente el kernel más utilizado es Linux, pero no es el único, ya que el software GNU también es usado en conjunto con otros kernels como son los \textbf{*BSD}, de ahí la importancia que cuando hacemos referencia al sistema operativo se haga uso de \hyperlink{gnu_linux}{GNU/Linux}.

    \hypertarget{gnu_linux}{}

    \itemimage{GNU/Linux:}{r}{0.21}
    {img/Gnulinux.svg.png}
    {\href{https://es.wikipedia.org/wiki/GNU/Linux\#/media/Archivo:Gnulinux.svg}{GNU/Linux: Wikipedia}}
    {
        Aunque comúnmente solemos llamar a las \hyperlink{distribucion_gnu_linux}{distribuciones} como “Linux” esto no suele ser correcto ya que en la distribución aparte del kernel va un conjunto enorme de software del proyecto GNU. Por lo tanto, lo ideal siempre es hacer uso del nombre completo GNU/Linux.

        El proyecto \hyperlink{gnu}{GNU} y sus herramientas y software son usados con otros kernels como son los *BSD en distribuciones como FreeBSD u OpenBSD. También existen versiones con kernel BSD para la distribución Debian, por lo que en ese caso sería “Debian GNU/BSD”.
    }


    \hypertarget{json}{}
    \item[JSON:] Es un formato de texto sencillo para el intercambio de datos. Aunque originalmente fue creado como notación de objetos para Javascript, su amplia utilización ha hecho que sea utilizado como alternativa a XML.


    \hypertarget{licencias_libres}{}
    \item[Licencias libres:] Una licencia de software es un contrato entre el creador (o el titular de los derechos de autor) del software y el usuario. Todo software que usamos suele exigir la lectura de esta licencia y es por ello muy importante conocer qué se puede y no se puede hacer con dicho software.

    Las licencias libres son aquellas que nos permiten hacer con el software lo que las cuatro libertades del \hyperlink{software_libre}{Software Libre} exige.

    Entre las licencias libres más utilizadas hoy en día están la GPL (General Public License del proyecto \hyperlink{gnu}{GNU}), la Apache License, algunas de las versiones de las licencias Creative Commons, …


    \hypertarget{linux}{}
    \item[Linux:] Creado originalmente por Linus Torvalds en 1991 y actualmente desarrollado por cientos de desarrolladores de todo el mundo, Linux es el núcleo (o kernel) gratuito y libre similar al núcleo de los sistemas operativos Unix.

    Comenzó como un proyecto personal de Linus (siendo estudiante universitario) para su ordenador 386 y actualmente está portado a \href{https://es.wikipedia.org/wiki/Portabilidad\_del\_n\%C3\%BAcleo\_Linux\_y\_arquitecturas\_soportadas}{decenas de plataformas hardware}. Es el proyecto más grande y ambicioso del \hyperlink{software_libre}{Software Libre}, aunque originalmente no se permitía el uso comercial del mismo (hasta la versión 0.12).

    Al poco tiempo de comenzar su desarrollo el proyecto \hyperlink{gnu}{GNU} lo adoptó como su kernel naciendo lo que actualmente conocemos como \hyperlink{gnu_linux}{GNU/Linux} y con ello cientos de \hyperlink{distribucion_gnu_linux}{distribuciones}.

    Es un núcleo de tipo monolítico que permite la carga de módulos en tiempo de ejecución


    \hypertarget{lts}{}
    \item[LTS:] Del inglés \textit{\textbf{L}ong \textbf{T}erm \textbf{S}upport} (en castellano “soporte a largo plazo”), es una característica en informática que hace referencia a versiones especiales de software que contarán con un soporte más largo del habitual, por lo que serán las versiones idóneas para usar en servidores.

    Estas versiones suelen contar con actualizaciones de seguridad, pero no con cambios notorios en la forma del software para fomentar la fiabilidad del mismo. Lo habitual es utilizar este tipo de versiones en servidores, que aunque puedan no tener las últimas modificaciones de las versiones más recientes del software, nos aseguramos la fiabilidad. Esto hace que tengamos que decidir si es necesario contar con las características de las últimas versiones (ya sea nuevos servicios, opciones nuevas, velocidad, … ) o si preferimos contar con una versión que tendrá un ciclo de vida más longevo pero con actualizaciones de seguridad.

    Es habitual verlo en proyectos de \hyperlink{software_libre}{Software Libre}, como ejemplos podemos tomar el kernel \hyperlink{linux}{Linux} (actualmente la versión 5.4.58 es la denominada LTS) y la distribución \hyperlink{ubuntu}{Ubuntu} (en este caso la versión 20.04).


    \hypertarget{paquete_de_software}{}
    \item[Paquetes de Software:] Un paquete de software no es más que una manera de poder distribuir el software creado. En \hyperlink{distribucion_gnu_linux}{distribuciones GNU/Linux} estos paquetes determinan las \hyperlink{dependencia_software}{dependencias} que necesitan para que su instalación sea lo más sencilla posible.

    Lo habitual es que estos paquetes estén gestionados mediante un sistema de gestión propio para conocer cuáles están instalados, sus dependencias, desinstalarlos de manera sencila...

    No sólo se usa en distribuciones GNU/Linux, ya que varios lenguajes de programación hacen lo propio para distribuir software en forma de paquetes. Como ejemplos:
    \begin{itemize}
        \item En distribuciones GNU/Linux tenemos APT, Yum, Zypper, Portage, ...
        \item En lenguajes de programación tenemos Gem para Ruby, Eggs para Python, CPAN en Perl, ...
    \end{itemize}


    \hypertarget{repositorio_de_software}{}
    \item[Repositorio de Software:] Se podría denominar repositorio como el almacén donde se guardan los \hyperlink{paquete_de_software}{paquetes de software}. Las \hyperlink{distribucion_gnu_linux}{distribuciones GNU/Linux} cuentan con sus repositorios oficiales, donde se almacena el software para cada versión que tiene la distribución.

    Aparte del software que podemos instalar, también cuentan con un índice para saber los paquetes y las versiones que se almacena en ellos. Este índice es necesario que lo actualicemos de manera periódica (en Ubuntu ejecutando: “apt update”) ya que gracias a él sabremos si tenemos que realizar actualizaciones de los paquetes instalados.

    También podemos utilizar repositorios externos al de la distribución, repositorios oficiales de un software por ejemplo, que nos permiten instalar la última versión de ese software sobre nuestra distribución. Cuando un paquete con el mismo nombre existe en distintos repositorios, siempre se instalará del repositorio que tenga la versión más nueva.

    No es buena práctica, y \textbf{está completamente desaconsejado}, mezclar repositorios de distribuciones distintas aunque el gestor de paquetes sea el mismo (usar repositorios de Debian en Ubuntu o viceversa).


    \hypertarget{spf}{}
    \item[Single Point of Failure:] O punto único de fallo, es un componente de un sistema que tras un fallo en su funcionamiento ocasiona un fallo global en el sistema completo, dejándolo inoperante. Un SPOF puede ser un componente de hardware, software o eléctrico.


    \hypertarget{software_libre}{}
    \item[Software Libre:] El movimiento del Software Libre fue creado por Richard Stallman a la par que creaba el proyecto \hyperlink{gnu}{GNU}. Para que un software sea considerado como Software Libre debe contener una \hyperlink{licencias_libres}{licencia libre} que debe otorgar las cuatro libertades siguientes:
    \begin{itemize}
        \item Libertad de usar el software para cualquier propósito.
        \item Libertad de estudiar el software y su funcionamiento interno (es por ello necesario poder acceder al código fuente).
        \item Libertad de distribuir el software con quien queramos.
        \item Libertad de poder modificar y mejorar el software según nos interese.
    \end{itemize}

    Es muy importante tener en cuenta que Software Libre no significa gratis, ya que en inglés el término viene de Free Software donde “Free” puede significar libre y gratis. Es cierto que la gran mayoría del Software Libre puede ser gratis, pero no todo el software gratis es Software Libre.


    \hypertarget{ssh_server}{}
    \item [SSH Server]: De \textbf{S}ecure \textbf{SH}ell, es el nombre de un protocolo y del programa (tanto servidor, como cliente) cuya función principal es la de acceder de manera remota a través de un canal seguro a un servidor.

    SSH permite no sólo la conexión a un servidor sino también la transferencia de ficheros y creación de túneles cifrados por los que pueden viajar otros protocolos. El puerto habitual de uso para este protocolo es el \textbf{22}.


    \hypertarget{systemd}{}
    \item [Systemd]: es un conjunto de demonios de administración de sistema, bibliotecas y herramientas diseñados como una plataforma de administración y configuración central para interactuar con el núcleo del Sistema operativo \hyperlink{gnu_linux}{GNU/Linux}.


    \hypertarget{ubuntu}{}

    \itemimage{Ubuntu:}{r}{0.21}
    {img/ubuntu.svg.png}
    {\href{https://es.wikipedia.org/wiki/Ubuntu}{Ubuntu: Wikipedia}}
    {
        Es una \hyperlink{distribucion_gnu_linux}{distribución de GNU/Linux} originalmente basada en Debian y creada por la compañía Canonical en el 2004. En su momento fue una de las distribuciones que apostaron por un sistema de instalación sencillo y con la intención de detectar el máximo hardware posible para acercarse a la gran cantidad de usuarios posibles.

        Hoy en día es una de las distribuciones más utilizadas tanto a nivel de escritorio como a nivel de servidores ya que cuenta con dos versiones separadas a la hora de realizar la instalación (aunque realmente es la misma distribución).

        Una de sus ventajas es la creación de versiones \hyperlink{lts}{LTS} cada dos años, que son versiones que garantizan su soporte técnico durante más tiempo por lo que supone una ventaja a la hora de realizar la instalación en servidores. Con ellos nos aseguramos que el software va a ser actualizado ante fallos de seguridad durante más tiempo que las versiones que no son LTS.
    }


\end{description}

\clearpage


\end{document}