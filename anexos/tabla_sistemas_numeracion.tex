\hypertarget{tabla_conversiones_directas}{}

\chapter{Sistemas de numeración: tabla de conversión}
La siguiente tabla sirve a modo de resumen de los sistemas de numeración.

\infobox{Esta tabla no hay que aprenderla de memoria.
    \textbf{HAY QUE SABER CREARLA}. Para ello hay que entender los sistemas de numeración. }

\begin{table}[H]
    \centering
    \tablestyle
    \begin{tabular}{|C{0.22\linewidth}|C{0.22\linewidth}|C{0.22\linewidth}|C{0.22\linewidth}|}
        \theadstart
        \thead \textbf{Decimal} &
        \thead \textbf{Binario} &
        \thead \textbf{Octal} &
        \thead \textbf{Hexadecimal} \tabularnewline
        \tbody
        $  0_{(10} $  & $     0_{(2} $  & $  0_{(8} $   & $  0_{(16} $  \\
        $  1_{(10} $  & $     1_{(2} $  & $  1_{(8} $   & $  1_{(16} $  \\
        $  2_{(10} $  & $    10_{(2} $  & $  2_{(8} $   & $  2_{(16} $  \\
        $  3_{(10} $  & $    11_{(2} $  & $  3_{(8} $   & $  3_{(16} $  \\
        $  4_{(10} $  & $   100_{(2} $  & $  4_{(8} $   & $  4_{(16} $  \\
        $  5_{(10} $  & $   101_{(2} $  & $  5_{(8} $   & $  5_{(16} $  \\
        $  6_{(10} $  & $   110_{(2} $  & $  6_{(8} $   & $  6_{(16} $  \\
        $  7_{(10} $  & $   111_{(2} $  & $  7_{(8} $   & $  7_{(16} $  \\
        $  8_{(10} $  & $  1000_{(2} $  & $ 10_{(8} $   & $  8_{(16} $  \\
        $  9_{(10} $  & $  1001_{(2} $  & $ 11_{(8} $   & $  9_{(16} $  \\
        $ 10_{(10} $  & $  1010_{(2} $  & $ 12_{(8} $   & $  A_{(16} $  \\
        $ 11_{(10} $  & $  1011_{(2} $  & $ 13_{(8} $   & $  B_{(16} $  \\
        $ 12_{(10} $  & $  1100_{(2} $  & $ 14_{(8} $   & $  C_{(16} $  \\
        $ 13_{(10} $  & $  1101_{(2} $  & $ 15_{(8} $   & $  D_{(16} $  \\
        $ 14_{(10} $  & $  1110_{(2} $  & $ 16_{(8} $   & $  E_{(16} $  \\
        $ 15_{(10} $  & $  1111_{(2} $  & $ 17_{(8} $   & $  F_{(16} $  \\
        $ 16_{(10} $  & $ 10000_{(2} $  & $ 20_{(8} $   & $ 10_{(16} $  \\
        \tend
    \end{tabular}
    \vspace{-10pt}
\end{table}

Tal como se puede ver, los primeros símbolos coinciden en algunas de las bases mientras el símbolo representado sea menor que las bases que comparemos, pero luego no tienen nada que ver. Por eso, hay que insistir, que lo importante de esta tabla no es aprenderla, sino saber hacerla.

\clearpage